\documentclass[format=acmsmall, review=false, screen=true]{acmart}

\renewcommand{\texttt}[1]{%
  \begingroup
  \ttfamily
  \begingroup\lccode`~=`/\lowercase{\endgroup\def~}{/\discretionary{}{}{}}%
  \begingroup\lccode`~=`[\lowercase{\endgroup\def~}{[\discretionary{}{}{}}%
  \begingroup\lccode`~=`.\lowercase{\endgroup\def~}{.\discretionary{}{}{}}%
  \catcode`/=\active\catcode`[=\active\catcode`.=\active
  \scantokens{#1\noexpand}%
  \endgroup
}

\usepackage{booktabs} % For formal tables
\usepackage{tikz}
\usetikzlibrary{positioning,fit,calc,shapes,shadows,arrows}
\usepackage{forest}
\usetikzlibrary{arrows.meta}
\usepackage[ruled]{algorithm2e} % For algorithms
\renewcommand{\algorithmcfname}{ALGORITHM}
\SetAlFnt{\small}
\SetAlCapFnt{\small}
\SetAlCapNameFnt{\small}
\SetAlCapHSkip{0pt}
\IncMargin{-\parindent}
\usepackage{hyperref}
\usepackage[normalem]{ulem}

% Metadata Information
\acmJournal{TOSEM}
%\acmVolume{}
%\acmNumber{}
%\acmArticle{}
%\acmYear{}
%\acmMonth{}
\copyrightyear{2018}
%\acmArticleSeq{9}

% Copyright
%\setcopyright{acmcopyright}
\setcopyright{acmlicensed}
%\setcopyright{rightsretained}
%\setcopyright{usgov}
%\setcopyright{usgovmixed}
%\setcopyright{cagov}
%\setcopyright{cagovmixed}

% DOI
\acmDOI{0000001.0000001}

% Paper history
%\received{February 2007}
%\received[revised]{March 2009}
%\received[accepted]{June 2009}

\newcommand{\dl}{drasil-lang}
\newcommand{\dc}{drasil-code}
\newcommand{\dd}{drasil-data}
\newcommand{\de}{drasil-example}
\newcommand{\dg}{drasil-gen}
\newcommand{\drp}{drasil-printers}

\newcommand{\edl}{\emph{\dl}}
\newcommand{\edc}{\emph{\dc}}
\newcommand{\edd}{\emph{\dd}}
\newcommand{\ede}{\emph{\de}}
\newcommand{\edg}{\emph{\dg}}
\newcommand{\edp}{\emph{\drp}}

%% Comments
\newif\ifcomments\commentstrue

\ifcomments
\newcommand{\authornotes}[3]{\textcolor{#1}{[#3 ---#2]}}
\newcommand{\todo}[1]{\textcolor{red}{[TODO: #1]}}
\else
\newcommand{\authornotes}[3]{}
\newcommand{\todo}[1]{}
\fi

\newcommand{\wss}[1]{\authornotes{blue}{SS}{#1}}
\newcommand{\jc}[1]{\authornotes{red}{JC}{#1}}
\newcommand{\ds}[1]{\authornotes{olive}{DS}{#1}}

% Document starts
\begin{document}

%%%%%%%%%%%%%%%%%%%%%
% Tikz Styles
\tikzstyle{class}=[rectangle, draw=black, rounded corners, fill=white!40, drop 
shadow,
text centered, anchor=north, text=black, text width=3cm]
\tikzstyle{myarrow}=[->, >=triangle 90, thick]
%%%%%%%%%%%%%%%%%%%%%

% Title portion. Note the short title for running heads
\title[headertitle]{TITLE}

\author{Dan Szymczak}
%\orcid{1234-5678-9012-3456}
\affiliation{%
  \institution{McMaster University}
  \streetaddress{1280 Main St. W.}
  \city{Hamilton}
  \state{ON}
  \postcode{L8S 4K1}
  \country{Canada}}
\email{szymczdm@mcmaster.ca}
\author{Jacques Carette}
%\affiliation{%
%  \institution{Inria Paris-Rocquencourt}
%  \city{Rocquencourt}
%  \country{France}
%}
%\email{beranger@inria.fr}
\author{Spencer Smith}
%\affiliation{%
% \institution{Rajiv Gandhi University}
% \streetaddress{Rono-Hills}
% \city{Doimukh}
% \state{Arunachal Pradesh}
% \country{India}}
%\email{aprna_patel@rguhs.ac.in}
%\author{Huifen Chan}
%\affiliation{%
%  \institution{Tsinghua University}
%  \streetaddress{30 Shuangqing Rd}
%  \city{Haidian Qu}
%  \state{Beijing Shi}
%  \country{China}
%}
%\email{chan0345@tsinghua.edu.cn}
%\author{Ting Yan}
%\affiliation{%
%  \institution{Eaton Innovation Center}
%  \city{Prague}
%  \country{Czech Republic}}
%\email{yanting02@gmail.com}
%\author{Tian He}
%\affiliation{%
%  \institution{University of Virginia}
%  \department{School of Engineering}
%  \city{Charlottesville}
%  \state{VA}
%  \postcode{22903}
%  \country{USA}
%}
%\affiliation{%
%  \institution{University of Minnesota}
%  \country{USA}}
%\email{tinghe@uva.edu}
%\author{Chengdu Huang}
%\author{John A. Stankovic}
%\author{Tarek F. Abdelzaher}
%\affiliation{%
%  \institution{University of Virginia}
%  \department{School of Engineering}
%  \city{Charlottesville}
%  \state{VA}
%  \postcode{22903}
%  \country{USA}
%}


\begin{abstract}

CONTEXT: Software (re-)certification requires the creation and maintenance of
many different software artifacts. Manually creating and maintaining \wss{and
reusing?} them is tedious and costly. \wss{and error prone}

OBJECTIVE: Improve software (re-)certification efforts by automating as much of
the artifact creation process as possible, while maintaining full traceability
within -- and between -- artifacts.  %DS Secondary objective -> Knowledge reuse 
%-- Don't know if I want
% to focus here as it muddies the waters.

METHOD: %Use grounded theory in the creation of a tool for software artifact
%generation. 
Start by analyzing the artifacts themselves from several case 
studies to understand what (semantically) is being said in each.
Capture the underlying knowledge and apply transformations to create each of 
the requisite artifacts through a generative approach.  

RESULTS: Case studies -- GlassBR to show capture and transformation. SWHS and
NoPCM for reuse (Something about Kolmogorov complexity / MDL here?).
% Moved from Method:
Captured knowledge can 
be re-used across projects as it represents the ``science''. Maintenance
involves updating the captured knowledge or transformations as necessary.
%Moved from Objective:
Creation of our tool -- Drasil -- facilitates this automation process using a 
knowledge-based approach to Software Engineering.  \wss{Maybe add something
  about the infrastructure now being in place to reuse/grow the scientific and
  computing knowledge base to cover new case studies?  It would be nice if we
  could make the connection between Drasil's knowledge and existing scientific
  knowledge ontologies, but maybe it is too early for that connection?}

CONCLUSIONS: With good tool support and a front-loaded time investment, we can 
automate the generation of software artifacts required for certification. 
(fill in later)?????

\wss{The abstract doesn't have anything from Jacques ``bottom up'' viewpoint.
  Can you work in something about consistency by construction?}

\end{abstract}


%
% The code below should be generated by the tool at
% http://dl.acm.org/ccs.cfm
% Please copy and paste the code instead of the example below.
%
%\begin{CCSXML}
%<ccs2012>
% <concept>
%  <concept_id>10010520.10010553.10010562</concept_id>
%  <concept_desc>Computer systems organization~Embedded systems</concept_desc>
%  <concept_significance>500</concept_significance>
% </concept>
% <concept>
%  <concept_id>10010520.10010575.10010755</concept_id>
%  <concept_desc>Computer systems organization~Redundancy</concept_desc>
%  <concept_significance>300</concept_significance>
% </concept>
% <concept>
%  <concept_id>10010520.10010553.10010554</concept_id>
%  <concept_desc>Computer systems organization~Robotics</concept_desc>
%  <concept_significance>100</concept_significance>
% </concept>
% <concept>
%  <concept_id>10003033.10003083.10003095</concept_id>
%  <concept_desc>Networks~Network reliability</concept_desc>
%  <concept_significance>100</concept_significance>
% </concept>
%</ccs2012>
%\end{CCSXML}

%\ccsdesc[500]{Computer systems organization~Embedded systems}
%\ccsdesc[300]{Computer systems organization~Redundancy}
%\ccsdesc{Computer systems organization~Robotics}
%\ccsdesc[100]{Networks~Network reliability}

%
% End generated code
%


\keywords{??}


\maketitle

% The default list of authors is too long for headers.
\renewcommand{\shortauthors}{D.\ Szymczak et al.}
\newcommand{\fgr}{\textcolor{red}{!FIGURE!}}
\section{Introduction}\label{S:Intro}

Writing non-executable software artifacts (requirements and design documents,
verification \& validation plans, etc.) can be tedious work, but is 
ultimately necessary when attempting to certify software. 
Similarly, maintenance of these artifacts, as necessary for re-certification as 
improvements are made, typically requires a large time investment.

Why, in a world of software tools, do we continue to undertake these
efforts manually? Literate programming had the right idea, but was too heavily 
focused on code.

We want to aid software (re-)certification efforts by automating as much of the 
artifact creation process as possible. By generating our software artifacts -- 
including code -- in the right way, we can implement changes much more quickly 
and easily for a modest up-front time investment. By front-loading the costs of 
maintenance and rolling them into the development cycle, we can save time and 
money in the long run.

\wss{The bottom up justification should also be part of the introduction.}

\ds{bottom up justification can go here?}

We \ds{and others -- citation ?} have observed a number of issues arise 
throughout a standard software development and maintenance life-cycle.
Information duplication is a major concern as several problems occur therein. 
The first, most trivial, and incredibly tedious is that of the necessity to 
duplicate information across software artifacts. A piece of knowledge that 
appears in the Software Requirements Specification (SRS) will undoubtedly 
appear in the design and implementation artifacts (documents/source code). This 
knowledge \emph{may} be transformed, but in many cases is an exact copy 
(Ctrl-C \& Ctrl-V anyone?) and a 
massive waste of time for a human to replicate throughout multiple 
artifacts.

Duplication also becomes a time-sink for maintenance, particularly as the 
duplicated information tends to fall out of sync across artifacts over the 
course of a piece of software's lifetime. For example, a common problem is that 
of updating the source code, but not updating any of the design documents to 
reflect the change. These inconsistencies are at best a minor headache and at 
worst, in the case of software certification, an expensive lesson. If the 
artifacts are inconsistent when submitted for (re-)certification, the software 
will not be (re-)certified and an extra cost will be incurred (both in labour, 
and from the certifying body) to attempt certification again.

\ds{Fill in this part about consistency, variabilities, and commonalities}

\begin{table}
	\caption{Problems and Proposed Solutions}
	\label{Tab:ProbSol}
	\begin{tabular}[width=0.5\textwidth]{l | l}		
		Problem 		& Technical Solution 					\\
		\hline
		Duplication		& Databases (global \& system-specific) \\
						& single-source \& generation *			\\
		\hline
		Consistency 	& 	Sanity checkers						\\
		(incl. Source/	& 										\\
		generated content)	&  									\\
		\hline
		Variability \&	& Software families \& 					\\
		Commonality		& Generation							\\
	\end{tabular}

\textit{* Also provides free \& easy reuse}
\end{table}

Table~\ref{Tab:ProbSol} gives a brief rundown of our proposed solutions to the 
problems mentioned above. We are leveraging our experience with generative 
programming to provide a solution that is applicable to a wide range of 
problems.

%DS Maybe throw in a figure here to show off a bunch of common artifacts.

%DS Should everything after this point go into Background? Maybe not

\wss{We can certainly wait to do it, but I like a ``roadmap'' paragraph at the
  end of the introduction.  The roadmap explains how the paper is organized.
  More importantly, it provides an opportunity to present your ``story,''
  assuming that the organization of the paper follows the story.}

\subsection{Software (Re-)certification}

When we talk about software certification, we are specifically discussing the 
goal of determining ``based on the principles of science, engineering and 
measurement theory, whether an artifact satisfies accepted, well defined and 
measurable criteria''~\cite{HatcliffEtAl2009}. Essentially, we are ensuring 
that software, or some piece of it, performs a given task to within an 
acceptable standard and can potentially be reused in other systems.

Software certification is necessary by law in certain fields. This is 
particularly evident in safety-critical applications such as control systems 
for nuclear power plants or X-ray machines. 

Different certifying bodies exist across domains and each has their own list of 
requirements to satisfy for certifying software. Looking at some 
examples \cite{CSA1999,CSA2009,CDRH2002,FDA2014} there are many pieces
of requisite documentation including, but not limited to:

\begin{itemize}
\item Problem definition
\item Theory manual
\item Requirements specification
\item Design description
\item Verification and Validation (V\&V) report
\end{itemize}

We should keep in mind that we require full traceability -- inter- and 
intra-artifact -- of the knowledge contained within these artifacts. That is, 
we should be able to find an explicit link between our problem definition and 
theory manual, down to our requirements, design, and other development planning 
artifacts. From there, we should be able to continue through our proposed 
verification and validation plans, and should eventually end up in the V\&V 
report.

\ds{rework the following paragraph}
Ensuring this traceability and, in fact, getting anything certified has many 
costs associated with it. There is a massive time investment, fees, and costs 
associated with contracting out a third-party verifier. Overall it is a 
very expensive process.

Re-certification of software following any change, no matter how minor, %DS 
%keep?
incurs a similar level of costs; all the artifacts must be updated to 
reflect the new change, and everything must be re-checked and verified to 
ensure no new errors have been introduced. We have an implicit burden of 
ensuring the consistency of related information across our artifacts.

We intend to alleviate some of this cost-burden through a strategic, generative 
approach to Software Engineering (SE). With the automated generation of 
artifacts we can ensure they are \emph{consistent by construction}, implement 
changes 
quickly, and automatically update relevant and/or dependent artifacts.  \wss{I
  was looking for consistent by construction here, and I found it.  :-)}

\ds{Work the following paragraph in somewhere related to consistency.}
Consistent documentation has its own advantages while developing or maintaining 
software~\cite{Hyman1990, Kotula2000}. Similarly, there are many downsides to
inconsistent documentation~\cite{Kotula2000,Thimbleby1986}.


\subsection{Scope} %DS Should this be here? Maybe, as long as the above 
%clarifies the problem.
\label{S:Scope}

We limit our scope to Scientific Computing Software (SCS) for our proof of 
concept. The SCS domain is first and foremost a very well understood domain, by 
which we mean there are many theories underpinning the work being done. This is 
a necessity if we wish to be able to explain the science to a machine for the 
purposes of generating our software artifacts. It also gives us a foundation 
from which to build our knowledge-base.

SCS is also a domain rich in specialized software requiring certification. For 
example, control software in nuclear power plants, x-ray machines, and other 
safety critical applications require software to be approved by a certifying 
body.

\ds{More to come}

\wss{We definitely want a scope section.  What do we do is important, but what
  we don't do is also important information to convey.  We discussed this
  recently.  Many people want us to jump to the more interesting/more
  challenging problems, but we recognize that there is much to be gained by
  first tackling the more mundane problems.  We allow for many classes of
  practical changes to be made quickly and easily.  This has real value.  We can
  discuss in the future work section the more interesting ideas.}

\wss{My suggestion is to stop writing Section 1 for now and focus on Sections 5
	and 6.  Section 1 is a good start, but it feels like we need more current
	information on MDL.  We might also want more information on scientific
	knowledge ontologies.}


\section{Background}
\label{S:BG}

Reducing the costs of (re-)certification efforts and automating the generation
of software artifacts has been attempted in differing scopes by many others. We
look to them for insight and in an attempt to combine the fruits of their
labours into something more versatile.

In this section we will first look at the state of SC software 
development, as we have limited our scope to that domain, and identify 
practices we can learn from or potentially improve. We 
then explore previous efforts in automating software artifact
generation, as well as look at the current state of Model Driven
Engineering, scientific knowledge ontologies, and other areas that may be of 
interest.

\subsection{SC Software Development}

In many instances of SC software development, the developers are scientists and 
tend to emphasize their science without necessarily following 
software development best practices~\cite{Kelly2007}. These developers tend to 
prefer an agile~\cite{AckroydEtAl2008, CarverEtAl2007, 
EasterbrookAndJohns2009, Segal2005} or knowledge-acquisition driven 
process~\cite{Kelly2015} instead. They consider process-heavy approaches 
too rigid and disagreeable~\cite{CarverEtAl2007}.

Tool use and adoption is also a problem in the SC software development 
community, especially the use of version control software~\cite{Wilson2006}. 
There is also a limited use of automated testing~\cite{PatrickEtAl2015} and 
lack of understanding of what good software testing entails~\cite{Merali2010} 
with quality assurance having ``a bad name among creative scientists and
engineers''~\cite[p.~352]{Roache1998}. 

With good science relying on replication and reuse, we see a lack of 
reuse on the software side. A survey~\cite{Owen1998} showed that of 81 
different mesh generator packages, 52 generated triangular meshes. Of those 52, 
37 used the same Delaunay triangulation algorithm. There is no 
reason for the exact same algorithm to be implemented 37 separate times when it 
could simply be reused.

Some SC software developers have used advanced techniques, like code 
generation, quite successfully. Examples that come to mind are 
FEniCS~\cite{LoggEtAl2012}, FFT~\cite{KiselyovEtAl2004}, Gaussian 
Elimination~\cite{Carette2006}, and Spiral\cite{PuschelEtAl2005}. The focus of 
generation techniques, thus far, have been solely on a single software 
artifact: the source code.

\subsection{Software Artifact Reuse and Generation} %DS Is this subsect 
%necessary?

Previous attempts at generating software artifacts were primarily focused on
reusability. One aim of these approaches was to remove the burden of
replicating some artifacts as, in many cases, replication can become impossible 
without the help of the original author. Typically this is due to undocumented 
assumptions, modifications to the finished product, or errors in the original 
work~\cite{IonescuAndJansson2013}.

\subsubsection{Reproducible Research}
\label{S:RR}

The term \emph{reproducible research} means embedding executable 
code in research papers to allow readers to reproduce the results 
described~\cite{SchulteEtAl2012}. However, combining research reports with 
relevant code and data is not easy, particularly when dealing with publication 
versions of an author's work, thus the idea of \emph{compendia} were 
introduced~\cite{GentlemanAndLang2012}.

Compendia provide a means of encapsulating the full scope of a work. They 
allow readers to see computational details and re-run computations performed by 
the original author. While Gentleman and Lang intended compendia to be used for 
peer review in scientific journals, we can also see their use in the realm of 
software certification. Any requisite artifacts for getting software certified 
could together be considered a type of compendium.

Alongside compendia, several other tools have been created for reproducible 
research. Examples include Sweave~\cite{Leisch2002},
SASweave~\cite{LenthEtAl2007}, Statweave~\cite{Lenth2009},
Scribble~\cite{FlattEtAl2009}, and Org-mode~\cite{SchulteEtAl2012}. These tools
maintain a focus on specific computational details and code in general. Sweave 
(the most popular of the aforementioned examples~\cite{SchulteEtAl2012}) allows 
for embedding code into a document to be run during typesetting so up-to-date 
results are always included. The majority of these tools aim to create a 
singular, linear document like a research report.

\subsubsection{Literate Programming}
\label{S:LP}

Introduced by Knuth, Literate Programming (LP) changes the focus from writing 
programs as a list of instructions to explaining (\emph{to humans}) what we 
want the computer to do~\cite{Knuth1984}.

Developing literate programs involves breaking algorithms down into
\emph{chunks}~\cite{JohnsonAndJohnson1997} or \emph{sections}~\cite{Knuth1984}
which are small and easily understandable. These chunks are organized into a 
``psychological order''~\cite{PieterseKourieAndBoake2004} to promote 
understanding. One key aspect of LP is that chunks do not have to be written in 
the order necessary for computation, as that may not be the most understandable.

It should also be noted that in a literate program, the code and
documentation are kept together in one source. Extracting working source code 
is done through the \emph{tangle} process. Similarly, \emph{weave}
is used to extract and typeset the documentation.

Beyond understandability, LP has some key advantages over traditional 
development. The intent of LP is to update documentation surrounding a piece of 
source code as a program is developed and updated. There is also some reduction 
in knowledge duplication through chunking. Adopting proper usage of LP ends up
with more consistent documentation and code~\cite{ShumAndCook1993}. Keeping in 
mind the benefits of artifact consistency we can see that more effective and 
maintainable code can be produced when using 
LP~\cite{PieterseKourieAndBoake2004}. 

LP has not been very popular due to a couple of main issues: language/text 
processor dependency and the lack of flexibility on output 
presentation/suppression~\cite{ShumAndCook1993}. Still, there are
several successful examples of literate programs in SC. Two such examples are
VNODE-LP~\cite{Nedialkov2006} and ``Physically Based Rendering: From Theory to
Implementation''~\cite{PharrAndHumphreys2004}. The latter being a textbook that 
can be run as a literate program. 

Many attempts to address the issues with LP's popularity have focused on
changing or removing the output language or text processor dependency. Several
new tools were developed such as: CWeb (for the C language), DOC++ (for C++),
noweb (programming language independent), and more. While these tools did not 
bring LP into the mainstream~\cite{Ramsey1994}, they did help drive the 
understanding behind what exactly LP tools must do. We can now see LP becoming 
more standardized in certain domains (for example: Agda, Haskell, and R support 
LP to some extent). R has good tool support, with the most popular being
Sweave~\cite{Leisch2002}, however it is mainly used for dynamically generating 
reports as mentioned in Section~\ref{S:RR}.

New tools led to the introduction of many new features including, but not 
limited to, a ``What You See Is What You Get'' (WYSIWYG) 
editor~\cite{FritzsonGunnarssonAndJirstrand2002}, phantom 
abstracting~\cite{ShumAndCook1993}, and even movement away from the ``one 
source'' idea~\cite{Simonis2003}.

Tools such as Haddock (for Haskell), javadoc (for Java), and Doxygen (for 
multiple languages) were also influenced by LP, but differ in that they are 
merely document extraction tools. They do not contain the chunking features 
which allow for re-ordering algorithms.

As a final note, LP does not overly simplify the software development process 
as documentation and code must be written as usual (barring chunk reuse), but 
with the additional effort of re-ordering chunks.

\subsubsection{Literate Software}

A combination of LP and Box Structure~\cite{Mills1986} was proposed as a new
method called ``Literate Software Development''
(LSD)~\cite{AlMatiiAndBoujarwah2002}.

Box structure can be summarized as using 
different abstractions (views) that communicate the same
information in differing levels of detail, for distinctive purposes. Box
structures consist of black box, state machine, and clear boxes. The
black box gives an external (user) view of the system and consists of stimuli
and responses; the state machine makes the state data of the system visible -- 
it defines the data stored between stimuli; and the clear box gives an internal
(designer's) view describing how data are processed~\cite{Mills1986}. These 
three structures are nested as necessary to describe a system.

LSD was developed with the intent to overcome the disadvantages of both LP and
box structure. It was intended to overcome LP's inability to specify interfaces
between modules; the inability to decompose boxes and implement designs
created by box structures; and overcome the lack of tool support for box
structure~\cite{Deck1996}.

``WebBox'' is a framework for LSD which expands LP and box structures in a
variety of ways. It includes new chunk types and functionality for refining 
chunks, specifying interfaces and communication between boxes, and decomposing 
boxes at any level. LSD remains primarily code-focused with little support for 
other software artifacts.

\ds{Things to try and work in RE:LP}
    - Identified the need for good documentation
- One other problem we've identified:
  - common underlying knowledge between projects is duplicated as there is no
   real cross-project reuse mechanism in place with these tools.

\ds{Really need an intro to the basics of grounded theory here, for section 
3.1 (S:IntroCases) and subsequently 4.2 (S:KReUse) if we keep the references to 
grounded theory}

\subsection{Knowledge-Based Software Engineering (KBSE)}
\label{S:KBSE}

\ds{Pull apart this section and move the pieces around as necessary. Some of 
this is intro, some is our approach, etc.}

Knowledge-Based Software Engineering (KBSE) was originally defined as an
``engineering discipline that includes the integration of knowledge into 
software systems in order to solve complex problems, which would normally 
require rather high level of human 
expertise''~\cite{FeigenbaumAndMcCorduck1983}.
This is a solid definition, provided we understand what ``knowledge'' is. So 
then, what exactly is knowledge?

Knowledge ``presents understanding of a subject area. It includes concepts and 
facts ... as well as relations ... and mechanisms for how to combine them to 
solve problems in that area''~\cite{Durkin1994}.

\ds{Need to write about how KBSE uses formal methods, particularly for defining 
the semantics and whatnot for software synthesis. Good stuff in 
\cite{Lowry1989} and \cite{DevanbuEtAl1990}.}

%%%%%%%%%%%%%%%%%%%%%%%%%%%%%%%%%%%%%%%%%%%%%%%%%%%%%%%%%%%%%%%%%%%%%%%%%%
\ds{This probably will have to move} With that in mind, we have decided to 
restrict our focus to KBSE for Scientific 
Computing Software (SCS) as it is a field rich in knowledge we can use.

\wss{Should MDL show up here?}

\section{A rational analysis of software artifacts}

To generate all software artifacts, we must first take stock of how they are 
composed. We have chosen to focus on the Smith Et Al. templates 
\cite{Smith????} \ds{grab proper citation later}. We made this choice because 
we have access to a number of case studies which were implemented using the 
given templates and thus give us a reasonable starting point for our analyses.

\subsection{Introducing our case studies}
\label{S:IntroCases}

To understand exactly what we are looking at in our software artifacts, we will 
now introduce the case studies that have driven the development of the Drasil 
framework.

%DS Use a style similar to the Materials paper for introducing case studies 
%here.

%DS This doesn't really need to be said.
%Our case studies are broken into two main categories: those developed by 
%scientists for a specific application (related to SC), and those developed not 
%necessarily by scientists using a variety of scientific knowledge. 

The first case study we will explore is \emph{GlassBR} -- software created for 
checking whether a given pane of glass will be able to withstand the force of a 
particular explosion. It has a number of glass and explosion-related parameters 
which are used to compute the probability of glass breakage. This will be our 
running example throughout this paper, with minor asides to demonstrate 
particular features used by or curiosities found in the other case studies.

\ds{Figure of the GlassBR system for context?}

The next two case studies -- \emph{SWHS} and \emph{NoPCM} -- are members of the 
same software 
family. They both model a solar water heating system with one non-trivial 
difference. SWHS incorporates Phase Change Material (PCM) which allows the 
system to store more thermal energy than a system without PCM like that 
modelled in NoPCM.

Our fourth case study, \emph{SSP}, is for modelling a slope stability problem 
intended 
to analyze whether a given slope (natural or excavated) can be considered safe. 
The software identifies the surface most likely to experience slip and gives 
its factor of safety -- an index of its relative stability.

Finally we have \emph{Gamephysics} which is a pared-down version of the 
Chipmunk2D 
game physics engine. We chose this project specifically for its rich use of 
physics knowledge, but as it is a very large physics library, we narrowed the 
scope in the interests of time for our proof of concept.

There is also a toy-example case study, nicknamed \emph{tiny}, which is taken 
from an example of a fuel-pin in a nuclear reactor. This toy example helped 
push the early development of Drasil, but has recently been primarily of use 
for testing our generated output.

With our case studies in hand, we must look for the commonalities between them, 
particularly between the software artifacts in each project and what they are 
truly saying. With that in mind we can extrapolate these commonalities to other 
pieces of SCS. One obvious commonality across many projects in SCS is that of 
the use of Système International (SI) and derived units.

\subsection{Common software artifacts}

As our case studies appear to follow a rational design process, the same 
software artifacts can be found for each. Each project includes a:

\begin{itemize}
	\item Software Requirements Specification (SRS)
	\item Detailed design documents such as: 
		\begin{itemize}
			\item Module Guide (MG) 
			\item Module Interface Specification (MIS)
		\end{itemize}
	\item Source code
	\item Test cases
	\item User Guide
	\item Verification and Validation (V\&V) Plan
	\item V\&V Report
\end{itemize}

Let us first look at the similarities and differences found in the SRS, the 
detailed design documents, and the source code. The specific system knowledge 
found in the SRS is duplicated throughout the design and source code, though 
not always verbatim. The knowledge may be transformed to produce different 
views relevant to the current document.

	- only some of that knowledge is necessarily relevant in those views
	- \fgr: SRS \& DD showing the same piece of knowledge \label{Fig:SRSDDComp}
	in diff contexts. Use a few different \fgr here.
	- \fgr: Attempt to show generalized overlap via Venn diagram?

\subsection{Emerging structures} 
\sout{\ds{want to fit this into the analysis but it would make more sense 
--after-- the intro to KBSE}}

\ds{Makes more sense now that intro to KBSE has been moved.}
\label{S:KnowStruct}

- As shown above, 

In the common software artifacts we see different ways of representing what 
are, semantically, the same things (for example, see 
Figure~\ref{Fig:SRSDDComp}). We are really seeing the pieces of underlying 
knowledge that have been composed from a variety of components. Each component 
tells us something about one aspect of that piece of knowledge. Particularly, 
they give examples of how we can transform, or view, the same semantic 
knowledge in different contexts.

\begin{figure}
%GLASSBR DD here with code example, possible it might be useful earlier.
\caption{Data Definition for !FIXME! from GlassBR SRS}
\label{Fig:GlassBRDDSRS}
\end{figure}

\begin{figure}
%Fig goes here. Or lsting
\caption{Data Definition code from GlassBR implementation}
\label{Fig:GlassBRDDCode}
\end{figure}

If we take a look at one particular example across artifacts from GlassBR 
(Figures~\ref{Fig:GlassBRDDSRS},\ref{Fig:GlassBRDDCode}), we can see that it is 
an aggregation of the following components:

\begin{itemize}
\item Unique Identifier (label)
\item Symbolic (theory) representation
\item Symbolic (implementation) representation
\item Concise natural language description (a term)
\item Verbose natural language description (a definition)
\item Equation 
\item Constraints %(Really relevant. Again show \fgr as math and code)
\item Units? %DS If applicable in the example?
\end{itemize}

The unique identifier is fairly straightforward (!FIXME id!), it is just a 
label that we associate with this particular piece of knowledge and nothing 
else. The symbolic representations are just the symbols we use when referring 
to this particular quantity in an equation (theory) or code (implementation) 
context. Our natural language descriptions are terms and their corresponding 
definitions (!FIXME! and !FIXME! respectively for this example). 

We also have a defining equation, which incorporates the symbolic 
representation for various other pieces of knowledge and relates them to
!FIXME name!. Similarly, we have constraints which are just relationships which 
must be maintained between !NAME! and some other quantities. Lastly, we have 
the units which our quantity is measured in, which are derived from the 
fundamental !SI UNITS!. %!FILL IN?!

	%DS (not sure which yet, probably something big -- like a quantity or 
	%bigger; maybe a DD or TM?).
	% Working from TM or DD as an example

Similar examples of knowledge crop up over all the artifacts. Some have the 
same depth of information, whereas others do not. Regardless, all of our 
knowledge shares some components in common. We will always have a label, and 
usually a term and definition. Depending on what we are looking at, there may 
not be a symbolic representation, or perhaps we have a quantity that is 
unit-less. These special cases help us see the underlying root structure from 
which our knowledge buds. %DS Tree metaphor! Wooo!

\wss{Going through the GlassBR example like this seems like a good motivating
  example to me.  I'm assuming that you will also be able to reuse it later when
  you discuss the advantages of our approach for consistency and maintainability
  (with respect to change).}

-Discuss the breakdown of knowledge into classes. Refer to 
Table~\ref{Tab:KnowledgeClasses} for more.  \wss{This table looks like a good
  way to summarize this information to me.}

\begin{table}
\caption{Knowledge Classes}
\label{Tab:KnowledgeClasses}
\begin{tabular}[]{ l | l | l | l | l | l | l | l | l}
Knowledge Class & ID & Term & Abbreviation & Definition & Symbol & 
Equation & Constraints & Units \\

			\hline{} & & & & & & & & \\
Labeled & X & & & & & & &\\
			\hline{} & & & & & & & &  \\
Named Idea & X & X & O & & & & &\\
		  	\hline{} & & & & & & & & \\
Common Idea & X & X & X & & & & &\\
			\hline{} & & & & & & & &\\
Concept & X & X & X & X & & & &\\
%TODO: Finish filling in
\end{tabular}

Legend: X - Mandatory; O - Optional

\end{table}

\section{Our solution -- A combined approach}
\label{S:solution}

With inspiration from similar problem domains, as mentioned in 
Section~\ref{S:BG}, we combine a number of ideas to tackle the problems of 
duplication, inter-/intra-artifact inconsistency, design for change, lack of 
reusability, and difficulty with (re-)certifiability.

For our purposes, we extend and tighten the definition of knowledge introduced 
in Section~\ref{S:KBSE} to include the additional constraint that a piece of 
knowledge has a structured encoding, as opposed to natural language encoding, 
which then allows it to be automatically reused. For example, the first law of 
thermodynamics is a piece of knowledge that can be simply expressed as ``total 
energy within a closed system must be conserved'', but this is not a structured 
encoding. One such encoding would allow us to view the knowledge in those 
relatively simple terms, or just as easily, we could view it as:
\begin{equation*}
\Delta{}U = Q - W
\end{equation*}
where we define a \emph{closed system} as one which cannot exchange matter with 
its surroundings, but energy can be transferred. $\Delta{}U$ is then the change 
in internal energy of a closed system, $Q$ is the amount of energy supplied to 
the system, and $W$ is the amount of energy lost to the system's surroundings 
as a result of work.

Regardless of our view, we have not changed the underlying structured knowledge 
encoding -- we merely project out what is relevant to our current audience.

Our approaches involves using this underlying idea in the creation and 
maintenance of a singular knowledge-base that we can pull from to implement our 
software. With knowledge coming from a single source, we have guaranteed 
consistency. We are able to mix and match knowledge as needed, which allows for 
greater reuse across projects, and we use generators to produce the software 
artifacts we need.

For our approach to succeed we must satisfy two major requirements. First off, 
we must capture the underlying knowledge in a meaningful and reusable 
(artifact-independent) way. We want a single source for our knowledge, 
regardless where it ends up or how it is viewed. Thus, using the right 
transformations %DS Conversions?
and projections, we can automatically generate our software artifacts from the 
knowledge-base.

The second requirement is that we restrict our scope to well-understood domains 
as we need a solid theoretical underpinning. Both mathematics and the physical 
sciences are good examples of well-understood domains as the knowledge has 
already been formalized and, to an extent, structured. These are also good 
candidate domains since we need to explain the underlying knowledge to 
computers in a nontrivial way, which from our experience is harder than it 
sounds. Hence the reduction in scope we mentioned in Section~\ref{S:Scope}

\subsection{Capturing Knowledge}
\label{S:KnowCapt}

From our work in Section~\ref{S:KnowStruct} we can create a knowledge-capture 
mechanism for encoding the requisite underlying science into a machine-usable 
form. By laying out the structure, we can see which information must be 
captured for each piece of knowledge.

Different types of information are required for encoding each of the various 
pieces of knowledge we intend to use. Some types of knowledge lack specific 
information bindings, for example a \emph{named idea} does not necessarily have 
a symbol associated with it, however, a \emph{quantity} \emph{must} have a 
symbol alongside its \emph{term} -- the fundamental information in a named
idea.  \wss{Should this information go closer to the GlassBR example above?}

We borrow, and expand, the idea of \emph{Chunks} from Literate Programming 
(LP)~\cite{Knuth1984} to facilitate our knowledge-capture. A chunk in its most 
rudimentary sense is simply a labeled piece of information. Given our 
understanding of how the knowledge should be structured, we have created a 
hierarchy of classes built up from the simplest of chunks, to fulfill our 
knowledge-capture requirements. This hierarchy as implemented in Drasil can be 
seen in Figure~\ref{Fig:ChunkHierarchy}. It mimics the structure mentioned in 
Section~\ref{S:KnowStruct}. We will delve deeper into the specifics of our 
hierarchy in Section~\ref{S:Drasil}.

\begin{figure}
\begin{tikzpicture}
[every text node part/.style={align=center}]
\begin{scope}[every node/.style={rectangle,draw}]

\node(Chunk) at (0,0) 
{\textbf{Chunk} \\ \small{uid :: String}};

    \node(NamedIdea) at (-3,-1.5) 
    {\textbf{NamedIdea} \\ \small{term :: NounPhrase}};

        \node(Idea) at (-1.5, -3)
		{\textbf{Idea} \\ \small{getA :: Maybe String}};

			\node(Concept) at (0, -4.5)
			{\textbf{Concept} \\ \small{FIXME}};

        \node(CommonIdea) at (-4.5, -3)
        {\textbf{CommonIdea} \\ \small{abrv :: String}};

	\node(ExprRelat) at (0, -1.5)
	{\textbf{ExprRelat} \\ \small{FIXME}};

	\node(Theory) at (2, -1.5)
	{\textbf{Theory} \\ \small{FIXME}};

\node(Definition) at (2.5, 0)
{\textbf{Definition} \\ \small{FIXME}};

\node(ConceptDomain) at (5,0)
{\textbf{ConceptDomain} \\ \small{FIXME}};

%TODO: Finish and make this look nice

\end{scope}

\draw [->] (Chunk.south west) -- (NamedIdea.north east);
\draw [->] (Chunk.south) -- (ExprRelat.north);
\draw [->] (Chunk.south east) -- (Theory.north west);
\draw (NamedIdea) [->] to (CommonIdea);
\draw (NamedIdea) [->] to (Idea);
\draw (Definition) [->, out=330, in=0] to (Concept);
\draw (ConceptDomain) [->, out=270, in=0] to (Concept);
\draw (Idea) [->] to (Concept);

\end{tikzpicture}

%\begin{forest}
%  for tree={
%    align=center,
%    parent anchor=south,
%    child anchor=north,
%    font=\sffamily,
%    edge={thick, -{Stealth[]}},
%    l sep+=10pt,
%    edge path={
%      \noexpand\path [draw, \forestoption{edge}] (!u.parent anchor) -- 
%      +(0,-10pt) -| (.child anchor)\forestoption{edge label};
%    }
%  }
%  [\textbf{Chunk} \\ \small{uid :: String}
%    [\textbf{NamedIdea} \\ \small{term :: NounPhrase}
%      [\textbf{Idea} \\ \small{getA :: Maybe String}]
%      [\textbf{CommonIdea} \\ \small{abrv :: String}]
%    ]
%    [\textbf{Theory} \\ \small{FIXME}]
%    [\textbf{ExprRelat} \\ \small{FIXME}]
%  ]
%  [\textbf{Definition} \\ \small{uid :: String}
%  ]
%\end{forest}
\caption{Chunk hierarchy in Drasil Today}
\label{Fig:ChunkHierarchy}
\end{figure}

When we capture knowledge, we try to encode all of the information surrounding 
that piece of knowledge in an artifact-agnostic manner. We are not concerned 
with which views will be used by our artifacts, only what the underlying 
knowledge is and how it should be captured.  \wss{Great point.}

Once we have properly captured the relevant knowledge, we should not have to 
capture it again to reuse it in a different project. Any given piece of 
knowledge should only be added to the knowledge-base once!

\subsection{(Re-)Using Knowledge}
\label{S:KReUse}

Capturing knowledge in itself helps us improve our understanding of the 
underlying theory by laying things out in a structured way. That is a benefit 
in itself, however, when we can actually use the captured knowledge we see many 
advantages to this approach.

The most obvious perk is that we no longer need to manually copy knowledge 
across artifacts, we can simply pull what we need from our knowledge-base. 
While this seems trivial, the ramifications are huge -- we have guaranteed 
consistency by construction. 

At this point you may be wondering, ``what if I want to do more than just copy 
information around?" Recall the example from the beginning of 
Section~\ref{S:KBSE}, the view of our knowledge can change without affecting 
our encoding. To project these views, we use transformations.

Transformations represent the different 'views' of the knowledge we want based 
on how abstract we need it to be, what audience we are targeting, and a host of 
other factors. We use transformations to translate knowledge into its requisite 
forms, whether they be equations, descriptions, code, or something else 
entirely.  \wss{Can we call what we are doing transformation?  I've always
  thought of what we are doing as model transformation, but in a conversation
  Wolfram Kahl implied that it isn't.  I didn't quite catch his point, but I
  think there was a concern that we do not do bidirectional transformation?}

We can also use transformations to expose variabilities. These are what define 
project families -- projects which solve the same general problem, but with 
differences in the specific goals and/or implementations of those solutions. 

For example, our case studies (introduced in Section~\ref{S:IntroCases}) for 
SWHS and NoPCM are members of the same \emph{software family} as they solve the 
same general problem with a variation on whether phase-change material is 
present in the system. A correct solution for each problem will look different, 
but there is a non-trivial amount of fundamental knowledge being shared by both 
solutions.

%      - Example: SWHS vs NoPCM
      \fgr{} Show portion of each SRS, one similarity, one difference?

Manually transforming knowledge in this way is tedious and would likely not end 
up cutting costs or saving time. If, on the other hand, we had a framework or 
tool to support the automation of these transformations for our software 
artifacts, those particular disadvantages disappear.

\section{Drasil}
\label{S:Drasil}
- To use KBSE to its potential we need a strong support framework
- Intro to Drasil
    \fgr{} Knowledge tree
  - What it is and does
    - Domain Specific Language(s)
    - Generate all the things!
  - Dev to date.
  - How is Knowledge Capture handled in Drasil? - chunks!
  - What do transformations look like? Recipes!
    \fgr{} SmithEtAl template for SRS = Drasil.DocumentLanguage
  - Key components of the generator / renderer

- Haddock % Drasil is documented -- available on the repo if you make docs.

\subsection{Developing Drasil - A grounded theory} %DS Should this be here or 
													%in bg?
- Following grounded theory (ish). Using data from case studies to guide 
development and implement new features.
- \fgr{}: Before and after System Information.
- \fgr{}: Before and after mini-DBs
- Majority of features developed after analyzing commonalities in the case 
studies and abstracting them out.
- Allows for rapid progress -> constant iteration based on what we find in the 
data.

\subsection{Packaging Drasil}

The current version (0.1.1 as of this writing) of Drasil is built as a 
group of Haskell packages (see Figure~\ref{Fig:Packages}). The core components, 
including those for document concepts, are stored in a package named 
\edl{} and code-related components are in 
\edc{}. The main generator code is in \edg{}, with the HTML/LaTeX printers in 
\edp{}. We maintain our current knowledge-base in 
\edd{} and our case study examples in \ede{}.

\begin{figure}
\begin{tikzpicture}

%TODO: Fix this

\node (drasil-lang) [class] {\dl};
\node (drasil-code) [class, below left=1 and 0.2cm of drasil-lang] {\dc};
\node (drasil-data) [class, below right=1 and 4cm of drasil-code] {\dd};
\node (drasil-example) [class, below left=1 and 0.54cm of drasil-data] {\de};
\node (drasil-printers) [class, below right=1 and 1cm of drasil-lang] {\drp};
\node (drasil-gen) [class, below left=of drasil-printers] {\dg};

%lang coords
\coordinate [above=2.785cm of drasil-data] (rightl);
\coordinate [above=1.28cm of drasil-code] (leftl);
%code coords
\coordinate [below=1.25 of drasil-code] (underc);
\coordinate [below left=1.53 and 0.5 of underc] (underleftc);
\coordinate [left=0.5 of drasil-code.south] (leftc);
%data coords
\coordinate [below=1.285 of drasil-data] (underd);

%code
\draw [myarrow] (drasil-code.north) -- (leftl) -- (drasil-lang.west);

%data
\draw [myarrow] (drasil-data.north) -- (rightl) -- (drasil-lang.east);
\draw [myarrow] (drasil-data.west)  -- (underc) -- (drasil-code.south);

%example
\draw [myarrow] (drasil-example.north) -- (drasil-lang);
\draw [myarrow] (drasil-example.west) -- (underleftc) -- (leftc) node[midway, 
left]{depends};
\draw [myarrow] (drasil-example.east) -- (underd) -- (drasil-data.south);

\end{tikzpicture}
\caption{Drasil package dependencies}
\label{Fig:Packages}
\end{figure}

\subsubsection{\dl}

The core language used within the Drasil framework, including all of the 
building blocks for our knowledge-base are stored within \edl{} under the 
exported module \texttt{Language.Drasil}. This package also exports a second 
module with functionality targeted to developers of Drasil known as 
\texttt{Language.Drasil.Development}.

\texttt{Language.Drasil} presently contains the expression DSL, document layout 
DSL, and knowledge capture classes \& data types. Every other \emph{drasil-*} 
package relies on, and builds off, this core.

With \texttt{Language.Drasil} alone we can capture knowledge for generating
artifacts nearly identical to those shown in Section~\ref{S:IntroCases}. 
However this is a much lower-level approach than we would like to use and could 
be seen as being akin to programming in a language like C, or even assembly. 

\subsubsection{\drp}

The printing DSL for HTML and LaTeX are stored here under the namespace 
\texttt{Language.Drasil.Printers}. These are the DSLs which translate Drasil 
specifications to the target language(s). Please note that the 
printers for source-code are kept in a separate location.

\subsubsection{\dc}

The code generation DSL used within the Drasil framework is stored here under 
the namespace \texttt{Language.Drasil.Code}. The code generation framework 
incorporates \emph{GOOL}, a Generic Object-Oriented 
Language~\cite{Costabile2012}, to give us the 
ability to target multiple languages -- C++, C\#, Objective C, Java, Lua, and 
Python.

\subsubsection{\dg}

The main generator functions used by Drasil are stored here under the 
\texttt{Language.Drasil.Generate} namespace. These functions are used to take a 
Drasil specification and use the appropriate printers to generate the final 
output file(s). Both code and document generation are handled through 
function-calls found here.

The actual body of \edg{} is very small, consisting of only approximately 60 
lines of Haskell code.

\subsubsection{\dd}

The knowledge-base common to all Drasil programs is curated and maintained 
within this package under the \texttt{Data.Drasil} name. Currently we have 
captured 
knowledge in a range of domains including, but not limited to, Computation, 
Education, Math, Physics, and Software. We have also captured meta-knowledge 
related to documentation, physical properties, and more.

A more detailed breakdown of \texttt{Data.Drasil} will be given in 
Section~\ref{S:Data}.

\subsubsection{\de}

All of the code required to generate artifacts for our case study examples is 
maintained in this one package. Each case study has a unique namespace 
containing everything, other than common knowledge from \edd{}, required to 
generate that particular case study's artifacts. These namespaces can be seen 
in Table~\ref{Tab:Namespaces}. 

\begin{table}
\caption{Case study namespaces in \ede{}}
\label{Tab:Namespaces}
\begin{tabular}[]{ l | l}
		Case Study 		& 				Namespace 						\\
\hline{}
		GlassBR 		& 	\texttt{	Drasil.GlassBR			} 		\\
\hline{}
		GamePhysics 	& 	\texttt{	Drasil.GamePhysics		}	 	\\
\hline{}
		SSP 			& 	\texttt{	Drasil.SSP				}		\\
\hline{}
		SWHS 			& 	\texttt{	Drasil.SWHS				}		\\
\hline{}
		NoPCM 			& 	\texttt{	Drasil.NoPCM			} 		\\
\hline{}
		Tiny 			& 	\texttt{	Drasil.HGHC				} 		\\
\end{tabular}
\end{table}

The \ede{} package also contains an example recipe, found in 
\texttt{Drasil.DocumentLanguage} targeted at recreating the 
SmithEtAl SRS template\cite{SmithEtAl????}. %DS Gotta add the year.
Keeping the recipe with the examples is less than ideal, hence why it will soon 
be moved out of \ede{} and into its own package soon, along with other recipes 
as they are created.
We will discuss the recipe and document language specifics in 
Section~\ref{S:DrasilToday}.

\wss{I like the explanation of the packages}

\subsection{Drasil Today}
\label{S:DrasilToday}

- Sentence and Document
- Explain the chunk hierarchy (refer to Section~\ref{S:KnowCapt} figure)
- Data.Drasil
  \fgr{} Knowledge areas we have started to capture (See: SE-CSE paper)
- Recipe Language(s) -- Refer to:
  \fgr{} Drasil.DocumentLanguage
- The generator
  - HTML and TeX rendering
  - GOOL for code
- System Information -> Get into it

\section{Case studies - in more depth}

- Re-introduce case studies
  - Our methods for reimplementing
  - CI for testing
- Start showing off re-use and automated generation.
  - Start with common knowledge (generalized \fgr{}?)
  - Then onto GlassBR example to show off the doc lang recipe (\fgr{}?)
  - Then let's see SRS vs. NoPCM for reuse (particularly NoPCM) (\fgr{}?)

\wss{I like how specific this section is.  You are highlighting specific
  lessons/findings from actual examples.  When you get stuck with writing other
  sections, this would be a good place to focus your energy.  You should be able
  to write this material almost independently of the other sections, at least to
  get started.}

\subsection{Data.Drasil}
\label{S:Data}

- Common knowledge
  \fgr{} SI\_Units
  \fgr{} Thermodynamics (ConsThermE?)

\subsection{GlassBR}

- Brief intro to problem GlassBR is solving - how it works
- Show off the doc language here
  \fgr{} GlassBR SRS in (truncated) DocLang format 
  - "Reads like a table of contents, with a few quirks"
- Show off some code generation
  \fgr{} Side-by-side of Chunk Eqn vs. Doc Eqn vs. Code 
  - "Easy to see that the code matches the equations"
- Talk about potential variabilities and how to make this a family
- Why is this interesting?
	- Fairly straightforward example of something a scientist would create/use 
	in their research

\subsection{NoPCM \& SWHS}

- Re-introduce the problems
- See how they're a family?
- Really drill in the similarities
  \fgr{} Figure showing NoPCM import(s)
- Lots of knowledge-reuse
- Very few 'new' chunks (count them?)
- Show example of variability in action
  \fgr{} Equation with/without PCM (rendered?)
- Why this example is interesting:
  - ODE solver -> We don't gen, just link to existing good one(s)

\subsection{Others}

- Mention SSP, Tiny, GamePhysics, but don't go too in-depth.
  - Useful examples as they give us a wider range of problems for analysis
- Testing
  - Physics is physics -> when we make updates, the underlying knowledge isn't
    changing, so neither should our output
  - Refer to CI

\subsection{Freebies - Compliments of System Information}

- Thanks to the recipe language and the way we structure out system information
  we can get
- Table of Symbols
- Table of Units
- Table of Abbreviations and Acronyms
- Bibliography

- All tedious to do by hand, but are free to automatically generate
- Generator includes sanity-checking -> Can't use something that isn't defined!
- Sanity-checks are 'free' -> we can check for errors with our symbols,
  ensure units are consistent, guard against constraints, and ensure we only
  reference those things which are defined in our system. 
- Sanity-checks are run every time artifacts are generated.

\section{Results}

- Here we discuss the results we've seen so far.
- Had some of these case studies attempted to be certified, they would (should) 
have failed.
	- A number of common problems.

\subsection{Common issues across case studies}

- A number of undefined symbols even after multiple passes by humans. 
(Auto-generating the symbol table and including sanity-checking revealed them)
\wss{You are giving specific examples below, right?}

\subsection{NoPCM and SWHS}

- Along with the common errors, there was some sharing of PCM-related knowledge
  - Found because PCM symbols were not in the ToS and the sanity-check caught 
  it.
  - No way to specifically exclude knowledge that shouldn't 'exist' in a project
- Work in Kolmogorov complexity / MDL for NoPCM + SWHS?
- Kolmogorov/MDL implies less writing for the same artifacts -> less to sift 
through = maybe better?

\subsection{SSP}

- Symbols for given quantities changed throughout the documentation
  - Went unnoticed by a human for years! Found almost instantly by Drasil
    - the new symbols were undefined.

\subsection{Pervasive Bugs}

One of the utmost benefits of the knowledge-based approach using Drasil is the 
introduction of ``pervasive bugs". These are typically mistakes made in the 
captured knowledge which propagate across all generated artifacts wherein that 
knowledge is used. Calling this a benefit may seem counter-intuitive, but when 
an error appears in a multitude of locations it is far more likely to be caught 
then if it were hiding in the corner of one artifact.

Not only is it more likely that we will find an error, it is also far easier to 
track down the source of said error -- we need only go to the knowledge base 
and find the requisite chunk. We can also tune error messages to point us at 
the exact chunk causing the problem if we so desire.

Correcting an error in a chunk of knowledge is also trivial. It only needs to 
be fixed once to be fixed across all of our software artifacts. No need to 
\texttt{grep}, find-and-replace, or the like.


\section{Future Work}

[*SS* - Once we are capable of true variability in the documentation, we can
really start asking the question about what is the "best" documentation for a
given context.  In the future experiments could be done with presenting the same
information in different ways to find which approach is the most effective.]

[*SS* - Related to the previous point, the act of formalizing the knowledge that
goes into the requirements documentation forces us to deeply understand the
distinctions between difference concepts, like scope, goal, theory, assumption,
simplification, etc.  With this knowledge we can improve the focus and
effectiveness of existing templates, and existing requirements solicitation and
analysis efforts.  Teaching it to a computer.]

- Run an experiment to determine how easy it is to create new software with 
Drasil.

- Run an experiment to see how easy it is to find and remove errors with Drasil

- Experiment to see time saved in maintenance while using Drasil vs. not

- Create drasil-gen package

- Design language

- Open issues (as of writing there are \#\#\# %DS 147
issues currently open on the Drasil repository).

\section{Conclusion}

- Easier to find errors (anecdotally) - future work will tell us if this holds.

\bibliography{drasil}
\bibliographystyle{acm}
\end{document}
