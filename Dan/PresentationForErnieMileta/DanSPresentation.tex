% $Header: /cvsroot/latex-beamer/latex-beamer/solutions/conference-talks/conference-ornate-20min.en.tex,v 1.6 2004/10/07 20:53:08 tantau Exp $

\documentclass{beamer}

\mode<presentation>
{
%  \usetheme{Hannover}
\usetheme[width=0.7in]{Hannover}
% or ...

  \setbeamercovered{transparent}
  % or whatever (possibly just delete it)
}

\usepackage[english]{babel}
% or whatever

\usepackage[latin1]{inputenc}
% or whatever

\usepackage{times}
%\usepackage[T1]{fontenc}
% Or whatever. Note that the encoding and the font should match. If T1
% does not look nice, try deleting the line with the fontenc.
%\usepackage{logictheme}

%\usepackage{hhline}
\usepackage{multirow}
%\usepackage{multicol}
%\usepackage{array}
%\usepackage{supertabular}
%\usepackage{amsmath}
%\usepackage{amsfonts}
\usepackage{totpages}
\usepackage{hyperref}
%\usepackage{booktabs}

%\usepackage{bm}

\pgfdeclareimage[height=0.7cm]{logo}{McMasterLogo}
\title[\pgfuseimage{logo}]  % (optional, use only with long paper titles)
{Literate Software Development}

%\subtitle
%{Include Only If Paper Has a Subtitle}

\author[Slide \thepage~of \pageref{TotPages}] % (optional, use only with lots of
                                              % authors)
{Dan Szymczak}
% - Give the names in the same order as the appear in the paper.
% - Use the \inst{?} command only if the authors have different
%   affiliation.

\institute[McMaster University] % (optional, but mostly needed)
{
  Computing and Software Department\\
  Faculty of Engineering\\
  McMaster University
}
% - Use the \inst command only if there are several affiliations.
% - Keep it simple, no one is interested in your street address.

\date[Dec 11, 2014] % (optional, should be abbreviation of conference name)
{Ernie Mileta Visit, Dec.\ 11, 2014}
% - Either use conference name or its abbreviation.
% - Not really informative to the audience, more for people (including
%   yourself) who are reading the slides online

\subject{computational science and engineering, software engineering, software
  quality, literate programming, software requirements specification, document
  driven design}
% This is only inserted into the PDF information catalog. Can be left
% out. 

% If you have a file called "university-logo-filename.xxx", where xxx
% is a graphic format that can be processed by latex or pdflatex,
% resp., then you can add a logo as follows:

%\pgfdeclareimage[height=0.5cm]{Mac-logo}{McMasterLogo}
%\logo{\pgfuseimage{Mac-logo}}

% Delete this, if you do not want the table of contents to pop up at
% the beginning of each subsection:
\AtBeginSubsection[]
{
  \begin{frame}<beamer>
    \frametitle{Outline}
    \tableofcontents[currentsection,currentsubsection]
  \end{frame}
}

% If you wish to uncover everything in a step-wise fashion, uncomment
% the following command: 

%\beamerdefaultoverlayspecification{<+->}

\beamertemplatenavigationsymbolsempty 

% have SRS and LP open during the presentation

\begin{document}

%%%%%%%%%%%%%%%%%%%%%%%%%%%%%%%%%%%%%%
\begin{frame}

\titlepage

\end{frame}

%%%%%%%%%%%%%%%%%%%%%%%%%%%%%%%%%%%%%%

\begin{frame}

\frametitle{Overview}
\tableofcontents
% You might wish to add the option [pausesections]

% make like a story - the phases - reason for, why works, advantages
% changing the history a bit to make a more rational narrative

\end{frame}

%%%%%%%%%%%%%%%%%%%%%%%%%%%%%%%%%%%%%%

\section[Literate Software]{Literate Software Development.}

% \subsection[Important Software Qualities]{Scientific Computing Software
% Qualities}

%%%%%%%%%%%%%%%%%%%%%%%%%%%%%%%%%%%%%%

\begin{frame}

\frametitle{Literate Software Development for Scientific Software}

\begin{itemize}
\item Motivation
\begin{itemize}
\item Improve the qualities of verifiability, maintainability and reusability.
\item Save money and time when managing change.
\end{itemize}
\item One ``source,'' multiple views
\begin{itemize}
\item Requirements, including or excluding derivations.
\item Design
\item Test Cases
\item Build instructions
\item ...
\end{itemize}
\item Advantages
\begin{itemize}
\item Reuse chunks and transformed chunks to avoid duplication.
\item Improve understandability, traceability and reproducibility.
\item Increased flexibility
\end{itemize}
\end{itemize}
\end{frame}

%%%%%%%%%%%%%%%%%%%%%%%%%%%%%%%%%%%%%%

\section[Example]{Example.}

%%%%%%%%%%%%%%%%%%%%%%%%%%%%%%%%%%%%%%

\begin{frame}

\frametitle{Example}

\end{frame}

%%%%%%%%%%%%%%%%%%%%%%%%%%%%%%%%%%%%%%

\section[Next Steps]{Next Steps.}

%%%%%%%%%%%%%%%%%%%%%%%%%%%%%%%%%%%%%%

\begin{frame}

\frametitle{Next Steps}

\end{frame}

%%%%%%%%%%%%%%%%%%%%%%%%%%%%%%%%%%%%%%

\end{document}