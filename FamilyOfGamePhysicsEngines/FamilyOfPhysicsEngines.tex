\documentclass{article}

\usepackage{graphicx}
\usepackage{paralist}
\usepackage{amsfonts}
\usepackage[round]{natbib}
\usepackage{supertabular}
\usepackage{hyperref}
\usepackage{bm}
\usepackage{colortbl}

%\topmargin -0.25in
%\oddsidemargin -0.25in
%\evensidemargin -0.25in
\textwidth 5in
%\textheight 9in
%\topskip 0in
%\footskip 0in

\title{Commonality Analysis for a Family of Material Models}

\author{Joseph Seger, Spencer Smith and Jacques Carette}

\newcommand{\colAwidth}{0.2\textwidth}
\newcommand{\colBwidth}{0.73\textwidth}

\newcounter{defnum} %Definition Number
\newcommand{\dthedefnum}{D\thedefnum}
\newcommand{\dref}[1]{D\ref{#1}}

\newcounter{goalnum} %Goal Number
\newcommand{\gthegoalnum}{G\thegoalnum}
\newcommand{\gref}[1]{G\ref{#1}}

\newcounter{theorynum} %Theory Number
\newcommand{\tthetheorynum}{T\thetheorynum}
\newcommand{\tref}[1]{T\ref{#1}}

\newcounter{assumpnum} %Assumption Number
\newcommand{\atheassumpnum}{P\theassumpnum}
\newcommand{\aref}[1]{A\ref{#1}}

\newcounter{varnum} %Variability Number
\newcommand{\vthevarnum}{V\thevarnum}
\newcommand{\vref}[1]{V\ref{#1}}

\newcounter{examplenum} %Example Number
\newcommand{\etheexamplenum}{E\theexamplenum}
\newcommand{\eref}[1]{E\ref{#1}}

\newcommand{\blt}{- } %used for bullets in a list

\renewcommand{\arraystretch}{1.1} %so that tables with equations do not look crowded

\begin{document}

\maketitle

\tableofcontents

\newpage

\section*{Table of Units}

Throughout this document consistent units are employed and a consistent notation
is used.  The unit system adopted is the ``MLtT'' dimension system, where M is
the dimension of mass, L is length, t is time and T is temperature.  This system
corresponds nicely with the SI (Syst\`{e}me International d'Unit\'{e}s), or
modern metric, system, which uses units of kilogram (kg), meter (m), second (s)
and Kelvin (K) for M, L, t and T, respectively.  By leaving the units in this
form any unit system can be adopted in the future, as long as the choice of
specific units is consistent between different quantities.  In addition to the
basic units, several derived units are employed as described below.  For each
unit the symbol is given followed by a description of the unit with the SI
equivalent in parentheses.  ~\newline

\begin{supertabular}{l p{11cm}}
  L & \blt length (metre, m)\\
  M & \blt mass (kilogram, kg)\\
  t & \blt time (second, s)\\
  T & \blt temperature (Kelvin, K)\\
  ForceU & \blt force, which has units of $\mbox{M} \cdot \mbox{L} \cdot
  \mbox{t}^{-2}$ (Newton, $\mbox{N} = \mbox{kg} \cdot
  \mbox{m} \cdot \mbox{s}^{-2}$)\\
  StressU & \blt stress, which has units of $\mbox{L}^{-1}$M$\mbox{t}^{-2}$, or
  $\mbox{ForceU}/\mbox{L}^2$ (Pascal, $\mbox{Pa} =
  \mbox{N}/\mbox{m}^2$)\\
  angleU & \blt radian (rad)\\
\end{supertabular}

\newpage

\section*{Table of Symbols}

The table that follows summarizes the symbols used in this document along with
their units.  The choice of symbols was made with the goal of being consistent
with the existing literature.  Accomplishing this goal requires that some
symbols are used for multiple purposes.  Where this is the case, all possible
meanings will be listed, with each on a separate line.  When the symbol is later
used in the body of the document, it will be defined on its first occurrence.
In the other instances where the symbol occurs, its meaning should be clear from
the context.  The units are listed in two sets of brackets following the
definition of the symbol.  The first set of brackets shows the MLtT dimension
system and the second set of brackets shows the equivalent SI units.  In the
cases where the symbol refers to entities with multiple components, such as
vectors and matrices, the units given apply to each individual component of the
entity.  In the cases where the units are not listed, this is for one of several
reasons.  The symbol may not have any units associated with it, such as for
strings and for a region of space (the location of the points within the space
have units of length, but the set of points itself does not have an associated
unit).  Another reason for not including units is that the choice of units may
be dependent on a particular instance of the symbol.  For instance, the fluidity
parameter ($\gamma$) is sometimes defined as $1/2\eta$, which will have units of
1/($\mbox{StressU} \cdot \mbox{t}$), but in other cases the fluidity parameter
will have a different definition and different units.  The potential ambiguity
in units is a consequence of using a generic model of material behaviour.
~\newline

\begin{supertabular}{l p{10.5cm}}
  $\Omega$ & \blt the region of space occupied by a body\\
  $x, y, z$ & \blt coordinates in Cartesian space (L) (m)\\
  $\bar{\textbf{T}}$ & \blt surface traction (StressU) (Pa)\\
\end{supertabular}\\
~\newline

\noindent \textbf{Superscripts}\\
~\newline

\noindent \begin{tabular}{l l}
  $\dot{~}$ & \blt time derivative\\
\end{tabular}\\
~\newline

\noindent \textbf{Subscripts}\\
~\newline

\noindent
\begin{tabular}{l p{12cm}}
  $0$ & \blt original configuration or initial stress\\
  $1, 2, 3$ & \blt used to indicate different materials and used for indicating different coordinate axes\\
\end{tabular}\\
~\newline

\noindent \textbf{Prefixes}\\
~\newline

\noindent
\begin{tabular}{l l}
  $\Delta$ & \blt finite change in following quantity\\
  $d$ & \blt infinitesimal change in the following quantity\\
\end{tabular}\\
~\newline

\section*{Abbreviations and Acronyms}

\begin{tabular}{l l}
  1D & \blt one dimensional\\
  2D & \blt two dimensional\\
  3D & \blt three dimensional\\
  CA & \blt Commonality Analysis\\
  DSL & \blt Domain Specific Language\\
\end{tabular}\\

\newpage

\section*{Types}

\begin{tabular}{l p{8cm}}
  $\mathbb{R}$ & \blt real numbers\\
  $\mathbb{R}^{+}: \{ x: \mathbb{R} | x \geq 0 : x \}$ & \blt positive real numbers\\
\end{tabular}\\

\newpage

\section{Introduction}
Physics engines are utilized in both research and industry and can be applied to many different types of problems. Precise physics engines are used in weather forecasting and fluid particle modeling while less precise real-time engines have long been used in the video game industry for modeling rigid and soft body dynamics along with collision detection. Because of their variety of applications, there can be major differences in the structure of these physics engines, including dimensionality, precision of collision accuracy, and type of body modeling. However, much of the underlying calculations are the same because the core function of all engines is the same: applying forces to a set of geometry. The similarities that exist between many of these physics engines allows us to consider them as a program family as described by \cite{SmithMcCutchanAndCao2007}.
\par
To gain insight into this family of physics engines, this document presents
a Commonality Analysis (CA) of the family. This CA for a family of physics engines follows the guidelines of a CA for a program family as found in \cite{CukaAndWeiss1997} and \cite{WeissAndLai1999}. The CA can be seen as a method for summarizing the requirements for all potential engines that are considered to be within the scope of the physics engine family. The CA includes documentation of terminology, commonalities (including goals, theoretical models and assumptions) and variabilities.
\par
The template used for documenting this CA is based on the template utilized in \cite{SmithMcCutchanAndCarette2014} and \cite{SmithAndChen2004b}. In the remainder of this section the purpose of the report is described, the scope of the family is delineated and an outline of the remaining sections of the document is provided.

\subsection{Purpose of Document}
The purpose of this document is to summarize a family of game physics engines, where the features included in each engine's construction characterizes the depth and bredth of scenarios which that engine can simulate. Often, the amount of features included in an engine contributes to the "flexibility" of the engine, or the degree to which it may model many different scenarios.

\subsection{Scope of the Family} \label{Sec_Scope}

 
\subsection{Organization of the Document}


\section{General System Description} \label{Sec_GenSystDescript}


\subsection{Potential System Context}


\subsection{Potential User Characteristics}

\subsection{Potential System Constraints}

\section{Commonalities} \label{Sec_Commonalities}

\subsection{Background Overview}

\subsection{Terminology Definition} \label{Sec_TerminologyDef}

Each definition in this section uses the same table structure, with the following rows:

\begin{description}

\item [Number:] All of the data definitions are assigned a unique number, which
  takes the form of a natural number with the prefix ``D.''  This number will be
  used for purposes of cross-referencing and traceability within this document.
\item [Label:] The label is a short identifying phrase, each with the prefix
  ``D\_.''  This label provides a mnemonic that helps with quickly remembering
  which definition is being presented.  Moreover, the label will useful when an
  external document needs to reference one of the definitions in this document.
\item [Symbol:] This field shows the symbol that is used to represent variables
  related to this concept.  For instance, the natural strain rate tensor is
  represented by the symbol $\dot{\bm{\epsilon}}$.  Later in the document when
  the symbol $\dot{\bm{\epsilon}}$ appears, possibly with superscripts such as
  $e$ for elastic or $vp$ for viscoplastic, this is an indication that the term
  in question is a measure of natural strain rate.  If the symbol should appear
  without the dot over it, then it is referring to a total strain and not the
  rate of strain.
\item [Type:] Each variable designated by a symbol has a type associated with
  it, which is listed in this field of the data definition template.  The type
  information helps to clarify the meaning of the symbol and the variables.
\item [Units:] Where applicable the units associated with the symbol are given.
  These units are given in terms of the mass (M), length (L), time (t) and
  temperature (T).  For convenience the units are also given in SI.
\item [Related Items:] A related item is a data definition or assumption that is
  used by the current definition.  That is, if the used data definition or
  assumption should change, then the current data definition will also need to
  be modified.  As mentioned above, data definitions have a prefix ``D.''
  Assumptions will have the prefix ``A.''
\item [Sources:] This field lists references that can be consulted for
  additional information on the concept in question.
\item [Description:] The actual definition is given here.  In some cases where
  the description is lengthy, some of the details are moved to a section
  following the table.  When appropriate the description will reference the
  related definitions and assumptions.
\item [History:] Each data definition ends with a history of the definition,
  including the creation date and any subsequent modifications.

\end{description}

~\newline

\noindent
\begin{minipage}{\textwidth}
\begin{tabular}{| p{\colAwidth} | p{\colBwidth} |}
\hline
\rowcolor[gray]{0.9}
Number:  & D\refstepcounter{defnum}\thedefnum \label{D_Stress}\\
\hline
Label: & D\_Stress\\ \hline
Symbol: & $\bm{\sigma}$\\ \hline
Type: & array1DT\\ \hline
Units: & Each component of the stress tensor has units of StressU (Pa)\\
\hline
Related Items: & \aref{A_ContinuumHypothesis}, \aref{A_NoDistribMoments}, \aref{A_CartesianCoord},
\aref{A_DescriptionOfMotion}\\
\hline Sources: &  \citet[pages 35--41]{MaterialsCA}; \citet[pages 64--119]{Malvern1969}; \citet[pages 44--76]{Mase1970}; \citet[pages 1--21]{BeerAndJohnston1985}\\
\hline
Description: & The stress provides a measure of force per unit area associated with different directions at a point 
within a body.  A detailed definition of the stress tensor is provided below.  This definition is for the true stress,
which is also sometimes called the Cauchy stress.\\
\hline History: & Created -- June 14, 2007\\
\hline
\end{tabular}
\end{minipage}\\
~\newline

\subsection{Goal Statement}

\noindent
\begin{minipage}{\textwidth}
\begin{tabular}{| p{\colAwidth} | p{\colBwidth}|}
\hline
\rowcolor[gray]{0.9}
Number:  & G\refstepcounter{goalnum}\thegoalnum  \label{G_StressDetermination}\\ 
\hline
%awkward work around for correct spacing - put G first so that spacing is correct, even though \gthegoalnum is defined
Label: & G\_StressDetermination\\ \hline
Description: & Given the initial stress and the deformation history of a material particle, determine the stress
within the material particle.\\
\hline
Related Items: & \tref{T_ConstitEquation}\\ \hline
History: & Created -- June 8, 2007\\ \hline
\end{tabular}
\end{minipage}\\
~\newline

\subsection{Assumptions}

\begin{description}
\item [Equation:] Some of the assumptions include an equation, when this makes the
description more precise.  For each equation the types of each of the terms is listed.
\item [Rationale:] This field justifies the appropriateness of the assumption within the context of the current
family.  If changes in the assumptions are made in the future, it will be because the rationale is inadequate in some
sense.
\end{description}

~\newline

% should there be a section or a table that summarizes the symbols in more detail than in the table of symbols?
% what about types? input versus output versus "temporary"

\noindent
\begin{minipage}{\textwidth}
\begin{tabular}{| p{\colAwidth} | p{\colBwidth}|}
\hline
\rowcolor[gray]{0.9}
Number:  & A\refstepcounter{assumpnum}\theassumpnum \label{A_ContinuumHypothesis}\\
\hline
Label: & A\_ContinuumHypothesis\\ \hline
Related Items: & --\\ \hline
Equation: & --\\ \hline
Description: & The underlying molecular structure of matter is not considered and gaps and empty spaces within a
material particle are ignored.  The material is assumed to be continuous.\\ \hline 
Rationale: & The continuum hypothesis allows for the definition of stress and strain at a point.  Although not
strictly true, the notion that matter is continuous fits with what one usually sees at the macroscopic scale.  Even at
small fractions of the scales that engineering problems typically deal with there is generally an enormous number of
molecules, which makes the averages of the physical properties stable.  As the final, and most important justification,
the theories built using the continuum hypothesis often provide quantitative predictions that agree closely with
experimental data.\\ \hline
Source: & \citet[33--34]{Long1961}; \citet[pages 1--2]{Malvern1969}\\ \hline
History: & Created -- Aug 23, 2007\\ \hline
\end{tabular}
\end{minipage}\\
~\newline


\subsection{Theoretical Model}


\section{Variabilities}


\section{Dependence Graphs}


\section{Sample Family Members} \label{Sec_SampleFamilyMembers}


\newpage

\bibliographystyle{plainnat}

\bibliography{FamilyOfPhysicsEngines}

\end{document}
