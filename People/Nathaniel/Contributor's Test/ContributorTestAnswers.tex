\documentclass[12pt,fleqn]{examtst}
\usepackage{graphicx}
\usepackage{amssymb}
\usepackage{amsmath}
\usepackage{listings}
\usepackage{multirow}
\usepackage{multicol}
\usepackage{hhline}
\usepackage{booktabs}

\begin{document}

\newcommand{\soln}{y} %y for yes and n for no

\lstset{language=python, basicstyle=\ttfamily, breaklines=true,
  showspaces=false, showstringspaces=false, breakatwhitespace=true, texcl=true,
  escapeinside={\%*}{*)}}

\newcommand{\codeit}[1]{\texttt{\textit{#1}}}

\begin{center}
  {\large \bf Drasil Future Contributors' Test}\\[1ex]
  {\large \bf McMaster University}\\[1ex]
  {\large Faculty of Engineering, Department of Computing and Software}\\[1ex]
  \ifthenelse{\equal{\soln}{y}}{\large {\bf Answer Key:} Large arrow
    ($\Longleftarrow$) for correct% , small ($\leftarrow$) for partially
    % correct
  }{}
\end{center}

\medskip

\noindent
Future Drasil Contributors, \textbf{Version 1}  \hfill Nathaniel Hu \\
DURATION OF EXAMINATION: 2.5 hours \\
MCMASTER UNIVERSITY TEST \hfill June 3rd, 2020

\medskip

\noindent
\rule[3 mm]{\textwidth}{0.5mm}

\begin{minipage}[t]{1.0\textwidth}
\textbf{Please CLEARLY print}:\\[2mm]

NAME:\\[1ex]

\newsavebox{\bb}\newsavebox{\bbb}
\sbox{\bb}{\framebox[1cm]{\rule{0mm}{7mm}}}
\sbox{\bbb}{\usebox{\bb}\usebox{\bb}\usebox{\bb}\usebox{\bb}\usebox{\bb}\usebox{\bb}\usebox{\bb}\usebox{\bb}\usebox{\bb}}

Student ID: \usebox{\bbb} \\[2mm]

\rule[3 mm]{\textwidth}{0.5mm}

This examination paper includes \noofpages pages and
16 % VARIABILITY
questions. You are responsible for ensuring that your copy of the examination
paper is complete. Bring any discrepancy to the attention
of your invigilator.\\

\textbf{Special Instructions}:

\begin{enumerate}

\item It is your responsibility to ensure that the answer sheet is properly
  completed. Your examination result depends upon proper attention to the
  instructions on the next page.
\item Calculators, computers, cell phones, and all other electronic devices are
  \textbf{not} to be utilized.
\item Read each question carefully.
\item Try to allocate your time sensibly and divide it appropriately between the
  questions.
\item Select the \textbf{best} answer for each question.
\item The set $\mathbb{N}$ is assumed to include $0$.
\end{enumerate}
\end{minipage}\\

\hspace{14cm}
\begin{minipage}[t]{0.2\textwidth}
\newcommand{\markheight}{\rule[-2mm]{0 mm}{7 mm}}
\begin{tabular}[t]{|c|p{1.5 cm}|r|}
\hline
1--12 & \markheight & 12\\
\hline
13--16 & \markheight & 8\\

\hline
Total & \markheight & 20 \\
\hline

\end{tabular}
\end{minipage}

\examheader{Drasil \ifthenelse{\equal{\soln}{y}} {\hfill SOLUTIONS} }

\renewcommand{\labelenumi}{\Alph{enumi}.}

%%%%%%%%%%%%%%%%%%%%%%%%%%%%%%%%%%

\noindent
\begin{minipage}{\textwidth}

\question{1 mark}
What command is used to create a local version of a remote repository?

\begin{enumerate}
    \item \lstinline{git copy} $<$link to repo$>$
    \item \lstinline{git clone} $<$link to repo$>$ \marker
    \item \lstinline{clone} $<$link to repo$>$
    \item \lstinline{copy} $<$link to repo$>$
\end{enumerate}

\question{1 mark}
The \lstinline{git status} command displays:

\begin{enumerate}
    \item verified status of all repo files
    \item list of last 5 commits to the repo
    \item paths with differences between current state of repo and last commit \marker
    \item changes made to files after last commit
\end{enumerate}

\question{1 mark}
The \lstinline{git pull} command is used to:

\begin{enumerate}
    \item sync your local version with remote version of the repo \marker
    \item displays changes made to the remote version of the repo
    \item sync remote version with your local version of the repo
    \item updates the remote repo with other people's changes
\end{enumerate}

\question{1 mark}
What is the correct order of the following \lstinline{git} commands?

\begin{enumerate}
    \item \lstinline{git push, git add, git clone, git commit}
    \item \lstinline{git add, git clone, git push, git commit}
    \item \lstinline{git clone, git add, git push, git commit}
    \item \lstinline{git clone, git add, git commit, git push} \marker
\end{enumerate}

\end{minipage}

%%%%%%%%%%%%%%%%%%%%%%%%%%%%%%%%%%

\newpage
\noindent
\begin{minipage}{\textwidth}

\question{1 mark}
Which of the below phrases can you use to link a relevant issue to a pull request on GitHub, without closing the issue once the PR is merged? (\#HASH is the issue hash \#)

\begin{enumerate}
    \item closes \#HASH
    \item contributes to \#HASH \marker
\end{enumerate}

\question{1 mark}
The preferred coding style describes how lines should not be more than \_\_\_ characters wide.

\begin{enumerate}
    \item 60
    \item 90
    \item 80 \marker
    \item 70
\end{enumerate}

\question{1 mark}
When making pull requests involving changes to multiple files (e.g. Haskell scripts and stable folder files), remember to:

\begin{enumerate}
    \item use multiple \lstinline{git add} to stage multiple files before doing a single commit \marker
    \item update the stable files first and push them, then repeat with the scripts using \lstinline{git add}
    \item only update all scripts in one commit using multiple \lstinline{git add}
    \item only update all changed 'stable' folder files in one commit using multiple \lstinline{git add}
\end{enumerate}

\question{1 mark}
The \lstinline{git branch} command:

\begin{enumerate}
    \item shows the items under your current branch on your local repo
    \item shows a list of all your current branches on your local repo \marker
    \item shows branches dependent on your current branch on your local repo
    \item shows a list of all current branches on the remote repo
\end{enumerate}

\end{minipage}

%%%%%%%%%%%%%%%%%%%%%%%%%%%%%%%%%%

\newpage
\noindent
\begin{minipage}{\textwidth}

\question{1 mark}
When closing an issue, please provide:

\begin{enumerate}
    \item Rationale
    \item Relevant Links to other related issues
    \item Linked Pull Requests
    \item All of the above \marker
\end{enumerate}

\question{1 mark}
To \textbf{only} build the 2D Rigid Body Physics Library example $($gamephysics\_diff$)$ using the Drasil framework, run the command:

\begin{enumerate}
    \item \lstinline{setup gamephysics_diff}
    \item \lstinline{make gamephysics_diff} \marker
    \item \lstinline{stack exec gamephysics_diff}
    \item \lstinline{make}
\end{enumerate}

\question{1 mark}
If you are a Windows OS user working with Git Bash App, always run the command each time you open an instance of git bash:

\begin{enumerate}
    \item \lstinline{chcp.com 65001} 
    \item \lstinline{set encoding=utf-8}
    \item \lstinline{chcp 65001}
    \item \lstinline{A. or C.} \marker
\end{enumerate}

\question{1 mark}
To run the Glass-BR example $($glassbr$)$ using the Drasil framework $($assume that the example has already been built$)$, run the command:

\begin{enumerate}
    \item \lstinline{make glassbr}
    \item \lstinline{exec glassbr}
    \item \lstinline{stack glassbr}
    \item \lstinline{stack exec glassbr} \marker
\end{enumerate}

\end{minipage}

%%%%%%%%%%%%%%%%%%%%%%%%%%%%%%%%%%

\newpage
\noindent
\begin{minipage}{\textwidth}

\question{2 marks}
When creating a new issue on GitHub, describe two tips to follow that help to ensure that the new issue includes enough information (context).

\answer{Answer here.}{0cm}{0cm}

Sample Answers (any two of the following work):
\begin{itemize}
    \item include (annotated) excerpts of PDF/HTML documentation when referring to output (desired or generated); highlighting specific portions of such relevant screenshots helps important info stand out to the reader
    \item linking related issues, pull requests, comments and commit hashes; provides easy navigation through related content and discussion significant to the issue
    \item creating permalinks (permanent links) to code sections/snippets relevant to the issue
    \item directly inserting code into Issue comments (using markdown format); useful for recommending/proposing solutions of multi-line code (mentioning the language allows for syntax highlighting, which is helpful)
\end{itemize}

\question{2 marks}


\answer{Answer here.}{0cm}{7cm}

Sample Answer

\end{minipage}

%%%%%%%%%%%%%%%%%%%%%%%%%%%%%%%%%%

\newpage
\noindent
\begin{minipage}{\textwidth}

\question{2 marks}


\answer{Answer here.}{0cm}{7cm}

Sample Answer

\question{2 marks}


\answer{Answer here.}{0cm}{7cm}

Sample Answer

\end{minipage}

\end{document}