
\section{Requirements}
\label{Sec:Requirements}
This section provides the functional requirements, the tasks and behaviours that the software is expected to complete, and the non-functional requirements, the qualities that the software is expected to exhibit.
\subsection{Functional Requirements}
\label{Sec:FRs}
This section provides the functional requirements, the tasks and behaviours that the software is expected to complete.
\begin{itemize}
\item[Input-Parameters:\phantomsection\label{inputParams}]Input the quantities from \hyperref[Table:ReqInputs]{Table:ReqInputs}, which define the launch angle, launch speed, and target position.
\item[Verify-Parameters:\phantomsection\label{verifyParams}]Check the entered input parameters to ensure that they do not exceed the data constraints mentioned in \hyperref[Sec:DataConstraints]{Section: Data Constraints}. If any of the input parameters are out of bounds, an error message is displayed and the calculations stop.
\item[Calculate-Values:\phantomsection\label{calcValues}]Calculate the following quantities: ${t_{flight}}$ (from \hyperref[IM:calOfLandingTime]{IM: calOfLandingTime}), ${p_{land}}$ (from \hyperref[IM:calOfLandingDist]{IM: calOfLandingDist}), ${d_{offset}}$ (from \hyperref[IM:offsetIM]{IM: offsetIM}), and $s$ (from \hyperref[IM:messageIM]{IM: messageIM}).
\item[Output-Values:\phantomsection\label{outputValues}]Output $s$ (from \hyperref[IM:messageIM]{IM: messageIM}) and ${d_{offset}}$ (from \hyperref[IM:offsetIM]{IM: offsetIM}).
\end{itemize}
\begin{longtable}{l l l}
\toprule
\textbf{Symbol} & \textbf{Description} & \textbf{Units}
\\
\midrule
\endhead
${p_{target}}$ & Target position & m
\\
${v_{launch}}$ & Launch speed & $\frac{\text{m}}{\text{s}}$
\\
$θ$ & Launch angle & rad
\\
\bottomrule
\caption{Required Inputs following \hyperref[inputParams]{FR: Input-Parameters}}
\label{Table:ReqInputs}
\end{longtable}
\subsection{Non-Functional Requirements}
\label{Sec:NFRs}
This section provides the non-functional requirements, the qualities that the software is expected to exhibit.
\begin{itemize}
\item[Correct:\phantomsection\label{correct}]The outputs of the code have the properties described in \hyperref[Sec:CorSolProps]{Section: Properties of a Correct Solution}.
\item[Verifiable:\phantomsection\label{verifiable}]The code is tested with complete verification and validation plan.
\item[Understandable:\phantomsection\label{understandable}]The code is modularized with complete module guide and module interface specification.
\item[Reusable:\phantomsection\label{reusable}]The code is modularized.
\item[Maintainable:\phantomsection\label{maintainable}]The traceability between requirements, assumptions, theoretical models, general definitions, data definitions, instance models, likely changes, unlikely changes, and modules is completely recorded in traceability matrices in the SRS and module guide.
\item[Portable:\phantomsection\label{portable}]The code is able to be run in different environments.
\end{itemize}
\section{Traceability Matrices and Graphs}
\label{Sec:TraceMatrices}
The purpose of the traceability matrices is to provide easy references on what has to be additionally modified if a certain component is changed. Every time a component is changed, the items in the column of that component that are marked with an ``X'' should be modified as well. \hyperref[Table:TraceMatAvsAll]{Table:TraceMatAvsAll} shows the dependencies of data definitions, theoretical models, general definitions, instance models, requirements, likely changes, and unlikely changes on the assumptions. \hyperref[Table:TraceMatRefvsRef]{Table:TraceMatRefvsRef} shows the dependencies of data definitions, theoretical models, general definitions, and instance models with each other. \hyperref[Table:TraceMatAllvsR]{Table:TraceMatAllvsR} shows the dependencies of requirements, goal statements on the data definitions, theoretical models, general definitions, and instance models.
\begin{longtable}{l l l l l l l l l l l l l l l}
\toprule
\textbf{} & \textbf{\hyperref[twoDMotion]{A: twoDMotion}} & \textbf{\hyperref[cartSyst]{A: cartSyst}} & \textbf{\hyperref[yAxisGravity]{A: yAxisGravity}} & \textbf{\hyperref[launchOrigin]{A: launchOrigin}} & \textbf{\hyperref[targetXAxis]{A: targetXAxis}} & \textbf{\hyperref[posXDirection]{A: posXDirection}} & \textbf{\hyperref[constAccel]{A: constAccel}} & \textbf{\hyperref[accelXZero]{A: accelXZero}} & \textbf{\hyperref[accelYGravity]{A: accelYGravity}} & \textbf{\hyperref[neglectDrag]{A: neglectDrag}} & \textbf{\hyperref[pointMass]{A: pointMass}} & \textbf{\hyperref[freeFlight]{A: freeFlight}} & \textbf{\hyperref[neglectCurv]{A: neglectCurv}} & \textbf{\hyperref[timeStartZero]{A: timeStartZero}}
\\
\midrule
\endhead
\hyperref[DD:vecMag]{DD: vecMag} &  &  &  &  &  &  &  &  &  &  &  &  &  & 
\\
\hyperref[DD:speedIX]{DD: speedIX} &  &  &  &  &  &  &  &  &  &  &  &  &  & 
\\
\hyperref[DD:speedIY]{DD: speedIY} &  &  &  &  &  &  &  &  &  &  &  &  &  & 
\\
\hyperref[TM:acceleration]{TM: acceleration} &  &  &  &  &  &  &  &  &  &  &  &  &  & 
\\
\hyperref[TM:velocity]{TM: velocity} &  &  &  &  &  &  &  &  &  &  &  &  &  & 
\\
\hyperref[GD:rectVel]{GD: rectVel} &  &  &  &  &  &  &  &  &  &  & X &  &  & X
\\
\hyperref[GD:rectPos]{GD: rectPos} &  &  &  &  &  &  &  &  &  &  & X &  &  & X
\\
\hyperref[GD:velVec]{GD: velVec} & X & X &  &  &  &  & X &  &  &  &  &  &  & X
\\
\hyperref[GD:posVec]{GD: posVec} & X & X &  &  &  &  & X &  &  &  &  &  &  & X
\\
\hyperref[IM:calOfLandingTime]{IM: calOfLandingTime} &  &  & X & X & X & X &  &  & X &  &  &  &  & X
\\
\hyperref[IM:calOfLandingDist]{IM: calOfLandingDist} &  &  & X & X &  & X &  & X &  &  &  &  &  & 
\\
\hyperref[IM:offsetIM]{IM: offsetIM} &  &  &  &  &  & X &  &  &  &  &  &  &  & 
\\
\hyperref[IM:messageIM]{IM: messageIM} &  &  &  &  &  & X &  &  &  &  &  &  &  & 
\\
\hyperref[inputParams]{FR: Input-Parameters} &  &  &  &  &  &  &  &  &  &  &  &  &  & 
\\
\hyperref[verifyParams]{FR: Verify-Parameters} &  &  &  &  &  &  &  &  &  &  &  &  &  & 
\\
\hyperref[calcValues]{FR: Calculate-Values} &  &  &  &  &  &  &  &  &  &  &  &  &  & 
\\
\hyperref[outputValues]{FR: Output-Values} &  &  &  &  &  &  &  &  &  &  &  &  &  & 
\\
\hyperref[correct]{NFR: Correct} &  &  &  &  &  &  &  &  &  &  &  &  &  & 
\\
\hyperref[verifiable]{NFR: Verifiable} &  &  &  &  &  &  &  &  &  &  &  &  &  & 
\\
\hyperref[understandable]{NFR: Understandable} &  &  &  &  &  &  &  &  &  &  &  &  &  & 
\\
\hyperref[reusable]{NFR: Reusable} &  &  &  &  &  &  &  &  &  &  &  &  &  & 
\\
\hyperref[maintainable]{NFR: Maintainable} &  &  &  &  &  &  &  &  &  &  &  &  &  & 
\\
\hyperref[portable]{NFR: Portable} &  &  &  &  &  &  &  &  &  &  &  &  &  & 
\\
\bottomrule
\caption{Traceability Matrix Showing the Connections Between Assumptions and Other Items}
\label{Table:TraceMatAvsAll}
\end{longtable}
\begin{longtable}{l l l l l l l l l l l l l l}
\toprule
\textbf{} & \textbf{\hyperref[DD:vecMag]{DD: vecMag}} & \textbf{\hyperref[DD:speedIX]{DD: speedIX}} & \textbf{\hyperref[DD:speedIY]{DD: speedIY}} & \textbf{\hyperref[TM:acceleration]{TM: acceleration}} & \textbf{\hyperref[TM:velocity]{TM: velocity}} & \textbf{\hyperref[GD:rectVel]{GD: rectVel}} & \textbf{\hyperref[GD:rectPos]{GD: rectPos}} & \textbf{\hyperref[GD:velVec]{GD: velVec}} & \textbf{\hyperref[GD:posVec]{GD: posVec}} & \textbf{\hyperref[IM:calOfLandingTime]{IM: calOfLandingTime}} & \textbf{\hyperref[IM:calOfLandingDist]{IM: calOfLandingDist}} & \textbf{\hyperref[IM:offsetIM]{IM: offsetIM}} & \textbf{\hyperref[IM:messageIM]{IM: messageIM}}
\\
\midrule
\endhead
\hyperref[DD:vecMag]{DD: vecMag} &  &  &  &  &  &  &  &  &  &  &  &  & 
\\
\hyperref[DD:speedIX]{DD: speedIX} & X &  &  &  &  &  &  &  &  &  &  &  & 
\\
\hyperref[DD:speedIY]{DD: speedIY} & X &  &  &  &  &  &  &  &  &  &  &  & 
\\
\hyperref[TM:acceleration]{TM: acceleration} &  &  &  &  &  &  &  &  &  &  &  &  & 
\\
\hyperref[TM:velocity]{TM: velocity} &  &  &  &  &  &  &  &  &  &  &  &  & 
\\
\hyperref[GD:rectVel]{GD: rectVel} &  &  &  & X &  &  &  &  &  &  &  &  & 
\\
\hyperref[GD:rectPos]{GD: rectPos} &  &  &  &  & X & X &  &  &  &  &  &  & 
\\
\hyperref[GD:velVec]{GD: velVec} &  &  &  &  &  & X &  &  &  &  &  &  & 
\\
\hyperref[GD:posVec]{GD: posVec} &  &  &  &  &  &  & X &  &  &  &  &  & 
\\
\hyperref[IM:calOfLandingTime]{IM: calOfLandingTime} &  &  & X &  &  &  &  &  & X &  &  &  & 
\\
\hyperref[IM:calOfLandingDist]{IM: calOfLandingDist} &  & X &  &  &  &  &  &  & X & X &  &  & 
\\
\hyperref[IM:offsetIM]{IM: offsetIM} &  &  &  &  &  &  &  &  &  &  & X &  & 
\\
\hyperref[IM:messageIM]{IM: messageIM} &  &  &  &  &  &  &  &  &  &  &  & X & 
\\
\bottomrule
\caption{Traceability Matrix Showing the Connections Between Items and Other Sections}
\label{Table:TraceMatRefvsRef}
\end{longtable}
\begin{longtable}{l l l l l l l l l l l l l l l l l l l l l l l l}
\toprule
\textbf{} & \textbf{\hyperref[DD:vecMag]{DD: vecMag}} & \textbf{\hyperref[DD:speedIX]{DD: speedIX}} & \textbf{\hyperref[DD:speedIY]{DD: speedIY}} & \textbf{\hyperref[TM:acceleration]{TM: acceleration}} & \textbf{\hyperref[TM:velocity]{TM: velocity}} & \textbf{\hyperref[GD:rectVel]{GD: rectVel}} & \textbf{\hyperref[GD:rectPos]{GD: rectPos}} & \textbf{\hyperref[GD:velVec]{GD: velVec}} & \textbf{\hyperref[GD:posVec]{GD: posVec}} & \textbf{\hyperref[IM:calOfLandingTime]{IM: calOfLandingTime}} & \textbf{\hyperref[IM:calOfLandingDist]{IM: calOfLandingDist}} & \textbf{\hyperref[IM:offsetIM]{IM: offsetIM}} & \textbf{\hyperref[IM:messageIM]{IM: messageIM}} & \textbf{\hyperref[inputParams]{FR: Input-Parameters}} & \textbf{\hyperref[verifyParams]{FR: Verify-Parameters}} & \textbf{\hyperref[calcValues]{FR: Calculate-Values}} & \textbf{\hyperref[outputValues]{FR: Output-Values}} & \textbf{\hyperref[correct]{NFR: Correct}} & \textbf{\hyperref[verifiable]{NFR: Verifiable}} & \textbf{\hyperref[understandable]{NFR: Understandable}} & \textbf{\hyperref[reusable]{NFR: Reusable}} & \textbf{\hyperref[maintainable]{NFR: Maintainable}} & \textbf{\hyperref[portable]{NFR: Portable}}
\\
\midrule
\endhead
\hyperref[targetHit]{GS: targetHit} &  &  &  &  &  &  &  &  &  &  &  &  &  &  &  &  &  &  &  &  &  &  & 
\\
\hyperref[inputParams]{FR: Input-Parameters} &  &  &  &  &  &  &  &  &  &  &  &  &  &  &  &  &  &  &  &  &  &  & 
\\
\hyperref[verifyParams]{FR: Verify-Parameters} &  &  &  &  &  &  &  &  &  &  &  &  &  &  &  &  &  &  &  &  &  &  & 
\\
\hyperref[calcValues]{FR: Calculate-Values} &  &  &  &  &  &  &  &  &  & X & X & X & X &  &  &  &  &  &  &  &  &  & 
\\
\hyperref[outputValues]{FR: Output-Values} &  &  &  &  &  &  &  &  &  &  &  & X & X &  &  &  &  &  &  &  &  &  & 
\\
\hyperref[correct]{NFR: Correct} &  &  &  &  &  &  &  &  &  &  &  &  &  &  &  &  &  &  &  &  &  &  & 
\\
\hyperref[verifiable]{NFR: Verifiable} &  &  &  &  &  &  &  &  &  &  &  &  &  &  &  &  &  &  &  &  &  &  & 
\\
\hyperref[understandable]{NFR: Understandable} &  &  &  &  &  &  &  &  &  &  &  &  &  &  &  &  &  &  &  &  &  &  & 
\\
\hyperref[reusable]{NFR: Reusable} &  &  &  &  &  &  &  &  &  &  &  &  &  &  &  &  &  &  &  &  &  &  & 
\\
\hyperref[maintainable]{NFR: Maintainable} &  &  &  &  &  &  &  &  &  &  &  &  &  &  &  &  &  &  &  &  &  &  & 
\\
\hyperref[portable]{NFR: Portable} &  &  &  &  &  &  &  &  &  &  &  &  &  &  &  &  &  &  &  &  &  &  & 
\\
\bottomrule
\caption{Traceability Matrix Showing the Connections Between Requirements, Goal Statements and Other Items}
\label{Table:TraceMatAllvsR}
\end{longtable}
\section{Values of Auxiliary Constants}
\label{Sec:AuxConstants}
This section contains the standard values that are used for calculations in Projectile.
\begin{longtable}{l l l l}
\toprule
\textbf{Symbol} & \textbf{Description} & \textbf{Value} & \textbf{Unit}
\\
\midrule
\endhead
$g$ & gravitational acceleration & $9.8$ & $\frac{\text{m}}{\text{s}^{2}}$
\\
$ε$ & hit tolerance & $2.0\%$ & --
\\
$π$ & ratio of circumference to diameter for any circle & $3.14159265$ & --
\\
\bottomrule
\caption{Auxiliary Constants}
\label{Table:TAuxConsts}
\end{longtable}
\section{References}
\label{Sec:References}
\begin{filecontents*}{bibfile.bib}
@book{hibbeler2004,
author={Hibbeler, R. C.},
title={Engineering Mechanics: Dynamics},
publisher={Pearson Prentice Hall},
year={2004}}

@misc{accelerationWiki,
author={Wikipedia Contributors},
title={Acceleration},
howpublished={\url{https://en.wikipedia.org/wiki/Acceleration}},
month=jun,
year={2019}}

@misc{cartesianWiki,
author={Wikipedia Contributors},
title={Cartesian coordinate system},
howpublished={\url{https://en.wikipedia.org/wiki/Cartesian\_coordinate\_system}},
month=jun,
year={2019}}

@misc{velocityWiki,
author={Wikipedia Contributors},
title={Velocity},
howpublished={\url{https://en.wikipedia.org/wiki/Velocity}},
month=jun,
year={2019}}
\end{filecontents*}
\nocite{*}
\bibstyle{ieeetr}
\printbibliography[heading=none]
\end{document}
