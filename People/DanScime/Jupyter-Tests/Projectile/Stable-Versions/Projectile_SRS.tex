\noindent \begin{minipage}{\textwidth}
\begin{tabular}{>{\raggedright}p{0.13\textwidth}>{\raggedright\arraybackslash}p{0.82\textwidth}}
\toprule \textbf{Refname} & \textbf{DD:speedIY}
\phantomsection 
\label{DD:speedIY}
\\ \midrule \\
Label & $y$-component of initial velocity
\\ \midrule \\
Symbol & ${{v_{y}}^{i}}$
\\ \midrule \\
Units & $\frac{\text{m}}{\text{s}}$
\\ \midrule \\
Equation & \begin{displaymath}
{{v_{y}}^{i}}={v^{i}} \sin\left(θ\right)
\end{displaymath}
\\ \midrule \\
Description & \begin{symbDescription}
\item{${{v_{y}}^{i}}$ is the $y$-component of initial velocity ($\frac{\text{m}}{\text{s}}$)}
\item{${v^{i}}$ is the initial speed ($\frac{\text{m}}{\text{s}}$)}
\item{$θ$ is the launch angle (rad)}
\end{symbDescription}
\\ \midrule \\
Notes & ${v^{i}}$ is from \hyperref[DD:vecMag]{DD: vecMag}.
$θ$ is shown in \hyperref[Figure:Launch]{Fig:Launch}.
\\ \midrule \\
Source & --
\\ \midrule \\
RefBy & \hyperref[IM:calOfLandingTime]{IM: calOfLandingTime}
\\ \bottomrule
\end{tabular}
\end{minipage}
\subsubsection{Instance Models}
\label{Sec:IMs}
This section transforms the problem defined in \hyperref[Sec:ProbDesc]{Section: Problem Description} into one which is expressed in mathematical terms. It uses concrete symbols defined in \hyperref[Sec:DDs]{Section: Data Definitions} to replace the abstract symbols in the models identified in \hyperref[Sec:TMs]{Section: Theoretical Models} and \hyperref[Sec:GDs]{Section: General Definitions}.
\par~

\noindent \begin{minipage}{\textwidth}
\begin{tabular}{>{\raggedright}p{0.13\textwidth}>{\raggedright\arraybackslash}p{0.82\textwidth}}
\toprule \textbf{Refname} & \textbf{IM:calOfLandingTime}
\phantomsection 
\label{IM:calOfLandingTime}
\\ \midrule \\
Label & Calculation of landing time
\\ \midrule \\
Input & ${v_{launch}}$, $θ$
\\ \midrule \\
Output & ${t_{flight}}$
\\ \midrule \\
Input Constraints & \begin{displaymath}
{v_{launch}}>0
\end{displaymath}
\begin{displaymath}
0<θ<\frac{π}{2}
\end{displaymath}
\\ \midrule \\
Output Constraints & \begin{displaymath}
{t_{flight}}>0
\end{displaymath}
\\ \midrule \\
Equation & \begin{displaymath}
{t_{flight}}=\frac{2 {v_{launch}} \sin\left(θ\right)}{g}
\end{displaymath}
\\ \midrule \\
Description & \begin{symbDescription}
\item{${t_{flight}}$ is the flight duration (s)}
\item{${v_{launch}}$ is the launch speed ($\frac{\text{m}}{\text{s}}$)}
\item{$θ$ is the launch angle (rad)}
\item{$g$ is the gravitational acceleration ($\frac{\text{m}}{\text{s}^{2}}$)}
\end{symbDescription}
\\ \midrule \\
Notes & The constraint $0<θ<\frac{π}{2}$ is from \hyperref[posXDirection]{A: posXDirection} and \hyperref[yAxisGravity]{A: yAxisGravity}, and is shown in \hyperref[Figure:Launch]{Fig:Launch}.
$g$ is defined in \hyperref[Sec:AuxConstants]{Section: Values of Auxiliary Constants}.
The constraint ${t_{flight}}>0$ is from \hyperref[timeStartZero]{A: timeStartZero}.
\\ \midrule \\
Source & --
\\ \midrule \\
RefBy & \hyperref[IM:calOfLandingDist]{IM: calOfLandingDist} and \hyperref[calcValues]{FR: Calculate-Values}
\\ \bottomrule
\end{tabular}
\end{minipage}
\paragraph{Detailed derivation of flight duration:}
\label{IM:calOfLandingTimeDeriv}
We know that ${{p_{y}}^{i}}=0$ (\hyperref[launchOrigin]{A: launchOrigin}) and ${{a_{y}}^{c}}=-g$ (\hyperref[accelYGravity]{A: accelYGravity}). Substituting these values into the y-direction of \hyperref[GD:posVec]{GD: posVec} gives us:
\begin{displaymath}
{p_{y}}={{v_{y}}^{i}} t-\frac{g t^{2}}{2}
\end{displaymath}
To find the time that the projectile lands, we want to find the $t$ value (${t_{flight}}$) where ${p_{y}}=0$ (since the target is on the $x$-axis from \hyperref[targetXAxis]{A: targetXAxis}). From the equation above we get:
\begin{displaymath}
{{v_{y}}^{i}} {t_{flight}}-\frac{g {t_{flight}}^{2}}{2}=0
\end{displaymath}
Dividing by ${t_{flight}}$ (with the constraint ${t_{flight}}>0$) gives us:
\begin{displaymath}
{{v_{y}}^{i}}-\frac{g {t_{flight}}}{2}=0
\end{displaymath}
Solving for ${t_{flight}}$ gives us:
\begin{displaymath}
{t_{flight}}=\frac{2 {{v_{y}}^{i}}}{g}
\end{displaymath}
From \hyperref[DD:speedIY]{DD: speedIY} (with ${v^{i}}={v_{launch}}$) we can replace ${{v_{y}}^{i}}$:
\begin{displaymath}
{t_{flight}}=\frac{2 {v_{launch}} \sin\left(θ\right)}{g}
\end{displaymath}
\par~

\noindent \begin{minipage}{\textwidth}
\begin{tabular}{>{\raggedright}p{0.13\textwidth}>{\raggedright\arraybackslash}p{0.82\textwidth}}
\toprule \textbf{Refname} & \textbf{IM:calOfLandingDist}
\phantomsection 
\label{IM:calOfLandingDist}
\\ \midrule \\
Label & Calculation of landing position
\\ \midrule \\
Input & ${v_{launch}}$, $θ$
\\ \midrule \\
Output & ${p_{land}}$
\\ \midrule \\
Input Constraints & \begin{displaymath}
{v_{launch}}>0
\end{displaymath}
\begin{displaymath}
0<θ<\frac{π}{2}
\end{displaymath}
\\ \midrule \\
Output Constraints & \begin{displaymath}
{p_{land}}>0
\end{displaymath}
\\ \midrule \\
Equation & \begin{displaymath}
{p_{land}}=\frac{2 {v_{launch}}^{2} \sin\left(θ\right) \cos\left(θ\right)}{g}
\end{displaymath}
\\ \midrule \\
Description & \begin{symbDescription}
\item{${p_{land}}$ is the landing position (m)}
\item{${v_{launch}}$ is the launch speed ($\frac{\text{m}}{\text{s}}$)}
\item{$θ$ is the launch angle (rad)}
\item{$g$ is the gravitational acceleration ($\frac{\text{m}}{\text{s}^{2}}$)}
\end{symbDescription}
\\ \midrule \\
Notes & The constraint $0<θ<\frac{π}{2}$ is from \hyperref[posXDirection]{A: posXDirection} and \hyperref[yAxisGravity]{A: yAxisGravity}, and is shown in \hyperref[Figure:Launch]{Fig:Launch}.
$g$ is defined in \hyperref[Sec:AuxConstants]{Section: Values of Auxiliary Constants}.
The constraint ${p_{land}}>0$ is from \hyperref[posXDirection]{A: posXDirection}.
\\ \midrule \\
Source & --
\\ \midrule \\
RefBy & \hyperref[IM:offsetIM]{IM: offsetIM} and \hyperref[calcValues]{FR: Calculate-Values}
\\ \bottomrule
\end{tabular}
\end{minipage}
\paragraph{Detailed derivation of landing position:}
\label{IM:calOfLandingDistDeriv}
We know that ${{p_{x}}^{i}}=0$ (\hyperref[launchOrigin]{A: launchOrigin}) and ${{a_{x}}^{c}}=0$ (\hyperref[accelXZero]{A: accelXZero}). Substituting these values into the x-direction of \hyperref[GD:posVec]{GD: posVec} gives us:
\begin{displaymath}
{p_{x}}={{v_{x}}^{i}} t
\end{displaymath}
To find the landing position, we want to find the ${p_{x}}$ value (${p_{land}}$) at flight duration (from \hyperref[IM:calOfLandingTime]{IM: calOfLandingTime}):
\begin{displaymath}
{p_{land}}=\frac{{{v_{x}}^{i}}\cdot{}2 {v_{launch}} \sin\left(θ\right)}{g}
\end{displaymath}
From \hyperref[DD:speedIX]{DD: speedIX} (with ${v^{i}}={v_{launch}}$) we can replace ${{v_{x}}^{i}}$:
\begin{displaymath}
{p_{land}}=\frac{{v_{launch}} \cos\left(θ\right)\cdot{}2 {v_{launch}} \sin\left(θ\right)}{g}
\end{displaymath}
Rearranging this gives us the required equation:
\begin{displaymath}
{p_{land}}=\frac{2 {v_{launch}}^{2} \sin\left(θ\right) \cos\left(θ\right)}{g}
\end{displaymath}
\par~

\noindent \begin{minipage}{\textwidth}
\begin{tabular}{>{\raggedright}p{0.13\textwidth}>{\raggedright\arraybackslash}p{0.82\textwidth}}
\toprule \textbf{Refname} & \textbf{IM:offsetIM}
\phantomsection 
\label{IM:offsetIM}
\\ \midrule \\
Label & Offset
\\ \midrule \\
Input & ${p_{land}}$, ${p_{target}}$
\\ \midrule \\
Output & ${d_{offset}}$
\\ \midrule \\
Input Constraints & \begin{displaymath}
{p_{land}}>0
\end{displaymath}
\begin{displaymath}
{p_{target}}>0
\end{displaymath}
\\ \midrule \\
Output Constraints & 
\\ \midrule \\
Equation & \begin{displaymath}
{d_{offset}}={p_{land}}-{p_{target}}
\end{displaymath}
\\ \midrule \\
Description & \begin{symbDescription}
\item{${d_{offset}}$ is the distance between the target position and the landing position (m)}
\item{${p_{land}}$ is the landing position (m)}
\item{${p_{target}}$ is the target position (m)}
\end{symbDescription}
\\ \midrule \\
Notes & ${p_{land}}$ is from \hyperref[IM:calOfLandingDist]{IM: calOfLandingDist}.
The constraints ${p_{land}}>0$ and ${p_{target}}>0$ are from \hyperref[posXDirection]{A: posXDirection}.
\\ \midrule \\
Source & --
\\ \midrule \\
RefBy & \hyperref[outputValues]{FR: Output-Values}, \hyperref[IM:messageIM]{IM: messageIM}, and \hyperref[calcValues]{FR: Calculate-Values}
\\ \bottomrule
\end{tabular}
\end{minipage}
\par~

\noindent \begin{minipage}{\textwidth}
\begin{tabular}{>{\raggedright}p{0.13\textwidth}>{\raggedright\arraybackslash}p{0.82\textwidth}}
\toprule \textbf{Refname} & \textbf{IM:messageIM}
\phantomsection 
\label{IM:messageIM}
\\ \midrule \\
Label & Output message
\\ \midrule \\
Input & ${d_{offset}}$, ${p_{target}}$
\\ \midrule \\
Output & $s$
\\ \midrule \\
Input Constraints & \begin{displaymath}
{p_{target}}>0
\end{displaymath}
\begin{displaymath}
{d_{offset}}>-{p_{land}}
\end{displaymath}
\\ \midrule \\
Output Constraints & 
\\ \midrule \\
Equation & \begin{displaymath}
s=\begin{cases}
\text{``The target was hit.''}, & |\frac{{d_{offset}}}{{p_{target}}}|<ε\\
\text{``The projectile fell short.''}, & {d_{offset}}<0\\
\text{``The projectile went long.''}, & {d_{offset}}>0
\end{cases}
\end{displaymath}
\\ \midrule \\
Description & \begin{symbDescription}
\item{$s$ is the output message as a string (Unitless)}
\item{${d_{offset}}$ is the distance between the target position and the landing position (m)}
\item{${p_{target}}$ is the target position (m)}
\item{$ε$ is the hit tolerance (Unitless)}
\end{symbDescription}
\\ \midrule \\
Notes & ${d_{offset}}$ is from \hyperref[IM:offsetIM]{IM: offsetIM}.
The constraint ${p_{target}}>0$ is from \hyperref[posXDirection]{A: posXDirection}.
The constraint ${d_{offset}}>-{p_{land}}$ is from the fact that ${p_{land}}>0$, from \hyperref[posXDirection]{A: posXDirection}.
$ε$ is defined in \hyperref[Sec:AuxConstants]{Section: Values of Auxiliary Constants}.
\\ \midrule \\
Source & --
\\ \midrule \\
RefBy & \hyperref[outputValues]{FR: Output-Values} and \hyperref[calcValues]{FR: Calculate-Values}
\\ \bottomrule
\end{tabular}
\end{minipage}
\subsubsection{Data Constraints}
\label{Sec:DataConstraints}
\hyperref[Table:InDataConstraints]{Table:InDataConstraints} shows the data constraints on the input variables. The column for physical constraints gives the physical limitations on the range of values that can be taken by the variable. The uncertainty column provides an estimate of the confidence with which the physical quantities can be measured. This information would be part of the input if one were performing an uncertainty quantification exercise. The constraints are conservative, to give the user of the model the flexibility to experiment with unusual situations. The column of typical values is intended to provide a feel for a common scenario.
\begin{longtable}{l l l l}
\toprule
\textbf{Var} & \textbf{Physical Constraints} & \textbf{Typical Value} & \textbf{Uncert.}
\\
\midrule
\endhead
${p_{target}}$ & ${p_{target}}>0$ & $1000$ m & 10$\%$
\\
${v_{launch}}$ & ${v_{launch}}>0$ & $100$ $\frac{\text{m}}{\text{s}}$ & 10$\%$
\\
$θ$ & $0<θ<\frac{π}{2}$ & $\frac{π}{4}$ rad & 10$\%$
\\
\bottomrule
\caption{Input Data Constraints}
\label{Table:InDataConstraints}
\end{longtable}
\subsubsection{Properties of a Correct Solution}
\label{Sec:CorSolProps}
\hyperref[Table:OutDataConstraints]{Table:OutDataConstraints} shows the data constraints on the output variables. The column for physical constraints gives the physical limitations on the range of values that can be taken by the variable.
\begin{longtable}{l l}
\toprule
\textbf{Var} & \textbf{Physical Constraints}
\\
\midrule
\endhead
${p_{land}}$ & ${p_{land}}>0$
\\
${d_{offset}}$ & ${d_{offset}}>-{p_{land}}$
\\
\bottomrule
\caption{Output Data Constraints}
\label{Table:OutDataConstraints}
\end{longtable}
\section{Requirements}
\label{Sec:Requirements}
This section provides the functional requirements, the tasks and behaviours that the software is expected to complete, and the non-functional requirements, the qualities that the software is expected to exhibit.
\subsection{Functional Requirements}
\label{Sec:FRs}
This section provides the functional requirements, the tasks and behaviours that the software is expected to complete.
\begin{itemize}
\item[Input-Parameters:\phantomsection\label{inputParams}]Input the quantities from \hyperref[Table:ReqInputs]{Table:ReqInputs}, which define the launch angle, launch speed, and target position.
\item[Verify-Parameters:\phantomsection\label{verifyParams}]Check the entered input parameters to ensure that they do not exceed the data constraints mentioned in \hyperref[Sec:DataConstraints]{Section: Data Constraints}. If any of the input parameters are out of bounds, an error message is displayed and the calculations stop.
\item[Calculate-Values:\phantomsection\label{calcValues}]Calculate the following quantities: ${t_{flight}}$ (from \hyperref[IM:calOfLandingTime]{IM: calOfLandingTime}), ${p_{land}}$ (from \hyperref[IM:calOfLandingDist]{IM: calOfLandingDist}), ${d_{offset}}$ (from \hyperref[IM:offsetIM]{IM: offsetIM}), and $s$ (from \hyperref[IM:messageIM]{IM: messageIM}).
\item[Output-Values:\phantomsection\label{outputValues}]Output $s$ (from \hyperref[IM:messageIM]{IM: messageIM}) and ${d_{offset}}$ (from \hyperref[IM:offsetIM]{IM: offsetIM}).
\end{itemize}
\begin{longtable}{l l l}
\toprule
\textbf{Symbol} & \textbf{Description} & \textbf{Units}
\\
\midrule
\endhead
${p_{target}}$ & Target position & m
\\
${v_{launch}}$ & Launch speed & $\frac{\text{m}}{\text{s}}$
\\
$θ$ & Launch angle & rad
\\
\bottomrule
\caption{Required Inputs following \hyperref[inputParams]{FR: Input-Parameters}}
\label{Table:ReqInputs}
\end{longtable}
\subsection{Non-Functional Requirements}
\label{Sec:NFRs}
This section provides the non-functional requirements, the qualities that the software is expected to exhibit.
\begin{itemize}
\item[Correct:\phantomsection\label{correct}]The outputs of the code have the properties described in \hyperref[Sec:CorSolProps]{Section: Properties of a Correct Solution}.
\item[Verifiable:\phantomsection\label{verifiable}]The code is tested with complete verification and validation plan.
\item[Understandable:\phantomsection\label{understandable}]The code is modularized with complete module guide and module interface specification.
\item[Reusable:\phantomsection\label{reusable}]The code is modularized.
\item[Maintainable:\phantomsection\label{maintainable}]The traceability between requirements, assumptions, theoretical models, general definitions, data definitions, instance models, likely changes, unlikely changes, and modules is completely recorded in traceability matrices in the SRS and module guide.
\item[Portable:\phantomsection\label{portable}]The code is able to be run in different environments.
\end{itemize}
\section{Traceability Matrices and Graphs}
\label{Sec:TraceMatrices}
The purpose of the traceability matrices is to provide easy references on what has to be additionally modified if a certain component is changed. Every time a component is changed, the items in the column of that component that are marked with an ``X'' should be modified as well. \hyperref[Table:TraceMatAvsAll]{Table:TraceMatAvsAll} shows the dependencies of data definitions, theoretical models, general definitions, instance models, requirements, likely changes, and unlikely changes on the assumptions. \hyperref[Table:TraceMatRefvsRef]{Table:TraceMatRefvsRef} shows the dependencies of data definitions, theoretical models, general definitions, and instance models with each other. \hyperref[Table:TraceMatAllvsR]{Table:TraceMatAllvsR} shows the dependencies of requirements, goal statements on the data definitions, theoretical models, general definitions, and instance models.
\begin{longtable}{l l l l l l l l l l l l l l l}
\toprule
\textbf{} & \textbf{\hyperref[twoDMotion]{A: twoDMotion}} & \textbf{\hyperref[cartSyst]{A: cartSyst}} & \textbf{\hyperref[yAxisGravity]{A: yAxisGravity}} & \textbf{\hyperref[launchOrigin]{A: launchOrigin}} & \textbf{\hyperref[targetXAxis]{A: targetXAxis}} & \textbf{\hyperref[posXDirection]{A: posXDirection}} & \textbf{\hyperref[constAccel]{A: constAccel}} & \textbf{\hyperref[accelXZero]{A: accelXZero}} & \textbf{\hyperref[accelYGravity]{A: accelYGravity}} & \textbf{\hyperref[neglectDrag]{A: neglectDrag}} & \textbf{\hyperref[pointMass]{A: pointMass}} & \textbf{\hyperref[freeFlight]{A: freeFlight}} & \textbf{\hyperref[neglectCurv]{A: neglectCurv}} & \textbf{\hyperref[timeStartZero]{A: timeStartZero}}
\\
\midrule
\endhead
\hyperref[DD:vecMag]{DD: vecMag} &  &  &  &  &  &  &  &  &  &  &  &  &  & 
\\
\hyperref[DD:speedIX]{DD: speedIX} &  &  &  &  &  &  &  &  &  &  &  &  &  & 
\\
\hyperref[DD:speedIY]{DD: speedIY} &  &  &  &  &  &  &  &  &  &  &  &  &  & 
\\
\hyperref[TM:acceleration]{TM: acceleration} &  &  &  &  &  &  &  &  &  &  &  &  &  & 
\\
\hyperref[TM:velocity]{TM: velocity} &  &  &  &  &  &  &  &  &  &  &  &  &  & 
\\
\hyperref[GD:rectVel]{GD: rectVel} &  &  &  &  &  &  &  &  &  &  & X &  &  & X
\\
\hyperref[GD:rectPos]{GD: rectPos} &  &  &  &  &  &  &  &  &  &  & X &  &  & X
\\
\hyperref[GD:velVec]{GD: velVec} & X & X &  &  &  &  & X &  &  &  &  &  &  & X
\\
\hyperref[GD:posVec]{GD: posVec} & X & X &  &  &  &  & X &  &  &  &  &  &  & X
\\
\hyperref[IM:calOfLandingTime]{IM: calOfLandingTime} &  &  & X & X & X & X &  &  & X &  &  &  &  & X
\\
\hyperref[IM:calOfLandingDist]{IM: calOfLandingDist} &  &  & X & X &  & X &  & X &  &  &  &  &  & 
\\
\hyperref[IM:offsetIM]{IM: offsetIM} &  &  &  &  &  & X &  &  &  &  &  &  &  & 
\\
\hyperref[IM:messageIM]{IM: messageIM} &  &  &  &  &  & X &  &  &  &  &  &  &  & 
\\
\hyperref[inputParams]{FR: Input-Parameters} &  &  &  &  &  &  &  &  &  &  &  &  &  & 
\\
\hyperref[verifyParams]{FR: Verify-Parameters} &  &  &  &  &  &  &  &  &  &  &  &  &  & 
\\
\hyperref[calcValues]{FR: Calculate-Values} &  &  &  &  &  &  &  &  &  &  &  &  &  & 
\\
\hyperref[outputValues]{FR: Output-Values} &  &  &  &  &  &  &  &  &  &  &  &  &  & 
\\
\hyperref[correct]{NFR: Correct} &  &  &  &  &  &  &  &  &  &  &  &  &  & 
\\
\hyperref[verifiable]{NFR: Verifiable} &  &  &  &  &  &  &  &  &  &  &  &  &  & 
\\
\hyperref[understandable]{NFR: Understandable} &  &  &  &  &  &  &  &  &  &  &  &  &  & 
\\
\hyperref[reusable]{NFR: Reusable} &  &  &  &  &  &  &  &  &  &  &  &  &  & 
\\
\hyperref[maintainable]{NFR: Maintainable} &  &  &  &  &  &  &  &  &  &  &  &  &  & 
\\
\hyperref[portable]{NFR: Portable} &  &  &  &  &  &  &  &  &  &  &  &  &  & 
\\
\bottomrule
\caption{Traceability Matrix Showing the Connections Between Assumptions and Other Items}
\label{Table:TraceMatAvsAll}
\end{longtable}
\begin{longtable}{l l l l l l l l l l l l l l}
\toprule
\textbf{} & \textbf{\hyperref[DD:vecMag]{DD: vecMag}} & \textbf{\hyperref[DD:speedIX]{DD: speedIX}} & \textbf{\hyperref[DD:speedIY]{DD: speedIY}} & \textbf{\hyperref[TM:acceleration]{TM: acceleration}} & \textbf{\hyperref[TM:velocity]{TM: velocity}} & \textbf{\hyperref[GD:rectVel]{GD: rectVel}} & \textbf{\hyperref[GD:rectPos]{GD: rectPos}} & \textbf{\hyperref[GD:velVec]{GD: velVec}} & \textbf{\hyperref[GD:posVec]{GD: posVec}} & \textbf{\hyperref[IM:calOfLandingTime]{IM: calOfLandingTime}} & \textbf{\hyperref[IM:calOfLandingDist]{IM: calOfLandingDist}} & \textbf{\hyperref[IM:offsetIM]{IM: offsetIM}} & \textbf{\hyperref[IM:messageIM]{IM: messageIM}}
\\
\midrule
\endhead
\hyperref[DD:vecMag]{DD: vecMag} &  &  &  &  &  &  &  &  &  &  &  &  & 
\\
\hyperref[DD:speedIX]{DD: speedIX} & X &  &  &  &  &  &  &  &  &  &  &  & 
\\
\hyperref[DD:speedIY]{DD: speedIY} & X &  &  &  &  &  &  &  &  &  &  &  & 
\\
\hyperref[TM:acceleration]{TM: acceleration} &  &  &  &  &  &  &  &  &  &  &  &  & 
\\
\hyperref[TM:velocity]{TM: velocity} &  &  &  &  &  &  &  &  &  &  &  &  & 
\\
\hyperref[GD:rectVel]{GD: rectVel} &  &  &  & X &  &  &  &  &  &  &  &  & 
\\
\hyperref[GD:rectPos]{GD: rectPos} &  &  &  &  & X & X &  &  &  &  &  &  & 
\\
\hyperref[GD:velVec]{GD: velVec} &  &  &  &  &  & X &  &  &  &  &  &  & 
\\
\hyperref[GD:posVec]{GD: posVec} &  &  &  &  &  &  & X &  &  &  &  &  & 
\\
\hyperref[IM:calOfLandingTime]{IM: calOfLandingTime} &  &  & X &  &  &  &  &  & X &  &  &  & 
\\
\hyperref[IM:calOfLandingDist]{IM: calOfLandingDist} &  & X &  &  &  &  &  &  & X & X &  &  & 
\\
\hyperref[IM:offsetIM]{IM: offsetIM} &  &  &  &  &  &  &  &  &  &  & X &  & 
\\
\hyperref[IM:messageIM]{IM: messageIM} &  &  &  &  &  &  &  &  &  &  &  & X & 
\\
\bottomrule
\caption{Traceability Matrix Showing the Connections Between Items and Other Sections}
\label{Table:TraceMatRefvsRef}
\end{longtable}
\begin{longtable}{l l l l l l l l l l l l l l l l l l l l l l l l}
\toprule
\textbf{} & \textbf{\hyperref[DD:vecMag]{DD: vecMag}} & \textbf{\hyperref[DD:speedIX]{DD: speedIX}} & \textbf{\hyperref[DD:speedIY]{DD: speedIY}} & \textbf{\hyperref[TM:acceleration]{TM: acceleration}} & \textbf{\hyperref[TM:velocity]{TM: velocity}} & \textbf{\hyperref[GD:rectVel]{GD: rectVel}} & \textbf{\hyperref[GD:rectPos]{GD: rectPos}} & \textbf{\hyperref[GD:velVec]{GD: velVec}} & \textbf{\hyperref[GD:posVec]{GD: posVec}} & \textbf{\hyperref[IM:calOfLandingTime]{IM: calOfLandingTime}} & \textbf{\hyperref[IM:calOfLandingDist]{IM: calOfLandingDist}} & \textbf{\hyperref[IM:offsetIM]{IM: offsetIM}} & \textbf{\hyperref[IM:messageIM]{IM: messageIM}} & \textbf{\hyperref[inputParams]{FR: Input-Parameters}} & \textbf{\hyperref[verifyParams]{FR: Verify-Parameters}} & \textbf{\hyperref[calcValues]{FR: Calculate-Values}} & \textbf{\hyperref[outputValues]{FR: Output-Values}} & \textbf{\hyperref[correct]{NFR: Correct}} & \textbf{\hyperref[verifiable]{NFR: Verifiable}} & \textbf{\hyperref[understandable]{NFR: Understandable}} & \textbf{\hyperref[reusable]{NFR: Reusable}} & \textbf{\hyperref[maintainable]{NFR: Maintainable}} & \textbf{\hyperref[portable]{NFR: Portable}}
\\
\midrule
\endhead
\hyperref[targetHit]{GS: targetHit} &  &  &  &  &  &  &  &  &  &  &  &  &  &  &  &  &  &  &  &  &  &  & 
\\
\hyperref[inputParams]{FR: Input-Parameters} &  &  &  &  &  &  &  &  &  &  &  &  &  &  &  &  &  &  &  &  &  &  & 
\\
\hyperref[verifyParams]{FR: Verify-Parameters} &  &  &  &  &  &  &  &  &  &  &  &  &  &  &  &  &  &  &  &  &  &  & 
\\
\hyperref[calcValues]{FR: Calculate-Values} &  &  &  &  &  &  &  &  &  & X & X & X & X &  &  &  &  &  &  &  &  &  & 
\\
\hyperref[outputValues]{FR: Output-Values} &  &  &  &  &  &  &  &  &  &  &  & X & X &  &  &  &  &  &  &  &  &  & 
\\
\hyperref[correct]{NFR: Correct} &  &  &  &  &  &  &  &  &  &  &  &  &  &  &  &  &  &  &  &  &  &  & 
\\
\hyperref[verifiable]{NFR: Verifiable} &  &  &  &  &  &  &  &  &  &  &  &  &  &  &  &  &  &  &  &  &  &  & 
\\
\hyperref[understandable]{NFR: Understandable} &  &  &  &  &  &  &  &  &  &  &  &  &  &  &  &  &  &  &  &  &  &  & 
\\
\hyperref[reusable]{NFR: Reusable} &  &  &  &  &  &  &  &  &  &  &  &  &  &  &  &  &  &  &  &  &  &  & 
\\
\hyperref[maintainable]{NFR: Maintainable} &  &  &  &  &  &  &  &  &  &  &  &  &  &  &  &  &  &  &  &  &  &  & 
\\
\hyperref[portable]{NFR: Portable} &  &  &  &  &  &  &  &  &  &  &  &  &  &  &  &  &  &  &  &  &  &  & 
\\
\bottomrule
\caption{Traceability Matrix Showing the Connections Between Requirements, Goal Statements and Other Items}
\label{Table:TraceMatAllvsR}
\end{longtable}
\section{Values of Auxiliary Constants}
\label{Sec:AuxConstants}
This section contains the standard values that are used for calculations in Projectile.
\begin{longtable}{l l l l}
\toprule
\textbf{Symbol} & \textbf{Description} & \textbf{Value} & \textbf{Unit}
\\
\midrule
\endhead
$g$ & gravitational acceleration & $9.8$ & $\frac{\text{m}}{\text{s}^{2}}$
\\
$ε$ & hit tolerance & $2.0\%$ & --
\\
$π$ & ratio of circumference to diameter for any circle & $3.14159265$ & --
\\
\bottomrule
\caption{Auxiliary Constants}
\label{Table:TAuxConsts}
\end{longtable}
\section{References}
\label{Sec:References}
\begin{filecontents*}{bibfile.bib}
@book{hibbeler2004,
author={Hibbeler, R. C.},
title={Engineering Mechanics: Dynamics},
publisher={Pearson Prentice Hall},
year={2004}}

@misc{accelerationWiki,
author={Wikipedia Contributors},
title={Acceleration},
howpublished={\url{https://en.wikipedia.org/wiki/Acceleration}},
month=jun,
year={2019}}

@misc{cartesianWiki,
author={Wikipedia Contributors},
title={Cartesian coordinate system},
howpublished={\url{https://en.wikipedia.org/wiki/Cartesian\_coordinate\_system}},
month=jun,
year={2019}}

@misc{velocityWiki,
author={Wikipedia Contributors},
title={Velocity},
howpublished={\url{https://en.wikipedia.org/wiki/Velocity}},
month=jun,
year={2019}}
\end{filecontents*}
\nocite{*}
\bibstyle{ieeetr}
\printbibliography[heading=none]
\end{document}
