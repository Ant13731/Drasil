% $Header: /cvsroot/latex-beamer/latex-beamer/solutions/conference-talks/conference-ornate-20min.en.tex,v 1.6 2004/10/07 20:53:08 tantau Exp $

\documentclass{beamer}

\mode<presentation>
{
%  \usetheme{Hannover}
\usetheme[width=0.7in]{Hannover}
% or ...

  \setbeamercovered{transparent}
  % or whatever (possibly just delete it)
}
\usepackage{longtable}
\usepackage{booktabs}

\usepackage[english]{babel}
% or whatever

\usepackage[latin1]{inputenc}
% or whatever

\usepackage{times}
%\usepackage[T1]{fontenc}
% Or whatever. Note that the encoding and the font should match. If T1
% does not look nice, try deleting the line with the fontenc.
%\usepackage{logictheme}

%\usepackage{hhline}
\usepackage{multirow}
%\usepackage{multicol}
%\usepackage{array}
%\usepackage{supertabular}
%\usepackage{amsmath}
%\usepackage{amsfonts}
\usepackage{totpages}
\usepackage{hyperref}
%\usepackage{booktabs}

%\usepackage{bm}

\usepackage{listings}
\newcommand{\blt}{- } %used for bullets in a list

\newcounter{datadefnum} %Datadefinition Number
\newcommand{\ddthedatadefnum}{DD\thedatadefnum}
\newcommand{\ddref}[1]{DD\ref{#1}}

\newcommand{\colAwidth}{0.2\textwidth}
\newcommand{\colBwidth}{0.73\textwidth}

\renewcommand{\arraystretch}{0.6} %so that tables with equations do not look crowded

\pgfdeclareimage[height=0.7cm]{logo}{McMasterLogo}
\pgfdeclareimage[width=10cm]{d0}{d0}
\pgfdeclareimage[width=10cm]{d2}{d2}
\pgfdeclareimage[width=10cm]{d1}{d1}
\pgfdeclareimage[width=10cm]{d3}{d3}
\pgfdeclareimage[width=10cm]{d5}{d5}
\title[\pgfuseimage{logo}]  % (optional, use only with long paper titles)
{PhD Committee Meeting \#1}

%\subtitle
%{Include Only If Paper Has a Subtitle}

\author[Slide \thepage~of \pageref{TotPages}] % (optional, use only with lots of
                                              % authors)
{Steven Palmer}
% - Give the names in the same order as the appear in the paper.
% - Use the \inst{?} command only if the authors have different
%   affiliation.

\institute[McMaster University] % (optional, but mostly needed)
{
  Computing and Software Department\\
  Faculty of Engineering\\
  McMaster University
}
% - Use the \inst command only if there are several affiliations.
% - Keep it simple, no one is interested in your street address.

\date[June 28, 2017] % (optional, should be abbreviation of conference name)
{June 28, 2017}
% - Either use conference name or its abbreviation.
% - Not really informative to the audience, more for people (including
%   yourself) who are reading the slides online

%\subject{computational science and engineering, software engineering, software
%  quality, literate programming, software requirements specification, document
%  driven design}
% This is only inserted into the PDF information catalog. Can be left
% out. 

% If you have a file called "university-logo-filename.xxx", where xxx
% is a graphic format that can be processed by latex or pdflatex,
% resp., then you can add a logo as follows:

%\pgfdeclareimage[height=0.5cm]{Mac-logo}{McMasterLogo}
%\logo{\pgfuseimage{Mac-logo}}

% Delete this, if you do not want the table of contents to pop up at
% the beginning of each subsection:
\AtBeginSubsection[]
{
  \begin{frame}<beamer>
    \frametitle{Outline}
    \tableofcontents[currentsection,currentsubsection]
  \end{frame}
}

% If you wish to uncover everything in a step-wise fashion, uncomment
% the following command: 

%\beamerdefaultoverlayspecification{<+->}

\beamertemplatenavigationsymbolsempty 

% have SRS and LP open during the presentation

\begin{document}

%%%%%%%%%%%%%%%%%%%%%%%%%%%%%%%%%%%%%%
\begin{frame}

\titlepage

\end{frame}

%%%%%%%%%%%%%%%%%%%%%%%%%%%%%%%%%%%%%%

\begin{frame}

\frametitle{Overview}
\tableofcontents
% You might wish to add the option [pausesections]

% make like a story - the phases - reason for, why works, advantages
% changing the history a bit to make a more rational narrative

\end{frame}

%%%%%%%%%%%%%%%%%%%%%%%%%%%%%%%%%%%%%%

\section[Introduction]{Introduction}

% \subsection[Important Software Qualities]{Scientific Computing Software
% Qualities}

%%%%%%%%%%%%%%%%%%%%%%%%%%%%%%%%%%%%%%

\begin{frame}

\frametitle{Education History}

\begin{itemize}
\item Ph.D. Computer Science
	\begin{itemize}
	\item McMaster University
    \item Started May 2016.
	\end{itemize}
\item B.A.Sc. Computer Science
	\begin{itemize}
	\item McMaster University 2016
	\end{itemize}
\item M.A.Sc. Chemical Engineering
    \begin{itemize}
    \item University of Toronto
    \item Withdrew ABD
    \end{itemize}
\item B.A.Sc. Chemical Engineering 
    \begin{itemize}
	\item University of Toronto 2011
	\end{itemize}

\end{itemize}
\end{frame}
%%%%%%%%%%%%%%%%%%%%%%%%%%%%%%%%%%%%%%

\section[Progress]{Current Progress}

%%%%%%%%%%%%%%%%%%%%%%%%%%%%%%%%%%%%%%

\begin{frame}

\frametitle{Requirements}

\begin{itemize}
\item Courses:
	\begin{itemize}
	\item CAS 701 -- Logic \& Discrete Math (A+)
	\item CAS 781 -- Category Theory (A)
	\item MATLS 711 -- Advanced Thermodynamics (transfer)
	\end{itemize}
\item[]
\item Comprehensive Exams:
    \begin{itemize}
    \item Computing Fundamentals -- Completed May 2017
    \item Computer Science -- Expected Nov 2017
    \end{itemize}
\end{itemize}
\end{frame}
%%%%%%%%%%%%%%%%%%%%%%%%%%%%%%%%%%%%%%

\begin{frame}

\frametitle{Research Project}

Drasil: A Framework for Literate Scientific Software

\begin{itemize}
\item Started in 2014 (Dan Szymczak, Ph.D. work)
\item[]
\item Overview:
\begin{itemize}
\item Drasil is a tool for generating documentation and code for scientific software
\item Captures expert knowledge in reusable ``chunks''
\item Uses ``recipes'' to organize chunks into documents 
\end{itemize}
\item[]
\item Benefits:
\begin{itemize}
\item Improve the qualities of verifiability, maintainability and reusability for scientific software
\item Simplify the certification/re-certification process by providing traceability across all documents and code
\end{itemize}
\end{itemize}
\end{frame}

%%%%%%%%%%%%%%%%%%%%%%%%%%%%%%%%%%%%%%

\begin{frame}

\frametitle{Dan's Work}

Dan's work has focused on:
\begin{itemize}
\item Developing the Drasil infrastructure 
\item Designing the chunk system
\item Designing recipes to produce a software requirements specification document from chunks
\end{itemize}

\end{frame}


%%%%%%%%%%%%%%%%%%%%%%%%%%%%%%%%%%%%%%

\begin{frame}

\frametitle{My Contribution}

My work will focus on the code generation side of Drasil.

\begin{itemize}
\item[] How do we get from requirements to an implementation?
\item[]
\end{itemize}

\pgfuseimage{d0}

\end{frame}


%%%%%%%%%%%%%%%%%%%%%%%%%%%%%%%%%%%%%%

\begin{frame}

\frametitle{My Contribution}

Using Dan's work as a starting point, this question can be restated:

\begin{itemize}
\item[] How do we get from {\bf a collection of knowledge} to an implementation?
\end{itemize}

\pgfuseimage{d2}

\end{frame}

%%%%%%%%%%%%%%%%%%%%%%%%%%%%%%%%%%%%%%

\begin{frame}

\frametitle{My Contribution}
In addition to the captured knowledge, there will need to be a 
way to accommodate choices:
\begin{itemize}
\item  Design Choices:
\begin{itemize}
\item What algorithms do we want to use?
\item Structure of input and output data?
\item ...
\end{itemize}
\item  Implementation Choices:
\begin{itemize}
\item What language do we want to generate?
\item What optimizations do we want to make?
\item ...
\end{itemize}
\end{itemize}

\pgfuseimage{d3}

\end{frame}

%%%%%%%%%%%%%%%%%%%%%%%%%%%%%%%%%%%%%%

\section[Work to Date]{Work to Date}

%%%%%%%%%%%%%%%%%%%%%%%%%%%%%%%%%%%%%%

\begin{frame}

\frametitle{Case Studies}

Bottom-up approach:  working backwards from an implementation to develop the code generation mechanism\\[12pt]

We currently have six case studies for use as models to be replicated using Drasil:
\begin{itemize}
\item GlassBR (risk assessment software for glass plates subjected to blast loading) 
\item Straight-forward example where code consists of inputs loaded from a file, a series of calculation function calls, and outputs saved to a file
\end{itemize}


\end{frame}

%%%%%%%%%%%%%%%%%%%%%%%%%%%%%%%%%%%%%%

\begin{frame}

\frametitle{Implementation Language}

\framesubtitle{GOOL}

First step:  Language Rendering\\[12pt]

Goal:  Provide rendering option for as many languages as we can!\\[12pt]

GOOL integrated into Drasil for use as primary Implementation Language:
\begin{itemize}
\item "Generic Object-Oriented Language"
\item Developed by Jason Costabile
\item Haskell DSL that allows rendering to different OO languages
\item Currently supports C++, C\#, Objective C, Java, Python, and Lua (can be extended to more)
\end{itemize}


\end{frame}

%%%%%%%%%%%%%%%%%%%%%%%%%%%%%%%%%%%%%%

\begin{frame}

\frametitle{GOOL}
Second step:  GOOL implementation of GlassBR (translated from Python)\\[12pt]

This required some modifications to GOOL:
\begin{itemize}
\item GOOL was extended to allow non-OO code rendering (function and variable declarations outside of classes) 
\item Support was added for common console and file IO routines
\item Other minor changes like library/module imports
\end{itemize}


\end{frame}

%%%%%%%%%%%%%%%%%%%%%%%%%%%%%%%%%%%%%%

\begin{frame}

\frametitle{GOOL}
\begin{itemize}
\item The extension for non-OO code should allow for rendering of non-OO imperative languages.
\item In the future we should be able to generate any imperative language!
\end{itemize}
\pgfuseimage{d5}

\end{frame}

%%%%%%%%%%%%%%%%%%%%%%%%%%%%%%%%%%%%%%

\begin{frame}

\frametitle{Design Language}

Third step:  Bridging the gap between knowledge chunks and GOOL code (currently here)\\[12pt]

\begin{itemize}
\item First attempt:  pulling out patterns in the GOOL implementation of GlassBR 
\item Generation of input, output, and calculations modules achieved
\item Code related to linear interpolation less obvious
\end{itemize}

\end{frame}


%%%%%%%%%%%%%%%%%%%%%%%%%%%%%%%%%%%%%%

\section[Future Plans]{Future Plans}

%%%%%%%%%%%%%%%%%%%%%%%%%%%%%%%%%%%%%%

\begin{frame}

\frametitle{Future Plans}
\framesubtitle{Requirements}


\begin{itemize}
\item Complete course requirements:
  \begin{itemize}
  \item 3/8 complete, 5 more required
  \item May look at taking some math courses outside of department
  \item 4 this year, 1 next year vs. 5 this year?
  \end{itemize}
\item Complete Computer Science comprehensive exam in November.
\end{itemize}
\end{frame}

%%%%%%%%%%%%%%%%%%%%%%%%%%%%%%%%%%%%%%

\begin{frame}

\frametitle{Future Plans}
\framesubtitle{Project}


\begin{itemize}
\item Develop a first revision of the design language:
\begin{itemize}
\item Finish code generation for GlassBR example to produce fully working code.
\item Extend code generation to other five examples.
\end{itemize}
\item[]
\item Implement C generation in GOOL
\item Work on test case generation
\end{itemize}
\end{frame}

%%%%%%%%%%%%%%%%%%%%%%%%%%%%%%%%%%%%%%

\begin{frame}
\begin{center}
\Huge Thank You!
\end{center}
\end{frame}

%%%%%%%%%%%%%%%%%%%%%%%%%%%%%%%%%%%%%%

\end{document}
