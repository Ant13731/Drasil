\documentclass{article}

\title{SRS for $h_g$ and $h_c$}

\author{Spencer Smith}

\usepackage{longtable}
\usepackage{booktabs}

\newcommand{\blt}{- } %used for bullets in a list

\newcounter{datadefnum} %Datadefinition Number
\newcommand{\ddthedatadefnum}{DD\thedatadefnum}
\newcommand{\ddref}[1]{DD\ref{#1}}

\newcommand{\colAwidth}{0.2\textwidth}
\newcommand{\colBwidth}{0.73\textwidth}

\renewcommand{\arraystretch}{1.2} %so that tables with equations do not look crowded

\begin{document}

\maketitle

\section*{Table of Units}

Throughout this document SI (Syst\`{e}me International d'Unit\'{e}s) is employed
as the unit system.  In addition to the basic units, several derived units are
employed as described below.  For each unit, the symbol is given followed by a
description of the unit with the SI name in parentheses.  ~\newline

\begin{longtable}{l p{11cm}}

m & \blt for length (metre)\\
kg & \blt for mass (kilogram)\\
s & \blt for time (second)\\
K & \blt for temperature (kelvin)\\
$^oC$ & \blt for temperature (centigrade)\\
J & \blt for energy (joule, J=$\mathrm{\frac{kg m^2}{s^2}}$)\\
cal & \blt for energy (calorie, cal $\approx$ 4.2 $\mathrm{\frac{kg m^2}{s^2}}$)\\
mol& \blt for amount of substance (mole)\\
W &\blt for power (watt, W=$\mathrm{\frac{kgm^2}{s^3}}$)\\

\end{longtable}

\section*{Table of Symbols}

The table that follows summarizes the symbols used in this document along with
their units.  The choice of symbols was made with the goal of being consistent
with the nuclear physics literature and that used in the FP manual.  The SI
units are listed in brackets following the definition of the symbol.

\begin{longtable}{l p{10.5cm}}
$h_c$ & \blt convective heat transfer coefficient between clad and coolant ($\mathrm{\frac{kW}{m^2C}}$)\\
$h_g$ & \blt effective heat transfer coefficient between clad and fuel surface ($\mathrm{\frac{kW}{m^2C}}$)\\
\end{longtable}

\section{Data Definitions}

~\newline
\noindent
\begin{minipage}{\textwidth}
\begin{tabular}{p{\colAwidth} p{\colBwidth}}
\toprule
\textbf{Number} & \textbf{DD\refstepcounter{datadefnum}\thedatadefnum} \label{hg}\\
\midrule
Label & $h_g$\\
\midrule
Units & $ML^0t^{-3}T^{-1}$\\
\midrule
SI equivalent & $\mathrm{\frac{kW}{m^{2\circ} C}}$\\
\midrule
Equation & $h_g$ =$ \frac{2k_{c}h_{p}}{2k_{c}+\tau_c h_{p}}$\\
\midrule
Description & $h_g$ is the  gap conductance\newline
$\tau_c$ is the clad thickness\newline
$h_p$ is initial gap film conductance\newline
$k_c$ is the clad conductivity\newline
NOTE: Equation taken from the code\\
\midrule
 Sources & source code\\
\bottomrule
\end{tabular}
\end{minipage}\\

~\newline
\noindent
\begin{minipage}{\textwidth}
\begin{tabular}{p{\colAwidth} p{\colBwidth}}
\toprule
\textbf{Number} & \textbf{DD\refstepcounter{datadefnum}\thedatadefnum \label{hc}}\\
\midrule
Label & $h_c$\\
\midrule
Units & $ML^0t^{-3}T^{-1}$\\
\midrule
SI equivalent & $\mathrm{\frac{kW}{m^{2o}C}}$\\
\midrule
Equation & $h_c$ =$\frac{ 2k_{c}h_{b}}{2k_{c}+\tau_ch_{b}}$\\
\midrule
Description & 
$h_c$ is the  effective heat transfer coefficient between the clad and the
coolant \newline
$\tau_c$ is the clad thickness\newline
$h_b$ is initial coolant film conductance\newline
$k_c$ is the clad conductivity\newline
NOTE: Equation taken from the code\\
\midrule
 Sources & source code \\
\bottomrule
\end{tabular}
\end{minipage}\\

\end{document}
