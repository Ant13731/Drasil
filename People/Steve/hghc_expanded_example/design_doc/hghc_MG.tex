\documentclass{article}

\title{MG for $h_g$ and $h_c$}

\author{Spencer Smith}

\usepackage{fullpage}
\usepackage{longtable}
\usepackage{booktabs}
\usepackage{multirow}
\usepackage{hyperref}

\hypersetup{
bookmarks=true,     % show bookmarks bar?
colorlinks=true,       % false: boxed links; true: colored links
linkcolor=red,          % color of internal links (change box color with linkbordercolor)
citecolor=blue,      % color of links to bibliography
filecolor=magenta,  % color of file links
urlcolor=cyan          % color of external links
}

\newcommand{\blt}{- } %used for bullets in a list

\newcounter{datadefnum} %Datadefinition Number
\newcommand{\ddthedatadefnum}{DD\thedatadefnum}
\newcommand{\ddref}[1]{DD\ref{#1}}

\newcommand{\colAwidth}{0.2\textwidth}
\newcommand{\colBwidth}{0.73\textwidth}

\newcounter{mnum}
\newcommand{\mthemnum}{M\themnum}
\newcommand{\mref}[1]{M\ref{#1}}

\renewcommand{\arraystretch}{1.2} %so that tables with equations do not look crowded

\begin{document}

\maketitle

\section*{Module Hierarchy}

This section provides an overview of the module design. Modules are summarized
in a hierarchy decomposed by secrets in Table \ref{TblMH}. The modules listed
below, which are leaves in the hierarchy tree, are the modules that will
actually be implemented.

\begin{description}
\item [\refstepcounter{mnum} \mthemnum \label{mHH}:] Hardware-Hiding Module
\item [\refstepcounter{mnum} \mthemnum \label{mInput}:] Calc Module
\end{description}

Note that \mref{mHH} is a commonly used module and is already implemented by the operating
system.  It will not be reimplemented.

\begin{table}[h!]
\centering
\begin{tabular}{p{0.3\textwidth} p{0.6\textwidth}}
\toprule
\textbf{Level 1} & \textbf{Level 2}\\
\midrule

{Hardware-Hiding Module} & ~ \\
\midrule

{Behaviour-Hiding Module} & Calc Module\\
\bottomrule

\end{tabular}
\caption{Module Hierarchy}
\label{TblMH}
\end{table}

\section*{Module Decomposition}

Modules are decomposed according to the principle of ``information hiding''
proposed by Parnas. The \emph{Secrets} field in a module
decomposition is a brief statement of the design decision hidden by the
module. The \emph{Services} field specifies \emph{what} the module will do
without documenting \emph{how} to do it. For each module, a suggestion for the
implementing software is given under the \emph{Implemented By} title. If the
entry is \emph{OS}, this means that the module is provided by the operating
system or by standard programming language libraries. If the entry is
\emph{Matlab}, this means that the module is provided by Matlab.  \emph{HGHC} means the
module will be implemented by the HGHC software.
% should reference a command for the name, in case it changes
Only the leaf modules in the
hierarchy have to be implemented. If a dash (\emph{--}) is shown, this means
that the module is not a leaf and will not have to be implemented. Whether or
not this module is implemented depends on the programming language
selected.

\subsection*{Hardware Hiding Modules (\mref{mHH})}

\begin{description}
\item[Secrets:]The data structure and algorithm used to implement the virtual
  hardware.
\item[Services:]Serves as a virtual hardware used by the rest of the
  system. This module provides the interface between the hardware and the
  software. So, the system can use it to display outputs or to accept inputs.
\item[Implemented By:] OS
\end{description}

\subsection*{Behaviour-Hiding Module}

\begin{description}
\item[Secrets:]The contents of the required behaviors.
\item[Services:]Includes programs that provide externally visible behavior of
  the system as specified in the software requirements specification (SRS)
  documents. This module serves as a communication layer between the
  hardware-hiding module and the software decision module. The programs in this
  module will need to change if there are changes in the SRS.
\item[Implemented By:] --
\end{description}

\subsubsection*{Input Format Module (\mref{mInput})}

\begin{description}
\item[Secrets:]The equations used to calculate $h_g$ and $h_c$
\item[Services:]Calculates the value of $h_g$ and $h_c$
\item[Implemented By:] HGHC
\end{description}

\end{document}
