% !TEX TS-program = pdflatex
% !TEX encoding = UTF-8 Unicode

% This is a simple template for a LaTeX document using the "article" class.
% See "book", "report", "letter" for other types of document.

\documentclass[twocolumn, 10pt]{article} % use larger type; default would be 10pt

\usepackage[utf8]{inputenc} % set input encoding (not needed with XeLaTeX)

\renewcommand{\rmdefault}{ptm}
%\usepackage[mtbold,noTS1]{mathtime}%

%%% PAGE DIMENSIONS
\usepackage{geometry} % to change the page dimensions
\geometry{letterpaper} % or letterpaper (US) or a5paper or....
% \geometry{margin=2in} % for example, change the margins to 2 inches all round
% \geometry{landscape} % set up the page for landscape
%   read geometry.pdf for detailed page layout information

\usepackage{graphicx} % support the \includegraphics command and options

% \usepackage[parfill]{parskip} % Activate to begin paragraphs with an empty line rather than an indent

%%% PACKAGES
\usepackage{booktabs} % for much better looking tables
\usepackage{array} % for better arrays (eg matrices) in maths
\usepackage{paralist} % very flexible & customisable lists (eg. enumerate/itemize, etc.)
\usepackage{verbatim} % adds environment for commenting out blocks of text & for better verbatim
\usepackage{subfig} % make it possible to include more than one captioned figure/table in a single float
% These packages are all incorporated in the memoir class to one degree or another...

%%% HEADERS & FOOTERS
\usepackage{fancyhdr} % This should be set AFTER setting up the page geometry
\pagestyle{empty} % options: empty , plain , fancy
\renewcommand{\headrulewidth}{0pt} % customise the layout...
\lhead{}\chead{}\rhead{}
\lfoot{}\cfoot{\thepage}\rfoot{}

%%% SECTION TITLE APPEARANCE
%\usepackage{sectsty}
%\allsectionsfont{\sffamily\mdseries\upshape} % (See the fntguide.pdf for font help)
% (This matches ConTeXt defaults)

%%% ToC (table of contents) APPEARANCE
%\usepackage[nottoc,notlof,notlot]{tocbibind} % Put the bibliography in the ToC
%\usepackage[titles,subfigure]{tocloft} % Alter the style of the Table of Contents
%\renewcommand{\cftsecfont}{\rmfamily\mdseries\upshape}
%\renewcommand{\cftsecpagefont}{\rmfamily\mdseries\upshape} % No bold!

%%% END Article customizations

%%% The "real" document content comes below...

\title{The Drasil Framework -- Poster}
	%{Captured Knowledge and Artifact Generation:\\The Drasil Framework}
\author{
  Daniel Szymczak
  \and
  Steven Palmer
  \and
  Jacques Carette
  \and
  Spencer Smith\
}

%\date{} % Activate to display a given date or no date (if empty),
         % otherwise the current date is printed 

\begin{document}
\maketitle

%%\abstract{Do we need an abstract for a one-page extended abstract?}
\section{Research Problem}

Software artifacts contain a vast amount of knowledge duplication and
transformation. A single piece of underlying knowledge should appear (in some
form) in every artifact for a given piece of software.

In certain domains,
like Scientific Computing (SC),
knowledge is duplicated across many different pieces of software. This
duplication is almost exclusively performed manually, whether by copying/pasting
from existing artifacts, or by transcription from another source such as a textbook.

Artifacts tend to fall out of sync as the software is updated 
(especially code versus documentation),
negatively impacting the overall software quality. This is most noticeable 
when it comes to maintainability, traceability, and verifiability.

\section{Approach}

We are using ideas from Literate Programming (LP) and Generative Programming
(GP) to capture knowledge, reduce manual duplication, and generate software
artifacts. We have created a framework, Drasil, to accomplish this through a
practical, example-driven approach involving frequent redesigns and refactoring.

In Drasil, knowledge is captured in \textit{chunks}, which are then used in
conjunction with our \textit{recipes} to generate software artifacts
(requirements specifications, design documents, source code, etc.) in various
formats.

Recipes are built from a number of custom Domain Specific Languages (DSLs)
embedded in Haskell. The recipes allow us to target our output to specific
audiences, creating different ``views'' of our knowledge-base with minimal
effort. These views can be large-scale (ex.\ different artifact types) or
small-scale (ex. showing/hiding verbose descriptions within an artifact).

We are currently implementing a new recipe DSL to gain a higher level of
abstraction, which will be easier to work with than the previous implementation.

%Chunks + Chunk Hierarchy
%  - Recipes 
%    - DSLs all the way down.
%    - Currently implementing new recipe DSL as intermediary
%  - Pictures: Chunk hierarchy + Abstraction of models, etc. (from meetings).
  
\section{Preliminary results}
As of this writing, there are currently six example projects being redeveloped
using Drasil. Advantages of the approach have already become clear.

Knowledge capture has lead to reduced knowledge duplication. Common knowledge
across examples is captured once and reused as necessary (ex. 
conservation of thermal energy appears in several examples).

Traceability is guaranteed as everything is generated from the knowledge-base.
The recipe is aware of all necessary chunks and can easily produce a trace to
a given knowledge source.

Drasil helps create more maintainable software as there is a single unique
source for knowledge. Any mistakes in the knowledge-base will appear everywhere
in the generated artifacts, making them easy to find and easier to fix. Only one
source must be updated to propagate a fix through all artifacts.

The one main disadvantage to a knowledge-based approach is the large up-front
investment in creating and expanding the knowledge base. However, we believe this
investment is worthwhile as it produces long-term gains.

%\bibliographystyle{abbrv}
%\bibliography{drasil}

\end{document}
