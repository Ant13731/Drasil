% !TEX TS-program = pdflatex
% !TEX encoding = UTF-8 Unicode

% This is a simple template for a LaTeX document using the "article" class.
% See "book", "report", "letter" for other types of document.

\documentclass[11pt]{article} % use larger type; default would be 10pt

\usepackage[utf8]{inputenc} % set input encoding (not needed with XeLaTeX)

%%% Examples of Article customizations
% These packages are optional, depending whether you want the features they provide.
% See the LaTeX Companion or other references for full information.

%%% PAGE DIMENSIONS
\usepackage{geometry} % to change the page dimensions
\geometry{a4paper} % or letterpaper (US) or a5paper or....
% \geometry{margin=2in} % for example, change the margins to 2 inches all round
% \geometry{landscape} % set up the page for landscape
%   read geometry.pdf for detailed page layout information

\usepackage{graphicx} % support the \includegraphics command and options

% \usepackage[parfill]{parskip} % Activate to begin paragraphs with an empty line rather than an indent

%%% PACKAGES
\usepackage{booktabs} % for much better looking tables
\usepackage{array} % for better arrays (eg matrices) in maths
\usepackage{paralist} % very flexible & customisable lists (eg. enumerate/itemize, etc.)
\usepackage{verbatim} % adds environment for commenting out blocks of text & for better verbatim
\usepackage{subfig} % make it possible to include more than one captioned figure/table in a single float
% These packages are all incorporated in the memoir class to one degree or another...

%%% HEADERS & FOOTERS
\usepackage{fancyhdr} % This should be set AFTER setting up the page geometry
\pagestyle{fancy} % options: empty , plain , fancy
\renewcommand{\headrulewidth}{0pt} % customise the layout...
\lhead{}\chead{}\rhead{}
\lfoot{}\cfoot{\thepage}\rfoot{}

%%% SECTION TITLE APPEARANCE
\usepackage{sectsty}
\allsectionsfont{\sffamily\mdseries\upshape} % (See the fntguide.pdf for font help)
% (This matches ConTeXt defaults)

%%% ToC (table of contents) APPEARANCE
\usepackage[nottoc,notlof,notlot]{tocbibind} % Put the bibliography in the ToC
\usepackage[titles,subfigure]{tocloft} % Alter the style of the Table of Contents
\renewcommand{\cftsecfont}{\rmfamily\mdseries\upshape}
\renewcommand{\cftsecpagefont}{\rmfamily\mdseries\upshape} % No bold!

%%% END Article customizations

%%% The "real" document content comes below...

\title{Literate Development of Families of Mathematical Models}
\author{Daniel Szymczak}
%\date{} % Activate to display a given date or no date (if empty),
         % otherwise the current date is printed 

\begin{document}
\maketitle

\section{Research Questions}

Scientists and engineers rely heavily on models that share common mathematical
knowledge, but many times in software these models are reimplemented for each
new project. With that in mind, the first question my research intends to
address is ``how can we improve the reuse of mathematical knowledge for
scientific and engineering models?''

Another issue that tends to come up in software development projects is that of
duplication. Across the different artifacts in a given software project
there is typically a non-trivial amount of knowledge duplication. This brings us
to the next question my research aims to address: ``how can we handle the 
massive amount of mathematical knowledge duplication across multiple artifacts
in a software development project?'' Being able to reduce (and hopefully
eliminate) the amount of duplication would also help to improve the 
maintainability of a project.

On that note, it is often the case that Non-functional requirements (NFRs) such
as traceability, maintainability, verifiability, and (re)usability are 
considered nice to have,
but not necessarily as critical as other traits like correctness, accuracy, and
performance. This is usually due to the relatively un-quantifiable nature of 
many of the NFRs (for example, how does one quantify maintainability?). That 
brings us to the final question: ``how can we improve software development to
ensure qualities such as traceability, maintainability, verifiability and
(re)usability are met?''

\section{Research Plans}

To address the aforementioned questions, I plan to create an 
infrastructure and/or tool for mathematical knowledge capture and reuse. The 
main focus of the tool will be to avoid any knowledge duplication across
artifacts while maintaining clear traceability between them. I plan to base it
off of many of the ideas and tools in the CICM community.

The tool will utilize generative programming for the creation of software
artifacts (including, but not limited to code) from the previously captured
knowledge. To do this, I will be expanding upon the ideas of literate
programming and implementing a domain-specific language for mathematical
knowledge capture (as well as the generator back-end).

The tool will also be focused on creating program families, and allowing the
user to configure which specific member should be created. In doing so, the tool
can create many different programs for a given task which will be specialized
depending on the specific problem at hand. For example, if an algorithm for
computing a determinant were to be generated, but one (or more) of the matrices
involved were known to have many zeroes, a specialized computation would be
created to avoid performing unnecessary operations. If the matrices were not
known ahead of time, a general algorithm for computing the determinant could be
produced instead.

Finally, I plan to perform a case study which will involve applying my tool to
a real problem in the realm of Scientific Computing as a proof of concept.
Specifically, it will be used for safety analysis software in nuclear power
systems.

\section{Completed and Remaining Research}

As of the time of this writing I have completed part one of the
comprehensive exams and I have completed the majority of my required
coursework. I am presently in the process of building a prototype of the tool
using an iterative approach to flesh out the design.

As the prototype is still in its early stages I
believe that now would be the perfect time to receive feedback and 
additional insights for the implementation of the tool/infrastructure.

\section{Evaluation Plans}

To evaluate the usefulness of my tool, I plan to apply it to a real-world
problem in the realm of safety analysis in nuclear power systems. The plan is to
have the tool used by experts in the field for a software development project
and then have them provide feedback regarding whether or not the tool would be
useful for them.

Another planned form of evaluation is to quantify the amount of effort
involved in making a likely change to a piece of code through traditional means
and comparing that to the amount of effort involved in making a change using my
tool.

\section{Publication Plans}

As of the time of this writing, a conference paper 
(\textit{Literate Development of Families of Mathematical Models} -- working title) 
is being written for 
the international conference on Generative Programming: Concepts \& Experience
(GPCE) and the international conference on Software Language Engineering (SLE).

I am also planning to write a paper 
(\textit{A Tool for Generating Families of Mathematical Models})
for the International Conference on Software
Engineering (ICSE).

As for Journals, I am currently planning to submit a paper 
(\textit{Evaluation of a Literate Process for Generating Mathematical Models})
to Advances in Engineering Software. Also, I plan to submit a paper 
to Science of Computer Programming. 
\end{document}
