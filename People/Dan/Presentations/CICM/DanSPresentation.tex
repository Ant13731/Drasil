% $Header: /cvsroot/latex-beamer/latex-beamer/solutions/conference-talks/conference-ornate-20min.en.tex,v 1.6 2004/10/07 20:53:08 tantau Exp $

\documentclass{beamer}

\mode<presentation>
{
%  \usetheme{Hannover}
\usetheme[width=0.7in]{Hannover}
% or ...

  \setbeamercovered{transparent}
  % or whatever (possibly just delete it)
}
\usepackage{longtable}
\usepackage{booktabs}

\usepackage[english]{babel}
% or whatever

\usepackage[latin1]{inputenc}
% or whatever

\usepackage{times}
%\usepackage[T1]{fontenc}
% Or whatever. Note that the encoding and the font should match. If T1
% does not look nice, try deleting the line with the fontenc.
%\usepackage{logictheme}

%\usepackage{hhline}
\usepackage{multirow}
%\usepackage{multicol}
%\usepackage{array}
%\usepackage{supertabular}
%\usepackage{amsmath}
%\usepackage{amsfonts}
\usepackage{totpages}
\usepackage{hyperref}
%\usepackage{booktabs}

%\usepackage{bm}

\usepackage{listings}
\newcommand{\blt}{- } %used for bullets in a list

\newcounter{datadefnum} %Datadefinition Number
\newcommand{\ddthedatadefnum}{DD\thedatadefnum}
\newcommand{\ddref}[1]{DD\ref{#1}}

\newcommand{\colAwidth}{0.2\textwidth}
\newcommand{\colBwidth}{0.73\textwidth}

\renewcommand{\arraystretch}{0.6} %so that tables with equations do not look crowded

\pgfdeclareimage[height=0.7cm]{logo}{McMasterLogo}
\pgfdeclareimage[height=6cm]{Dan}{Dan.jpg}
\pgfdeclareimage[height=6cm]{Dan2}{Dan_Suit.jpg}
\pgfdeclareimage[width=10.55cm]{ex1}{h_g_ex}
\pgfdeclareimage[width=6cm]{rec1}{SRS_ex}
\pgfdeclareimage[width=6cm]{rec2}{LPM_ex}
\title[\pgfuseimage{logo}]  % (optional, use only with long paper titles)
{Literate Development of Families of Mathematical Models}

%\subtitle
%{Include Only If Paper Has a Subtitle}

\author[Slide \thepage~of \pageref{TotPages}] % (optional, use only with lots of
                                              % authors)
{Dan Szymczak}
% - Give the names in the same order as the appear in the paper.
% - Use the \inst{?} command only if the authors have different
%   affiliation.

\institute[McMaster University] % (optional, but mostly needed)
{
  Computing and Software Department\\
  Faculty of Engineering\\
  McMaster University
}
% - Use the \inst command only if there are several affiliations.
% - Keep it simple, no one is interested in your street address.

\date[] % (optional, should be abbreviation of conference name)
{July 13, 2015}
% - Either use conference name or its abbreviation.
% - Not really informative to the audience, more for people (including
%   yourself) who are reading the slides online

\subject{computational science and engineering, software engineering, software
  quality, literate programming, software requirements specification, document
  driven design}
% This is only inserted into the PDF information catalog. Can be left
% out. 

% If you have a file called "university-logo-filename.xxx", where xxx
% is a graphic format that can be processed by latex or pdflatex,
% resp., then you can add a logo as follows:

%\pgfdeclareimage[height=0.5cm]{Mac-logo}{McMasterLogo}
%\logo{\pgfuseimage{Mac-logo}}

% Delete this, if you do not want the table of contents to pop up at
% the beginning of each subsection:
\AtBeginSubsection[]
{
  \begin{frame}<beamer>
    \frametitle{Outline}
    \tableofcontents[currentsection,currentsubsection]
  \end{frame}
}

% If you wish to uncover everything in a step-wise fashion, uncomment
% the following command: 

%\beamerdefaultoverlayspecification{<+->}

\beamertemplatenavigationsymbolsempty 

% have SRS and LP open during the presentation

\begin{document}

%%%%%%%%%%%%%%%%%%%%%%%%%%%%%%%%%%%%%%
\begin{frame}

\titlepage

\end{frame}

%%%%%%%%%%%%%%%%%%%%%%%%%%%%%%%%%%%%%%

\begin{frame}

\frametitle{Overview}
\tableofcontents
% You might wish to add the option [pausesections]

% make like a story - the phases - reason for, why works, advantages
% changing the history a bit to make a more rational narrative

\end{frame}

%%%%%%%%%%%%%%%%%%%%%%%%%%%%%%%%%%%%%%

\section[Introduction]{Introduction}

% \subsection[Important Software Qualities]{Scientific Computing Software
% Qualities}

%%%%%%%%%%%%%%%%%%%%%%%%%%%%%%%%%%%%%%

%D Remove this slide? Should I even include an introduction to me?

\begin{frame}

\frametitle{Who am I?}

Dan Szymczak

\vspace{0.5cm}
\pgfuseimage{Dan2}\hspace{1cm}\pgfuseimage{Dan}
\end{frame}

%%%%%%%%%%%%%%%%%%%%%%%%%%%%%%%%%%%%%%

\begin{frame}

\frametitle{Education History}

\begin{itemize}
\item Ph.D. Software Engineering
	\begin{itemize}
	\item Currently in progress. Started Autumn 2014.
	\end{itemize}
\item M.A.Sc. Software Engineering
	\begin{itemize}
	\item McMaster University 2014
	\item Thesis -- \textit{Generating Learning Algorithms: Hidden Markov Models as a Case Study}
	\end{itemize}
\item B.Eng Software (Game Design) 
\begin{itemize}
	\item McMaster University 2011
	\end{itemize}

\end{itemize}
\end{frame}

%%%%%%%%%%%%%%%%%%%%%%%%%%%%%%%%%%%%%%

%D Remove this slide?

\begin{frame}
\frametitle{Current Program Progress}
\begin{itemize}
\item Completed 3/4 necessary graduate courses
%D Remove the course details?
%	\begin{itemize}
%	\item CAS703 -- Software Design: A (11)
%	\item CAS708 -- Scientific Computation: A+ (12)
%	\item CAS761 -- Generative Programming: A+ (12)
%	\end{itemize}
\item Completed part one of comprehensive exam
\item Research and prototype system development are underway
\end{itemize}
\end{frame}

%%%%%%%%%%%%%%%%%%%%%%%%%%%%%%%%%%%%%%

\section[Research]{Research Plan}

%%%%%%%%%%%%%%%%%%%%%%%%%%%%%%%%%%%%%%

%D Move this to a section on publication plans/future stuff?

%\begin{frame}
%
%\frametitle{Current Program Progress}
%\framesubtitle{Cont'd}
%
%\begin{itemize}
%\item Attending conferences over the coming months
%	\begin{itemize}
%	\item CICM 2015 Doctoral Programme
%	\item ICFP 2015 Programming Languages Mentoring Workshop
%	\end{itemize}
%\item Preparing to submit for SPLASH and SEHPCCSE
%\end{itemize}
%\end{frame}

%%%%%%%%%%%%%%%%%%%%%%%%%%%%%%%%%%%%%%

\begin{frame}

\frametitle{Research}
\framesubtitle{Key Problem(s)}


%D Slide looks too cluttered with all this information, going to remove the 
% enumerated items and only leave the questions. Will say the numbered parts
% aloud.

%\begin{enumerate}
%\item Many scientific and engineering models %in the abstract sense
%	tend to reuse the same underlying mathematical knowledge, yet are
%	reimplemented for new software projects.
How can we
	\begin{itemize}
	\item improve the reuse of mathematical, scientific, and engineering knowledge?
%	\end{itemize}

%\item Across software artifacts there is typically a non-trivial amount of 
%	knowledge duplication.

%	\begin{itemize}
	\item handle knowledge duplication across software artifacts?
%	\end{itemize}

%\item It is often the case (cite: Kelly) that non-functional requirements (NFRs)
%	are
%	considered nice to have, but not as critical as correctness, accuracy, and
%	performance.
	
%	\begin{itemize}
	\item improve the qualities of
					traceability, maintainability, verifiability and (re)usability?
	\end{itemize}

%\end{enumerate}

%D Yup, way too much text. Commented out as per above note.

\end{frame}

%%%%%%%%%%%%%%%%%%%%%%%%%%%%%%%%%%%%%%
%D Insert slides addressing the problems: 
%		How can we improve reuse? - Open question, lots of possible answers, but
%			which is the ``best'' one? No idea. It's time to take a practical approach.
%			Trying not to ``overdesign'' and underimplement.
%   Why is duplication not solved yet? - Knowledge is being shared across 
%			languages/artifacts/views.
%		Are these specific to math software? - No. However, math \& scientific
%			software domains tend to be well-understood, making it much easier to
%			address the problems in this domain first.

%D Where should I bring in some lit review? I'd like to mention MMT at some point.
%		Formal content of the knowledge is not (yet) the focus. Reusing the knowledge
%		via recipes is the true focus.

%D Intentional Programming to come up somewhere?


\begin{frame}

\frametitle{Research}
\framesubtitle{Musings}

How can we improve the reuse of mathematical, scientific, and engineering knowledge? 
%D -Open question, lots of possible answers, but
%			which is the ``best'' one? No idea. It's time to take a practical approach.
%			Trying not to ``overdesign'' and underimplement. With that in mind...
%\begin{enumerate} enum or itemize?
\begin{itemize}
\item Simplify the knowledge store
%D Not sure if I should include these as bullets or just talking points
\item Create a means of obtaining knowledge relevant to a project
%	\begin{itemize}
%	\item Use a single source for the creation of artifacts
%	\item Make it language/view independent
%	\item Make it sexy
%	\end{itemize}
\item Make it accessible %So anyone can use it regardless of what they're 
														%developing.
\end{itemize}
%\end{enumerate}

\end{frame}

%%%%%%%%%%%%%%%%%%%%%%%%%%%%%%%%%%%%%%

\begin{frame}

\frametitle{Research}
\framesubtitle{Musings Cont'd}

Why is the duplication problem not solved yet? 
%D Especially since many languages have useful abstraction features and tools 
%		for reducing the amount of code duplication? Short answer: These features 
%		only go so far.

\begin{itemize}
\item Existing tools and abstraction features only go so far
%D They might be very useful for source code, and even sometimes extend to some
%   of the documentation (through tools like doxygen, literate programming
%   practices, etc.). These tools and practices don't help as much when dealing
%   with multiple languages or types of artifacts.
\item Knowledge is shared across languages/artifacts/views
%D Example?

\item No standard method for encoding knowledge or reusing it
%D How do you write something in a way that is both human and computer readable
%	  in a simple enough manner that clients can read it? They may be experts in 
%	  the domain of interest, but not necessarily in the computer-readable 
%   language.
\end{itemize}
%\end{enumerate}

\end{frame}

%%%%%%%%%%%%%%%%%%%%%%%%%%%%%%%%%%%%%%
\begin{frame}

\frametitle{Research}
\framesubtitle{Musings Cont'd}

Are these problems specific to math software?
%D No. However, we're focusing on math \& scientific
%			software domains as they tend to be well-understood, making it much easier
%    to address the problems in this domain first. Specifically, in this domain,
%			We can take advantage of the power of mathematical abstraction to simplify
%			things.
%%%D (Reword the above)
% Also, we are working with a power generation company that wants to capture the
%			knowledge they use as their current documentation will potentially change.
%			This is one of the biggest motivators for a practical approach as well,
%			since we have concrete examples to work from and strive towards.

\Huge{No}

\end{frame}

%%%%%%%%%%%%%%%%%%%%%%%%%%%%%%%%%%%%%%


\begin{frame}

\frametitle{Research}
\framesubtitle{Solution Plan}

%D Create an infrastructure / tool for knowledge capture and reuse.

\begin{itemize}
\item Focus: avoid knowledge duplication across artifacts through reuse%D copy/pasting
\item Maintain: clear traceability between artifacts
\item Utilize: generative programming to create artifacts from captured
	knowledge
\item Expand: ideas from literate programming

\end{itemize}
\end{frame}

%D This slide and the next are more of an ``overview'' of the plan of attack.
%		 More in-depth info is to follow.

%%%%%%%%%%%%%%%%%%%%%%%%%%%%%%%%%%%%%%

\begin{frame}

\frametitle{Research}
\framesubtitle{Solution Plan Cont'd}

%D Create an infrastructure / tool for knowledge capture and reuse.

\begin{itemize}
\item Create: a domain-specific language for both knowledge capture \& the 
	artifact	generator
\item Implement: a feature for creating program families %Allows us to
				%easily design for change.
%D Example time -> determinant calculation involving a known matrix with 
%		many zeroes vs. unknown input matrices.
\item Test: apply the tool to real world problems in scientific computing

\end{itemize}
\end{frame}

%%%%%%%%%%%%%%%%%%%%%%%%%%%%%%%%%%%%%%

\section[Prototype]{The Prototype System}

%%%%%%%%%%%%%%%%%%%%%%%%%%%%%%%%%%%%%%

\begin{frame}

\frametitle{Prototype}
\framesubtitle{Design and Development}
%D Practical approach -> Don't overdesign and underdevelop
%   Build off examples. Flesh out the details as we go.

\begin{itemize}
\item Taking a practical approach
%D Many reasons for this. Avoids overdesigning the solution since this is a
%   fairly ambitious project which could stay in design for a decade without
%   making much progress. Also, since we're working with industry examples we
%   have a solid starting point and clear goals for the output.
\item Focusing on knowledge reuse, as opposed to the formal nature of the
				knowledge itself
\end{itemize}

\end{frame}

%%%%%%%%%%%%%%%%%%%%%%%%%%%%%%%%%%%%%%
%D Maybe bring up the stuff below once getting to the prototype goals?
%\begin{frame}
%\begin{itemize}
%\item One ``source,'' multiple views
%\begin{itemize}
%\item Requirements%, including or excluding derivations.
%\item Design
%\item Test Cases
%\item Build instructions
%\item ...
%\end{itemize}
%\item Motivation
%\begin{itemize}
%\item Improve verifiability, maintainability and reusability.
%\item Save money and time% when managing change.
%\end{itemize}
%\end{itemize}
%\end{frame}

%%%%%%%%%%%%%%%%%%%%%%%%%%%%%%%%%%%%%%

\begin{frame}

\frametitle{Prototype}
\framesubtitle{How?}

A practical approach

\begin{itemize}
\item Use existing artifacts as knowledge sources 
%D We don't need to reinvent the knowledge held within these sources, just
%		extract it and remove any irrelevant information (mostly English language
%		segues and the like).
\item Motivated by concrete examples 
%D We have something to work towards that we can measure against. A good
%		indicator of progress is ``how closely does the tool output match the
%		original source?'' In some cases, it won't be 100% (and that's a good thing!)
%		as the tool can find errors that may have gone unnoticed until now.
\item Avoid overdesigning and underdeveloping
\end{itemize}
\end{frame}

%D I've mentioned examples a lot to this point, so how about we actually take a
%		look at a real-world example (this is from the power generation company I
%		mentioned earlier.

%%%%%%%%%%%%%%%%%%%%%%%%%%%%%%%%%%%%%%

\section[Example]{Example}

%%%%%%%%%%%%%%%%%%%%%%%%%%%%%%%%%%%%%%

\begin{frame}

\frametitle{Example: $h_g$}

\framesubtitle{A simple example taken from the SRS for FP}
%D This is an example taken from a nuclear reactor control system. Specifically
%   the SRS for a fuel pin.
%		h_g is gap conductance

%D To start off,
$h_g$ is a symbol which appears in several locations including:
\begin{itemize}
\item The Software Requirements Specification (SRS)
%D Majority should be familiar with SRS
\item The Literate Programmer's Manual (LPM)
%D The LPM is a lit. prog. doc. where implemented code segments are placed
%		alongside the description of the code (more on that in 2 slides).
\item The Source Code
\end{itemize}

Let's take a look!

\end{frame}

%%%%%%%%%%%%%%%%%%%%%%%%%%%%%%%%%%%%%%

\begin{frame}
%D - Give a brief overview of this slide.
\frametitle{Example: $h_g$}

\framesubtitle{SRS Definition for $h_g$ (original)}

\noindent
\begin{minipage}{\textwidth}
\begin{tabular}{p{\colAwidth} p{\colBwidth}}
\toprule
\textbf{Number} & \textbf{DD\refstepcounter{datadefnum}\thedatadefnum} \label{hg}\\
\midrule
Label & $h_g$\\
\midrule
Units & $ML^0t^{-3}T^{-1}$\\
\midrule
SI & $\mathrm{\frac{kW}{m^{2} (^{\circ}C)}}$\\
\midrule
Equation & $h_g$ =$ \frac{2k_{c}h_{p}}{2k_{c}+\tau_c h_{p}}$\\
\midrule
Description & $h_g$ is the  gap conductance\newline
$\tau_c$ is the clad thickness\newline
$h_p$ is initial gap film conductance\newline
$k_c$ is the clad conductivity\newline
NOTE: Equation taken from the code\\
\midrule
Sources & source code\\
\bottomrule
\end{tabular}
\end{minipage}\\

\end{frame}

%%%%%%%%%%%%%%%%%%%%%%%%%%%%%%%%%%%%%%

\begin{frame}[fragile]

\frametitle{Example: $h_g$}

\framesubtitle{LPM Definition for $h_g$ (original)}
%D The point of this document (the LPM) may not be that apparent, however,
%		its ppurpose is for verification.
%		Once we trust the generator, documents like this one may become obsolete.
%		Currently this type of document may be a necessity for regulation and/or
%		certification.
\begin{equation} 
h_{g} =\frac{2k_{c}h_{p}}{2k_{c}+\tau_c h_{p}}
\end{equation}

The corresponding C code is given by:

\begin{lstlisting}[basicstyle=\tiny]
double calc_hg(double k_c,double h_b,double tau_c)
{
 return (2*(k_c)*(h_p)) / ((2*(k_c)) + (tau_c*(h_p)));
}
\end{lstlisting} %D Make sure to talk about the source code.

\end{frame}

%%%%%%%%%%%%%%%%%%%%%%%%%%%%%%%%%%%%%%

%\begin{frame}
%
%\frametitle{Example: $h_g$ and $h_c$}
%
%\framesubtitle{Code source for $h_g$}
%
%\end{frame}

%%%%%%%%%%%%%%%%%%%%%%%%%%%%%%%%%%%%%%

\begin{frame}

\frametitle{Example: $h_g$}

\framesubtitle{A simple example taken from the SRS for FP}

Modifying $h_g$ to reflect changes in requirements is not %a 
simple.
%matter
It involves
%,at the very least, 
the following steps:
\begin{itemize}
\item Update the definition in the SRS, LPM, etc.
% and all other documents which reference the symbol
\item Modify the source code %to reflect the new requirements
\item Trace all dependencies
\item Modify dependents %to accomodate the change
\item Ensure each artifact is now up to date and consistent
\end{itemize}

\end{frame}

%%%%%%%%%%%%%%%%%%%%%%%%%%%%%%%%%%%%%%

\begin{frame}

\frametitle{Example: $h_g$}

\framesubtitle{Simplifying the process}

%D This is where knowledge capture comes into play. 
Here is an example of a ``chunk'' for $h_g$:
%D Pay special attention to the Dependency as that is used to fill in the entire
%		description field. ``get_dep'' determines this chunk's dependencies based
%		solely on its equation.

%D Note that ALL of the information we need to generate both the SRS definition,
%		the LPM, and the source code is contained right here in this little chunk.

\pgfuseimage{ex1}

\end{frame}

%%%%%%%%%%%%%%%%%%%%%%%%%%%%%%%%%%%%%%

\begin{frame}

\frametitle{Example: $h_g$}

\framesubtitle{How do we generate?}
What do we do with the ``chunk''?

That depends on the ``recipe''!

\begin{itemize}
\item To create our SRS we use the following recipe:
\end{itemize}

\pgfuseimage{rec1}

\begin{itemize}
\item To create our LPM we use the following recipe:
\end{itemize}

\pgfuseimage{rec2}

\end{frame}

%%%%%%%%%%%%%%%%%%%%%%%%%%%%%%%%%%%%%%

\begin{frame}

\frametitle{Example: $h_g$}

\framesubtitle{Generated SRS Output}

%Let's look at (excerpts of the) generated output:

\noindent \begin{minipage}{\textwidth}
\begin{tabular}{p{\colAwidth} p{\colBwidth}}
\toprule \textbf{Number} & \textbf{DD\refstepcounter{datadefnum}\thedatadefnum}
\label{hg}\\ \midrule
Label & $h_{g}$\\ \midrule
Units & $ML^0t^{-3}T^{-1}$\\ \midrule
SI & $\mathrm{\frac{kW}{m^{2\circ} C}}$\\
\midrule Equation & $h_{g}$ = $\frac{2k_{c}h_{p}}{2k_{c}+\tau_{c}h_{p}}$\\ \midrule
Description & $h_{g}$ is the effective heat transfer coefficient between clad and fuel surface
\newline
$k_{c}$ is the
clad conductivity \newline
$h_{p}$ is the
initial gap film conductance \newline
$\tau_{c}$ is the
clad thickness \newline
NOTE: Equation taken from the code\\ \midrule
Sources & source code\\
\bottomrule \end{tabular} \end{minipage}\\
\end{frame}

%D Note the description of h_g, if we go back to the chunk definition, you can
%		see that there is a description in place, but it only describes hg itself.
%		The generator determined the dependencies in the equation and described them
%		automatically as per the recipe's instructions.
%		There are also a few minor differences between the two descriptions as the
%		original document used two different sets of terminology to refer to the same
%		symbol. To improve understanding, usability, etc. that oversight has been
%   corrected (and automatically at that).


%%%%%%%%%%%%%%%%%%%%%%%%%%%%%%%%%%%%%%

\begin{frame}[fragile]

\frametitle{Example: $h_g$}

\framesubtitle{Generated LPM Output}

%Let's look at (excerpts of the) generated output:

\begin{equation}
h_{g} =\frac{2k_{c}h_{p}}{2k_{c}+\tau_{c}h_{p}}\label{eq:hg}
\end{equation}

The corresponding C code is given by:

\begin{lstlisting}[basicstyle=\tiny]
double calc_h_g(double k_c, double h_p, double tau_c)
{
	return 2*k_c*h_p/(2*k_c+tau_c*h_p);
}
\end{lstlisting}
\end{frame}

%D Here you'll notice that the equation is exactly the same as the previous one.
%		Also, from this equation alone, the source code has been generated. There are
% 	a few minor differences in the generated code, but the biggest (imo) is the
%		generated code is actually MORE readable (imo) as it removed extraneous use
%		of parentheses. The other differences are cosmetic, as the code still
%		performs the exact same function.

%D All of that documentation was created from so little information! The most
%		important piece of information in the chunk is the equation as it directly
%		affects so many artifacts. Of course the other fields are important too, but
%		they simply aren't used in the same way as the equation.

%%%%%%%%%%%%%%%%%%%%%%%%%%%%%%%%%%%%%%

\section[Next Steps]{Next Steps}

%%%%%%%%%%%%%%%%%%%%%%%%%%%%%%%%%%%%%%

\begin{frame}

\frametitle{Next Steps}
\framesubtitle{The next 12 months}


\begin{Large}
What next?
\end{Large}

\begin{itemize}
\item Comprehensive examination part two
\item Complete final graduate level course
\begin{itemize} %D Specifically I'm ... since I don't know category theory and
									%   it comes up a lot in code generation literature.
\item Looking for a category theory course, but open to suggestions
\end{itemize}
\item Complete paper for SPLASH conference %vision paper
\item Complete SEHPCCSE conference paper %showing prelim results to SC community
\item Complete default ``recipe'' for each software artifact %reqs, design,
%	code, test cases, makefiles, etc. that should be gen'd
\item Have at least one large example working from the prototype
% output mult. artifacts, gen code, ensure generated items are consistent
\item Create the external language for using the prototype %cleaner syntax,
%avoid requiring haskell experts to ensure accessibility
\item Communicate with industry regarding prototype and example(s) %Find out
%what changes, if any, would make them want to adopt the system. Also
%investigate using the prototype tool for other document types
%(i.e. engineering calculation reports).
\end{itemize}
\end{frame}

%%%%%%%%%%%%%%%%%%%%%%%%%%%%%%%%%%%%%%

%%%%%%%%%%%%%%%%%%%%%%%%%%%%%%%%%%%%%%

\end{document}
