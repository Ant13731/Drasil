%\documentclass[review]{elsarticle}
\documentclass{elsarticle}

\usepackage{lineno}
\usepackage{hyperref}
\usepackage{amsmath}   % From the American Mathematical Society
\usepackage{amssymb}
\usepackage{amsfonts}
\usepackage{paralist}
\usepackage{booktabs}
\usepackage{xspace}
\usepackage{enumitem}

\newcommand{\quotes}[1]{\lq\lq #1\rq\rq}
\newcommand{\CWEB}{{\sc cweb}\xspace}
\newcommand{\ft}{\textit{fuel\_temp\_}\xspace}

\newcommand{\colAwidth}{0.2\columnwidth}
\newcommand{\colBwidth}{0.755\columnwidth}

\renewcommand{\arraystretch}{1.1} %so that tables with equations do not look crowded

\modulolinenumbers[5]
\journal{Working Paper}

\begin{document}

\begin{frontmatter}

\title{A Recipe for Scientific Computing Software Development}

%% or include affiliations in footnotes:
\author[mymainaddress]{W.\ Spencer Smith\corref{mycorrespondingauthor}}
\cortext[mycorrespondingauthor]{Corresponding author}
\ead[url]{http://www.cas.mcmaster.ca/~smiths/}
\ead{smiths@mcmaster.ca}

\author[mymainaddress]{?}

\author[mymainaddress]{?}

\address[mymainaddress]{Computing and Software Department, McMaster
University, Hamilton, Ontario L8S 4L7, Canada}

\begin{abstract}
abstract
\end{abstract}

\begin{keyword}
  Software Certification, Literate Programming, Document Driven Design,
  Scientific Computing
\end{keyword}

\end{frontmatter}

%\linenumbers

\section {Introduction} \label{Sec_Introduction}

- technology and processes exist, but still not used - need something more
prescriptive

- Recipe for dev SC software - includes technology, process, templates etc.

- goals:
\begin{itemize}
\item Correct
\item Verifiable
\item ...
\item Reproducible
\item Simple technology
\end{itemize}

Figure~\ref{Fig_Process} shows a high level view of the documentation to be produced.

\begin{figure}
\fbox{
\begin{minipage}{0.96\textwidth}
\begin{enumerate}
\item Problem statement.
\item Software requirements specification.
\item Verification and validation plan.
\item Module guide.
\item Module interface specification.
\item Literate code.
\item Verification report
\item Validation report.
\end{enumerate}
\end{minipage}
}
\caption{Top-down view of software documentation following a
  rational process}
\label{Fig_Process}
\end{figure} 

Bottom up pieces are shown in Figure~\ref{Fig_BottomUpPieces}.

\begin{figure}
\scalebox{0.8}{
\fbox{
\begin{minipage}{1.19\textwidth}
\begin{enumerate}

\item{Problem Statement}

\item{SRS:}
\begin{inparaenum}[a\upshape)]
\item{Symbols}
\item{Goals}
\item{Theoretical Models}
\item{Instanced Models}
\item{Data Definitions}
\item{General Data Definitions}
\item{Assumptions}
\item{Requirements}
\item{Likely Changes}
\end{inparaenum}

\item{Verification and Validation plan:} 
\begin{inparaenum}[a\upshape)]
\item{V\&V techniques}
\item{System level verification test cases}
\item{System level validation test cases}
\item{Traceability to SRS}
\end{inparaenum}

\item{MG:}
\begin{inparaenum}[a\upshape)]
\item{Modules (name, secret, description)}
\item{Traceability to SRS}
\end{inparaenum}

\item{MIS}:
\begin{inparaenum}[a\upshape)]
\item{Interface}
\item{Description (formal or informal)}
\item{Trace to MG}
\end{inparaenum}

\item Literate Code:
\begin{inparaenum}[a\upshape)]
\item Code
\item Traceability to numerical algorithm
\item Traceability to requirements
\item Traceability to MG
\end{inparaenum}

\item {Verification Report}
\begin{inparaenum}[a\upshape)]
\item{Interface}
\item{Unit tests}
\item{System tests}
\end{inparaenum}

\item {Validation Report}
\begin{inparaenum}[a\upshape)]
\item{Interface}
\item{Comparison to experiments}
\end{inparaenum}

\end{enumerate}
\end{minipage}
}
}
\caption{Constituent parts of documents}
\label{Fig_BottomUpPieces}
\end{figure}

Introduce Figure~\ref{Fig_TofC_SRS}, which shows template for the  SRS.  Adding
a section for likely changes.

\begin{figure}

\scalebox{0.8}{
\fbox{
\begin{minipage}{1.19\textwidth}
\begin{enumerate}

\item{Reference Material:}
\begin{inparaenum}[a\upshape)]
\item{Table of Symbols}
\item{Abbreviations and Acronyms}
\end{inparaenum}

\item{Introduction:} 
\begin{inparaenum}[a\upshape)]
\item{Purpose of the Document}
\item{Scope of the Software Product}
\item{Organization of the Document}
\item{Intended Audience}
\end{inparaenum}

\item{General System Description:}
\begin{inparaenum}[a\upshape)]
\item{System Context}
\item{User Characteristics}
\item{System Constraints}
\end{inparaenum}

\item Specific System Description:
\begin{enumerate}
\item Problem Description: 
\begin {inparaenum}[i\upshape)] 
\item Background Overview,
\item Terminology Definition,
\item Physical System Description,
\item Goal Statements
\end {inparaenum}
\item Solution specification:
\begin {inparaenum}[i\upshape)]
\item Assumptions,
\item Theoretical Models, 
\item General Definitions,
\item Data Definitions,
\item Instanced Models, 
\item Data Constraints, 
\item System Behaviour
\end {inparaenum}
\item Non-functional Requirements:
\begin{inparaenum}[i\upshape)]
\item{Accuracy of Input Data}
\item{Sensitivity of Model}
\item{Tolerance of Solution}
\item{Solution Validation Strategies}
\item{Look and Feel Requirements}
\item{Usability Requirements}
\item{Performance Requirements}
\item{Maintainability Requirements}
\item{Portability Requirements}
\item{Security Requirements}
\end{inparaenum}
\end{enumerate}

\item Other System Issues:
\begin{inparaenum}[a\upshape)]
\item Likely Changes
\item Open Issues
\item Off the Shelf Solutions
\item Waiting Room
\end{inparaenum}

\item{Traceability Matrix}

\item List of Possible Changes in the Requirements

\item{Values of Auxiliary Constants}

\end{enumerate}
\end{minipage}
}
}
\caption{Table of Contents of the SRS for FP}
\label{Fig_TofC_SRS}
\end{figure}



\section{Concluding Remarks} \label{Sec_ConcRemarks}


\section*{Acknowledgements}

acknowledgements

\bibliographystyle{plainnat}

\bibliography{RecipeForSCSoftDev}

\end {document}
