\documentclass[review]{elsarticle}

\usepackage{lineno,hyperref}
\modulolinenumbers[5]

\journal{Science of Computer Programming}

\bibliographystyle{elsarticle-num}

\begin{document}

\begin{frontmatter}

\title[Generate Everything]{A Generic OO Language as a backend for a
  ``Generate Everything'' System}

\author[mymainaddress]{Jacques Carette\corref{mycorrespondingauthor}}
\cortext[mycorrespondingauthor]{Corresponding author}

\author[mysecondaryaddress]{Brooks MacLachlan}

\author[mymainaddress]{Spencer Smith}

\address[mymainaddress]{Computing and Software, McMaster University}
\address[mysecondaryaddress]{Brooke's address}

\begin{abstract}
Abstract here
\end{abstract}

\begin{keyword}
code generation, document generation, knowledge capture,
  software engineering, research software
\end{keyword}

\end{frontmatter}

\linenumbers

\section{Introduction}

- extol the virtues of a “generate all things” approach
- introduce Drasil
- requirements for Drasil to be effective
	- multiple back-ends, since research software developers want code in
        their chosen language, has to fit into an ecosystem of existing code
	- a design language that will allow for variabilities between the
        generated code and documentation
	- generative back-end needs to be extensible to new languages with
        minimal effort
- the above requirements motivate the design and implementation of GOOL

\section{Background}

- more on Drasil

\section{Design of GOOL}

- borrow from Brooks’ MASc thesis document

\section{Implementation of GOOL}

- borrow from Brooks’ MASc thesis document
- maybe this would be where we would mention the OO Patterns in GOOL (Chapter 6
of Brooks’ thesis)?

\section{Design Language}

- some motivating examples

\section{Literature Review}

\section{Future Work}

\section{Conclusion}

\section*{References}

\bibliography{References}
\end{document}
