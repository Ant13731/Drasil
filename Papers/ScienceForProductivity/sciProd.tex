\documentclass[sigconf, authorversion, nonacm]{acmart}

\settopmatter{printacmref=false}

% \makeatletter
% \def\@copyrightspace{\relax}
% \makeatother

% \setcopyright{none} 

\begin{document}

\title{Long-Term Productivity Based on Science, not Preference}

\author{Spencer Smith and Jacques Carette}
\orcid{0000-0002-0760-0987}
\affiliation{
  \department{Computing and Software, McMaster University, Canada\\ smiths@mcmaster.ca and carette@mcmaster.ca}
  %\streetaddress{1280 Main Street West}
  %\institution{McMaster University}
  %\city{Hamilton}
  %\state{Ontario}
  %\postcode{L8S 4L8}
  %\country{Canada}
  }
%\email{smiths@mcmaster.ca}

% \author{Jacques Carette}
% \orcid{0000-0001-8993-9804}
% \affiliation{
%   \department{Computing and Software}
%   %\streetaddress{1280 Main Street West}
%   \institution{McMaster University}
%   %\city{Hamilton}
%   %\state{Ontario}
%   %\postcode{L8S 4L8}
%   \country{Canada}}
% \email{carette@mcmaster.ca}

\maketitle

Our goal is to identify inhibitors and catalysts for productive long-term
scientific software development.  The inhibitors and catalysts could take the
form of processes, tools, techniques, and software artifacts (such as user
manuals, unit tests, design documents and code). The effort (time) invested in
catalysts will pay off in the long-term, while inhibitors will take up
resources, but may actually lower product quality.

If developers are surveyed on inhibitors and catalysts, their answers will be as
varied as the education and experiential backgrounds of the respondents. Their
responses will be well-meaning, but they will undoubtedly come with problems and
biases.  For instance, developers may be guilty of the \emph{sunk cost fallacy},
promoting a technology that they have invested considerable hours in learning,
even if the current costs outweigh the benefits. As another example, developers
may recommend against spending time on proper requirements, but this lack of
support doesn't imply requirements are an inhibitor, only that current practice
doesn't promote requirements~\cite{HeatonAndCarver2015}. Another
perceived inhibitor is time spent in meetings. For instance, departmental
retreats can be unpopular because of a lack of short-term benefits, but
relationship building and strategic decision making may provide significant
future rewards. The difficult trick is to know which meetings are useful, and
which are not.  As these examples illustrate, \emph{we need to take developer
and personal preference out of the development process and instead pick the
artifacts/processes/tools that have a long-term impact.}

\section{Building Blocks}

A scientific approach requires a solid foundation.  The building blocks for
scientific discourse is an unambiguous language for communicating concepts,
formulating hypotheses, planning data collection, and analyzing models and
theories.  To start with, we need to classify the software under discussion.
Likely dimensions for classification include the following: general purpose
scientific tools versus special purpose physical models, scientific domain, open
source versus commercial software, project maturity, project size, and level of
safety criticality.

Another important building block is defining what we hope to achieve in
terms of scientific software quality. Qualities that need to be unambiguously
defined include reliability, sustainability, reproducibility and productivity.
Software engineers have frequently attempted to define quality since the
1970s~\cite{McCallEtAl1977}, but the definitions aren't usually specific to
scientific software (as shown by the confusion between precision and accuracy is
the ISO/IEC definitions~\cite{ISO9126}). Moreover, the definitions often focus
on measurability, where the first priority should be conceptual clarity,
analogous to the unmeasurable, but conceptually clear, definition of forward
error, which requires knowing the (usually unknown) true answer.

For each relevant quality we recommend collecting as many distinct instances of
the definition as possible.  Once all the definitions are collected, they can be
assessed against the following criteria (based on~\citet{IEEE1998}):
completeness, consistency, modifiability, traceability, unambiguity and
abstractness. The understanding gained from this systematic survey and analysis
can then be used to propose a new set of quality definitions that allow for
reasoning about quality.

\section{Productivity}

Our definition of long-term productivity~\cite{SmithAndCarette2020arXiv}
provides an example of how we envision quality definitions in the future.  The
definition meets the criteria listed in the previous section.  Moreover, it
provides a starting point for reasoning about improving productivity, as
explored in subsequent sections.  We define productivity as:

$$P = O / I$$ 
$$ I = \int_{0}^{T} H(t)\ dt $$
$$ O = \int_{0}^{T} \sum_{c \in C} S_c(t) K_c(t)\ dt $$

\noindent where $P$ is productivity, $I$ is the inputs, $O$ is the outputs, $0$
is the time the project started, $T$ is the time \emph{in the future} where we
want to take stock, $H$ is the total number of hours available by all personnel,
$C$ represents different classes of users (external as well as internal), $S$ is
satisfaction and $K$ is \emph{practical knowledge}.  Thus productivity is
measured in ``satisfying reusable knowledge per hour.''  The equation can be
applied to measure the productivity of a single developer or a group of
developers. Alternatively, the equation can also be used to assess a given
intervention, where an intervention is a process or an output intended to
improve productivity.  An intervention process can be assessed to determine if
it reduces $I$.  For outputs, we can back calculate the inputs required to
achieve the touted benefits.

\section{Measuring}

Proper science requires measurement.  We can only determine whether a given
intervention is a catalyst or inhibitor by measuring its impact.  We can
illustrate this via the above productivity equation.  Although it cannot be
directly measured, examining the parts provides insight.

The equation shows an integral over time to emphasize the time-frame.  Some
developers may show short-term productivity, say by producing a poor design that
repeats the same, or very similar, code in multiple places. However, long-term
productivity should favour the developer that refactors the design by replacing
the repeated code with a function. A productivity measure for knowledge workers
cannot just focus on quantity, quality is at least as
important~\cite{Drucker1999}.  The proposed definition captures this by
combining knowledge with satisfaction.

The input $H$ is the number of hours worked.  Depending on the context, this
will be the total hours worked by a developer, or the total hours required
(possibly by multiple developers) to realize a given intervention (process or
output).

The output depends on user satisfaction ($S$), where user satisfaction acts as a
proxy for effective quality. How best to assess satisfaction should be studied
in the future.  As a starting point, satisfaction can potentially be
approximated with such measures as numbers of users, number of citations, number
of forks of a repository, the number of ``stars'', surveys of existing users,
number of mentions in the issue tracker, and experiments to measure usability.

Productivity means increasing the effective knowledge ($K$) delivered to users
over time, at a lesser cost. As the knowledge produced will be used by different
kinds of users (such as internal developers and external users), it is important
to weigh the satisfaction of each class separately. This explicit emphasis on
all knowledge produced, rather than just the operationalizable knowledge (aka
code) implies that human-reusable knowledge, i.e. documentation, is crucial.
This reinforces the importance of the long-lived context.  The best measure for
knowledge is an area for future exploration.

% \section{State of the Practice}

% Maybe remove this section?

% Understanding the state of the practice, including finding a methodology
% to assess the state of practice (SOP work).

% Understanding the gap between what is recommended and what is practiced (Olu’s
% work) 

% Research questions.

\section{Artifacts}

The software development process outputs different forms of knowledge. This
knowledge is typically distributed across multiple software artifacts.
Potential artifacts include requirements, specifications, user manual, unit
tests, system tests, usability tests, build scripts, API (Application
Programming Interface) documentation, READMEs, license documents, process
documents, and code.

%Our goal is to determine which artifacts are long-term productivity catalysts.
Although documentation is often neglected for scientific software, we anticipate
that explicit evidence, and a long-term viewpoint, will show which documents are
catalysts for different classes of users.  As an example, for long-lived project
with significant contributor turnover, the users tasked with training new
developers will consider onboarding documents as catalysts.

As we gain understanding on measures of productivity, those measures can be used
to determine the state of the practice for different scientific software
domains. We can estimate $K$ and $S$ for existing artifacts. (Unfortunately, it
is unlikely that records will be available for the corresponding $H$ values.) We
can then potentially compare real projects to the $K$ and $S$ values for the
artifacts recommended by software engineering textbooks.  The combined
information can then be analyzed to determine how $K$ is distributed.  We know
that knowledge will be duplicated.  The data will allow us to understand
usability reasons (related to $S$) for the different views of the same
knowledge.  Putting this all together, we hope researchers will find the most
valuable $K$ and when and how it should appear in future artifacts for different
classes of users.

% The goal is to invest effort in artifacts when the investment can be justified by the benefits.  
Besides looking at $K$ and $S$, another way to judge the utility of
documentation is to look at the documentation necessary to make an assurance
case. An assurance case~\cite{RinehartEtAl2015} presents an organized and
explicit argument for correctness (or whatever other software quality is deemed
important) through a series of sub-arguments and evidence.  Documentation will
be necessary, but through assurance cases the developers will only create the
documentation that is relevant and necessary. %maybe remove this paragraph?

\section{Production Methods}

The previous section focused on increasing $O$, but another way to improve
productivity is to reduce $I$.  The production methods, or process, used to
build software has a significant impact on $I$.  One of the likely reasons that
developers focus on code, and code related artifacts (like testing files and
build scripts), is that documentation is difficult to keep in sync as the
project evolves over time.  Out of sync documentation is arguably worse than no
documentation; therefore, a case could be made that developers are ``improving''
productivity by not producing documentation.  Of course, a better approach to
improve productivity is to capture valuable knowledge in relevant documentation
that is continually in sync with the code.

In the future, we recommend developing an approach that synchronizes code and
documentation.  A promising approach is to generate all artifacts from a
knowledge base~\cite{SzymczakEtAl2016}.  Using the understanding of artifacts
gained from the work previously proposed, a generator can be taught to create
artifacts with satisfactory knowledge. Since the approach is generative,
repetition of knowledge is fine.  This means that documents can be tailored to
different classes of users, thus further improving productivity.  A generative
approach can potentially produce high $S$ and high $K$, while reducing $H$.

\section{Concluding Remarks}

Our position is that decisions on processes, tools, techniques and software
artifacts should be driven by science, not by personal preference.  Decisions
should not be based on anecdotal evidence, gut instinct or the path of least
resistance.  Moreover, decisions should vary depending on the users and the
context.  In many cases this will mean that a longer term view should be
adopted.  We need to use a scientific approach based on unambiguous definitions,
empirical evidence, hypothesis testing and rigorous processes.

By developing an understanding of how input hours, satisfaction and knowledge
are related to productivity, in the future we will be able to determine what
interventions have the greatest return on investment.  We will be able to
recommend software production methods and artifacts that justify their value
because the output benefits are high compared to the required input resources.  

% - version control can track individual programmers, what lines of code have
%   been added and deleted.  link to productivity.

%% Bibliography
\bibliographystyle{ACM-Reference-Format}
\bibliography{References}

\end{document}