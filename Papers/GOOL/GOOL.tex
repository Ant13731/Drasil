\documentclass[10pt,conference]{IEEEtran}
\bibliographystyle{plainurl}% the recommnded bibstyle

\usepackage{listings}
\usepackage{amsmath}
\usepackage{hyperref}
\usepackage{cite}

\begin{document}
\title{GOOL: A Generic Object-Oriented Language}

%\titlerunning{Dummy short title}%optional, please use if title is longer than 
%%one line

\author{\IEEEauthorblockN{Brooks MacLachlan}
\IEEEauthorblockA{Department of Computing and Software\\
McMaster University, Canada\\
Email: maclachb@mcmaster.ca}
\and
\IEEEauthorblockN{Jacques Carette}
\IEEEauthorblockA{Department of Computing and Software\\
McMaster University, Canada\\
Email: carette@mcmaster.ca}
\and
\IEEEauthorblockN{W. Spencer Smith}
\IEEEauthorblockA{Department of Computing and Software\\
McMaster University, Canada\\
Email: smiths@mcmaster.ca}
}
%\authorrunning{J. Carette and S. Smith}

% \Copyright{Jacques Carette and W. Spencer Smith}

%\subjclass{Dummy classification}% mandatory: Please choose ACM 2012 classifications from https://www.acm.org/publications/class-2012 or https://dl.acm.org/ccs/ccs_flat.cfm . E.g., cite as "General and reference $\rightarrow$ General literature" or \ccsdesc[100]{General and reference~General literature}. 

%\keywords{Dummy keyword}%mandatory

%\category{}%optional, e.g. invited paper

%\relatedversion{}%optional, e.g. full version hosted on arXiv, HAL, or other respository/website

%\supplement{}%optional, e.g. related research data, source code, ... hosted on a repository like zenodo, figshare, GitHub, ...

%\funding{}%optional, to capture a funding statement, which applies to all authors. Please enter author specific funding statements as fifth argument of the \author macro.

%\acknowledgements{I want to thank \dots}%optional

\maketitle

\begin{abstract}
To be written when paper is mostly done.
\end{abstract}

\section{Introduction}
Given a task, before writing any code a programmer must select a programming 
language to use. Whatever they may base their choice upon, almost any
programming language will work. While a program may be more difficult to
express in one language over another, it should at least be possible to write
the program in either language. Just as the same sentence can be translated to
any spoken language, the same program can be written in any programming
language. Though they will accomplish the same tasks semantically, the
expressions of a program in different programming languages can appear
substantially different due to the unique syntax of each language. Within a
single programming language paradigm, such as object-oriented (OO) programming,
these differences should not be so extreme. OO programs, no matter the
language, share certain structural properties. They are built from variables,
methods, classes, and objects. Some OO languages even have very similar syntax.
But however similar they may be, no two programming languages are identical.

If a programmer wishes to write a program that will integrate into 
existing systems written in different languages, they will likely need to 
write a different version of the program for each. This 
requires investing the time to learn the idiosyncrasies of each language and 
give attention to the operational details where languages differ. Repeatedly 
writing the same program in different languages is entirely inefficient. 
Languages in the same paradigm have many similarities, and there is an 
excellent opportunity to take advantage of these similarities to improve the
efficiency of writing code. If a program could be written in one language and
automatically translated to any other language in the same paradigm, this would
greatly facilitate program reuse. Directly translating between existing 
OO languages will not always be possible because some languages 
require more information than others. A dynamically typed language like Python, 
for instance, cannot be straightforwardly translated to a statically typed 
language like Java, because additional type information would need to be 
provided. But if there was a language that contained all of the information 
that any of the other OO languages would need, it could be used as the source 
language for translation. This source language should also be completely 
language-agnostic, free of any of the idiosyncratic ``noise'' required by
specific languages.

The similarities between OO programs do not end with syntax and structural 
components. Additionally, there are tasks and patterns commonly performed by OO
programs in any language, from simple tasks like splitting a string or 
patterns like defining functions on inputs to produce outputs, to higher-level 
design patterns like those described in \cite{DesignPatterns}. A language 
that provided abstractions for these tasks and patterns would make the process 
of writing OO code even easier.

A Domain-Specific Language (DSL) is a high-level programming language with 
syntax tailored to a specific domain \cite{DSLs}. DSLs allow domain experts 
to write code without having to concern themselves with the syntactical and
operational requirements of general-purpose programming languages. A DSL
abstracts over the details of the code, providing notation for a user to
specify domain-specific knowledge in a natural manner. DSL code is typically
compiled to a more traditional target language. Abstracting over code details
and compiling into traditional OO languages is exactly what we want our source
OO language to do. The code details to abstract over in this case include both
the operational details of using a specific language as well as the higher-level
patterns that commonly show up in OO programs. So the source language we are
looking for is just a DSL in the domain of OO programming languages!

We have developed a Generic Object-Oriented Language (GOOL), proving that such 
a language indeed exists. GOOL is a DSL embedded in Haskell that can currently
generate OO code in Python, Java, C\#, and C++. Theoretically, any OO language 
could be added as a target language for GOOL. This paper presents GOOL, 
starting with the syntax of the language in Section \ref{syntax}. Section 
\ref{implementation} describes how GOOL is implemented. GOOL provides some 
higher-level functions for convenient generation of code following commonly 
used patterns, examples of which are presented in Section \ref{patterns}. We 
close with a discussion of related work in Section \ref{related}, plans for 
future improvements in Section \ref{future}, and conclusions in Section 
\ref{conclusions}.

\section{GOOL Syntax} \label{syntax}
\section{GOOL Implementation} \label{implementation}
\section{Higher-level GOOL functions} \label{patterns}
\section{Related Work} \label{related}
\section{Future Work} \label{future}
\section{Conclusion} \label{conclusions}

%%
%% Bibliography
%%

\bibliography{References}

\end{document}
