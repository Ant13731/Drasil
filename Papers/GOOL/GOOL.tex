%% For double-blind review submission, w/o CCS and ACM Reference (max submission space)
\documentclass[sigplan,review,anonymous,prologue,dvipsnames]{acmart}
\settopmatter{printfolios=true,printccs=false,printacmref=false}
%% For double-blind review submission, w/ CCS and ACM Reference
%\documentclass[sigplan,review,anonymous]{acmart}\settopmatter{printfolios=true}
%% For single-blind review submission, w/o CCS and ACM Reference (max submission space)
%\documentclass[sigplan,review]{acmart}\settopmatter{printfolios=true,printccs=false,printacmref=false}
%% For single-blind review submission, w/ CCS and ACM Reference
%\documentclass[sigplan,review]{acmart}\settopmatter{printfolios=true}
%% For final camera-ready submission, w/ required CCS and ACM Reference
%\documentclass[sigplan]{acmart}\settopmatter{}


%% Conference information
%% Supplied to authors by publisher for camera-ready submission;
%% use defaults for review submission.
\acmConference[PL'18]{ACM SIGPLAN Conference on Programming Languages}{January 01--03, 2018}{New York, NY, USA}
\acmYear{2018}
\acmISBN{} % \acmISBN{978-x-xxxx-xxxx-x/YY/MM}
\acmDOI{} % \acmDOI{10.1145/nnnnnnn.nnnnnnn}
\startPage{1}

%% Copyright information
%% Supplied to authors (based on authors' rights management selection;
%% see authors.acm.org) by publisher for camera-ready submission;
%% use 'none' for review submission.
\setcopyright{none}
%\setcopyright{acmcopyright}
%\setcopyright{acmlicensed}
%\setcopyright{rightsretained}
%\copyrightyear{2018}           %% If different from \acmYear

%% Bibliography style
\bibliographystyle{ACM-Reference-Format}
%% Citation style
%\citestyle{acmauthoryear}  %% For author/year citations
%\citestyle{acmnumeric}     %% For numeric citations
%\setcitestyle{nosort}      %% With 'acmnumeric', to disable automatic
                            %% sorting of references within a single citation;
                            %% e.g., \cite{Smith99,Carpenter05,Baker12}
                            %% rendered as [14,5,2] rather than [2,5,14].
%\setcitesyle{nocompress}   %% With 'acmnumeric', to disable automatic
                            %% compression of sequential references within a
                            %% single citation;
                            %% e.g., \cite{Baker12,Baker14,Baker16}
                            %% rendered as [2,3,4] rather than [2-4].


%%%%%%%%%%%%%%%%%%%%%%%%%%%%%%%%%%%%%%%%%%%%%%%%%%%%%%%%%%%%%%%%%%%%%%
%% Note: Authors migrating a paper from traditional SIGPLAN
%% proceedings format to PACMPL format must update the
%% '\documentclass' and topmatter commands above; see
%% 'acmart-pacmpl-template.tex'.
%%%%%%%%%%%%%%%%%%%%%%%%%%%%%%%%%%%%%%%%%%%%%%%%%%%%%%%%%%%%%%%%%%%%%%


%% Some recommended packages.
\usepackage{booktabs}   %% For formal tables:
                        %% http://ctan.org/pkg/booktabs
\usepackage{subcaption} %% For complex figures with subfigures/subcaptions
                        %% http://ctan.org/pkg/subcaption
\usepackage{listings}

\usepackage{xargs}                      % Use more than one optional parameter in a new commands
%
\usepackage[colorinlistoftodos,prependcaption,textsize=tiny]{todonotes}

% for nice TODO notes
\newcommandx{\unsure}[2][1=]{\todo[inline,linecolor=red,backgroundcolor=red!25,bordercolor=red,#1]{#2}}
\newcommandx{\change}[2][1=]{\todo[inline,linecolor=blue,backgroundcolor=blue!25,bordercolor=blue,#1]{#2}}
\newcommandx{\info}[2][1=]{\todo[inline,linecolor=OliveGreen,backgroundcolor=OliveGreen!25,bordercolor=OliveGreen,#1]{#2}}
\newcommandx{\improvement}[2][1=]{\todo[inline,linecolor=Plum,backgroundcolor=Plum!25,bordercolor=Plum,#1]{#2}}

%%%%%%%%%%%%%%%%%%%%%
%% Useful abbreviations
\newcommand{\Csharp}{C$^{\sharp}$}
\newcommand{\Cplusplus}{C\texttt{++}}

\newcommand{\abbrev}[1]{\textbf{#1}}
\newcommand{\mainstream}{\abbrev{mainstream}}
\newcommand{\readable}{\abbrev{readable}}
\newcommand{\idiomatic}{\abbrev{idiomatic}}
\newcommand{\documented}{\abbrev{documented}}
\newcommand{\oopatterns}{\abbrev{patterns}}
\newcommand{\common}{\abbrev{common}}
\newcommand{\expressivity}{\abbrev{expressivity}}

\begin{document}

%% Title information
\title{GOOL: A Generic Object-Oriented Language}         %% [Short
%%Title] is optional;
                                        %% when present, will be used in
%                                        %% header instead of Full Title.
%\titlenote{with title note}             %% \titlenote is optional;
%                                        %% can be repeated if necessary;
%                                        %% contents suppressed with 'anonymous'
%\subtitle{Subtitle}                     %% \subtitle is optional
%\subtitlenote{with subtitle note}       %% \subtitlenote is optional;
%                                        %% can be repeated if necessary;
%                                        %% contents suppressed with 'anonymous'


%% Author information
%% Contents and number of authors suppressed with 'anonymous'.
%% Each author should be introduced by \author, followed by
%% \authornote (optional), \orcid (optional), \affiliation, and
%% \email.
%% An author may have multiple affiliations and/or emails; repeat the
%% appropriate command.
%% Many elements are not rendered, but should be provided for metadata
%% extraction tools.

%% Author with single affiliation.
\author{Brooks MacLachlan}
%\authornote{with author1 note}          %% \authornote is optional;
                                        %% can be repeated if necessary
%\orcid{nnnn-nnnn-nnnn-nnnn}             %% \orcid is optional
\affiliation{
  \position{Position1}
  \department{Department of Computing and Software}              %% \department
  %%is recommended
  \institution{McMaster University}            %% \institution is required
  \streetaddress{1280 Main Street West}
  \city{Hamilton}
  \state{Ontario}
  \postcode{L8S 4L8}
  \country{Canada}                    %% \country is recommended
}
\email{maclachb@mcmaster.ca}          %% \email is recommended

\author{Jacques Carette}
%\authornote{with author1 note}          %% \authornote is optional;
%% can be repeated if necessary
%\orcid{nnnn-nnnn-nnnn-nnnn}             %% \orcid is optional
\affiliation{
  \position{Position1}
  \department{Department of Computing and Software}              %%
  %%\department is recommended
  \institution{McMaster University}            %% \institution is required
  \streetaddress{1280 Main Street West}
  \city{Hamilton}
  \state{Ontario}
  \postcode{L8S 4L8}
  \country{Canada}                    %% \country is recommended
}
\email{carette@mcmaster.ca}          %% \email is recommended

\author{Spencer Smith}
%\authornote{with author1 note}          %% \authornote is optional;
%% can be repeated if necessary
%\orcid{nnnn-nnnn-nnnn-nnnn}             %% \orcid is optional
\affiliation{
  \position{Position1}
  \department{Department of Computing and Software}              %%
  %%\department is recommended
  \institution{McMaster University}            %% \institution is required
  \streetaddress{1280 Main Street West}
  \city{Hamilton}
  \state{Ontario}
  \postcode{L8S 4L8}
  \country{Canada}                    %% \country is recommended
}
\email{smiths@mcmaster.ca}          %% \email is recommended


%% Abstract
%% Note: \begin{abstract}...\end{abstract} environment must come
%% before \maketitle command
\begin{abstract}
Text of abstract \ldots.
\end{abstract}


%% 2012 ACM Computing Classification System (CSS) concepts
%% Generate at 'http://dl.acm.org/ccs/ccs.cfm'.
\begin{CCSXML}
<ccs2012>
<concept>
<concept_id>10011007.10011006.10011008</concept_id>
<concept_desc>Software and its engineering~General programming languages</concept_desc>
<concept_significance>500</concept_significance>
</concept>
<concept>
<concept_id>10003456.10003457.10003521.10003525</concept_id>
<concept_desc>Social and professional topics~History of programming languages</concept_desc>
<concept_significance>300</concept_significance>
</concept>
</ccs2012>
\end{CCSXML}

\ccsdesc[500]{Software and its engineering~General programming languages}
\ccsdesc[300]{Social and professional topics~History of programming languages}
%% End of generated code


%% Keywords
%% comma separated list
\keywords{keyword1, keyword2, keyword3}  %% \keywords are mandatory in final camera-ready submission


%% \maketitle
%% Note: \maketitle command must come after title commands, author
%% commands, abstract environment, Computing Classification System
%% environment and commands, and keywords command.
\maketitle


\section{Introduction}

Given a task, before writing any code a programmer must select a programming
language to use. Whatever they may base their choice upon, almost any
programming language will work. While a program may be more difficult to
express in one language over another, it should at least be possible to write
the program in either language. Just as the same sentence can be translated to
any spoken language, the same program can be written in any programming
language. Though they will accomplish the same tasks, the
expressions of a program in different programming languages can appear
substantially different due to the unique syntax of each language. Within a
single language paradigm, such as object-oriented (OO),
these differences should not be as extreme -- at least the global structuring
mechanisms and the local idioms will be shared.  Mainstream OO languages
generally contain (mutable) variables, methods, classes, objects and a core
imperative set of primitives. Some OO languages even have very similar syntax
(such as Java and \Csharp say).

When faced with the task to write a program meant to fit into multiple existing
infrastructure, which might be written in different languages, frequently that
entails writing different versions of the program, one for each. While not
necessarily difficult, it nevertheless
requires investing the time to learn the idiosyncrasies of each language and
pay attention to the operational details where languages differ. Ultimately,
the code will likely be marred by influences of the language the programmer
knows best. They may consistently use techniques that they are familiar with
from one language, while unaware that the language in which they are currently
writing offers a better or cleaner way of doing the same task
\cite{scholtz1990learning, alexandron2012effect}. Besides this
likelihood of writing sub-optimal code, repeatedly writing the same program in
different languages is entirely inefficient, both as an up-front development cost,
and even more so for maintenance.

Since languages from the same paradigm share many semantic similarities,
it is tempting to try to leverage this; perhaps the program could be written in
one language and
automatically translated to the others?  But a direct translation is often
difficult, as different languages require the programmer to provide different
levels of information, even to achieve the same taks.  For example, a dynamically
typed language like Python
cannot be straightforwardly translated to a statically typed
language like Java, as additional type information generally needs to be
provided\footnote{Type inference for Python notwithstanding}.

What if, instead, there was a single meta-language which was designed to
contain the common semantic concepts of a number of OO languages, encoded
in such a way that all the necessary information for translation was always
present?  This source language could be made to be agnostic about what
eventual target language was used -- free of the idiosyncratic details of
any given language.  This would be quite the boon for the translator.
In fact, we could try to go even further, and attempt to teach the translator
about idiomatic patterns of each target language.

Why would this even be possible? There are commonly performed tasks and
patterns of OO solutions, from idioms to architecture patterns, as outlined
in~\cite{gamma1995design}. A meta-language
that provided abstractions for these tasks and patterns would make the process
of writing OO code even easier.

But is this even feasible? In some sense, this is already old hat: most
modern compilers have a single internal Intermediate Representation (IR)
which is used to target multiple processors. Compilers can generate
human-readable symbolic assembly code for a large family of CPUs. But this
is not quite the same as generating human-readable, idiomatic high-level
languages.

There is another area where something like this has been looked at:
the production of high-level code from Domain-Specific Languages (DSL).
A DSL is a high-level programming language with syntax and semantics tailored
to a specific domain \cite{mernik2005and}. DSLs allow domain experts
to write code without having to concern themselves with the details of
General-Purpose programming Languages (GPL). A DSL
abstracts over the details of the code, providing notation for a user to
specify domain-specific knowledge in a natural manner. Such DSL code is typically
translated to a GPL for execution. Abstracting over code details
and compiling into traditional OO languages is exactly what we want to do!
The details to abstract over include both syntactic and operational details of
any specific language, but also higher-level idioms in common use.  Thus
the language we are looking for
is just a DSL in the domain of OO programming languages!

There are some DSLs that already generate code in multiple languages, to be
further discussed in Section \ref{sec:related}, but none of them have the
combination of features we want. We are indeed trying to do something odd:
writing a ``DSL'' for what is essentially the domain of OO GPLs. Furthermore,
we have additional requirements:
\begin{enumerate}
\item The generated code should be human-readable,
\item The generated code should be idiomatic,
\item The generated code should be documented,
\item The generator should allow one to express common OO patterns.
\end{enumerate}

We have developed a Generic Object-Oriented Language (GOOL)%
\footnote{GOOL is publicly available; the exact link will be
given once the paper is no longer anonymous}, demonstrating
that all these requirements can be met.  GOOL is a DSL embedded in Haskell that
can currently generate code in Python, Java, \Csharp, and \Cplusplus%
\footnote{and is close to generating Lua and Objective-C, but those backends
have fallen into disuse}. Others could be added, with the implementation
effort being commensurate to their (semantic) distance to the languages
already supported.

First we present the high-level requirements for such an endeavour, in
Section~\ref{sec:req}.  To be able to give illustrated examples, we
next show the syntax of GOOL in Section~\ref{sec:creating}. The details of
the implementations, namely the internal representation and the
family of pretty-printers, is in Section~\ref{sec:implementation}.
Common patterns are illustrated in Section~\ref{sec:patterns}.  We
close with a discussion of related work in Section \ref{sec:related}, plans for
future improvements in Section \ref{sec:future}, and conclusions in Section
\ref{sec:conclusions}.

\section{Requirements} \label{sec:req}

While we outlined some of our requirements above, here we will give a
complete list, along with acronyms (to make referring to them simpler),
as well as some reasoning behind each requirement.

\begin{description}
\item[mainstream] Generate code in mainstream object-oriented languages.
\item[readable] The generated code should be human-readable,
\item[idiomatic] The generated code should be idiomatic,
\item[documented] The generated code should be documented,
\item[patterns] The generator should allow one to express common OO patterns.
\item[common] Language commonalities should be abstracted.
\item[expressivity] The resulting language should be rich enough to express a
certain set of test cases, drawn from scientific computation software.
\end{description}

Targetting OO languages (\mainstream) is primarily because of their popularity, and thus
would enjoy the most potential users --- in much the same way that the makers
of Scala and Kotlin chose to target the JVM to leverage the Java ecosystem, and
Typescript for Javascript.

The \readable~requirement is not as obvious. As DSL users are typically
domain experts who are not ``programmers'', why generate readable code?
Few Java programmers ever look at JVM bytecode, and few \Cplusplus programmers
look at assembly. But GOOL's aim is different: to allow writing
high-level OO code once, but have it be available in many GPLs. One use case
would be to generate libraries of utilities for a narrow domain. As needs
evolve and language popularity changes, it is useful to have it immediately
available in a number of languages. Another use, which is a core part of our
own motivation, is to have \emph{extremely well documented} code, indeed to
a level that would be unrealistic to do by hand. But this documentation is
crucial in domains where \emph{certification} of code is required.

The same underlying reasons for \readable~also drive \idiomatic~and \documented,
as they contribute to the human-understandability of the generated code.
\idiomatic~is important as many human readers would find the code ``foreign''
otherwise, and would not be keen on using it.
Note that documentation can span from informal comments meant for humans, to
formal, structured comments useful for generating API documentation with tools
like Doxygen, or with a variety of static analysis tools.
Readability (and thus understandability) are improved when code is pretty-printed%
\cite{buse2009learning}. Thus taking care of layout, redundant parentheses,
well-chosen variable names, using a common style with lines which are not too
long, are just as valid for generated code as for human-written code.
GOOL does not prevent users from writing undocumented or complex code, if they
choose to do so. It just makes it easy to have \readable, \idiomatic~and
\documented~code in multiple languages.

The \oopatterns~requirement is typical of DSLs: common programming patterns
can be reified into a proper linguistic form instead of being merely
informal. In particular some of the \emph{design patterns} of~\cite{gamma1995design}
can become part of the language itself. This does make writing some OO
code even easier in GOOL than in GPLs, it also helps quite a lot with
keeping GOOL language-agnostic and generating idiomatic code. Illustrative
examples will be given in Section~\ref{sec:patterns}.  But we can give an
indication now as to why this helps: Consider Python's
ability to return multiple values with a single return statement, which
is uncommon in other languages.  Two choices might be to disallow this
feature in GOOL, or throw an error on use when generating code in languages
that do not support this feature. In the first case, this would likely mean
unidiomatic Python code, or increased complexity in the Python generator to
infer that pattern. The second option is worse still: one might have to resort
to writing language-specific GOOL, obviating the whole reason for the language!
Multiple-value return statements are always used when a function returns multiple
outputs; what we can do in GOOL is to support such multiple-output functions,
and
then generate the idiomatic pattern of implementation in each target language.

The last two requirements, that language commonalities (\common) be abstracted,
and that we can phrase a certain collection of test cases (\expressivity) are
internal requirements: we didn't set out to create GOOL as a primary artifact,
but as a side-effect of other work on different methods of creating long-lived
scientific software.  Part of long-lived means that we need to be flexible about
the technology, thus needing to be polymorphic on the underlying language.
Regarding commonalities, we noticed a lot of repeated code in our initial
backends, something that ought to be distasteful to most programmers. For
example, writing a generator for both Java and \Csharp makes it incredibly
clear how similar the two languages are.

\section{Creating GOOL} \label{sec:creating}

How do we go about creating a ``generic'' object-oriented language?
We chose an incremental abstraction approach: start from two languages,
and unify them \emph{conceptually}.  In other words, pay very close
attention to the \emph{denotational} semantics of the features, some
attention to the operational semantics, and ignore syntactic details.

This is most easily done from the core imperative language outwards.
Most languages provide similar basic types (variations on integers,
floating point numbers, characters, strings, etc) and functions to deal
with them. The core expression language tends to be extremely similar
cross languages. One then moves up to the statement language ---
assignments, conditionals, loops, etc.  Here we start to encounter
variations, and choices can be made, and we'll cover that later.

For ease of experimentation, we chose to make GOOL an embedded
domain specic language (EDSL) inside Haskell.  Haskell is very well-suited
for this task, offering a variety of features (GADTs, type classes,
parametric polymorphism, kind polymorphism, etc) which is extremely useful
for building languages.  Its syntax is also fairly liberal, so that it is
possible to create \emph{smart constructors} that somewhat mimic the
usual syntax of OO languages.

\subsection{GOOL Syntax: Imperative core} \label{ssec:syntax}

As our exposition has been somewhat abstract until now, it is useful to
dive in and give some concrete syntax, so as to be able to illustrate
our ideas with valid code.

Specifically, basic types in GOOL are \verb|bool| for Booleans,
\verb|int| for integers, \verb|float| for doubles, \verb|char|
for characters, \verb|string| for strings, \verb|infile| for a file
in read mode, and \verb|outfile| for a file in write mode. Lists can be
specified with \verb|listType|. For example, \verb|listType int|
specifies a list of integers. Types of objects are specified using
\verb|obj| followed by the class name, so \verb|obj "FooClass"| is the type of
an object of a class called ``FooClass''.

Variables are specified with \verb|var| followed by the variable name and type.
For example, \verb|var "ages" (listType int)| represents a variable called
``ages'' that is a list of integers. This illustrates a (necessary) design
decision: even though we target languages like Python, as we also target
Java, types are necessary. As type inference for OO languages is too
difficult, we chose to be explicitly typed.

As some constructions are common, it is useful to offer shortcuts for
defining them; for example, the above can also be done via
\verb|listVar "ages" int|. Typical use would be
\begin{lstlisting}
let ages = listVar "ages" int in
\end{lstlisting}
so that \verb|ages| can be used directly from then on. Other GOOL syntax for
specifying variables is shown in Table~\ref{tab:variables}.

\begin{table}[!h]
\caption{Syntax for specifying variables}
\begin{tabular}{p{0.12\textwidth} p{0.33\textwidth}}
  \textbf{GOOL Syntax} & \textbf{Semantics} \\
  \midrule
  \verb|extVar| & for a variable from an external library \\
  \verb|classVar| & for a variable belonging to a class \\
  \verb|objVar| & for a variable belonging to an object \\
  \verb|$->| & infix operator form of \verb|objVar| \\
  \verb|self| & for referring to an object in the definition of its class \\
\end{tabular}
\label{tab:variables}
\end{table}

Note that GOOL distinguishes a variable from its value\footnote{
as befits the use-mention distinction from analytic philosophy}. To get
the value of~\verb|ages|, one must write \verb|valueOf ages|. The reason for
this distinction will be made clear in section~\ref{where?}, driven
by semantic considerations. This is beneficial for stricter typing and
enables convenient syntax for \oopatterns~that translate to more idiomatic
code.

Syntax for literal values is shown in Table~\ref{tab:literals} and for
operators on values is shown in Table~\ref{tab:operators}. In GOOL, each
operator is prefixed with an additional symbol based on type. Operators that
return Booleans are prefixed by a \verb|?|, operators on numeric values are
prefixed by \verb|#|, and other operators are prefixed by \verb|$|.

\begin{table}[!h]
  \caption{Syntax for literal values}
  \begin{tabular}{p{0.12\textwidth} p{0.33\textwidth}}
    \textbf{GOOL Syntax} & \textbf{Semantics} \\
    \midrule
    \verb|litTrue| & literal Boolean true \\
    \verb|litFalse| & literal Boolean false \\
    \verb|litInt| \verb|i| & literal integer \verb|i| \\
    \verb|litFloat| \verb|f| & literal float \verb|f| \\
    \verb|litChar| \verb|c| & literal character \verb|c| \\
    \verb|litString| \verb|s| & literal string \verb|s| \\
  \end{tabular}
  \label{tab:literals}
\end{table}

\begin{table}[!h]
  \caption{Operators for making expressions}
  \begin{tabular}{p{0.12\textwidth} p{0.33\textwidth}}
    \textbf{GOOL Syntax} & \textbf{Semantics} \\
    \midrule
    \verb|?!| & Boolean negation \\
    \verb|?&&| & conjunction \\
    \verb|?||| & disjunction \\
    \verb|?<| & less than \\
    \verb|?<=| & less than or equal \\
    \verb|?>| & greater than \\
    \verb|?>=| & greater than or equal \\
    \verb|?==| & equality \\
    \verb|?!=| & inequality \\
    \verb|#~| & numeric negation \\
    \verb|#/^| & square root \\
    \verb|#|| & absolute value \\
    \verb|#+| & addition \\
    \verb|#-| & subtraction \\
    \verb|#*| & multiplication \\
    \verb|#/| & division \\
    \verb|#%| & modulus \\
    \verb|#^| & exponentiation \\
  \end{tabular}
  \label{tab:operators}
\end{table}

\begin{table}[!h]
  \caption{Syntax for conditionals and function application}
  \begin{tabular}{p{0.12\textwidth} p{0.33\textwidth}}
    \textbf{GOOL Syntax} & \textbf{Semantics} \\
    \midrule
    \verb|inlineIf| & conditional expression \\
    \verb|funcApp| & function application, to a list of parameters \\
    \verb|extFuncApp| & function application, for external library
    functions\\
    \verb|newObj| & for calling an object constructor (\verb|extNewObj|
    exists too) \\
    \verb|objMethodCall| & for calling a method on an object \\
  \end{tabular}
  \label{tab:values}
\end{table}

Syntax for defining values with conditional expressions or function
applications is shown in Table~\ref{tab:values}. \verb|selfFuncApp| and
\verb|objMethodCallNoParams| are two shortcuts for the common cases when a
method is being called on \verb|self| or when the method takes no parameters.

Variable declarations are statements, and take a variable specification
as argument. For \verb|foo = var "foo" int|, the corresponding variable
declaration would be \verb|varDec foo|, and to also initialize it
\verb|varDecDef foo (litInt 5)| can be used.

Assignments are represented by \verb|assign a (litInt 5)|. Convenient
infix and postfix operators are also provided, prefixed by \verb|&|:
\verb|&=| is a synonym for \verb|assign|, and C-like
\verb|&+=|, \verb|&++|, \verb|&-=| and \verb|&~-| (the more intuitive
\verb|&--| cannot be used as \verb|--| starts a comment in Haskell).

Other simple statements in GOOL include \verb|break| and \verb|continue|,
\verb|returnState| followed by a value to return, \verb|throw| followed by an
error message to throw, \verb|free| followed by a variable to free from
memory, and \verb|comment| followed by a string to be displayed as a
single-line comment.

Most languages have statement blocks, introduced by \verb|block| with
a list of statements in GOOL. Bodies (\verb|body|) are composed
of a list of blocks, and can be used as a function body, conditional body, loop
body, etc. The purpose of blocks as an intermediate between statement and body
is to allow for more organized, readable generated code. For example, the
generator can choose to insert a blank line between blocks so lines of code
related to the same task are visually grouped together. Naturally shortcuts
are provided for single-block bodies (\verb|bodyStatements|) and
for the common single-statement case, \verb|oneLiner|.


GOOL has two forms of conditionals: if-then-else via \verb|ifCond| (
which takes a list of pairs of conditions and bodies) and
if-then via \verb|ifNoElse|.  For example:
\begin{lstlisting}
ifCond [
  (foo ?> litInt 0, oneLiner (
    printStrLn "foo is positive")),
  (foo ?< litInt 0, oneLiner (
    printStrLn "foo is negative"))]
  (oneLiner $ printStrLn "foo is zero")
\end{lstlisting}
GOOL also supports \verb|switch| statements.

There are a variety of loops: for-loops (\verb|for|), which are
parametrized by a statement to
initialize the loop variable, a condition, a statement to update the loop
variable, and a body; \verb|forRange| loops, which are given a
starting value, ending value, and step size; as well as \verb|forEach|
loops.
For example:
\begin{lstlisting}
for (varDecDef age (litInt 0))
  (age < litInt 10) (age &++) loopBody
forRange age (litInt 0) (litInt 9)
  (litInt 1) loopBody
forEach age ages loopBody
\end{lstlisting}
While-loops (\verb|while|) are parametrized by a condition and a body. Finally,
try-catch statements (\verb|tryCatch|) are parametrized by two bodies.

\subsection{GOOL Syntax: OO features}

A \verb|function| declaration is followed by the function
name, scope, binding type (static or dynamic), type, list of parameters, and
body. Methods (\verb|method|) are defined similarly, with the addition of the
specification of the containing class' name.  Parameters are built from
variables, using \verb|param| or \verb|pointerParam|. For example, assuming
variables ``num1'' and ``num2'' have been defined, one can define an
\textsf{add} function as follows:
\begin{lstlisting}
function "add" public dynamic_ int
  [param num1, param num2]
  (oneLiner (returnState (num1 #+ num2)))
\end{lstlisting}
The  \verb|pubMethod| and \verb|privMethod| shortcuts are useful for public
dynamic and private dynamic methods, respectively. \verb|mainFunction|
followed by a body defines the main function of a program. \verb|docFunc|
generates a documented function from a function description and
a list of parameter descriptions, an optional description of the return
value, and the function itself.  This generates Doxygen-style comments.

Classes are defined with \verb|buildClass| followed by the class name, name of
the parent class (if applicable), scope, list of state variables, and list of
methods. State variables can be built by \verb|stateVar| followed by an
integer, scope, static or dynamic binding, and the variable itself. The integer
is a measure of delete priority. \verb|constVar| can be used for constant state
variables. Shortcuts for state variables include \verb|privMVar| for private
dynamic, \verb|pubMVar| for public dynamic, and \verb|pubGVar| for public
static variables. For example:
\begin{lstlisting}
buildClass "FooClass" Nothing public
  [pubMVar 0 var1, privMVar 0 var2]
  [mth1, mth2]
\end{lstlisting}
\verb|Nothing| here indicates that this class does not have a parent,
\verb|privClass| and \verb|pubClass| are shortcuts for private and public
classes, respectively. \verb|docClass| serves a similar purpose as \verb|docFunc|.

\subsection{GOOL syntax: modules and programs}

Akin to Java packages and other similar constructs, GOOL has modules
(\verb|buildModule|) consisting of a module name, a list of libraries to import,
a list of functions, and a list of classes. Module-level comments are done
with \verb|docMod|.

\begin{comment}
Passing a module to
\verb|fileDoc| will put the finishing touches on the generated file, such as
imports of standard libraries and preprocessor guards for \Cplusplus header files.
\begin{lstlisting}
fileDoc \$ buildModule "mod1" ["library1"]
  [func1, func2] [class1, class2]
\end{lstlisting}
\end{comment}

Finally, at the top of the GOOL hierarchy are programs, auxiliary files, and
packages. A program (\verb|prog|) has a name and a list of files.  A
\verb|package| is a program and a list of auxiliary files. These files are non
code files that augment the program. Examples are a Doxygen configuration file
(\verb|doxConfig|), and a makefile (\verb|makefile|).  One of the parameters of
\verb|makefile| toggles generation of a \verb|make| \verb|doc| rule, which will
compile the Doxygen documentation with the generated Doxygen configuration
file.

\section{GOOL Implementation} \label{sec:implementation}

There are two ``obvious'' means of dealing with large embedded DSLs in
Haskell: either as a set of Generalized Algebraic Data Types (GADTs), or
using a set of classes, in the ``finally tagless'' style~\cite{carette2009finally}
(we will refer to it as simply \emph{tagless} from now on).
The current implementation uses a ``sophisticated'' version of tagless. A
first implementation~\footnote{citation omitted for anonymization}
used a straightforward version of tagless
which did not allow for enough generic routines to be properly implemented.
This was replaced by a version based on GADTs, which fixed that problem, but
did not allow for \emph{patterns} to be easily encoded. Thus the current
version has gone back to tagless, but also uses \emph{type families} in a crucial
way.

It is worth recalling that in tagless, the means of encoding a language,
through methods from a set of classes, really encodes a generalized
\emph{fold} over any \emph{representation} of the language.  Thus what
looks like GOOL ``keywords'' are either class methods or generic functions
that await the specification of a dictionary to decide on the final
interpretation of the representation.  We typically instantiate these to
language renderers, but we're also free to do various analysis passes if
we wish.

Because tagless representations give an embedded syntax to a DSL while
being polymorphic on the eventual semantic interpretation of the terms,
\cite{carette2009finally} dubs the resulting classes ``symantic''.
Our language is defined by a hierarchy of $43$ of these symantic classes,
grouped by functionality.  For example, there are classes for programs,
bodies, control blocks, types, unary operators, variables, values, selectors,
statements, control statements, blocks, scopes, classes, modules, and so
on.  These define $328$ different methods --- GOOL is not a small language!

\improvement{Perhaps a graph of all $43$ classes as a picture would be a
nice illustration?}

For example, here is how variables are defined:
\begin{lstlisting}
class (TypeSym repr)
  => VariableSym repr where
    type Variable repr
    var :: Label -> repr (Type repr)
             -> repr (Variable repr)
\end{lstlisting}
As variables are typed, a representation of variables much also know how
to represent types, thus we constrain our representation with that capability,
here the~\verb|TypeSym| class.  We also notice the use of an
\emph{associated type}~\verb|type Variable repr|. This is a type-level
function which is representation-dependent.  Each instance of this class
is free to define its own internal representation of what a
\verb|Variable| is. \verb|var| is then a constructor for variables,
which takes a \verb|Label| and a representation of a type, returning
a representation of a variable.  Specifically, \verb|repr| has kind
\verb|* -> *|, and thus \verb|Variable| has kind \verb|(* -> *) -> *|.
In \verb|repr| \verb|(X repr)|, the type variable \verb|repr| appears
twice because there are two layers of abstraction: over the target
language, handled by the outer \verb|repr|, and over the underlying
types to which GOOL's types map, represented by the inner \verb|repr|.

The principal use we make of the flexibility of type families on a
per-target-language basis is to record more (or less) information for
successful code generation. For example, the internal representation for a 
state variable in \Cplusplus stores the corresponding destructor code for the 
variable, but in the other languages destructors are not needed so the internal 
representation of a state variable is just a \verb|Doc|.

For example, for Java, we instantiate the class as follows:
\begin{lstlisting}
instance VariableSym JavaCode where
  type Variable JavaCode = VarData
  var = varD
\end{lstlisting}
where \verb|JavaCode| is essentially the \verb|Identity| monad
by another name:
\begin{lstlisting}
newtype JavaCode a = JC {unJC :: a}
\end{lstlisting}
The \verb|unJC| record field is useful for type inference: when applied to
an otherwise generic term, it lets Haskell infer that we're then wishing
to only consider the \verb|JavaCode| instances.  \verb|VarData| is defined as
\begin{lstlisting}
data VarData = VarD {
  varBind :: Binding,
  varName :: String,
  varType :: TypeData,
  varDoc :: Doc}
\end{lstlisting}
In other words, for every (Java) variable, we store its binding time, either
\verb|Static| or \verb|Dynamic|, the name of the variable as a \verb|String|,
and its type as a \verb|TypeData|, which is the representation for
\verb|Type|s for Java, and finally how the variable should appear in the
generated code, represented as a \verb|Doc|. \verb|Doc| comes from the
package \verb|Text.PrettyPrint.HughesPJ| and represents formatted text.

All representing structures contain at least a \verb|Doc|. It can be considered
to be our \emph{dynamic} representation of code, from a partial-evaluation
perspective. The other fields are generally \emph{static} information used to
optimize the code generation.

Generally, GOOL prefers to work generically. So there is as little code as
possible that works on \verb|VarData| directly.  Instead, there are
methods like \verb|variableDoc|, part of the \verb|VariableSym| type class,
with signature:
\begin{lstlisting}
variableDoc :: repr (Variable repr)
  -> Doc
\end{lstlisting}
which acts as an accessor.  For \verb|JavaCode|, its instance is
straightforward:
\begin{lstlisting}
variableDoc = varDoc . unJC
\end{lstlisting}

Here are a few more examples of the kinds of additional information stored by
each representation. \verb|Statement| stores a
\verb|Terminator| which is how that language indicates how a statement is to be
terminated (frequently this is a semi-colon).  For \verb|Method|, a Boolean
indicates whether it is the main method.  For \verb|Value|, \verb|UnaryOp|
and \verb|BinaryOp|, precendence information is stored so that printing can
elide parentheses whenever possible, leading to more readable code.

Note that the \verb|JavaCode| instance of \verb|VariableSym| defines the
\verb|var| function via the \verb|varD| function:
\begin{lstlisting}
varD :: (RenderSym repr) => Label ->
  repr (Type repr) -> repr (Variable repr)
varD n t = varFromData Dynamic n t
  (varDocD n)

varDocD :: Label -> Doc
varDocD = text
\end{lstlisting}
\verb|varD| is generic, i.e. works for all instances, via dispatching to other
generic functions, such as \verb|varFromData|:
\begin{lstlisting}
varFromData :: Binding -> String ->
  repr (Type repr) -> Doc ->
  repr (Variable repr)
\end{lstlisting}
This method is in a type class \verb|InternalVariable|. Several of these
``internal'' classes exist, none of which are exported from GOOL's interface.
They however contain functions useful for the various language renderers, but
not meant to be used to construct code representations, as they reveal too
much of the internals (and are rather tedious to use too).  One important
example is the \verb|cast| method, which is never needed by user-level code,
but frequently used by higher-level functions.

\verb|varDocD| can simply be \verb|text| as \verb|Label| is simply an
alias for a \verb|String| -- and Java variables are simply their names,
which is indeed the case for most OO languages.  Exceptions can use the
class mechanism to override this in their specific case.

This genericity
makes writing new renderers for new languages fairly straightforward.
GOOL's Java and \Csharp renderers demonstrate this fact well. Out of $328$
methods across all of GOOL's type classes, the instances of $228$ of them are
shared between the Java and \Csharp renderers, in that they are just calls to the
same common function. A further $37$ are partially shared, for example they call
the same common function but with different parameters. $143$ methods are
actually the same between all $4$ languages GOOL currently targets. This might
indicate that some should be generic functions rather than class methods,
but we have not investigated this in detail yet.

Examples from Python and \Csharp are not shown here because they both
work very similarly to the Java renderer. There are \verb|PythonCode| and
\verb|CSharpCode| analogs to \verb|JavaCode|, the underlying types are all the
same, and the methods are defined by calling common functions where possible
or by constructing the GOOL value directly in the instance definition, if the
definition is unique to that language.

\Cplusplus is different since most modules are split between a source and
header file. To generate \Cplusplus, we traverse the code twice,
once to generate the header file and a second time to generate the
source file corresponding to the same module. This is done via two instances
of the classes, for two different types: \verb|CppSrcCode| for source code and
\verb|CppHdrCode| for header code. Since a main function does not require a
header file, the \verb|CppHdrCode| instance for a module containing only a main
function is empty. The renderer optimizes empty modules/files away --- for
all renderers.

As \Cplusplus source and header should always be generated together, a third
type, \verb|CppCode| achieves this:
\begin{lstlisting}
data CppCode x y a =
  CPPC {src :: x a, hdr :: y a}
\end{lstlisting}
The type variables \verb|x| and \verb|y| are intended to be instantiated with
\verb|CppSrcCode| and \verb|CppHdrCode|, but they are left generic
so that we may use an even more generic \verb|Pair| class:
\begin{lstlisting}
class Pair (p :: (* -> *) -> (* -> *)
  -> (* -> *)) where
  pfst :: p x y a -> x a
  psnd :: p x y b -> y b
  pair :: x a -> y a -> p x y a

instance Pair CppCode where
  pfst (CPPC xa _) = xa
  psnd (CPPC _ yb) = yb
  pair = CPPC
\end{lstlisting}
\verb|Pair| is a \emph{type constructor} pairing, one level up from
Haskell's own \verb|(,) :: * -> * -> *|.  It is given by one constructor
and two destructors, much as the Church-encoding of pairs into the
$\lambda$-calculus.

To understand how this works, here is the instance of \verb|VariableSym|
but for \Cplusplus:
\begin{lstlisting}
instance (Pair p) => VariableSym 
  (p CppSrcCode CppHdrCode) where
  type Variable 
    (p CppSrcCode CppHdrCode) = VarData
  var n t = pair 
    (var n $ pfst t) (var n $ psnd t)
\end{lstlisting}
The instance is generic in the pair representation \verb|p| but
otherwise concrete, because \verb|VarData| is concrete. The actual
instance code is straightforward, as it just dispatches to the 
underlying instances, using the generic wrapping/unwrapping
methods from \verb|Pair|.  This pattern is used for all instances,
so adapting it to any other language with two (or more) files per
module is straightforward.

At the program level, the difference between source and header is no
longer relevant, so they are joined together into a single component.
For technical reasons, currently \verb|Pair| is still used, and we arbitrarily
choose to put the results in the first component.

While ``old'' features of OO languages --- basically features that
were already present in ancestor procedural languages like Algol ---
have fairly similar renderings, more recent (to OO languages) features
such as for-each loops show more variations.  More precisely,
the first line of a for-each loop in Python, Java, \Csharp and
\Cplusplus~are (respectively):
\begin{lstlisting}
for age in ages:
\end{lstlisting}
\begin{lstlisting}
for (int age : ages) {
\end{lstlisting}
\begin{lstlisting}
foreach (int age in ages) {
\end{lstlisting}
\begin{lstlisting}
for (std::vector<int>::iterator age \
  = ages.begin(); age != ages.end(); \
  age++) {
\end{lstlisting}
By providing \verb|forEach|, GOOL abstracts over these differences.

\section{Encoding Patterns} \label{sec:patterns}

There are various levels of ``patterns'' to encode. The previous section
documented how to encode the programming language aspects. Now we
move on to other patterns, from simple library-level functions, to
simple tasks (command-line arguments, list processing, printing), on to
more complex patterns such as methods with a mixture of input, output
and in-out parameters, and finally on to design patterns.

\subsection{Internalizing library functions}

Consider the simple trigonometric sine function, called \verb|sin| in
GOOL. It is common enough to warrant its own name, even though in most
languages it is part of a library.  A GOOL expression \verb|sin foo|
can then be seamlessly translated to 
yield \verb|math.sin(foo)| in Python, \verb|Math.sin(foo)| in Java,
\verb|Math.Sin(foo)| in \Csharp, and \verb|sin(foo)| in \Cplusplus. Other
functions are handled similarly.  This part is easily extensible, but does
require adding to GOOL classes.

\subsection{Command line arguments}

A slightly more complex task is accessing arguments passed on the command
line. This tends to differ more significantly accross languages. GOOL
offers an abstraction of these mechanisms, through an \verb|argList| function
that represents the list of arguments, as well as convenience functions for
common tasks such as indexing into \verb|argList| and checking if an argument
at a particular position exists.

\subsection{Lists}

Variations on lists are frequently used in OO code.  But the actual API
in each language tends to vary quite a lot, so we need to provide a single
abstraction that provides sufficient functionality to do useful list
computations.  Rather than abstracting from the functionality provided
in the libraries of each language to find some common ground, we instead
reverse engineer the ``useful'' API from actual use cases in scientific
code.  

One thing we immediately notice from such an exercise is that lists in
OO languages are rarely \emph{linked lists} (unlike in Haskell, our host
language), but rather more like a dynamically sized vector. In particular,
indexing a list by position, which is a horrifying idea for linked lists,
is extremely common.

This narrows things down to a small set of functions and statements:
\begin{table}[!h]
\caption{List functions}
\begin{tabular}{p{0.15\textwidth} p{0.3\textwidth}}
  \textbf{GOOL Syntax} & \textbf{Semantics} \\
  \midrule
  \verb|listAccess| & access a list element at a given index \\
  \verb|listSet| & set a list element at a given index to a given value \\
  \verb|at| & same as \verb|listAccess| \\
  \verb|listSize| & get the size of a list \\
  \verb|listAppend| & append a value to the end of a list \\
  \verb|listIndexExists| & check whether the list has a value at a given index 
  \\
  \verb|indexOf| & get the index of a given value in a list \\
\end{tabular}
\label{tab:listfuncs}
\end{table}
For example, \verb|listAccess| \verb|(valueOf ages)| \verb|(litInt 1)| will 
generate
\verb|ages[1]| in Python and \Csharp, \verb|ages.get(1)| in Java, and
\verb|ages.at(1)| in \Cplusplus.  List slicing is a very convenient
higher-level primitive.  The \verb|listSlice| \emph{statement} gets
a variable for the rest, a list to slice, and three
values representing the starting and ending indices for the slice and the step
size. These last three values are all optional (we use Haskell's \verb|Maybe|
for this) and default to the start of the list, end of the list and $1$
respectively.  To take elements from index 1 to 2 of \verb|ages| and
assign the result to \verb|someAges|, we can use
\begin{lstlisting}
listSlice someAges (valueOf ages)
  (Just $ litInt 1) (Just $ litInt 3)
  Nothing
\end{lstlisting}
List slicing is of particular note because the generated Python is particularly
simple, unlike in other languages; the Python:
\begin{lstlisting}
someAges = ages[1:3:]
\end{lstlisting}
while in Java it is
\begin{lstlisting}
ArrayList<Double> temp = \
  new ArrayList<Double>(0);
for (int i_temp = 1; i_temp < 3; \
  i_temp++) {
    temp.add(ages.get(i_temp));
}
someAges = temp;
\end{lstlisting}
where we use backslashes in generated code to indicate manually inserted
line breaks so that the code fits in this paper's narrow column margins.
This demonstrates GOOL's idiomatic code generation, enabled by having the
appropriate high-level information to drive the generation process.

\subsection{Printing}

Printing is another such important feature, which generates quite
different code depending on the target language.  Here again Python
is more ``expressive'' so that printing a list (via
\verb|printLn ages|) generates \verb|print(ages)|, but in other languages
must generate a loop; for example, in Java:
\unsure{Can't better code be generated in Java for this? Not that I want
GOOL changed now, what I mean is perhaps using C++ as an example, where
this is not going to be a feature (I think), the loop is indeed
needed.}
\begin{lstlisting}
System.out.print("[");
for (int list_i1 = 0; \
  list_i1 < ages.size() - 1; \
  list_i1++) {
    System.out.print(ages.get(list_i1));
    System.out.print(", ");
}
if (ages.size() > 0) {
    System.out.print(ages.get( \
      ages.size() - 1));
}
System.out.println("]");
\end{lstlisting}
In addition to printing, there is also functionality for reading input.

\subsection{Procedures with input, output and input-output parameters}

Moving to larger-scale patterns, we noticed that our codes had methods that
used its parameters differently: some were used as inputs, some as outputs
and some for both purposes.  This was a \emph{semantic} pattern that was
not necessarily obvious in any of the implementations. But once we noticed it,
we could use that information to generate better, more idiomatic code in
each language, while still capturing the higher-level semantics of the
functionality we were trying to implement.  More concretely, consider a
function \verb|applyDiscount| that takes a price and a discount, subtracts the
discount from the price, and returns both the new price and a Boolean for
whether the price is below $20$. In GOOL, using \verb|inOutFunc|, assuming
all variables mentioned have been defined:
\begin{lstlisting}
inOutFunc "applyDiscount" public static_
  [discount] [isAffordable] [price]
  (bodyStatements [
    price &-= valueOf discount,
    isAffordable &=
      valueOf price ?< litFloat 20.0])
\end{lstlisting}
\verb|inOutFunc| takes three lists of parameters, the input, output and
input-output respectively.  This function has two outputs
---\verb|price| and \verb|isAffordable|--- and multiple outputs are
not directly supports in al target languages.  Thus we need to use
different features to represent these.  For example, in Python,
return statement with multiple values is used:
\begin{lstlisting}
def applyDiscount(price, discount):
    price = price - discount
    isAffordable = price < 20

    return price, isAffordable
\end{lstlisting}
In Java, the outputs are returned in an array of \verb|Object|s:
\begin{lstlisting}
public static Object[] applyDiscount( \
  int price, int discount) \
  throws Exception {
    Boolean isAffordable;

    price = price - discount;
    isAffordable = price < 20;

    Object[] outputs = new Object[2];
    outputs[0] = price;
    outputs[1] = isAffordable;
    return outputs;
  }
}
\end{lstlisting}
In \Csharp, the outputs are passed as parameters, using the \verb|out| keyword if
it is only an output or the \verb|ref| keyword if it is both an input and an
output:
\begin{lstlisting}
public static void applyDiscount( \
  ref int price, int discount, \
  out Boolean isAffordable) {
    price = price - discount;
    isAffordable = price < 20;
}
\end{lstlisting}
And in \Cplusplus, the outputs are passed as pointer parameters:
\begin{lstlisting}
void applyDiscount(int &price, \
  int discount, bool &isAffordable) {
    price = price - discount;
    isAffordable = price < 20;
}
\end{lstlisting}
Here again we see how a natural task-level ``feature'', namely the
desire to have different kinds of parameters, end up being rendered differently,
but hopefully idiomatically, in each target language.  GOOL manages the
tedious aspects of generated any needed variable declarations and return
statements.  To call an \verb|inOutFunc| function, one must use
\verb|inOutCall| so that GOOL can ``line up'' all the pieces properly.

\subsection{Getters and setters}

Getters and setters are a mainstay of OO programming.  Whether these achieve
encapsulation or not, it is certainly the case that saying to an OO programmer
``variable \verb|foo| from class \verb|FooClass| should have getters and setters''
is enough information for them to write the code. And so it is in GOOL as well.
Saying \verb|getMethod "FooClass" foo| and \verb|setMethod "FooClass" foo|. 
The generated set method in Python, Java, \Csharp and \Cplusplus are:
\begin{lstlisting}
def setFoo(self, foo):
    self.foo = foo
\end{lstlisting}

\begin{lstlisting}
public void setFoo(int foo) \
  throws Exception {
    this.foo = foo;
  }
}
\end{lstlisting}

\begin{lstlisting}
public void setFoo(int foo) {
    this.foo = foo;
}
\end{lstlisting}

\begin{lstlisting}
void FooClass::setFoo(int foo) {
    this->foo = foo;
}
\end{lstlisting}
The point is that the conceptually simple ``set method'' contains a number
of idiosyncracies in each target language. These details are irrelevant for
the task at hand, and this tedium can be automated. As before, there are
specific means of calling these functions, \verb|get| and \verb|set|.

\subsection{Design Patterns}
Finally we get to the design patterns of ~\cite{gamma1995design}. GOOL
currently handles three design patterns: Observer,
State, and Strategy. 

For Strategy, we draw from partial evaluation, and ensure that the set of
strategies that will effectively be used are statically known at generation
time.  This way we can ensure to only generate code for those that will
actually be used.  \verb|runStrategy| is the user-facing function; it needs the
name of the strategy to use, a list of pairs of strategy names and bodies, and
an optional variable and value to assign to upon termination of the strategy.

For Observer , \verb|initObserverList| generates an observer for a list.  More
specifically, given a list of (initial values), it generates a declaration of
an observer list variable, initially containing the given values.
\verb|addObserver| can be used to add a value to the observer list, and
\verb|notifyObservers| will call a method on each of the observers. Currently,
the name of the observer list variable is fixed, so there can only be one
observer list in a given scope.

The State pattern is here specialized to implement \emph{Finite State Machines}
with fairly general transition functions.  Transitions happen on checking, not
on changing the state.  \verb|initState| takes a name and a state label and
generate a declaration of a variable with the given name and initial state.
\verb|changeState| changes the state of the variable to a new state.
\verb|checkState| is more complex.  It takes the name of the state variable, a
list of value-body pairs, and a fallback body; and it generates a conditional
(usually a switch statement) that checks the state and runs the corresponding
body, or the fallback body if none of the states match.

Of course the design patterns could already have been coded in GOOL, but
having these as language features is useful for two reasons: 1) the GOOL-level
code is clearer in its intent (and more concise), and 2) the resulting code
can be more idiomatic.

\section{Related Work} \label{sec:related}

We divide the Related Work into the following categories
\begin{itemize}
\item General-purpose code generation
\item Multi-language OO code generation
\item Design pattern modeling and code generation
\end{itemize}
which we present in turn.

\subsection{General-purpose code generation}

\textbf{Haxe}~\cite{Haxe} is a general-purpose multi-paradigm language and cross-platform
compiler.  It compiles to all of the languages GOOL does, and many
others.  However, it is designed as a more traditional programming language, and
thus does not offer the high-level abstractions GOOL that provides. Furthermore
Haxe strips comments and generates source code around a custom framework; 
the effort of learning this framework and the lack of comments makes the generated
code not particularly readable.

\textbf{Protokit}~\cite{kovesdan2017multi} is a DSL and code generator for Java and
\Cplusplus, where the generator is designed to produce
general-purpose imperative or object-oriented code. The Protokit generator is
model-driven and uses a final ``output model'' from which actual code can be
generated. Since the ``output model'' is quite similar to the generated
code, it presented challenges with regards to semantic, conventional, and
library-related differences between the target language
\cite{kovesdan2017multi}. GOOL's finally-tagless approach and syntax for
high-level tasks, on the other hand, helped it overcome differences between
target languages.

\textbf{ThingML}~\cite{harrand2016thingml} is a DSL for model-driven engineering
targeting C, \Cplusplus, Java, and JavaScript. It is specialized to deal with
distributed reactive systems (a nevertheless broad range of application domains).
This means that this not quite a general-purpose DSL, unlike GOOL.
ThingML's modelling-related syntax and abstractions stand in contrast to GOOL's
object-oriented syntax and abstractions. The generated code lacks some of the
pretty-printing provided by GOOL, specifically indentation, which detracts from
readability.

\subsection{Object-oriented generators}

\info{here}
Moving on to OO-specific code generators with multiple target languages, there
are many examples of such DSLs, but for more restricted domains than GOOL.
Google protocol buffers is a DSL for serializing structured data, which can
then be compiled into Java, Python, Objective C, and \Cplusplus \cite{Protobuf}.
Thrift is a Facebook-developed tool for generating code in multiple languages
and even multiple paradigms based on language-neutral descriptions of data
types and interfaces \cite{slee2007thrift}. Clearwater is an approach for
implementing DSLs with multiple target languages for components of distributed
systems \cite{swint2005clearwater}. The Time Weaver tool uses a multi-language
code generator to generate ``glue'' code for real-time embedded systems
\cite{de2004glue}. The domain of mobile applications is host to a bevy of DSLs
with multiple target languages, of which MobDSL \cite{kramer2010mobdsl} and
XIS-Mobile \cite{ribeiro2014xis} are two examples. Conjure is a DSL for
generating APIs. It reads YML descriptions of APIs and can generate code in
Java, TypeScript, Python, and Rust \cite{Conjure} (include this?). All of these
are examples of multi-language code generation, but none of them generate
general-purpose code like GOOL does.

A number of languages for modeling design patterns have been developed. The
Design Pattern Modeling Language (DPML) \cite{mapelsden2002design} is similar
to the Unified Modeling Language (UML) but designed specifically to overcome
UML's shortcomings to be able to model all design patterns. DPML consists of
both specification diagrams and instance diagrams for instantiations of design
patterns, but does not attempt to generate actual source code from the models.
The Role-Based Metamodeling Language \cite{kim2003uml} is also based on UML but
with changes to allow for better models of design patterns, with specifications
for the structure, interactions, and state-based behaviour in patterns. Again,
source code generation is not attempted. Another metamodel for design patterns
includes generation of Java code \cite{albin2001meta}, and IBM developed a DSL
for generation of OO code based on design patterns
\cite{budinsky1996automatic}. IBM's DSL was in the form of a visual user
interface rather than a programming or modeling language. The languages that
generate code do so only for design patterns, not for any general-purpose code
like GOOL does.

\section{Future Work} \label{sec:future}

Currently GOOL code is typed based on what kind of code it represents:
variable, value, type, or method, for example. Code that represents a variable
or value is not further typed based on what the type of the variable or value
would be in a traditional programming language: Boolean or integer, for
example. There is nothing to stop a user from passing a non-list value to a
function specifically intended for lists, like \verb|listSize|, or from passing
string values to numeric operators like \verb|#+|. We plan on adding this
additional layer of typing, making GOOL a statically typed programming
language. We have started work to this end by making the underlying types for
GOOL's \verb|Variable|s and \verb|Value|s Generalized Algebraic Data Types
(GADTs), such as this one for \verb|Variable|s:
\begin{lstlisting}
data TypedVar a where
  BVr :: VarData -> TypedVar Boolean
  IVr :: VarData -> TypedVar Integer
  ...
\end{lstlisting}
Members of the type family \verb|Variable| \verb|repr| would now map to
\verb|TypedVar| and the type for a \verb|Variable| in the generic \verb|repr|
context would now be something like \verb|repr| \verb|(Variable repr Boolean)|
instead of just \verb|repr| \verb|(Variable repr)|. All instances of
\verb|TypedVar| are built from the same \verb|VarData| structure, but variables
built with the different GADT constructors will have different types and
Haskell's compiler will throw errors if a wrongly-typed variable is passed to a
function.

There are many improvements we plan on making to the generated code, especially
with regards to not generating code that is not necessary. For example, we
currently always generate import statements for certain libraries, like math
and IO-related libraries, but we'd instead like to detect when a library is
used and only import those that are actually used. We plan on using Haskell's
\verb|State| monad to have a list of libraries that gets updated whenever a new
library is used, and can then be read by the \verb|buildModule| function to
generate only the necessary imports. This technique of using the \verb|State|
monad will be useful for other improvements as well, such as to only generate
\verb|throws Exception| in the header of a Java method that actually may throw
an exception, rather than in every method as is done currently.

We plan on adding syntax to interface with more kinds of external libraries,
similar to how we currently interface with math libraries with functions like
\verb|sin|. We would like to be able to interface with libraries for solving
Ordinary Differential Equations (ODEs), for instance, with different functions
available for using different ODE-solving algorithms. Since interfacing with
ODE-solving libraries can have major impact on the structure of a program,
implementing these high-level functions may require having multiple passes over
a GOOL program: an initial pass to collect information, such as the information
that an ODE library is used, and then a second pass to generate the code.

Currently GOOL forces certain design decisions on the user, but we plan on
providing configuration options to give the user more control over the code
they generate. For example, GOOL uses \verb|ArrayList|s to represent lists in
Java, but there are many other list implementations that could be used, such as
\verb|Vector|. This is something we would like the user to have control over.
Another example, building on the aforementioned plan to provide functions to
interface with more external libraries, is to allow the user to choose which
specific external library they want to be used. These kinds of choices could be
accomplished by having the user pass a configuration (such as a Haskell record)
to GOOL, which GOOL will then read to decide how the code should be generated.

GOOL's current set of high-level functions for generating common OO patterns is
by no means exhaustive. We plan to continue to identify patterns for which GOOL
can provide abstractions, and to implement those as more high-level functions,
with the aim of writing OO programs in GOOL as efficiently as possible without
losing expressivity.

\section{Conclusion} \label{sec:conclusions}

OO programming languages are similar enough that it is feasible to have a
single OO language that can be compiled to any other OO language. GOOL is a DSL
with multiple target languages for the domain of OO GPLs. Like most DSLs, GOOL
provides abstractions suited to its domain, in this case abstractions of common
OO patterns. Unlike most DSLs, GOOL considers the code it generates to be a
product for human consumption in addition to computer consumption, so it
focuses on generating idiomatic, human-readable, and documented code. The goal
of generating idiomatic code is helped by the abstractions: GOOL provides
syntax for expressing OO patterns naturally and efficiently, and these
high-level functions afford each target language renderer the freedom to
generate code following its own idioms. The goal of generating documented code
is realized by GOOL's syntax for generating informal documentation in the form
of code comments, or more formal documentation in the form of Doxygen-style
structured comments for functions, classes, and modules. GOOL can even generate
Doxygen configuration files and makefiles to facilitate compiling the
documentation into PDF or HTML formats. The generated code is pretty-printed so
that it is readable, and GOOL allows organization of code related to the same
task into blocks to further increase readability. The idiomaticity and presence
of documentation in the generated code also contribute to readability.

%% Acknowledgments
%\begin{acks}                            %% acks environment is optional
                                        %% contents suppressed with 'anonymous'
  %% Commands \grantsponsor{<sponsorID>}{<name>}{<url>} and
  %% \grantnum[<url>]{<sponsorID>}{<number>} should be used to
  %% acknowledge financial support and will be used by metadata
  %% extraction tools.
%\end{acks}


%% Bibliography
\bibliography{References}


%% Appendix
% \appendix
% \section{Appendix}
% 
% Text of appendix \ldots
% 
\end{document}
