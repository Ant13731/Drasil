%% For double-blind review submission, w/o CCS and ACM Reference (max submission space)
\documentclass[sigplan,review,anonymous]{acmart}\settopmatter{printfolios=true,printccs=false,printacmref=false}
%% For double-blind review submission, w/ CCS and ACM Reference
%\documentclass[sigplan,review,anonymous]{acmart}\settopmatter{printfolios=true}
%% For single-blind review submission, w/o CCS and ACM Reference (max submission space)
%\documentclass[sigplan,review]{acmart}\settopmatter{printfolios=true,printccs=false,printacmref=false}
%% For single-blind review submission, w/ CCS and ACM Reference
%\documentclass[sigplan,review]{acmart}\settopmatter{printfolios=true}
%% For final camera-ready submission, w/ required CCS and ACM Reference
%\documentclass[sigplan]{acmart}\settopmatter{}


%% Conference information
%% Supplied to authors by publisher for camera-ready submission;
%% use defaults for review submission.
\acmConference[PL'18]{ACM SIGPLAN Conference on Programming Languages}{January 01--03, 2018}{New York, NY, USA}
\acmYear{2018}
\acmISBN{} % \acmISBN{978-x-xxxx-xxxx-x/YY/MM}
\acmDOI{} % \acmDOI{10.1145/nnnnnnn.nnnnnnn}
\startPage{1}

%% Copyright information
%% Supplied to authors (based on authors' rights management selection;
%% see authors.acm.org) by publisher for camera-ready submission;
%% use 'none' for review submission.
\setcopyright{none}
%\setcopyright{acmcopyright}
%\setcopyright{acmlicensed}
%\setcopyright{rightsretained}
%\copyrightyear{2018}           %% If different from \acmYear

%% Bibliography style
\bibliographystyle{ACM-Reference-Format}
%% Citation style
%\citestyle{acmauthoryear}  %% For author/year citations
%\citestyle{acmnumeric}     %% For numeric citations
%\setcitestyle{nosort}      %% With 'acmnumeric', to disable automatic
                            %% sorting of references within a single citation;
                            %% e.g., \cite{Smith99,Carpenter05,Baker12}
                            %% rendered as [14,5,2] rather than [2,5,14].
%\setcitesyle{nocompress}   %% With 'acmnumeric', to disable automatic
                            %% compression of sequential references within a
                            %% single citation;
                            %% e.g., \cite{Baker12,Baker14,Baker16}
                            %% rendered as [2,3,4] rather than [2-4].


%%%%%%%%%%%%%%%%%%%%%%%%%%%%%%%%%%%%%%%%%%%%%%%%%%%%%%%%%%%%%%%%%%%%%%
%% Note: Authors migrating a paper from traditional SIGPLAN
%% proceedings format to PACMPL format must update the
%% '\documentclass' and topmatter commands above; see
%% 'acmart-pacmpl-template.tex'.
%%%%%%%%%%%%%%%%%%%%%%%%%%%%%%%%%%%%%%%%%%%%%%%%%%%%%%%%%%%%%%%%%%%%%%


%% Some recommended packages.
\usepackage{booktabs}   %% For formal tables:
                        %% http://ctan.org/pkg/booktabs
\usepackage{subcaption} %% For complex figures with subfigures/subcaptions
                        %% http://ctan.org/pkg/subcaption


\begin{document}

%% Title information
\title[Short Title]{GOOL: A Generic Object-Oriented Language}         %% [Short 
%%Title] is optional;
                                        %% when present, will be used in
%                                        %% header instead of Full Title.
%\titlenote{with title note}             %% \titlenote is optional;
%                                        %% can be repeated if necessary;
%                                        %% contents suppressed with 'anonymous'
%\subtitle{Subtitle}                     %% \subtitle is optional
%\subtitlenote{with subtitle note}       %% \subtitlenote is optional;
%                                        %% can be repeated if necessary;
%                                        %% contents suppressed with 'anonymous'


%% Author information
%% Contents and number of authors suppressed with 'anonymous'.
%% Each author should be introduced by \author, followed by
%% \authornote (optional), \orcid (optional), \affiliation, and
%% \email.
%% An author may have multiple affiliations and/or emails; repeat the
%% appropriate command.
%% Many elements are not rendered, but should be provided for metadata
%% extraction tools.

%% Author with single affiliation.
\author{Brooks MacLachlan}
%\authornote{with author1 note}          %% \authornote is optional;
                                        %% can be repeated if necessary
%\orcid{nnnn-nnnn-nnnn-nnnn}             %% \orcid is optional
\affiliation{
  \position{Position1}
  \department{Department of Computing and Software}              %% \department 
  %%is recommended
  \institution{McMaster University}            %% \institution is required
  \streetaddress{1280 Main Street West}
  \city{Hamilton}
  \state{Ontario}
  \postcode{L8S 4L8}
  \country{Canada}                    %% \country is recommended
}
\email{maclachb@mcmaster.ca}          %% \email is recommended

\author{Jacques Carette}
%\authornote{with author1 note}          %% \authornote is optional;
%% can be repeated if necessary
%\orcid{nnnn-nnnn-nnnn-nnnn}             %% \orcid is optional
\affiliation{
	\position{Position1}
	\department{Department of Computing and Software}              %% 
	%%\department is recommended
	\institution{McMaster University}            %% \institution is required
	\streetaddress{1280 Main Street West}
	\city{Hamilton}
	\state{Ontario}
	\postcode{L8S 4L8}
	\country{Canada}                    %% \country is recommended
}
\email{carette@mcmaster.ca}          %% \email is recommended

\author{Spencer Smith}
%\authornote{with author1 note}          %% \authornote is optional;
%% can be repeated if necessary
%\orcid{nnnn-nnnn-nnnn-nnnn}             %% \orcid is optional
\affiliation{
	\position{Position1}
	\department{Department of Computing and Software}              %% 
	%%\department is recommended
	\institution{McMaster University}            %% \institution is required
	\streetaddress{1280 Main Street West}
	\city{Hamilton}
	\state{Ontario}
	\postcode{L8S 4L8}
	\country{Canada}                    %% \country is recommended
}
\email{smiths@mcmaster.ca}          %% \email is recommended


%% Abstract
%% Note: \begin{abstract}...\end{abstract} environment must come
%% before \maketitle command
\begin{abstract}
Text of abstract \ldots.
\end{abstract}


%% 2012 ACM Computing Classification System (CSS) concepts
%% Generate at 'http://dl.acm.org/ccs/ccs.cfm'.
\begin{CCSXML}
<ccs2012>
<concept>
<concept_id>10011007.10011006.10011008</concept_id>
<concept_desc>Software and its engineering~General programming languages</concept_desc>
<concept_significance>500</concept_significance>
</concept>
<concept>
<concept_id>10003456.10003457.10003521.10003525</concept_id>
<concept_desc>Social and professional topics~History of programming languages</concept_desc>
<concept_significance>300</concept_significance>
</concept>
</ccs2012>
\end{CCSXML}

\ccsdesc[500]{Software and its engineering~General programming languages}
\ccsdesc[300]{Social and professional topics~History of programming languages}
%% End of generated code


%% Keywords
%% comma separated list
\keywords{keyword1, keyword2, keyword3}  %% \keywords are mandatory in final camera-ready submission


%% \maketitle
%% Note: \maketitle command must come after title commands, author
%% commands, abstract environment, Computing Classification System
%% environment and commands, and keywords command.
\maketitle


\section{Introduction}

Given a task, before writing any code a programmer must select a programming 
language to use. Whatever they may base their choice upon, almost any
programming language will work. While a program may be more difficult to
express in one language over another, it should at least be possible to write
the program in either language. Just as the same sentence can be translated to
any spoken language, the same program can be written in any programming
language. Though they will accomplish the same tasks semantically, the
expressions of a program in different programming languages can appear
substantially different due to the unique syntax of each language. Within a
single programming language paradigm, such as object-oriented (OO) programming,
these differences should not be so extreme. OO programs, no matter the
language, share certain structural properties. They are built from variables,
methods, classes, and objects. Some OO languages even have very similar syntax.
But however similar they may be, no two programming languages are identical.

If a programmer wishes to write a program that will integrate into 
existing systems written in different languages, they will likely need to 
write a different version of the program for each. This 
requires investing the time to learn the idiosyncrasies of each language and 
give attention to the operational details where languages differ. Repeatedly 
writing the same program in different languages is entirely inefficient. 
Languages in the same paradigm have many similarities, and there is an 
excellent opportunity to take advantage of these similarities to improve the
efficiency of writing code. If a program could be written in one language and
automatically translated to any other language in the same paradigm, this would
greatly facilitate program reuse. Directly translating between existing 
OO languages will not always be possible because some languages 
require more information than others. A dynamically typed language like Python, 
for instance, cannot be straightforwardly translated to a statically typed 
language like Java, because additional type information would need to be 
provided. But if there was a language that contained all of the information 
that any of the other OO languages would need, it could be used as the source 
language for translation. This source language should also be completely 
language-agnostic, free of any of the idiosyncratic ``noise'' required by
specific languages.

The similarities between OO programs do not end with syntax and structural 
components. Additionally, there are tasks and patterns commonly performed by OO
programs in any language, from simple tasks like splitting a string or 
patterns like defining functions on inputs to produce outputs, to higher-level 
design patterns like those described in \cite{DesignPatterns}. A language 
that provided abstractions for these tasks and patterns would make the process 
of writing OO code even easier.

A Domain-Specific Language (DSL) is a high-level programming language with 
syntax tailored to a specific domain \cite{DSLs}. DSLs allow domain experts 
to write code without having to concern themselves with the syntactical and
operational requirements of general-purpose programming languages. A DSL
abstracts over the details of the code, providing notation for a user to
specify domain-specific knowledge in a natural manner. DSL code is typically
compiled to a more traditional target language. Abstracting over code details
and compiling into traditional OO languages is exactly what we want our source
OO language to do. The code details to abstract over in this case include both
the operational details of using a specific language as well as the higher-level
patterns that commonly show up in OO programs. So the source language we are
looking for is just a DSL in the domain of OO programming languages!

We have developed a Generic Object-Oriented Language (GOOL), proving that such 
a language indeed exists. GOOL is a DSL embedded in Haskell that can currently
generate OO code in Python, Java, C\#, and C++. Theoretically, any OO language 
could be added as a target language for GOOL. This paper presents GOOL, 
starting with the syntax of the language in Section \ref{syntax}. Section 
\ref{implementation} describes how GOOL is implemented. GOOL provides some 
higher-level functions for convenient generation of code following commonly 
used patterns, examples of which are presented in Section \ref{patterns}. We 
close with a discussion of related work in Section \ref{related}, plans for 
future improvements in Section \ref{future}, and conclusions in Section 
\ref{conclusions}.

\section{GOOL Syntax} \label{syntax}
\section{GOOL Implementation} \label{implementation}
\section{Higher-level GOOL functions} \label{patterns}
\section{Related Work} \label{related}
\section{Future Work} \label{future}
\section{Conclusion} \label{conclusions}


%% Acknowledgments
\begin{acks}                            %% acks environment is optional
                                        %% contents suppressed with 'anonymous'
  %% Commands \grantsponsor{<sponsorID>}{<name>}{<url>} and
  %% \grantnum[<url>]{<sponsorID>}{<number>} should be used to
  %% acknowledge financial support and will be used by metadata
  %% extraction tools.
  This material is based upon work supported by the
  \grantsponsor{GS100000001}{National Science
    Foundation}{http://dx.doi.org/10.13039/100000001} under Grant
  No.~\grantnum{GS100000001}{nnnnnnn} and Grant
  No.~\grantnum{GS100000001}{mmmmmmm}.  Any opinions, findings, and
  conclusions or recommendations expressed in this material are those
  of the author and do not necessarily reflect the views of the
  National Science Foundation.
\end{acks}


%% Bibliography
\bibliography{References}


%% Appendix
\appendix
\section{Appendix}

Text of appendix \ldots

\end{document}
