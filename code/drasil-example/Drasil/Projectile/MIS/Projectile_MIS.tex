\documentclass[12pt, titlepage]{article}

\usepackage{amsmath, mathtools}

\usepackage[round]{natbib}
\usepackage{amsfonts}
\usepackage{amssymb}
\usepackage{graphicx}
\usepackage{colortbl}
\usepackage{xr}
\usepackage{hyperref}
\usepackage{longtable}
\usepackage{xfrac}
\usepackage{tabularx}
\usepackage{float}
\usepackage{siunitx}
\usepackage{booktabs}
\usepackage{multirow}
\usepackage[section]{placeins}
\usepackage{caption}
\usepackage{fullpage}
\usepackage[nottoc, numbib]{tocbibind}

\hypersetup{
bookmarks=true,     % show bookmarks bar?
colorlinks=true,       % false: boxed links; true: colored links
linkcolor=red,          % color of internal links (change box color with linkbordercolor)
citecolor=blue,      % color of links to bibliography
filecolor=magenta,  % color of file links
urlcolor=cyan          % color of external links
}

\usepackage{array}

%% Comments
\newif\ifcomments\commentstrue

\ifcomments
\newcommand{\authornote}[3]{\textcolor{#1}{[#3 ---#2]}}
\newcommand{\todo}[1]{\textcolor{orange}{[TODO: #1]}}
\else
\newcommand{\authornote}[3]{}
\newcommand{\todo}[1]{}
\fi

\newcommand{\wss}[1]{\authornote{blue}{SS}{#1}}
\newcommand{\bmac}[1]{\authornote{red}{BM}{#1}}
\newcommand{\sam}[1]{\authornote{magenta}{SC}{#1}}

\newcommand{\progname}{Projectile}

\begin{document}

\title{Module Interface Specification for Projectile}

\author{Samuel J. Crawford}

\date{\today}

\maketitle

\pagenumbering{roman}

\newpage

\tableofcontents

\newpage

\section{Symbols, Abbreviations and Acronyms}

See SRS Documentation at \url{https://jacquescarette.github.io/Drasil/examples/Projectile/srs/Projectile_SRS.pdf}

\pagenumbering{arabic}

\section{Introduction}

The following document details the Module Interface Specifications for the
implemented modules in a program simulating projectile motion. It is intended to 
ease navigation through the program for design and maintenance purposes.

Complementary documents include the System Requirement Specifications
and Module Guide.
%The full documentation and implementation can be
%found at \sam{No manual version of Projectile? Should this link be removed or  
%should there be a link?}.

% The specification is given in terms of functions, rather than sequences.  For
% instance, the predicted temperature of the water is given as a function of time
% ($\mathbb{R} \rightarrow \mathbb{R})$, not as a sequence ($\mathbb{R}^n$). This
% approach is more straightforward for the specification, but in the
% implementation stage, it will likely be necessary to introduce a
% sequence, assuming that a numerical solver is used for the system of ODEs.

\section{Notation}

The structure of the MIS for modules comes from \citet{HoffmanAndStrooper1995},
with the addition that template modules have been adapted from
\cite{GhezziEtAl2003}.  The mathematical notation comes from Chapter 3 of
\citet{HoffmanAndStrooper1995}.  For instance, the symbol := is used for a
multiple assignment statement and conditional rules follow the form $(c_1
\Rightarrow r_1 | c_2 \Rightarrow r_2 | ... | c_n \Rightarrow r_n )$.

\progname\ uses strings and real numbers. For information on these data types 
(and others), see \cite{GhezziEtAl2003}, \cite{GriesAndSchneider1993}, and 
\cite{HoffmanAndStrooper1995}. In addition, \progname\ uses functions, which are
defined by the data types of their inputs and outputs. Local functions are
described by giving their type signature followed by their specification.

\newpage
\section{Module Decomposition}

The following table is taken directly from the Module Guide document for this project.

\begin{table}[h!]
\centering
\begin{tabular}{p{0.3\textwidth} p{0.6\textwidth}}
\toprule
\textbf{Level 1} & \textbf{Level 2}\\
\midrule

{Hardware-Hiding} & ~ \\
\midrule

\multirow{5}{0.3\textwidth}{Behaviour-Hiding} & Input Parameters\\
& Calculations\\
& Output Format\\
% & Output Verification\\
& Control Module\\
& Specification Parameters Module\\
\midrule

\multirow{1}{0.3\textwidth}{Software Decision} & Sequence Data Structure\\
% & Plotting\\
\bottomrule

\end{tabular}
\caption{Module Hierarchy}
\label{TblMH}
\end{table}

\newpage

\section{MIS of Control Module} \label{Main}

\subsection{Module}

main

\subsection{Uses}

Param (Section~\ref{Parameters}), Calculations (Section~\ref{Calc}),
% verify\_output (Section~\ref{VerifyOutput}), plot(Section~\ref{Plot}),
Output (Section~\ref{Output})

\subsection{Syntax}

\subsubsection{Exported Access Programs}

\begin{center}
\begin{tabular}{p{2cm} p{4cm} p{4cm} p{2cm}}
\hline
\textbf{Name} & \textbf{In} & \textbf{Out} & \textbf{Exceptions} \\
\hline
main & - & - & - \\
\hline
\end{tabular}
\end{center}

\subsection{Semantics}

\subsubsection{State Variables}

None

\subsubsection{Access Routine Semantics}

\noindent main():
\begin{itemize}
\item transition: Modify the state of Param module and the environment variables
  for the Output module by following these steps\\
\end{itemize}

\noindent Get (filenameIn: string) from user.\\
\noindent get\_input(filenameIn, Param)\\
\noindent $p_{\text{land}}$ := func\_p\_land(Param)\\
\noindent $d_{\text{offset}}$ := func\_d\_offset(Param, $p_{\text{land}}$)\\
\noindent $s$ := func\_s(Param, $d_{\text{offset}}$)\\
\noindent \#\textit{Output calculated values to a file.}\\
% \noindent verify\_output($T_w$, $T_p$, $E_w$, $E_p$, $t_\text{final}$)\\
% \noindent plot($T_w$, $T_p$, $E_w$, $E_p$, $t_\text{final}$)\\
\noindent write\_output($s$, $d_{\text{offset}}$)\\

\newpage

\section{MIS of Input Parameters Module} \label{Parameters}

The secrets of this module are the data structure for input parameters and how 
the values are input and verified.

\subsection{Module}

Param

\subsection{Uses}

SpecParam (Section~\ref{SpecParam})

\subsection{Syntax}

\begin{tabular}{p{3cm} p{2.5cm} p{1cm} >{\raggedright\arraybackslash}p{7.5cm}}
\toprule
\textbf{Name} & \textbf{In} & \textbf{Out} & \textbf{Exceptions} \\
\midrule
get\_input & string, Param & - &  FileError \\
input\_constraints & Param & - & badSpeed, badAngle, badTargetPosition\\
$v_{\text{launch}}$ & - & $\mathbb{R}$ & -\\
$\theta$ & - & $\mathbb{R}$ & -\\
$p_{\text{target}}$ & - & $\mathbb{R}$ & -\\
\bottomrule
\end{tabular}

\subsection{Semantics}

\subsubsection{Environment Variables}

inputFile: sequence of string \#\textit{f[i] is the ith string in the text file f}

\subsubsection{State Variables}

\# To Support IM1 and IM2 \\
$v_{\text{launch}}$: $\mathbb{R}$\\
$\theta$: $\mathbb{R}$\\
\# To Support IM3 and IM4 \\
$p_{\text{target}}$: $\mathbb{R}$\\

\subsubsection{Assumptions}

\begin{itemize}

\item get\_input will be called before the values of any state variables will 
be accessed or verified.

\item The file contains the string equivalents of the numeric values for
each input parameter in order, each on a new line. Any comments in the input file 
should be denoted with a `\#' symbol.

\end{itemize}

\subsubsection{Access Routine Semantics}

\noindent get\_input(filename, Param):
\begin{itemize}
\item transition: The file name ``filename" is first associated with the file $f$. 
The following procedural specification is followed:
\begin{enumerate}
\item Read data sequentially from $f$ to populate the state variables from
FR1 ($v_{\text{launch}}$, $\theta$, and $p_{\text{target}}$) in Param.
\item input\_constraints()
\end{enumerate}

\item exception: exc := a file name ``filename" cannot be found OR the format of
inputFile is incorrect $\Rightarrow$  FileError
\end{itemize}

\noindent input\_constraints():
\begin{itemize}
\item output: \textit{out} := none
\item exception: exc := ( \\
$\neg (0 < v_{\text{launch}}) \Rightarrow$ badSpeed $|$\\
$\neg (0 < \theta < \frac{\pi}{2}) \Rightarrow$ badAngle $|$\\
$\neg (0 < p_{\text{target}}) \Rightarrow$ badTargetPosition)
\end{itemize}
 
\noindent Param.$v_{\text{launch}}$:
\begin{itemize}
\item output: \textit{out} := $v_{\text{launch}}$
\item exception: none
\end{itemize}

\noindent Param.$\theta$:
\begin{itemize}
\item output: \textit{out} := $\theta$
\item exception: none
\end{itemize}

\noindent Param.$p_{\text{target}}$:
\begin{itemize}
\item output: \textit{out} := $p_{\text{target}}$
\item exception: none
\end{itemize}

\subsection{Considerations}

The value of each state variable can be accessed through its name (getter).  An
access program is available for each state variable.  There are no setters for
the state variables, since the values will be set by get\_input and
not changed for the life of the program.

See Appendix (Section~\ref{Appendix}) for the complete list of exceptions and
 associated error messages.

\newpage

\section{MIS of Calculations Module} \label{Calc}

The secret of this module is how the required values are calculated.

\subsection{Module}

Calculations

\subsection{Uses}

Param (Section~\ref{Parameters}), SpecParam (Section~\ref{SpecParam})

\subsection{Syntax}

\subsubsection{Exported Access Programs}

\begin{center}
 \begin{tabular}{p{3cm} p{3cm} p{1cm} p{8cm}}
 \hline
 \textbf{Name} & \textbf{In} & \textbf{Out} & \textbf{Exceptions} \\
 \hline
 func\_p\_land & Param & $\mathbb{R}$ & None \\
 \hline
 func\_d\_offset & Param, $\mathbb{R}$ & $\mathbb{R}$ & None \\
 \hline
 func\_s & Param, $\mathbb{R}$ & String & None \\
 \hline
 \end{tabular}
 \end{center}

\subsection{Semantics}

\subsubsection{Assumptions}

All of the fields Param have been assigned values before any of the access
 routines for this module are called.

\subsubsection{Access Routine Semantics}

func\_p\_land(Param):
 \begin{itemize}
 \item out: $\textit{out} := \frac{2 * v_{\text{launch}}^2 * sin(\theta) * cos(\theta)}{g}$
 \item exception: exc := none
 \end{itemize}
 
func\_d\_offset(Param, $p_{\text{land}}$):
 \begin{itemize}
 \item out: $\textit{out} := p_{\text{land}} - p_{\text{target}}$
 \item exception: exc := none
 \end{itemize}
 
func\_s(Param, $d_{\text{offset}}$):
 \begin{itemize}
 \item out: $\textit{out} := $ ( \\
$|\frac{d_{\text{offset}}}{p_{\text{target}}}| < \epsilon \Rightarrow$ ``The target was hit." $|$\\
$d_{\text{offset}} < 0 \Rightarrow$ ``The projectile fell short."$|$\\
$\text{True} \Rightarrow$ ``The projectile went long.")
 \item exception: exc := none
 \end{itemize}

%\newpage
%\section{MIS of Output Verification Module} \label{VerifyOutput}
%
%\subsection{Module}
%
%verify\_output
%
%\subsection{Uses}
%
%Param (Section~\ref{Parameters})
%
%\subsection{Syntax}
%
%\subsubsection{Exported Constant}
%
%ADMIS\_ER = $1 \times 10^{-6}$
%
%\subsubsection{Exported Access Programs}
%
%\begin{center}
%\begin{tabular}{p{3cm} p{7cm} p{1cm} p{2cm}}
%\hline
%\textbf{Name} & \textbf{In} & \textbf{Out} & \textbf{Exceptions} \\
%\hline
%verify\_output & $T_W(t):\mathbb{R} \rightarrow \mathbb{R},
%                 T_P(t):\mathbb{R} \rightarrow \mathbb{R},
%                 E_W(t):\mathbb{R} \rightarrow \mathbb{R},
%                 E_P(t):\mathbb{R} \rightarrow \mathbb{R},
%                 t_\text{final}: \mathbb{R}$ & - & EWAT\_NOT\_CONSERVE, EPCM\_NOT\_CONSERVE \\
%\hline
%\end{tabular}
%\end{center}
%
%\subsection{Semantics}
%
%\subsubsection{State Variables}
%
%None
%
%\subsubsection{Assumptions}
%
%All of the fields of the input parameters structure have been assigned a
%value.  
%
%\subsubsection{Access Routine Semantics}
%
%\noindent verify\_output($T_W, T_P, E_W, E_P$, $t_\text{final}$):
%\begin{itemize}
%\item exception: exc := (
%\end{itemize}
%
%\noindent $
%(\forall t | 0 \leq t \leq t_\text{final} : \text{relErr}(E_W,
%\int_{0}^{t} h_C A_C (T_C - T_W(t)) dt - \int_{0}^{t} h_P A_P (T_W(t)
%- T_P(t)) dt) < \text{ADMIS\_ER}) \Rightarrow \text{EWAT\_NOT\_CONSERVE}
%$
%
%$|$
%
%\noindent $ 
%(\forall t | 0 \leq t \leq t_\text{final} : \text{relErr}(E_{P}, \int_{0}^{t}
%h_{P} A_{P} (T_{W}(t) - T_{P}(t)) dt) < \text{ADMIS\_ER}) \Rightarrow
%\text{EPCM\_NOT\_CONSERVE} 
%$
%)
%
%\subsubsection{Local Functions}
%
%relErr: $\mathbb{R}$ $\times$ $\mathbb{R}$ $\rightarrow$ $\mathbb{R}$ \\
%$\text{relErr}(t, e) \equiv \frac{|t - e|}{|t|}$ \\
%\newline

%
%\newpage
%\section{MIS of Plotting Module} \label{Plot}
%
%\subsection{Module}
%
%plot
%
%\subsection{Uses}
%
%N/A
%
%\subsection{Syntax}
%
%\subsubsection{Exported Access Programs}
%
%\begin{center}
%\begin{tabular}{p{2cm} p{8cm} p{2cm} p{2cm}}
%\hline
%\textbf{Name} & \textbf{In} & \textbf{Out} & \textbf{Exceptions} \\
%\hline
%plot & $T_W(t):\mathbb{R} \rightarrow \mathbb{R},
%                 T_P(t):\mathbb{R} \rightarrow \mathbb{R},
%                 E_W(t):\mathbb{R} \rightarrow \mathbb{R},
%       E_P(t):\mathbb{R} \rightarrow \mathbb{R}$, $t_\text{final}: \mathbb{R}$ & - & - \\
%\hline
%\end{tabular}
%\end{center}
%
%\subsection{Semantics}
%
%\subsubsection{State Variables}
%
%None
%
%\subsubsection{Environment Variables}
%
%win: 2D sequence of pixels displayed on the screen\\
%
%\subsubsection{Assumptions}
%
%None
%
%\subsubsection{Access Routine Semantics}
%
%\noindent plot($T_w$, $T_p$, $E_w$, $E_p$, $t_\text{final}$):
%\begin{itemize}
%\item transition: Modify win to display a plot where the vertical axis
%  is time and one horizontal axis is temperature and the other
%  horizontal axis is energy.  The time should run from $0$ to $t_\text{final}$
%\item exception: none
%\end{itemize}

\newpage
\section{MIS of Output Module} \label{Output}

\subsection{Module}

Output

\subsection{Uses}

Param (Section~\ref{Parameters})

\subsection{Syntax}

\subsubsection{Exported Access Program}

\begin{center}
\begin{tabular}{p{3cm} p{7cm} p{2cm} p{2cm}}
\hline
\textbf{Name} & \textbf{In} & \textbf{Out} & \textbf{Exceptions} \\
\hline
write\_output & String, String, $\mathbb{R}$ & - & - \\
\hline
\end{tabular}
\end{center}

\subsection{Semantics}

\subsubsection{State Variables}

None

\subsubsection{Environment Variables}

file: The file named ``output".

\subsubsection{Access Routine Semantics}

\noindent write\_output($s$, $d_{\text{offset}}$):
\begin{itemize}
\item transition:  Write to environment variable named ``file" the calculated values $s$ and $d_{\text{offset}}$.
\item exception: none
\end{itemize}

\newpage

\section{MIS of Specification Parameters} \label{SpecParam}

The secrets of this module is the value of the specification parameters.

\subsection{Module}

SpecParam

\subsection{Uses}

N/A

\subsection{Syntax}

\subsubsection{Exported Constants}

\renewcommand{\arraystretch}{1.2}
\begin{longtable*}[l]{l} 
\# From Table 10 in SRS\\
  $g$ := 9.8\\
  $\epsilon$ := 0.02\\
\end{longtable*}

\subsection{Semantics}

N/A

\newpage

\bibliographystyle {plainnat}
\bibliography {MIS}

\section{Appendix} \label{Appendix}

\renewcommand{\arraystretch}{1.2}

\begin{longtable}{l p{12cm}}
\toprule
\textbf{Message ID} & \textbf{Error Message} \\
\midrule
badSpeed & InputError: Speed must be positive.\\
badAngle & InputError: Angle must be between zero and pi over two radians.\\
badTargetPosition & InputError: Target position must be positive.\\
\bottomrule
\caption{Possible Exceptions} \\
\end{longtable}


\end{document}
