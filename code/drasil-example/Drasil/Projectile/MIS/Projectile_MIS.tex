\documentclass[12pt, titlepage]{article}

\usepackage{amsmath, mathtools}

\usepackage[round]{natbib}
\usepackage{amsfonts}
\usepackage{amssymb}
\usepackage{graphicx}
\usepackage{colortbl}
\usepackage{xr}
\usepackage{hyperref}
\usepackage{longtable}
\usepackage{xfrac}
\usepackage{tabularx}
\usepackage{float}
\usepackage{siunitx}
\usepackage{booktabs}
\usepackage{multirow}
\usepackage[section]{placeins}
\usepackage{caption}
\usepackage{fullpage}

\hypersetup{
bookmarks=true,     % show bookmarks bar?
colorlinks=true,       % false: boxed links; true: colored links
linkcolor=red,          % color of internal links (change box color with linkbordercolor)
citecolor=blue,      % color of links to bibliography
filecolor=magenta,  % color of file links
urlcolor=cyan          % color of external links
}

\usepackage{array}

%% Comments
\newif\ifcomments\commentstrue

\ifcomments
\newcommand{\authornote}[3]{\textcolor{#1}{[#3 ---#2]}}
\newcommand{\todo}[1]{\textcolor{orange}{[TODO: #1]}}
\else
\newcommand{\authornote}[3]{}
\newcommand{\todo}[1]{}
\fi

\newcommand{\wss}[1]{\authornote{blue}{SS}{#1}}
\newcommand{\bmac}[1]{\authornote{red}{BM}{#1}}
\newcommand{\sam}[1]{\authornote{magenta}{SC}{#1}}

\newcommand{\progname}{Projectile}

\begin{document}

\title{Module Interface Specification for Projectile}

\author{Samuel J. Crawford}

\date{\today}

\maketitle

\pagenumbering{roman}

\newpage

\tableofcontents

\newpage

\section{Symbols, Abbreviations and Acronyms}

See SRS Documentation at \url{https://jacquescarette.github.io/Drasil/examples/Projectile/srs/Projectile_SRS.pdf}

\pagenumbering{arabic}

\section{Introduction}

The following document details the Module Interface Specifications for the
implemented modules in a program simulating projectile motion. It is intended to 
ease navigation through the program for design and maintenance purposes.

Complementary documents include the System Requirement Specifications
and Module Guide.  The full documentation and implementation can be
found at \sam{No manual version of Projectile? Should this link be removed or  should there be a link?}.

% The specification is given in terms of functions, rather than sequences.  For
% instance, the predicted temperature of the water is given as a function of time
% ($\mathbb{R} \rightarrow \mathbb{R})$, not as a sequence ($\mathbb{R}^n$). This
% approach is more straightforward for the specification, but in the
% implementation stage, it will likely be necessary to introduce a
% sequence, assuming that a numerical solver is used for the system of ODEs.

\section{Notation}

The structure of the MIS for modules comes from \citet{HoffmanAndStrooper1995},
with the addition that template modules have been adapted from
\cite{GhezziEtAl2003}.  The mathematical notation comes from Chapter 3 of
\citet{HoffmanAndStrooper1995}.  For instance, the symbol := is used for a
multiple assignment statement and conditional rules follow the form $(c_1
\Rightarrow r_1 | c_2 \Rightarrow r_2 | ... | c_n \Rightarrow r_n )$.

The following table summarizes the primitive data types used by \progname. 

\begin{center}
\renewcommand{\arraystretch}{1.2}
\noindent 
\begin{tabular}{l l p{7.5cm}} 
\toprule 
\textbf{Data Type} & \textbf{Notation} & \textbf{Description}\\ 
\midrule
character & char & a single symbol or digit\\
real & $\mathbb{R}$ & any number in (-$\infty$, $\infty$)\\
\bottomrule
\end{tabular} 
\end{center}

\noindent
The specification of \progname\ uses strings, a derived data type. Strings
are lists of characters. In addition, \progname\ uses functions, which
are defined by the data types of their inputs and outputs. Local functions are
described by giving their type signature followed by their specification.

\section{Module Decomposition}

The following table is taken directly from the Module Guide document for this project.

\begin{table}[h!]
\centering
\begin{tabular}{p{0.3\textwidth} p{0.6\textwidth}}
\toprule
\textbf{Level 1} & \textbf{Level 2}\\
\midrule

{Hardware-Hiding} & ~ \\
\midrule

\multirow{7}{0.3\textwidth}{Behaviour-Hiding} & Input Parameters\\
& Output Format\\
& Output Verification\\
& Temperature ODEs\\
& Energy Equations\\ 
& Control Module\\
& Specification Parameters Module\\
\midrule

\multirow{3}{0.3\textwidth}{Software Decision} & {Sequence Data Structure}\\
& ODE Solver\\
& Plotting\\
\bottomrule

\end{tabular}
\caption{Module Hierarchy}
\label{TblMH}
\end{table}

\newpage
~\newpage

\section{MIS of Control Module} \label{Main}

\subsection{Module}

main

\subsection{Uses}

Param (Section~\ref{Parameters}), Temperature
(Section~\ref{Temperature}), Solver
(Section~\ref{ODE}), Energy (Section~\ref{Energy}), verify\_output (Section~\ref{VerifyOutput}), plot
(Section~\ref{Plot}), output (Section~\ref{Output})

\subsection{Syntax}

\subsubsection{Exported Access Programs}

\begin{center}
\begin{tabular}{p{2cm} p{4cm} p{4cm} p{2cm}}
\hline
\textbf{Name} & \textbf{In} & \textbf{Out} & \textbf{Exceptions} \\
\hline
main & - & - & - \\
\hline
\end{tabular}
\end{center}

\subsection{Semantics}

\subsubsection{State Variables}

None

\subsubsection{Access Routine Semantics}

\noindent main():
\begin{itemize}
\item transition: Modify the state of Param module and the environment variables
  for the Plot and Output modules by following these steps\\
\end{itemize}

\noindent Get (filenameIn: string) and (filenameOut: string) from user\\

\noindent load\_params(filenameIn)\\

\noindent \#\textit{Find temperature function} ($T_W^\text{Solid},
  T_W^\text{Melting}, T_W^\text{Liquid}$, $T_P^\text{Solid},
  T_P^\text{Melting}, T_P^\text{Liquid}$), \textit{and energy} ($Q_P$) \textit{and times of transition between solid,
  melting and liquid phases} ($t_\text{melt}^{\text{init}}, t_\text{melt}^{\text{final}}$)\\

\noindent $t_\text{melt}^\text{init}, [T_W^{\text{Solid}}, T_P^{\text{Solid}}]^T := \text{solve}(\text{ODE\_SolidPCM}, 0.0,
[T_\text{init}, T_\text{init}]^T, \text{event\_StartMelt} , t_\text{final})$\\

\noindent $t_\text{melt}^\text{final}$, $[T_W^{\text{Melting}},
T_P^{\text{Melting}}, Q_p]^T$ := solve(ODE\_MeltingPCM,
$t_\text{melt}^\text{init}$, $[T_W^{\text{Solid}}(t_\text{melt}^\text{init}),
  T_P^{\text{Solid}}(t_\text{melt}^\text{init}), 0.0]^T$,
  event\_EndMelt, $t_\text{final}$)\\

\noindent $[ T_W^{\text{Liquid}}, T_P^{\text{Liquid}}]^T$ := solveNoE(ODE\_LiquidPCM, $t_\text{melt}^\text{final}$,
[$T_W^{\text{Melting}}(t_\text{melt}^\text{final})$,
$T_P^{\text{Melting}}(t_\text{melt}^\text{final})$]$^T$, $t_\text{final}$)\\

\noindent \#\textit{Combine temperatures for $0 \leq t \leq t_\text{final}$}\\

\noindent $T_W(t) = (0 \leq t < t_\text{melt}^{\text{init}} \Rightarrow
T_W^{\text{Solid}} | t_\text{melt}^{\text{init}} \leq t <
t_\text{melt}^{\text{final}} \Rightarrow T_W^{\text{Melting}} |
t_\text{melt}^{\text{final}} \leq t \leq
t_\text{final} \Rightarrow T_W^{\text{Liquid}} )$\\

\noindent $T_P(t) = (0 \leq t < t_\text{melt}^{\text{init}} \Rightarrow
T_P^{\text{Solid}} | t_\text{melt}^{\text{init}} \leq t <
t_\text{melt}^{\text{final}} \Rightarrow T_P^{\text{Melting}} | t_\text{melt}^{\text{final}} \leq t \leq
t_\text{final} \Rightarrow T_P^{\text{Liquid}} )$\\

\noindent \#\textit{Energy values} $(E_W(t), E_P(t))$ \textit{for} $0 \leq t \leq t_\text{final}$\\

\noindent $E_W(t) = (0 \leq t < t_\text{melt}^{\text{init}} \Rightarrow
\text{energyWater}(T_W^{\text{Solid}}) | t_\text{melt}^{\text{init}} \leq t <
t_\text{melt}^{\text{final}} \Rightarrow \text{energyWater}(T_W^{\text{Melting}}) | t_\text{melt}^{\text{final}} \leq t \leq
t_\text{final} \Rightarrow \text{energyWater}(T_W^{\text{Liquid}}) )$\\

\noindent $E_P(t) = (0 \leq t < t_\text{melt}^{\text{init}} \Rightarrow
\text{energySolidPCM}(T_P^{\text{Solid}}) | t_\text{melt}^{\text{init}} \leq t <
t_\text{melt}^{\text{final}} \Rightarrow \text{energyMeltingPCM}(Q_P) | t_\text{melt}^{\text{final}} \leq t \leq
t_\text{final} \Rightarrow \text{energyLiquidPCM}(T_P^{\text{Liquid}}) )$\\

\noindent \#\textit{Output calculated values to a file and to a plot.  Verify
  calculated values obey conservation of energy.}\\

\noindent verify\_output($T_w$, $T_p$, $E_w$, $E_p$, $t_\text{final}$)\\

\noindent plot($T_w$, $T_p$, $E_w$, $E_p$, $t_\text{final}$)\\

\noindent output(filenameOut, $T_w$, $T_p$, $E_w$, $E_p$, $t_\text{final}$)\\

\newpage

\section{MIS of Input Parameters Module} \label{Parameters}

The secrets of this module are the data structure for input parameters, how the
values are input and how the values are verified.  The load and verify secrets
are isolated to their own access programs.

\subsection{Module}

Param

\subsection{Uses}

SpecParam (Section~\ref{SpecParam})

\subsection{Syntax}

\begin{tabular}{p{3cm} p{1cm} p{1cm} >{\raggedright\arraybackslash}p{9cm}}
\toprule
\textbf{Name} & \textbf{In} & \textbf{Out} & \textbf{Exceptions} \\
\midrule
load\_params & string & - &  FileError \\
verify\_params & - & - & InputError \\
$v$ & - & $\mathbb{R}$\\
$\theta$ & - & $\mathbb{R}$\\
$p_{\text{target}}$ & - & $\mathbb{R}$\\
$p_{\text{land}}$ & - & $\mathbb{R}$\\
$\text{offset}$ & - & $\mathbb{R}$\\
$\text{message}$ & - & $\text{string}$\\
\bottomrule
\end{tabular}

\subsection{Semantics}

\subsubsection{Environment Variables}

inputFile: sequence of string \#\textit{f[i] is the ith string in the text file f}

\subsubsection{State Variables}

\renewcommand{\arraystretch}{1.2}
\begin{longtable*}[l]{l} 
\# To Support IM1 and IM2 \\
$v$: $\mathbb{R}$\\
$\theta$: $\mathbb{R}$\\
\# To Support IM3 and IM4 \\
$p_{\text{target}}$: $\mathbb{R}$\\
\# From FR4 \\
$p_{\text{land}}$: $\mathbb{R}$\\
$\text{offset}$: $\mathbb{R}$\\
$\text{message}$: $\text{string}$\\
\end{longtable*}

\subsubsection{Assumptions}

\begin{itemize}

\item load\_params will be called before the values of any state variables will 
be accessed.

\item The file contains the string equivalents of the numeric values for
each input parameter in order, each on a new line. Any comments in the input file 
should be denoted with a '\#' symbol.

\end{itemize}

\subsubsection{Access Routine Semantics}

\noindent Param.$v$:
\begin{itemize}
\item output: \textit{out} := $v$
\item exception: none
\end{itemize}

\noindent Param.$\theta$:
\begin{itemize}
\item output: \textit{out} := $\theta$
\item exception: none
\end{itemize}

\noindent Param.$p_{\text{target}}$:
\begin{itemize}
\item output: \textit{out} := $p_{\text{target}}$
\item exception: none
\end{itemize}

\noindent Param.$p_{\text{land}}$:
\begin{itemize}
\item output: \textit{out} := $p_{\text{land}}$
\item exception: none
\end{itemize}

\noindent Param.$\text{offset}$:
\begin{itemize}
\item output: \textit{out} := $\text{offset}$
\item exception: none
\end{itemize}

\noindent Param.$\text{message}$:
\begin{itemize}
\item output: \textit{out} := $\text{message}$
\item exception: none
\end{itemize}

\noindent load\_params($s$):
\begin{itemize}
\item transition: The filename $s$ is first associated with the file f. 
{inputFile} is used to modify the state variables using the following procedural 
specification:
\begin{enumerate}
\item Read data sequentially from inputFile to populate the state variables from
  FR1 ($v$, $\theta$, and $p_{\text{target}}$).
\item Calculate the derived quantities ($p_{\text{land}}$, $\text{offset}$, and $
\text{message}$) as follows:
\begin{itemize}
\item $p_{\text{land}} := \frac{2 v^2 sin(\theta) cos(\theta)}{9.8}$
\item $\text{offset} := p_{\text{land}} - p_{\text{target}} $
\item $\text{message} := $
\begin{longtable*}[l]{l l} 
$|\frac{\text{offset}}{p_{\text{target}}}| < 0.02$ & $\Rightarrow$ ``The target was hit."\\
$\text{offset} < 0$ & $\Rightarrow$ ``The projectile fell short."\\
$\text{True}$ & $\Rightarrow$ ``The projectile went long."\\
\end{longtable*}
\end{itemize}
\item verify\_params()
\end{enumerate}

\item exception: exc := a file name $s$ cannot be found OR the format of
  inputFile is incorrect $\Rightarrow$  FileError
\end{itemize}

\noindent verify\_params():
\begin{itemize}
\item out: \textit{out} := none
\item exception: exc := 
\end{itemize}
\noindent \begin{longtable*}[l]{l l} 
$\neg (0 < v)$ & $\Rightarrow$ InputError\\
$\neg (0 < \text{offset})$ & $\Rightarrow$ InputError\\
$\neg (0 < p_{\text{target}})$ & $\Rightarrow$ InputError\\
\end{longtable*}

etc.  See Appendix (Section~\ref{Appendix}) for the complete list of exceptions 
and associated error messages. \sam{I think the error messages should be more 
descriptive, and be defined in the Appendix.}

\subsection{Considerations}

The value of each state variable can be accessed through its name (getter).  An
access program is available for each state variable.  There are no setters for
the state variables, since the values will be set and checked by load params and
not changed for the life of the program.

\newpage

% \section{MIS of Input Verification Module} \label{VerifyInput}

% \subsection{Module}

% verify\_params

% \subsection{Uses}

% Param (Section~\ref{Parameters})

% \subsection{Syntax}

% \subsubsection{Exported Access Programs}

% \begin{center}
% \begin{tabular}{p{3cm} p{1cm} p{1cm} p{9cm}}
% \hline
% \textbf{Name} & \textbf{In} & \textbf{Out} & \textbf{Exceptions} \\
% \hline
% verify\_valid & - & - & badLength, badDiam, badPCMVolume, badPCMAndTankVol,
%                         badPCMArea, badPCMDensity, badMeltTemp,
%                         badCoilAndInitTemp, badCoilTemp, badPCMHeatCapSolid,
%                         badPCMHeatCapLiquid, badHeatFusion, badCoilArea,
%                         badWaterDensity, badWaterHeatCap, badCoilCoeff,
%                         badPCMCoeff, badInitTemp, badFinalTime,
%                         badInitAndMeltTemp \\
% \hline
% verify\_recommend & - & - & - \\
% \hline
% \end{tabular}
% \end{center}

% \subsection{Semantics}

% \subsubsection{Environment Variables}

% $win$: 2D array of pixels displayed on the screen.

% \subsubsection{Assumptions}

% All of the fields Param have been assigned values before any of the access
% routines for this module are called.

% \subsubsection{Access Routine Semantics}

% verify\_valid(): 
% \begin{itemize}
% \item transition: none
% \item exceptions: exc := (\\
% Param.getL() $\leq 0 \Rightarrow$ badLength $|$\\
% Param.get\_diam() $\leq 0 \Rightarrow$ badDiam $|$\\
% Params.get\_Vp() $\leq 0 \Rightarrow$ badPCMVolume $|$\\
% Params.getVp() $\geq$ Params.Vt $\Rightarrow$ badPCMAndTankVol $|$\\
% Params.getAp() $\leq 0 \Rightarrow$ badPCMArea $|$\\
% Params.get\_rho\_p() $\leq 0 \Rightarrow$ badPCMDensity $|$\\
% Params.getTmelt() $\leq 0 \Rightarrow$ badMeltTemp $|$\\
% Params.getTmelt() $\geq$ Params.getTc() $\Rightarrow$ badMeltTemp $|$\\
% Params.getTc() $\leq$ Params.getTinit() $\Rightarrow$ badCoilAndInitTemp $|$\\
% Params.getTc() $\geq 100 \lor$ Params.getTc() $\leq 0 \Rightarrow$ badCoilTemp $|$\\
% Params.getC\_ps() $\leq 0 \Rightarrow$ badPCMHeatCapSolid $|$\\
% Params.getC\_pl() $\leq 0 \Rightarrow$ badPCMHeatCapLiquid $|$\\
% Params.getHf() $\leq 0 \Rightarrow$ badHeatFusion $|$\\
% Params.getAc() $\leq 0 \Rightarrow$ badCoilArea $|$\\
% Params.get\_rho()\_w $\leq 0 \Rightarrow$ badWaterDensity $|$\\
% Params.getC\_w() $\leq 0 \Rightarrow$ badWaterHeatCap $|$\\
% Params.get\_hc() $\leq 0 \Rightarrow$ badCoilCoeff $|$\\
% Params.get\_hp() $\leq 0 \Rightarrow$ badPCMCoeff $|$\\
% Params.getTinit() $\leq 0 \lor$ Params.getTinit() $\geq 100 \Rightarrow$
% badInitTemp $|$\\
% Params.get\_tfinal() $\leq 0 \Rightarrow$ badFinalTime $|$\\
% Params.getTinit() $\geq$ Params.getTmelt() $\Rightarrow$ badInitAndMeltTemp)  
% \end{itemize}

% verify\_recommend():
% \begin{itemize}
% \item transition: none
% \item exceptions: exc := (\\
% Params.getL() $< 0.1 \lor$ Params.getL() $> 50 \Rightarrow$ warnLength $|$\\
% Params.getdiam() / Params.getL() $< 0.002 \lor$ Params.getdiam() / Params.getL() $> 200
% \Rightarrow$ warnDiam $|$\\
% Params.getVp() $<$ Params.getVt() $\times 10 ^ -6 \Rightarrow$ warnPCMVol $|$\\
% Params.getVp() $>$ Params.getAp() $\lor$ Params.getAp $> (2/0.001) \times$ Params.getVp()
% $\Rightarrow$ warnVolArea $|$\\
% (Params.get\_rho\_p() $\leq 500) \lor ($ Params.get\_rho\_p() $\geq 20000) \Rightarrow$
% warnPCMDensity $|$ ... )\\
% \# \textit{Need to continue for the rest of the example - tabular form?}
% \# \textit{Should add a module (Configuration Module) to store symbolic constants}
%  % Params.getC\_ps \leq 100 \lor Params.getC\_ps \geq 4000 \Rightarrow$
%  % warnPCMHeatCapSolid $|$\\
%  % Params.getC\_pl \leq 100 \lor Params.getC\_pl \geq 5000 \Rightarrow$
%  % warnPCMHeatCapLiquid $|$\\
%  % Params.getAc > \pi \times (Params.getdiam / 2) ^ 2 \Rightarrow$ warnCoilArea
%  % $|$\\
%  % Params.getrho\_w \leq 950 \lor Params.getrho\_w > 1000 \Rightarrow$
%  % warnWaterDensity $|$\\
%  % Params.getC\_w \leq 4170 \lor Params.getC\_w \geq 4210 \Rightarrow$
%  % warnWaterHeatCap $|$\\
%  % Params.gethc \leq 10 \lor Params.gethc \geq 10000 \Rightarrow$ warnCoilCoeff $|$\\
%  % Params.gethp \leq 10 \lor Params.gethp \geq 10000 \Rightarrow$ warnPCMCoeff $|$\\
%  % Params.gettfinal \leq 0 \lor Params.gettfinal \geq 86400 \Rightarrow$ warnFinalTime)
% \end{itemize}

% \subsection{Considerations}

% See Appendix (Section~\ref{Appendix}) for the complete list of exceptions and
% associated error messages.

\newpage
\section{MIS of Temperature ODEs Module} \label{Temperature}

\subsection{Module}

Temperature

\subsection{Uses}

Param (Section~\ref{Parameters})

\subsection{Syntax}

\subsubsection{Exported Access Programs}

\begin{center}
\begin{tabular}{p{3.5cm} p{1cm} p{7cm} p{2cm}}
\hline
\textbf{Name} & \textbf{In} & \textbf{Out} & \textbf{Exceptions} \\
\hline
ODE\_SolidPCM & -- & $(\mathbb{R}^{3} \rightarrow \mathbb{R})^2$ & - \\
\hline
ODE\_MeltingPCM & -- &  $(\mathbb{R}^{4} \rightarrow \mathbb{R})^3$ & - \\
\hline
ODE\_LiquidPCM & -- & $(\mathbb{R}^{3} \rightarrow \mathbb{R})^2$ & - \\
\hline
event\_StartMelt & -- & $\mathbb{R}^2 \rightarrow \mathbb{R}$ & - \\
\hline
event\_EndMelt & -- & $\mathbb{R}^3 \rightarrow \mathbb{R}$ & - \\
\hline
\end{tabular}
\end{center}

\subsection{Semantics}

\subsubsection{State Variables}

none

\subsubsection{Assumptions}

none

\subsubsection{Access Routine Semantics}

ODE\_SolidPCM(): 
\renewcommand*{\arraystretch}{1.5}
\begin{itemize}
\item output: $out := 
\dfrac{d}{dt} \left [ 
\begin{array}{c}
T_W\\
T_P 
\end{array} 
\right] =
\left [ 
\begin{array}{c}
\frac{1}{\tau_W}[(T_C - T_W(t)) + {\eta}(T_P(t) - T_W(t))]\\
\frac{1}{\tau^S_P}(T_W(t) - T_P(t)) 
\end{array} 
\right]
$
\item exception: none
\end{itemize}

ODE\_MeltingPCM(): 
\renewcommand*{\arraystretch}{1.5}
\begin{itemize}
\item output: $out := 
\dfrac{d}{dt} \left [ 
\begin{array}{c}
T_W\\
T_P\\
Q_P 
\end{array} 
\right] =
\left [ 
\begin{array}{c}
\frac{1}{\tau_W}[(T_C - T_W(t)) + {\eta}(T_P(t) - T_W(t))]\\
0 \\
h_P A_P (T_W(t) - T_\text{melt}^P)
\end{array} 
\right]
$
\item exception: none
\end{itemize}

ODE\_LiquidPCM(): 
\renewcommand*{\arraystretch}{1.5}
\begin{itemize}
\item output: $out := 
\dfrac{d}{dt} \left [ 
\begin{array}{c}
T_W\\
T_P
\end{array} 
\right] =
\left [ 
\begin{array}{c}
\frac{1}{\tau_W}[(T_C - T_W(t)) + {\eta}(T_P(t) - T_W(t))]\\
\frac{1}{\tau^L_P}(T_W(t) - T_P(t))
\end{array} 
\right]
$
\item exception: none
\end{itemize}

event\_StartMelt(): 
\begin{itemize}
\item output: $out := g([T_W, T_P]^T) = T_\text{melt}^P - T_P$
\item exception: none
\end{itemize}

event\_EndMelt(): 
\begin{itemize}
\item output: $out := g([T_W, T_P, Q_P]^T) = 1 - \phi$, where $\phi = \frac{Q_P}{E_{P\text{melt}}^{\text{all}}}$
\item exception: none
\end{itemize}

\newpage
\section{MIS of ODE Solver Module} \label{ODE}

\#\textit{Bold font is used to indicate variables that are a sequence
  type}

\subsection{Module}

Solver($n: \mathbb{N}$) \#\textit{$n$ is the length of the sequences}

\subsection{Uses}

None

\subsection{Syntax}

\subsubsection{Exported Access Programs}

\begin{center}
\begin{tabular}{p{1.5cm} >{\raggedright\arraybackslash}p{9.25cm} >{\raggedright\arraybackslash}p{2.4cm} p{1.5cm}}
  \hline
  \textbf{Name} & \textbf{In} & \textbf{Out} & \textbf{Except.} \\
  \hline
  solve & $\textbf{f}: (\mathbb{R}^{n+1} \rightarrow \mathbb{R})^n, t_0 : \mathbb{R},
          \textbf{y}_0: \mathbb{R}^n, g: \mathbb{R}^n \rightarrow \mathbb{R},
          t_\text{fin}: \mathbb{R}$ & $t_1: \mathbb{R}, 
                                      \textbf{y}:
                                      (\mathbb{R}
                                      \rightarrow
                                      \mathbb{R})^n$ & ODE\_ERR, NO\_EVENT\\
  solveNoE & $\textbf{f}: (\mathbb{R}^{n+1} \rightarrow \mathbb{R})^n, t_0 : \mathbb{R},
             \textbf{y}_0: \mathbb{R}^n, t_\text{fin}: \mathbb{R}$ & $\textbf{y}: (\mathbb{R}
                                                            \rightarrow
                                                            \mathbb{R})^n$ & ODE\_ERR\\

  \hline 
\end{tabular}
\end{center}

\subsection{Semantics}

\subsubsection{State Variables}

None

\subsubsection{Access Routine Semantics}

\#\textit{Solving} $\frac{d}{dt} \mathbf{y} = \mathbf{f}(t,
\mathbf{y}(t))$\\

\noindent solve($\textbf{f}, t_0, \textbf{y}_0, g, t_\text{fin}$): 
\begin{itemize}
\item output: $out := t_1, \textbf{y}(t)$ where 
$$\textbf{y}(t) = \textbf{y}_0 + \int_{t_0}^{t} \textbf{f}(s, \textbf{y}(s)) ds$$ 
with $t_1$ determined by the first time where $g(\textbf{y}(t_1)) = 0$.  $\textbf{y}(t)$ is
calculated from $t = t_0$ to $t = t_1$.
\item exception: $exc :=$ ( $\neg (\exists t: \mathbb{R}| t_0 \leq t
  \leq t_\text{fin} : g(\textbf{y}(t)) = 0) \Rightarrow$ NO\_EVENT $|$ ODE Solver Fails $\Rightarrow$
  ODE\_ERR)
\end{itemize}

solveNoE($\textbf{f}$, $t_0$, $\textbf{y}_0$, $t_\text{fin}$): 
\begin{itemize}
\item output: $out := \textbf{y}(t)$ where 
$$\textbf{y}(t) = \textbf{y}_0 + \int_{t_0}^{t_\text{fin}} \textbf{f}(s, \textbf{y}(s)) ds$$ 
$y(t)$ is calculated from $t = t_0$ to $t = t_\text{fin}$.
\item exception: $exc :=$ ( ODE Solver Fails $\Rightarrow$ ODE\_ERR)
\end{itemize}

\newpage
\section{MIS of Energy Module} \label{Energy}

\subsection{Module}

Energy

\subsection{Uses}

Param (Section~\ref{Parameters})

\subsection{Syntax}

\subsubsection{External Access Programs}

\begin{center}
\begin{tabular}{p{4cm} p{4cm} p{3cm} p{2cm}}
\hline
\textbf{Name} & \textbf{In} & \textbf{Out} & \textbf{Exceptions} \\
\hline
energyWater & $\mathbb{R} \rightarrow \mathbb{R}$ & $\mathbb{R} \rightarrow \mathbb{R}$ & - \\
\hline
energySolidPCM & $\mathbb{R} \rightarrow \mathbb{R}$ & $\mathbb{R} \rightarrow \mathbb{R}$ & - \\
\hline
energyMeltingPCM & $\mathbb{R} \rightarrow \mathbb{R}$ & $\mathbb{R} \rightarrow \mathbb{R}$ & - \\
\hline
energyLiquidPCM & $\mathbb{R} \rightarrow \mathbb{R}$ & $\mathbb{R} \rightarrow \mathbb{R}$ & - \\
\hline
\end{tabular}
\end{center}

\subsection{Semantics}

\subsubsection{State Variables}

None

\subsubsection{Assumptions}

None

\subsubsection{Access Routine Semantics}

energyWater($T_W$): 
\renewcommand*{\arraystretch}{1.5}
\begin{itemize}
\item output: $out := C_W m_W (T_W - T_\text{init})$
\item exception: none
\end{itemize}

\noindent energySolidPCM($T_P$): 
\renewcommand*{\arraystretch}{1.5}
\begin{itemize}
\item output: $out := C^S_P m_P (T_P - T_\text{init})$
\item exception: none
\end{itemize}

\noindent energyMeltingPCM($Q_P$): 
\renewcommand*{\arraystretch}{1.5}
\begin{itemize}
\item output: $out := E_{P\text{melt}}^{\text{init}} + Q_P$
\item exception: none
\end{itemize}

\noindent energyLiquidPCM($T_P$): 
\renewcommand*{\arraystretch}{1.5}
\begin{itemize}
\item output: $out := E_{P\text{melt}}^{\text{init}}+H_f m_p + C_P^L m_P(T_P(t) - T_\text{melt}^P)$
\item exception: none
\end{itemize}

\newpage
\section{MIS of Output Verification Module} \label{VerifyOutput}

\subsection{Module}

verify\_output

\subsection{Uses}

Param (Section~\ref{Parameters})

\subsection{Syntax}

\subsubsection{Exported Constant}

ADMIS\_ER = $1 \times 10^{-6}$

\subsubsection{Exported Access Programs}

\begin{center}
\begin{tabular}{p{3cm} p{7cm} p{1cm} p{2cm}}
\hline
\textbf{Name} & \textbf{In} & \textbf{Out} & \textbf{Exceptions} \\
\hline
verify\_output & $T_W(t):\mathbb{R} \rightarrow \mathbb{R},
                 T_P(t):\mathbb{R} \rightarrow \mathbb{R},
                 E_W(t):\mathbb{R} \rightarrow \mathbb{R},
                 E_P(t):\mathbb{R} \rightarrow \mathbb{R},
                 t_\text{final}: \mathbb{R}$ & - & EWAT\_NOT\_CONSERVE, EPCM\_NOT\_CONSERVE \\
\hline
\end{tabular}
\end{center}

\subsection{Semantics}

\subsubsection{State Variables}

None

\subsubsection{Assumptions}

All of the fields of the input parameters structure have been assigned a
value.  

\subsubsection{Access Routine Semantics}

\noindent verify\_output($T_W, T_P, E_W, E_P$, $t_\text{final}$):
\begin{itemize}
\item exception: exc := (
\end{itemize}

\noindent $
(\forall t | 0 \leq t \leq t_\text{final} : \text{relErr}(E_W,
\int_{0}^{t} h_C A_C (T_C - T_W(t)) dt - \int_{0}^{t} h_P A_P (T_W(t)
- T_P(t)) dt) < \text{ADMIS\_ER}) \Rightarrow \text{EWAT\_NOT\_CONSERVE}
$

$|$

\noindent $ 
(\forall t | 0 \leq t \leq t_\text{final} : \text{relErr}(E_{P}, \int_{0}^{t}
h_{P} A_{P} (T_{W}(t) - T_{P}(t)) dt) < \text{ADMIS\_ER}) \Rightarrow
\text{EPCM\_NOT\_CONSERVE} 
$
)

\subsubsection{Local Functions}

relErr: $\mathbb{R}$ $\times$ $\mathbb{R}$ $\rightarrow$ $\mathbb{R}$ \\
$\text{relErr}(t, e) \equiv \frac{|t - e|}{|t|}$ \\
\newline

\newpage
\section{MIS of Plotting Module} \label{Plot}

\subsection{Module}

plot

\subsection{Uses}

N/A

\subsection{Syntax}

\subsubsection{Exported Access Programs}

\begin{center}
\begin{tabular}{p{2cm} p{8cm} p{2cm} p{2cm}}
\hline
\textbf{Name} & \textbf{In} & \textbf{Out} & \textbf{Exceptions} \\
\hline
plot & $T_W(t):\mathbb{R} \rightarrow \mathbb{R},
                 T_P(t):\mathbb{R} \rightarrow \mathbb{R},
                 E_W(t):\mathbb{R} \rightarrow \mathbb{R},
       E_P(t):\mathbb{R} \rightarrow \mathbb{R}$, $t_\text{final}: \mathbb{R}$ & - & - \\
\hline
\end{tabular}
\end{center}

\subsection{Semantics}

\subsubsection{State Variables}

None

\subsubsection{Environment Variables}

win: 2D sequence of pixels displayed on the screen\\

\subsubsection{Assumptions}

None

\subsubsection{Access Routine Semantics}

\noindent plot($T_w$, $T_p$, $E_w$, $E_p$, $t_\text{final}$):
\begin{itemize}
\item transition: Modify win to display a plot where the vertical axis
  is time and one horizontal axis is temperature and the other
  horizontal axis is energy.  The time should run from $0$ to $t_\text{final}$
\item exception: none
\end{itemize}

\newpage

\section{MIS of Output Module} \label{Output}

\subsection{Module}

output

\subsection{Uses}

Param (Section~\ref{Parameters})

\subsection{Syntax}

\subsubsection{Exported Constants}

$max\_width$: integer

\subsubsection{Exported Access Program}

\begin{center}
\begin{tabular}{p{3cm} p{7cm} p{2cm} p{2cm}}
\hline
\textbf{Name} & \textbf{In} & \textbf{Out} & \textbf{Exceptions} \\
\hline
output & fname: string, $T_W(t):\mathbb{R} \rightarrow \mathbb{R},
                 T_P(t):\mathbb{R} \rightarrow \mathbb{R},
                 E_W(t):\mathbb{R} \rightarrow \mathbb{R},
       E_P(t):\mathbb{R} \rightarrow \mathbb{R}$, $t_\text{final}: \mathbb{R}$ & - & - \\
\hline
\end{tabular}
\end{center}

\subsection{Semantics}

\subsubsection{State Variables}

None

\subsubsection{Environment Variables}

file: A text file

\subsubsection{Access Routine Semantics}

\noindent output(fname, $T_w$, $T_p$, $E_w$, $E_p$, $t_\text{final}$):
\begin{itemize}
\item transition:  Write to environment variable named fname the
  following: the input
    parameters from Param, and the calculated values $T_w$, $T_p$, $E_w$,
    $E_p$ from times $0$ to $t_\text{final}$.  The functions will be output as
    sequences in this file.  The spacing between points in the sequence should
    be selected so that the heating behaviour is captured in the data.
\item exception: none
\end{itemize}

\newpage

\section{MIS of Specification Parameters} \label{SpecParam}

The secrets of this module is the value of the specification parameters.

\subsection{Module}

SpecParam

\subsection{Uses}

N/A

\subsection{Syntax}

\subsubsection{Exported Constants}

\renewcommand{\arraystretch}{1.2}
\begin{longtable*}[l]{l} 
\# From Table 2 in SRS\\
  $L_\text{min}$ := 0.1\\
  $L_\text{max}$ := 50\\
  ${\frac{D}{L}}_\text{min}$ := 0.002 \\
  ${\frac{D}{L}}_\text{max}$ := 200 \\
  $\text{minfrac} $ := $10^{-6}$\\
  $h_\text{min}$ := 0.001 \\
  $\rho_P^{\text{min}}$ := 500\\
  $\rho_P^{\text{max}}$ := 20000\\
  $C_{P\text{min}}^S$ := 100 \\
  $C_{P\text{max}}^S$ := 4000\\
  $C_{P\text{min}}^L$ := 100 \\
  $C_{P\text{max}}^L$ := 5000\\
  $A_C^{\text{max}}$ := $\pi(\frac{D}{2})^2$\\
  $\rho_W^{\text{min}}$ := 950\\
  $\rho_W^{\text{max}}$ := 1000\\
  $C_W^{\text{min}}$ := 4170\\
  $C_W^{\text{max}}$ := 4210\\
  $h_C^{\text{min}}$ := 10\\
  $h_C^{\text{max}}$ := 10000\\
  $h_P^{\text{min}}$ := 10\\
  $h_P^{\text{max}}$ := 10000\\
  $t_{\text{final}}^{\text{max}}$ := 86400\\
\end{longtable*}

\wss{$A_C^{\text{max}}$ shouldn't be in this table of constants}

\subsection{Semantics}

N/A

\newpage

\bibliographystyle {plainnat}
\bibliography {MIS}

\newpage

\section{Appendix} \label{Appendix}

\renewcommand{\arraystretch}{1.2}

\begin{longtable}{l p{12cm}}
\caption{Possible Exceptions} \\
\toprule
\textbf{Message ID} & \textbf{Error Message} \\
\midrule
badLength & Error: Tank length must be $> 0$ \\
badDiam & Error: Tank diameter must be $> 0$ \\
badPCMVolume & Error: PCM volume must be $> 0$ \\
badPCMAndTankVol & Error: PCM volume must be $<$ tank volume \\
badPCMArea & Error: PCM area must be $> 0$ \\
badPCMDensity & Error: rho\_p must be $> 0$ \\
badMeltTemp & Error: Tmelt must be $> 0$ and $< Tc$ \\
badCoilAndInitTemp & Error: Tc must be $>$ Tinit \\
badCoilTemp & Error: Tc must be $> 0$ and $< 100$ \\
badPCMHeatCapSolid & Error: C\_ps must be $> 0$ \\
badPCMHeatCapLiquid & Error: C\_pl must be $> 0$ \\
badHeatFusion & Error: Hf must be $> 0$ \\
badCoilArea & Error: Ac must be $> 0$ \\
badWaterDensity & Error: rho\_w must be $> 0$ \\
badWaterHeatCap & Error: C\_w must be $> 0$ \\
badCoilCoeff & Error: hc must be $> 0$ \\
badPCMCoeff & Error: hp must be $> 0$ \\
badInitTemp & Error: Tinit must be $> 0$ and $< 100$ \\
badFinalTime & Error: tfinal must be $> 0$ \\
badInitAndMeltTemp & Error: Tinit must be $<$ Tmelt \\
ODE\_ACCURACY & $reltol$ and $abstol$ were not satisfied by the ODE solver for a given solution step. \\
ODE\_BAD\_INPUT & Invalid input to ODE solver \\
ODE\_MAXSTEP & ODE solver took $MaxStep$ steps and did not find solution \\
warnLength & Warning: It is recommended that 0.1 $<$= L $<$= 50 \\
warnDiam & Warning: It is recommended that 0.002 $<$= D/L $<$= 200 \\
warnPCMVol & Warning: It is recommended that Vp be $>$= 0.0001\% of Vt \\
warnVolArea & Warning: It is recommended that Vp $<$= Ap $<$= (2/0.001) * Vp \\
warnPCMDensity & Warning: It is recommended that 500 $<$ rho\_p $<$ 20000 \\
warnPCMHeatCapSolid & Warning: It is recommended that 100 $<$ C\_ps $<$ 4000 \\
warnPCMHeatCapLiquid & Warning: It is recommended that 100 $<$ C\_pl $<$ 5000 \\
warnCoilArea & Warning: It is recommended that Ac $<$= pi * (D/2) $\wedge$ 2 \\
warnWaterDensity & Warning: It is recommended that 950 $<$ rho\_w $<$= 1000 \\
warnWaterHeatCap & Warning: It is recommended that 4170 $<$ C\_w $<$ 4210 \\
warnCoilCoeff & Warning: It is recommended that 10 $<$ hc $<$ 10000 \\
warnPCMCoeff & Warning: It is recommended that 10 $<$ hp $<$ 10000 \\
warnFinalTime & Warning: It is recommended that 0 $<$ tfinal $<$ 86400 \\
warnWaterError & Warning: There is greater than $x$\% relative error between the energy in the water output and the expected output based on the law of conservation of energy. (Where $x$ is the value of $ConsTol$) \\
warnPCMError & Warning: There is greater than $x$\% relative error between the energy in the PCM output and the expected output based on the law of conservation of energy. (Where $x$ is the value of $ConsTol$) \\
\bottomrule
\end{longtable}


\end{document}
