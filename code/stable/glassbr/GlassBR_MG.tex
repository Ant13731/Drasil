\documentclass[12pt]{article}
\usepackage{fullpage}
\usepackage{hyperref}
\hypersetup{bookmarks=true,colorlinks=true,linkcolor=red,citecolor=blue,filecolor=magenta,urlcolor=cyan}
\usepackage{amsmath}
\usepackage{longtable}
\usepackage{booktabs}
\usepackage{caption}
\usepackage{luatex85}
\def\pgfsysdriver{pgfsys-pdftex.def}
\usepackage{tikz}
\usetikzlibrary{arrows.meta}
\usetikzlibrary{graphs}
\usetikzlibrary{graphdrawing}
\usegdlibrary{layered}
\newcounter{lcnum}
\newcommand{\lcthelcnum}{LC\thelcnum}
\newcounter{ucnum}
\newcommand{\uctheucnum}{UC\theucnum}
\newcounter{modnum}
\newcommand{\mthemodnum}{M\themodnum}
\title{Module Guide for Glass-BR}
\author{Spencer Smith and Thulasi Jegatheesan}
\begin{document}
\maketitle
\tableofcontents
\newpage
\section{Introduction}
\label{Sec:I}
Decomposing a system into modules is a commonly accepted approach to developing software.  A module is a work assignment for a programmer or programming team. In the best practices for scientific computing, Wilson et al advise a modular design, but are silent on the criteria to use to decompose the software into modules.  We advocate a decomposition based on the principle of information hiding. This principle supports design for change, because the ``secrets" that each module hides represent likely future changes.  Design for change is valuable in SC, where modifications are frequent, especially during initial development as the solution space is explored.
Our design follows the rules laid out by Parnas, as follows:
\begin{enumerate}
\item{System details that are likely to change independently should be the secrets of separate modules.}
\item{Any other program that requires information stored in a module's data structures must obtain it by calling access programs belonging to that module.}
\end{enumerate}
After completing the first stage of the design, the Software Requirements Specification (SRS), the Module Guide (MG) is developed. The MG specifies the modular structure of the system and is intended to allow both designers and maintainers to easily identify the parts of the software.  The potential readers of this document are as follows:
\begin{enumerate}
\item{New project members: This document can be a guide for a new project member to easily understand the overall structure and quickly find the relevant modules they are searching for.}
\item{Maintainers: The hierarchical structure of the module guide improves the maintainers' understanding when they need to make changes to the system. It is important for a maintainer to update the relevant sections of the document after changes have been made.}
\item{Designers: Once the module guide has been written, it can be used to check for consistency, feasibility and flexibility. Designers can verify the system in various ways, such as consistency among modules, feasibility of the decomposition, and flexibility of the design.}
\end{enumerate}
Section~\ref{Sec:LaUC}  lists the likely and unlikely changes of the software requirements. Section~\ref{Sec:MH}  summarizes the module decomposition that was constructed according to the likely changes. Section~\ref{Sec:MD}  gives a detailed description of the modules. Section~\ref{Sec:TM}  includes two traceability matrices. One checks the completeness of the design against the requirements provided in the SRS. The other shows the relation between anticipated changes and the modules. Section~\ref{Sec:UH}  describes the use relation between modules.
\section{Likely and Unlikely Changes}
\label{Sec:LaUC}
This section lists possible changes to the system. According to the likeliness of the change, the possible changes are classified into two categories. Likely changes are listed in Section~\ref{Sec:LC} and unlikely changes are listed in Section~\ref{Sec:UC}
\subsection{Likely Changes}
\label{Sec:LC}
Likely changes are the source of the information that is to be hidden inside the modules. Ideally, changing one of the likely changes will only require changing the one module that hides the associated decision. The approach adapted here is called design for change.
\begin{description}
\item[\refstepcounter{lcnum}\lcthelcnum\label{LChardware}:]The specific hardware on which the software is running.
\end{description}
\begin{description}
\item[\refstepcounter{lcnum}\lcthelcnum\label{LCinput}:]The format of the initial input data.
\end{description}
\begin{description}
\item[\refstepcounter{lcnum}\lcthelcnum\label{LCparameters}:]The format of the input parameters.
\end{description}
\begin{description}
\item[\refstepcounter{lcnum}\lcthelcnum\label{LCoutput}:]The format of the final output data.
\end{description}
\begin{description}
\item[\refstepcounter{lcnum}\lcthelcnum\label{LCequations}:]How the equations are defined using the input parameters.
\end{description}
\begin{description}
\item[\refstepcounter{lcnum}\lcthelcnum\label{LCcontrol}:]How the overall control of the calculations is orchestrated.
\end{description}
\begin{description}
\item[\refstepcounter{lcnum}\lcthelcnum\label{LCinterpd}:]The format of the data used for interpolation.
\end{description}
\begin{description}
\item[\refstepcounter{lcnum}\lcthelcnum\label{LCinterp}:]The algorithm used for interpolation.
\end{description}
\subsection{Unikely Changes}
\label{Sec:UC}
The module design should be as general as possible. However, a general system is more complex. Sometimes this complexity is not necessary. Fixing some design decisions at the system architecture stage can simplify the software design. If these decision should later need to be changed, then many parts of the design will potentially need to be modified. Hence, it is not intended that these decisions will be changed.  As an example, the model is assumed to follow the definition in the SRS.  If a new model is used, this will mean a change to the input format, fit parameters module, control, and output format modules.
\begin{description}
\item[\refstepcounter{ucnum}\uctheucnum\label{UCIO}:]Input/Output devices (Input: File and/or Keyboard, Output: File, Memory, and/or Screen).
\end{description}
\begin{description}
\item[\refstepcounter{ucnum}\uctheucnum\label{UCinputsource}:]There will always be a source of input data external to the software.
\end{description}
\begin{description}
\item[\refstepcounter{ucnum}\uctheucnum\label{UCoutput}:]Output data are displayed to the output device.
\end{description}
\begin{description}
\item[\refstepcounter{ucnum}\uctheucnum\label{UCgoal}:]The goal of the system is to predict whether the glass slab under consideration can withstand an explosion of a certain degree.
\end{description}
\begin{description}
\item[\refstepcounter{ucnum}\uctheucnum\label{UCequations}:]The equations for Safety can be defined using parameters defined in the input parameters module.
\end{description}
\section{Module Hierarchy}
\label{Sec:MH}
This section provides an overview of the module design. Modules are summarized in a hierarchy decomposed by secrets in Table~\ref{Table:MH}. The modules listed below, which are leaves in the hierarchy tree, are the modules that will actually be implemented.
\begin{description}
\item[\refstepcounter{modnum}\mthemodnum\label{Mhardwarehiding}:]Hardware Hiding Module
\end{description}
\begin{description}
\item[\refstepcounter{modnum}\mthemodnum\label{Minputformat}:]Input Format Module
\end{description}
\begin{description}
\item[\refstepcounter{modnum}\mthemodnum\label{Minputparameters}:]Input Parameters Module
\end{description}
\begin{description}
\item[\refstepcounter{modnum}\mthemodnum\label{Minputconstraints}:]Input Constraints Module
\end{description}
\begin{description}
\item[\refstepcounter{modnum}\mthemodnum\label{Moutputformat}:]Output Format Module
\end{description}
\begin{description}
\item[\refstepcounter{modnum}\mthemodnum\label{Mderivedvalues}:]Derived Values Module
\end{description}
\begin{description}
\item[\refstepcounter{modnum}\mthemodnum\label{Mcalculations}:]Calculations Module
\end{description}
\begin{description}
\item[\refstepcounter{modnum}\mthemodnum\label{Mcontrol}:]Control Module
\end{description}
\begin{description}
\item[\refstepcounter{modnum}\mthemodnum\label{Minterpolationdata}:]Interpolation Data Module
\end{description}
\begin{description}
\item[\refstepcounter{modnum}\mthemodnum\label{Minterpolation}:]Interpolation Module
\end{description}
\begin{longtable}{l l}
\toprule
Level 1 & Level 2
\\
\midrule
Hardware Hiding Module & 
\\
Behaviour Hiding Module & Input Format Module
\\
 & Input Parameters Module
\\
 & Input Constraints Module
\\
 & Output Format Module
\\
 & Derived Values Module
\\
 & Calculations Module
\\
 & Control Module
\\
 & Interpolation Data Module
\\
Software Decision Module & Interpolation Module
\\
\bottomrule
\caption{Module Hierarchy}
\label{Table:MH}
\end{longtable}
\section{Module Decomposition}
\label{Sec:MD}
Modules are decomposed according to the principle of ``information hiding" proposed by Parnas. The Secrets field in a module decomposition is a brief statement of the design decision hidden by the module. The Services field specifies what the module will do without documenting how to do it. For each module, a suggestion for the implementing software is given under the Implemented By title. If the entry is OS, this means that the module is provided by the operating system. If the entry is Glass-BR, this means that the module is provided by the Glass-BR program. Only the leaf modules in the hierarchy have to be implemented. If a dash (--) is shown, this means that the module is not a leaf and will not have to be implemented. Whether or not this module is implemented depends on the programming language selected.
\subsection{Hardware Hiding Module (M\ref{Mhardwarehiding})}
\label{Sec:HHM()}
\begin{description}
\item[Secrets:]The data structure and algorithm used to implement the virtual hardware.
\item[Services:]Serves as a virtual hardware used by the rest of the system. This module provides the interface between the hardware and the software. So, the system can use it to display outputs or to accept inputs.
\item[Implemented By:]OS
\end{description}
\subsection{Behaviour Hiding Module}
\label{Sec:BHM}
\begin{description}
\item[Secrets:]The contents of the required behaviors.
\item[Services:]Includes programs that provide externally visible behaviour of the system as specified in the software requirements specification (SRS) documents. This module serves as a communication layer between the hardware-hiding module and the software decision module. The programs in this module will need to change if there are changes in the SRS.
\item[Implemented By:]--
\end{description}
\subsection{Input Format Module (M\ref{Minputformat})}
\label{Sec:IFM()}
\begin{description}
\item[Secrets:]The format and structure of the input data.
\item[Services:]Converts the input data into the data structure used by the input parameters module.
\item[Implemented By:]Glass-BR
\end{description}
\subsection{Input Parameters Module (M\ref{Minputparameters})}
\label{Sec:IPM()}
\begin{description}
\item[Secrets:]The format and structure of the input parameters.
\item[Services:]Stores the parameters needed for the program, including material properties, processing conditions, and numerical parameters. The values can be read as needed. This module knows how many parameters it stores.
\item[Implemented By:]Glass-BR
\end{description}
\subsection{Input Constraints Module (M\ref{Minputconstraints})}
\label{Sec:ICM()}
\begin{description}
\item[Secrets:]The constraints on the input data.
\item[Services:]Defines the constraints on the input data and gives an error if a constraint is violated.
\item[Implemented By:]Glass-BR
\end{description}
\subsection{Output Format Module (M\ref{Moutputformat})}
\label{Sec:OFM()}
\begin{description}
\item[Secrets:]The format and structure of the output data.
\item[Services:]Outputs the results of the calculations, including the input parameters, the demand, the capacity, the probability of breakage, and both safety requirements.
\item[Implemented By:]Glass-BR
\end{description}
\subsection{Derived Values Module (M\ref{Mderivedvalues})}
\label{Sec:DVM()}
\begin{description}
\item[Secrets:]The transformations from initial inputs to derived quantities.
\item[Services:]Defines the equations transforming the initial inputs into derived quantities.
\item[Implemented By:]Glass-BR
\end{description}
\subsection{Calculations Module (M\ref{Mcalculations})}
\label{Sec:CM()}
\begin{description}
\item[Secrets:]The equations for predicting the probability of glass breakage, capacity, and demand, using the input parameters.
\item[Services:]Defines the equations for solving for the probability of glass breakage, demand, and capacity using the parameters in the input parameters module.
\item[Implemented By:]Glass-BR
\end{description}
\subsection{Control Module (M\ref{Mcontrol})}
\label{Sec:CM()}
\begin{description}
\item[Secrets:]The algorithm for coordinating the running of the program.
\item[Services:]Provides the main program.
\item[Implemented By:]Glass-BR
\end{description}
\subsection{Interpolation Data Module (M\ref{Minterpolationdata})}
\label{Sec:IDM()}
\begin{description}
\item[Secrets:]The format and structure of the data used for interpolation.
\item[Services:]Converts the input interpolation data into the data structure used by the interpolation module.
\item[Implemented By:]Glass-BR
\end{description}
\subsection{Software Decision Module}
\label{Sec:SDM}
\begin{description}
\item[Secrets:]The design decision based on mathematical theorems, physical facts, or programming considerations. The secrets of this module are not described in the SRS.
\item[Services:]Includes data structures and algorithms used in the system that do not provide direct interaction with the user.
\item[Implemented By:]--
\end{description}
\subsection{Interpolation Module (M\ref{Minterpolation})}
\label{Sec:IM()}
\begin{description}
\item[Secrets:]The interpolation algorithm.
\item[Services:]Provides the equations that take the input parameters and interpolation data and return an interpolated value.
\item[Implemented By:]Glass-BR
\end{description}
\section{Traceability Matrix}
\label{Sec:TM}
This section shows two traceability matrices: between the modules and the requirements in Table~\ref{Table:TBRaM} and between the modules and the likely changes in Table~\ref{Table:TBLCaM}.
\begin{longtable}{l l}
\toprule
Requirement & Modules
\\
\midrule
R1 & M\ref{Mhardwarehiding}, M\ref{Minputformat}, M\ref{Minputparameters}, M\ref{Mcontrol}
\\
R2 & M\ref{Minputformat}, M\ref{Minputparameters}
\\
R3 & M\ref{Minputconstraints}
\\
R4 & M\ref{Moutputformat}
\\
R5 & M\ref{Moutputformat}, M\ref{Mcalculations}
\\
R6 & M\ref{Moutputformat}
\\
\bottomrule
\caption{Trace Between Requirements and Modules}
\label{Table:TBRaM}
\end{longtable}
\begin{longtable}{l l}
\toprule
Likely Change & Modules
\\
\midrule
LC\ref{LChardware} & M\ref{Mhardwarehiding}
\\
LC\ref{LCinput} & M\ref{Minputformat}
\\
LC\ref{LCparameters} & M\ref{Minputparameters}
\\
LC\ref{LCoutput} & M\ref{Moutputformat}
\\
LC\ref{LCequations} & M\ref{Mcalculations}
\\
LC\ref{LCcontrol} & M\ref{Mcontrol}
\\
LC\ref{LCinterpd} & M\ref{Minterpolationdata}
\\
LC\ref{LCinterp} & M\ref{Minterpolation}
\\
\bottomrule
\caption{Trace Between Likely Changes and Modules}
\label{Table:TBLCaM}
\end{longtable}
\section{Uses Hierarchy}
\label{Sec:UH}
In this section, the uses hierarchy between modules is provided. Parnas said of two programs A and B that A uses B if correct execution of B may be necessary for A to complete the task described in its specification. That is, A uses B if there exist situations in which the correct functioning of A depends upon the availability of a correct implementation of B. Figure~\ref{Figure:UsesHierarchy} illustrates the uses hierarchy between the modules. The graph is a directed acyclic graph (DAG). Each level of the hierarchy offers a testable and usable subset of the system, and modules in the higher level of the hierarchy are essentially simpler because they use modules from the lower levels.
\begin{figure}
\centering
\resizebox{\textwidth}{!}{
\tikz [>=stealth, shorten >=1pt]
\graph [layered layout, components go right top aligned, minimum layers=3, nodes={ draw, thick, align=center, inner xsep=0.5em, inner ysep=0.5em, text width=4em, minimum height=5em, font=\scriptsize, fill=white, text opacity=1, fill opacity=0.8, typeset={\tikzgraphnodetext\\M\ref{\tikzgraphnodename}}}, edges={thick, rounded corners}]
{
Minputformat/Input Format Module -> Mhardwarehiding/Hardware Hiding Module;
Minputformat/Input Format Module -> Minputparameters/Input Parameters Module;
Minputparameters/Input Parameters Module -> Minputconstraints/Input Constraints Module;
Moutputformat/Output Format Module -> Mhardwarehiding/Hardware Hiding Module;
Moutputformat/Output Format Module -> Minputparameters/Input Parameters Module;
Mderivedvalues/Derived Values Module -> Minputparameters/Input Parameters Module;
Mcalculations/Calculations Module -> Minputparameters/Input Parameters Module;
Mcontrol/Control Module -> Minputformat/Input Format Module;
Mcontrol/Control Module -> Minputparameters/Input Parameters Module;
Mcontrol/Control Module -> Minputconstraints/Input Constraints Module;
Mcontrol/Control Module -> Mderivedvalues/Derived Values Module;
Mcontrol/Control Module -> Mcalculations/Calculations Module;
Mcontrol/Control Module -> Minterpolation/Interpolation Module;
Mcontrol/Control Module -> Moutputformat/Output Format Module;
Minterpolation/Interpolation Module -> Minterpolationdata/Interpolation Data Module;
};
}
\caption{Uses Hierarchy}
\label{Figure:UsesHierarchy}
\end{figure}
\end{document}
