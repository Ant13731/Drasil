\documentclass[12pt]{article}
\usepackage{fullpage}
\usepackage{hyperref}
\hypersetup{bookmarks=true,colorlinks=true,linkcolor=red,citecolor=blue,filecolor=magenta,urlcolor=cyan}
\usepackage{amsmath}
\usepackage{longtable}
\usepackage{booktabs}
\usepackage{caption}
\usepackage{luatex85}
\def\pgfsysdriver{pgfsys-pdftex.def}
\usepackage{tikz}
\usetikzlibrary{arrows.meta}
\usetikzlibrary{graphs}
\usetikzlibrary{graphdrawing}
\usegdlibrary{layered}
\newcounter{lcnum}
\newcommand{\lcthelcnum}{LC\thelcnum}
\newcounter{ucnum}
\newcommand{\uctheucnum}{UC\theucnum}
\newcounter{modnum}
\newcommand{\mthemodnum}{M\themodnum}
\title{Module Guide for Slope Stability Analysis}
\author{Henry Frankis}
\begin{document}
\maketitle
\tableofcontents
\newpage
\section{Introduction}
\label{Sec:I}
Decomposing a system into modules is a commonly accepted approach to developing software.  A module is a work assignment for a programmer or programming team. In the best practices for scientific computing, Wilson et al advise a modular design, but are silent on the criteria to use to decompose the software into modules.  We advocate a decomposition based on the principle of information hiding. This principle supports design for change, because the ``secrets" that each module hides represent likely future changes.  Design for change is valuable in SC, where modifications are frequent, especially during initial development as the solution space is explored.
Our design follows the rules laid out by Parnas, as follows:
\begin{enumerate}
\item{System details that are likely to change independently should be the secrets of separate modules.}
\item{Any other program that requires information stored in a module's data structures must obtain it by calling access programs belonging to that module.}
\end{enumerate}
After completing the first stage of the design, the Software Requirements Specification (SRS), the Module Guide (MG) is developed. The MG specifies the modular structure of the system and is intended to allow both designers and maintainers to easily identify the parts of the software.  The potential readers of this document are as follows:
\begin{enumerate}
\item{New project members: This document can be a guide for a new project member to easily understand the overall structure and quickly find the relevant modules they are searching for.}
\item{Maintainers: The hierarchical structure of the module guide improves the maintainers' understanding when they need to make changes to the system. It is important for a maintainer to update the relevant sections of the document after changes have been made.}
\item{Designers: Once the module guide has been written, it can be used to check for consistency, feasibility and flexibility. Designers can verify the system in various ways, such as consistency among modules, feasibility of the decomposition, and flexibility of the design.}
\end{enumerate}
Section~\ref{Sec:LaUC}  lists the likely and unlikely changes of the software requirements. Section~\ref{Sec:MH}  summarizes the module decomposition that was constructed according to the likely changes. Section~\ref{Sec:MD}  gives a detailed description of the modules. Section~\ref{Sec:TM}  includes two traceability matrices. One checks the completeness of the design against the requirements provided in the SRS. The other shows the relation between anticipated changes and the modules. Section~\ref{Sec:UH}  describes the use relation between modules.
\section{Likely and Unlikely Changes}
\label{Sec:LaUC}
This section lists possible changes to the system. According to the likeliness of the change, the possible changes are classified into two categories. Likely changes are listed in Section~\ref{Sec:LC} and unlikely changes are listed in Section~\ref{Sec:UC}
\subsection{Likely Changes}
\label{Sec:LC}
Likely changes are the source of the information that is to be hidden inside the modules. Ideally, changing one of the likely changes will only require changing the one module that hides the associated decision. The approach adapted here is called design for change.
\begin{description}
\item[\refstepcounter{lcnum}\lcthelcnum\label{LChardware}:]The specific hardware on which the software is running.
\end{description}
\begin{description}
\item[\refstepcounter{lcnum}\lcthelcnum\label{LCcontrol}:]The algorithm for the overall operation procedure of the program.
\end{description}
\begin{description}
\item[\refstepcounter{lcnum}\lcthelcnum\label{LCinput}:]The format of the initial input data.
\end{description}
\begin{description}
\item[\refstepcounter{lcnum}\lcthelcnum\label{LCoutput}:]The format of the final output data.
\end{description}
\begin{description}
\item[\refstepcounter{lcnum}\lcthelcnum\label{LCgenalg}:]Algorithm for determining the critical slip surface.
\end{description}
\begin{description}
\item[\refstepcounter{lcnum}\lcthelcnum\label{LCkin}:]Criteria for a slip surface to be considered physically realistic/kinematically admissible.
\end{description}
\begin{description}
\item[\refstepcounter{lcnum}\lcthelcnum\label{LCslicer}:]The algorithm for creating the slice points for slip surface analysis.
\end{description}
\begin{description}
\item[\refstepcounter{lcnum}\lcthelcnum\label{LCfsweight}:]The weighting scheme for comparing slip surfaces factors of safety.
\end{description}
\begin{description}
\item[\refstepcounter{lcnum}\lcthelcnum\label{LCmp}:]The algorithm for implementation of the Morgenstern Price Solver.
\end{description}
\begin{description}
\item[\refstepcounter{lcnum}\lcthelcnum\label{LCrfem}:]The algorithm for implementation of the RFEM Solver.
\end{description}
\begin{description}
\item[\refstepcounter{lcnum}\lcthelcnum\label{LCprop}:]The algorithm for assigning soil properties to slices.
\end{description}
\begin{description}
\item[\refstepcounter{lcnum}\lcthelcnum\label{LCarray}:]The implementation for the sequence (array) data structure.
\end{description}
\begin{description}
\item[\refstepcounter{lcnum}\lcthelcnum\label{LCrand}:]The method of generating pseudo-random numbers.
\end{description}
\begin{description}
\item[\refstepcounter{lcnum}\lcthelcnum\label{LCplot}:]The method of displaying the final output.
\end{description}
\subsection{Unikely Changes}
\label{Sec:UC}
The module design should be as general as possible. However, a general system is more complex. Sometimes this complexity is not necessary. Fixing some design decisions at the system architecture stage can simplify the software design. If these decision should later need to be changed, then many parts of the design will potentially need to be modified. Hence, it is not intended that these decisions will be changed.  As an example, the model is assumed to follow the definition in the SRS.  If a new model is used, this will mean a change to the input format, fit parameters module, control, and output format modules.
\begin{description}
\item[\refstepcounter{ucnum}\uctheucnum\label{UCIO}:]Input/Output devices (Input: File and/or Keyboard, Output: File, Memory, and/or Screen).
\end{description}
\begin{description}
\item[\refstepcounter{ucnum}\uctheucnum\label{UCinputsource}:]There will always be a source of input data external to the software.
\end{description}
\begin{description}
\item[\refstepcounter{ucnum}\uctheucnum\label{UCoutput}:]Output data are displayed to the output device.
\end{description}
\section{Module Hierarchy}
\label{Sec:MH}
This section provides an overview of the module design. Modules are summarized in a hierarchy decomposed by secrets in Table~\ref{Table:MH}. The modules listed below, which are leaves in the hierarchy tree, are the modules that will actually be implemented.
\begin{description}
\item[\refstepcounter{modnum}\mthemodnum\label{Mhardwarehiding}:]Hardware Hiding Module
\end{description}
\begin{description}
\item[\refstepcounter{modnum}\mthemodnum\label{Mcontrol}:]Control Module
\end{description}
\begin{description}
\item[\refstepcounter{modnum}\mthemodnum\label{Minputformat}:]Input Format Module
\end{description}
\begin{description}
\item[\refstepcounter{modnum}\mthemodnum\label{Moutputformat}:]Output Format Module
\end{description}
\begin{description}
\item[\refstepcounter{modnum}\mthemodnum\label{Mgeneticalgorithm}:]Genetic Algorithm Module
\end{description}
\begin{description}
\item[\refstepcounter{modnum}\mthemodnum\label{Mkineticadmissibility}:]Kinetic Admissibility Module
\end{description}
\begin{description}
\item[\refstepcounter{modnum}\mthemodnum\label{Mslipslicer}:]Slip Slicer Module
\end{description}
\begin{description}
\item[\refstepcounter{modnum}\mthemodnum\label{Mslipweighting}:]Slip Weighting Module
\end{description}
\begin{description}
\item[\refstepcounter{modnum}\mthemodnum\label{Mmorgensternpricesolver}:]Morgenstern Price Solver Module
\end{description}
\begin{description}
\item[\refstepcounter{modnum}\mthemodnum\label{MRFEMsolver}:]RFEM Solver Module
\end{description}
\begin{description}
\item[\refstepcounter{modnum}\mthemodnum\label{Mslicepropertysorter}:]Slice Property Sorter Module
\end{description}
\begin{description}
\item[\refstepcounter{modnum}\mthemodnum\label{Msequencedatastructure}:]Sequence Data Structure Module
\end{description}
\begin{description}
\item[\refstepcounter{modnum}\mthemodnum\label{Mrandomnumbergenerator}:]Random Number Generator Module
\end{description}
\begin{description}
\item[\refstepcounter{modnum}\mthemodnum\label{Mplotting}:]Plotting Module
\end{description}
\begin{longtable}{l l}
\toprule
Level 1 & Level 2
\\
\midrule
Hardware Hiding Module & 
\\
Behaviour Hiding Module & Control Module
\\
 & Input Format Module
\\
 & Output Format Module
\\
 & Genetic Algorithm Module
\\
 & Kinetic Admissibility Module
\\
 & Slip Slicer Module
\\
 & Slip Weighting Module
\\
 & Morgenstern Price Solver Module
\\
 & RFEM Solver Module
\\
 & Slice Property Sorter Module
\\
Software Decision Module & Sequence Data Structure Module
\\
 & Random Number Generator Module
\\
 & Plotting Module
\\
\bottomrule
\caption{Module Hierarchy}
\label{Table:MH}
\end{longtable}
\section{Module Decomposition}
\label{Sec:MD}
Modules are decomposed according to the principle of ``information hiding" proposed by Parnas. The Secrets field in a module decomposition is a brief statement of the design decision hidden by the module. The Services field specifies what the module will do without documenting how to do it. For each module, a suggestion for the implementing software is given under the Implemented By title. If the entry is OS, this means that the module is provided by the operating system. If the entry is program, this means that the module is provided by the program. If the entry is MATLAB, this means that the module is provided by the MATLAB programming language. Only the leaf modules in the hierarchy have to be implemented. If a dash (--) is shown, this means that the module is not a leaf and will not have to be implemented. Whether or not this module is implemented depends on the programming language selected.
\subsection{Hardware Hiding Module (M\ref{Mhardwarehiding})}
\label{Sec:HHM()}
\begin{description}
\item[Secrets:]The data structure and algorithm used to implement the virtual hardware.
\item[Services:]Serves as a virtual hardware used by the rest of the system. This module provides the interface between the hardware and the software. So, the system can use it to display outputs or to accept inputs.
\item[Implemented By:]OS
\end{description}
\subsection{Behaviour Hiding Module}
\label{Sec:BHM}
\begin{description}
\item[Secrets:]The contents of the required behaviors.
\item[Services:]Includes programs that provide externally visible behaviour of the system as specified in the software requirements specification (SRS) documents. This module serves as a communication layer between the hardware-hiding module and the software decision module. The programs in this module will need to change if there are changes in the SRS.
\item[Implemented By:]--
\end{description}
\subsection{Control Module (M\ref{Mcontrol})}
\label{Sec:CM()}
\begin{description}
\item[Secrets:]The algorithm for coordinating the running of the program.
\item[Services:]Provides the main program.
\item[Implemented By:]program
\end{description}
\subsection{Input Format Module (M\ref{Minputformat})}
\label{Sec:IFM()}
\begin{description}
\item[Secrets:]The format and structure of the input data.
\item[Services:]Reads the input data from an input file, and/or prompted command line inputs. Input data includes the x,y coordinates of the slope, with a set of coordinates for each layer. For each layer it's soil properties of effective angle of friction, effective cohesion, dry unit weight, saturated unit weight, elastic modulus, and Poisson's ratio are stored in vectors of soil properties. If a piezometric surface exists in the slope it's coordinates and the unit weight of water are also included in the input. Lastly an expected range for the entrance and exit points of the critical slip surface are inputted.
\item[Implemented By:]program
\end{description}
\subsection{Output Format Module (M\ref{Moutputformat})}
\label{Sec:OFM()}
\begin{description}
\item[Secrets:]The format and structure of the output data.
\item[Services:]Outputs the results of the calculations, including the factor of safety for the critical slip calculated by the Morgenstern Price Module and Rigid Finite Element Method Module, and a plot of the critical slip surface on the slope geometry, with the showing the element displacements as calculated by the RFEM Module.
\item[Implemented By:]program
\end{description}
\subsection{Genetic Algorithm Module (M\ref{Mgeneticalgorithm})}
\label{Sec:GAM()}
\begin{description}
\item[Secrets:]Algorithm to identify the slip surface that has the minimum factor of safety, based on the inputs.
\item[Services:]Searches the slope for the critical slip surface with the minimum factor of safety
\item[Implemented By:]program
\end{description}
\subsection{Kinetic Admissibility Module (M\ref{Mkineticadmissibility})}
\label{Sec:KAM()}
\begin{description}
\item[Secrets:]Algorithm to determine if a given slip surface passes or fails a set of admissibility criteria.
\item[Services:]Some slip surfaces are physically unlikely or impossible to occur in a slip surface, such as slip surfaces containing sharp angles, or going above the slope surface. Ensures randomly generated or mutated slopes from the Genetic Algorithm Module are physically possible according to the criteria of the Kinematic Admissibility Module.
\item[Implemented By:]program
\end{description}
\subsection{Slip Slicer Module (M\ref{Mslipslicer})}
\label{Sec:SSM()}
\begin{description}
\item[Secrets:]Algorithm to determine the coordinates of where the slip surface interslice nodes occur.
\item[Services:]When preparing a slip surface for analysis by the Morgenstern Price Module or the RFEM Module, the x-coordinates defining the boundaries of the slices are identified and stored in a vector.
\item[Implemented By:]program
\end{description}
\subsection{Slip Weighting Module (M\ref{Mslipweighting})}
\label{Sec:SWM()}
\begin{description}
\item[Secrets:]The weighting for each slip surface in a set of slip surfaces, based on each slip surfaces factor of safety.
\item[Services:]Weights a set of slip surfaces generated by the Genetic Algorithm Module based on their factors of safety. A slip surface with a low factor of safety will have a high weight as it is more likely to be or to lead to generation of the critical slip surface.
\item[Implemented By:]program
\end{description}
\subsection{Morgenstern Price Solver Module (M\ref{Mmorgensternpricesolver})}
\label{Sec:MPSM()}
\begin{description}
\item[Secrets:]The factor of safety of a given slip surface.
\item[Services:]Calculates the factor of safety of a given slip surface, through implementation of a Morgenstern Price slope stability analysis method.
\item[Implemented By:]program
\end{description}
\subsection{RFEM Solver Module (M\ref{MRFEMsolver})}
\label{Sec:RSM()}
\begin{description}
\item[Secrets:]The algorithm to perform a Rigid Finite Element Method analysis of the slope.
\item[Services:]Calculate the global factor of safety, local slice factors of safety, and local slice displacements of a given slip surface under given conditions, through implementation of a rigid finite element slope stability analysis method.
\item[Implemented By:]program
\end{description}
\subsection{Slice Property Sorter Module (M\ref{Mslicepropertysorter})}
\label{Sec:SPSM()}
\begin{description}
\item[Secrets:]Algorithm to assigns soil properties to slices based on the location of the slice with respect to the different soil layers.
\item[Services:]When performing slip analysis with the RFEM Solver Module or Morgenstern Price Module, the base and interslice surfaces of each slice in the analysis requires a soil constant. Slice Property Sorter Module identifies which soil layer the surface is in to assign properties from that soil layer, and uses a weighting scheme when the surface crosses multiple soil layers.
\item[Implemented By:]program
\end{description}
\subsection{Software Decision Module}
\label{Sec:SDM}
\begin{description}
\item[Secrets:]The design decision based on mathematical theorems, physical facts, or programming considerations. The secrets of this module are not described in the SRS.
\item[Services:]Includes data structures and algorithms used in the system that do not provide direct interaction with the user.
\item[Implemented By:]--
\end{description}
\subsection{Sequence Data Structure Module (M\ref{Msequencedatastructure})}
\label{Sec:SDSM()}
\begin{description}
\item[Secrets:]The data structure for a sequence data type.
\item[Services:]Provides array manipulation, including building an array, accessing a specific entry, slicing an array etc.
\item[Implemented By:]MATLAB
\end{description}
\subsection{Random Number Generator Module (M\ref{Mrandomnumbergenerator})}
\label{Sec:RNGM()}
\begin{description}
\item[Secrets:]Pseudo-random number generation algorithm.
\item[Services:]Randomly produces numbers between 0 and 1, using a chaotic function with an external seed. Used when generating slip surfaces in the Genetic Algorithm Module.
\item[Implemented By:]MATLAB
\end{description}
\subsection{Plotting Module (M\ref{Mplotting})}
\label{Sec:PM()}
\begin{description}
\item[Secrets:]The data structures and algorithms for plotting data graphically
\item[Services:]Provides a plot function.
\item[Implemented By:]MATLAB
\end{description}
\section{Traceability Matrix}
\label{Sec:TM}
This section shows two traceability matrices: between the modules and the requirements in Table~\ref{Table:TBRaM} and between the modules and the likely changes in Table~\ref{Table:TBLCaM}.
\begin{longtable}{l l}
\toprule
Requirement & Modules
\\
\midrule
R1 & M\ref{Mcontrol}, M\ref{Minputformat}, M\ref{Msequencedatastructure}
\\
R2 & M\ref{Mgeneticalgorithm}, M\ref{Msequencedatastructure}, M\ref{Mrandomnumbergenerator}
\\
R3 & M\ref{Mkineticadmissibility}, M\ref{Msequencedatastructure}
\\
R4 & M\ref{Mgeneticalgorithm}, M\ref{Mslipslicer}, M\ref{Msequencedatastructure}
\\
R5 & M\ref{Mmorgensternpricesolver}, M\ref{Msequencedatastructure}
\\
R6 & M\ref{Mslipweighting}, M\ref{Msequencedatastructure}
\\
R7 & M\ref{Mgeneticalgorithm}, M\ref{Msequencedatastructure}, M\ref{Mrandomnumbergenerator}
\\
R8 & M\ref{Mgeneticalgorithm}, M\ref{Msequencedatastructure}
\\
R9 & M\ref{Mcontrol}, M\ref{Moutputformat}, M\ref{Mslipslicer}, M\ref{Msequencedatastructure}
\\
R10 & M\ref{Moutputformat}, M\ref{Mmorgensternpricesolver}, M\ref{MRFEMsolver}, M\ref{Msequencedatastructure}
\\
R11 & M\ref{Moutputformat}, M\ref{Mplotting}
\\
\bottomrule
\caption{Trace Between Requirements and Modules}
\label{Table:TBRaM}
\end{longtable}
\begin{longtable}{l l}
\toprule
Likely Change & Modules
\\
\midrule
LC\ref{LChardware} & M\ref{Mhardwarehiding}
\\
LC\ref{LCcontrol} & M\ref{Mcontrol}
\\
LC\ref{LCinput} & M\ref{Minputformat}
\\
LC\ref{LCoutput} & M\ref{Moutputformat}
\\
LC\ref{LCgenalg} & M\ref{Mgeneticalgorithm}
\\
LC\ref{LCkin} & M\ref{Mkineticadmissibility}
\\
LC\ref{LCslicer} & M\ref{Mslipslicer}
\\
LC\ref{LCfsweight} & M\ref{Mslipweighting}
\\
LC\ref{LCmp} & M\ref{Mmorgensternpricesolver}
\\
LC\ref{LCrfem} & M\ref{MRFEMsolver}
\\
LC\ref{LCprop} & M\ref{Mslicepropertysorter}
\\
LC\ref{LCarray} & M\ref{Msequencedatastructure}
\\
LC\ref{LCrand} & M\ref{Mrandomnumbergenerator}
\\
LC\ref{LCplot} & M\ref{Mplotting}
\\
\bottomrule
\caption{Trace Between Likely Changes and Modules}
\label{Table:TBLCaM}
\end{longtable}
\section{Uses Hierarchy}
\label{Sec:UH}
In this section, the uses hierarchy between modules is provided. Parnas said of two programs A and B that A uses B if correct execution of B may be necessary for A to complete the task described in its specification. That is, A uses B if there exist situations in which the correct functioning of A depends upon the availability of a correct implementation of B. Figure~\ref{Figure:UsesHierarchy} illustrates the uses hierarchy between the modules. The graph is a directed acyclic graph (DAG). Each level of the hierarchy offers a testable and usable subset of the system, and modules in the higher level of the hierarchy are essentially simpler because they use modules from the lower levels.
\begin{figure}
\centering
\resizebox{\textwidth}{!}{
\tikz [>=stealth, shorten >=1pt]
\graph [layered layout, components go right top aligned, minimum layers=3, nodes={ draw, thick, align=center, inner xsep=0.5em, inner ysep=0.5em, text width=4em, minimum height=5em, font=\scriptsize, fill=white, text opacity=1, fill opacity=0.8, typeset={\tikzgraphnodetext\\M\ref{\tikzgraphnodename}}}, edges={thick, rounded corners}]
{
Mcontrol/Control Module -> Minputformat/Input Format Module;
Mcontrol/Control Module -> Moutputformat/Output Format Module;
Mcontrol/Control Module -> Mgeneticalgorithm/Genetic Algorithm Module;
Minputformat/Input Format Module -> Mhardwarehiding/Hardware Hiding Module;
Moutputformat/Output Format Module -> Mplotting/Plotting Module;
Moutputformat/Output Format Module -> Mslipslicer/Slip Slicer Module;
Moutputformat/Output Format Module -> Mmorgensternpricesolver/Morgenstern Price Solver Module;
Moutputformat/Output Format Module -> MRFEMsolver/RFEM Solver Module;
Mgeneticalgorithm/Genetic Algorithm Module -> Mslipslicer/Slip Slicer Module;
Mgeneticalgorithm/Genetic Algorithm Module -> Mkineticadmissibility/Kinetic Admissibility Module;
Mgeneticalgorithm/Genetic Algorithm Module -> Mrandomnumbergenerator/Random Number Generator Module;
Mgeneticalgorithm/Genetic Algorithm Module -> Mslipweighting/Slip Weighting Module;
Mgeneticalgorithm/Genetic Algorithm Module -> Mmorgensternpricesolver/Morgenstern Price Solver Module;
Mmorgensternpricesolver/Morgenstern Price Solver Module -> Mslicepropertysorter/Slice Property Sorter Module;
MRFEMsolver/RFEM Solver Module -> Mslicepropertysorter/Slice Property Sorter Module;
Mplotting/Plotting Module -> Mhardwarehiding/Hardware Hiding Module;
};
}
\caption{Uses Hierarchy}
\label{Figure:UsesHierarchy}
\end{figure}
\end{document}
