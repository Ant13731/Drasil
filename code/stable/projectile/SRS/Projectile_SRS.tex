\documentclass[12pt]{article}
\usepackage{fontspec}
\usepackage{fullpage}
\usepackage{hyperref}
\hypersetup{bookmarks=true,colorlinks=true,linkcolor=red,citecolor=blue,filecolor=magenta,urlcolor=cyan}
\usepackage{amsmath}
\usepackage{amssymb}
\usepackage{mathtools}
\usepackage{unicode-math}
\usepackage{tabu}
\usepackage{longtable}
\usepackage{booktabs}
\usepackage{caption}
\usepackage{enumitem}
\setmathfont{Latin Modern Math}
\global\tabulinesep=1mm
\newlist{symbDescription}{description}{1}
\setlist[symbDescription]{noitemsep, topsep=0pt, parsep=0pt, partopsep=0pt}
\title{Software Requirements Specification for Projectile}
\author{Samuel J. Crawford, Brooks MacLachlan, and W. Spencer Smith}
\begin{document}
\maketitle
\tableofcontents
\newpage
\section{Reference Material}
\label{Sec:RefMat}
This section records information for easy reference.
\subsection{Table of Units}
\label{Sec:ToU}
The unit system used throughout is SI (Système International d'Unités). In addition to the basic units, several derived units are also used. For each unit, the table lists the symbol, a description and the SI name.
\begin{longtable}{l l}
\toprule
Symbol & Description
\\
\midrule
\endhead
\bottomrule
\caption{}
\label{Table:ToU}
\end{longtable}
\subsection{Table of Symbols}
\label{Sec:ToS}
The table that follows summarizes the symbols used in this document along with their units. Throughout the document, symbols in bold will represent vectors, and scalars otherwise. The symbols are listed in alphabetical order.
\begin{longtable}{l l l}
\toprule
Symbol & Description & Units
\\
\midrule
\endhead
\bottomrule
\caption{}
\label{Table:ToS}
\end{longtable}
\subsection{Abbreviations and Acronyms}
\label{Sec:TAbbAcc}
\begin{longtable}{l l}
\toprule
Abbreviation & Full Form
\\
\midrule
\endhead
2D & two-dimensional
\\
A & Assumption
\\
DD & Data Definition
\\
GD & General Definition
\\
GS & Goal Statement
\\
IM & Instance Model
\\
PS & Physical System Description
\\
R & Requirement
\\
SRS & Software Requirements Specification
\\
TM & Theoretical Model
\\
Uncert. & Typical Uncertainty
\\
\bottomrule
\caption{}
\label{Table:TAbbAcc}
\end{longtable}
\section{Specific System Description}
\label{Sec:SpecSystDesc}
This section first presents the problem description, which gives a high-level view of the problem to be solved. This is followed by the solution characteristics specification, which presents the assumptions, theories, and definitions that are used.
\subsection{Problem Description}
\label{Sec:ProbDesc}
A system is needed to efficiently and correctly predict the landing position of a projectile. Projectile is a computer program developed to interpret the inputs to give out the outputs which predict the landing position of a projectile.
\subsubsection{Goal Statements}
\label{Sec:GoalStmt}
Given the angle and speed of the projectile, the goal statements are:
\begin{itemize}
\item[calcLandingPosition:\phantomsection\label{calcPosition}]Calculate the landing position of the projectile.
\end{itemize}
\subsection{Solution Characteristics Specification}
\label{Sec:SolCharSpec}
The instance models that govern Projectile are presented in \hyperref[Sec:IMs]{Section: Instance Models}. The information to understand the meaning of the instance models and their derivation is also presented, so that the instance models can be verified.
\subsubsection{Assumptions}
\label{Sec:Assumps}
This section simplifies the original problem and helps in developing the theoretical model by filling in the missing information for the physical system. The numbers given in the square brackets refer to the Theoretical Models \hyperref[Sec:TMs]{Section: Theoretical Models}, General Definitions \hyperref[Sec:GDs]{Section: General Definitions}, Data Definitions \hyperref[Sec:DDs]{Section: Data Definitions}, Instance Models \hyperref[Sec:IMs]{Section: Instance Models}, Likely Changes \hyperref[Sec:LCs]{Section: Likely Changes}, or Unlikely Changes \hyperref[Sec:UCs]{Section: Unlikely Changes}, in which the respective assumption is used.
\begin{itemize}
\item[twoDMotion:\phantomsection\label{twoDMotion}]The projectile motion is in 2D. \hyperref[GD:velVec]{GD: velVec} \hyperref[GD:posVec]{GD: posVec}.
\item[cartSyst:\phantomsection\label{cartSyst}]A Cartesian coordinate system is used (from \hyperref[neglectCurv]{A: neglectCurv}). \hyperref[GD:velVec]{GD: velVec} \hyperref[GD:posVec]{GD: posVec}.
\item[yAxisPerpend:\phantomsection\label{yAxisPerpend}]The y-axis is perpendicular to the x-axis.
\item[rightHandAxes:\phantomsection\label{rightHandAxes}]The axes are defined using a right-handed coordinate system.
\item[launchOrigin:\phantomsection\label{launchOrigin}]The launcher is coincident with the origin. \hyperref[IM:calOfLandingTime]{IM: calOfLandingTime} \hyperref[IM:calOfLandingDist]{IM: calOfLandingDist}.
\item[targetXAxis:\phantomsection\label{targetXAxis}]The target lies on the x-axis (from \hyperref[neglectCurv]{A: neglectCurv}). \hyperref[IM:calOfLandingTime]{IM: calOfLandingTime}.
\item[posXDirection:\phantomsection\label{posXDirection}]The positive x-direction is from the launcher to the target. \hyperref[IM:calOfLandingDist]{IM: calOfLandingDist}.
\item[constAccel:\phantomsection\label{constAccel}]The acceleration is constant (from \hyperref[accelXZero]{A: accelXZero}, \hyperref[accelYGravity]{A: accelYGravity}, \hyperref[neglectDrag]{A: neglectDrag}, and \hyperref[freeFlight]{A: freeFlight}). \hyperref[GD:velVec]{GD: velVec} \hyperref[GD:posVec]{GD: posVec}.
\item[accelXZero:\phantomsection\label{accelXZero}]The acceleration in the x-direction is zero. \hyperref[IM:calOfLandingDist]{IM: calOfLandingDist} \hyperref[constAccel]{A: constAccel}.
\item[accelYGravity:\phantomsection\label{accelYGravity}]The acceleration in the y-direction is the acceleration due to gravity. \hyperref[IM:calOfLandingTime]{IM: calOfLandingTime} \hyperref[constAccel]{A: constAccel}.
\item[neglectDrag:\phantomsection\label{neglectDrag}]Air drag is neglected. \hyperref[constAccel]{A: constAccel}.
\item[constMass:\phantomsection\label{constMass}]The mass of the projectile is constant.
\item[pointMass:\phantomsection\label{pointMass}]The size and shape of the projectile are negligible, so that it can be modelled as a point mass. \hyperref[GD:rectVel]{GD: rectVel} \hyperref[GD:rectPos]{GD: rectPos}.
\item[freeFlight:\phantomsection\label{freeFlight}]The flight is free; there are no collisions during the trajectory of the projectile. \hyperref[constAccel]{A: constAccel}.
\item[neglectCurv:\phantomsection\label{neglectCurv}]The distance is small enough that the curvature of the Earth can be neglected. \hyperref[targetXAxis]{A: targetXAxis} \hyperref[cartSyst]{A: cartSyst}.
\item[timeStartZero:\phantomsection\label{timeStartZero}]Time starts at zero. \hyperref[GD:rectVel]{GD: rectVel} \hyperref[GD:rectPos]{GD: rectPos}.
\end{itemize}
\subsubsection{Theoretical Models}
\label{Sec:TMs}
This section focuses on the general equations and laws that Projectile is based on.
\par~

\noindent \begin{minipage}{\textwidth}
\begin{tabular}{p{0.2\textwidth} p{0.73\textwidth}}
\toprule \textbf{Refname} & \textbf{TM:acceleration}
\phantomsection 
\label{TM:acceleration}
\\ \midrule \\
Label & Acceleration
\\ \midrule \\
Equation & \begin{displaymath}
           \mathbf{a}=\frac{d\,\mathbf{v}}{d\,t}
           \end{displaymath}
\\ \midrule \\
Description & \begin{symbDescription}
              \item{$\mathbf{a}$ is the acceleration ($\frac{\text{m}}{\text{s}^{2}}$)}
              \item{$t$ is the time (s)}
              \item{$\mathbf{v}$ is the velocity ($\frac{\text{m}}{\text{s}}$)}
              \end{symbDescription}
\\ \midrule \\
Source & --
\\ \midrule \\
RefBy & \hyperref[GD:rectVel]{GD: rectVel} \hyperref[GD:rectVel]{GD: rectVel}.
\\ \bottomrule \end{tabular}
\end{minipage}
\par~

\noindent \begin{minipage}{\textwidth}
\begin{tabular}{p{0.2\textwidth} p{0.73\textwidth}}
\toprule \textbf{Refname} & \textbf{TM:velocity}
\phantomsection 
\label{TM:velocity}
\\ \midrule \\
Label & Velocity
\\ \midrule \\
Equation & \begin{displaymath}
           \mathbf{v}=\frac{d\,\mathbf{p}}{d\,t}
           \end{displaymath}
\\ \midrule \\
Description & \begin{symbDescription}
              \item{$\mathbf{v}$ is the velocity ($\frac{\text{m}}{\text{s}}$)}
              \item{$t$ is the time (s)}
              \item{$\mathbf{p}$ is the position (m)}
              \end{symbDescription}
\\ \midrule \\
Source & --
\\ \midrule \\
RefBy & \hyperref[GD:rectPos]{GD: rectPos} \hyperref[GD:rectPos]{GD: rectPos}.
\\ \bottomrule \end{tabular}
\end{minipage}
\subsubsection{General Definitions}
\label{Sec:GDs}
This section collects the laws and equations that will be used to build the instance models.
\par~

\noindent \begin{minipage}{\textwidth}
\begin{tabular}{p{0.2\textwidth} p{0.73\textwidth}}
\toprule \textbf{Refname} & \textbf{GD:rectVel}
\phantomsection 
\label{GD:rectVel}
\\ \midrule \\
Label & Rectilinear velocity as a function of time for constant acceleration
\\ \midrule \\
Units & $\frac{\text{m}}{\text{s}}$
\\ \midrule \\
Equation & \begin{displaymath}
           v={v^{i}}+{a^{c}} t
           \end{displaymath}
\\ \midrule \\
Description & \begin{symbDescription}
              \item{$v$ is the speed ($\frac{\text{m}}{\text{s}}$)}
              \item{${v^{i}}$ is the initial speed ($\frac{\text{m}}{\text{s}}$)}
              \item{${a^{c}}$ is the constant acceleration ($\frac{\text{m}}{\text{s}^{2}}$)}
              \item{$t$ is the time (s)}
              \end{symbDescription}
\\ \midrule \\
Notes & 
\\ \midrule \\
Source & --
\\ \midrule \\
RefBy & \hyperref[GD:velVec]{GD: velVec} \hyperref[GD:rectPos]{GD: rectPos}.
\\ \bottomrule \end{tabular}
\end{minipage}
Detailed derivation of rectilinear velocity:
Assume we have rectilinear motion of a particle (of negligible size and shape \hyperref[pointMass]{A: pointMass}); that is, motion in a straight line. The velocity is $v$ and the acceleration is $a$. The motion in \hyperref[TM:acceleration]{TM: acceleration} is now one-dimensional with a constant acceleration, represented by ${a^{c}}$. The initial velocity (at $t=0$, from \hyperref[timeStartZero]{A: timeStartZero}) is represented by ${v^{i}}$. From \hyperref[TM:acceleration]{TM: acceleration}, using the above symbols we have:
\begin{displaymath}
{a^{c}}=\frac{d\,v}{d\,t}
\end{displaymath}
Rearranging and integrating, we have:
\begin{displaymath}
\int_{{v^{i}}}^{v}{1}\,dv=\int_{0}^{t}{{a^{c}}}\,dt
\end{displaymath}
Performing the integration, we have:
\begin{displaymath}
v={v^{i}}+{a^{c}} t
\end{displaymath}
\par~

\noindent \begin{minipage}{\textwidth}
\begin{tabular}{p{0.2\textwidth} p{0.73\textwidth}}
\toprule \textbf{Refname} & \textbf{GD:rectPos}
\phantomsection 
\label{GD:rectPos}
\\ \midrule \\
Label & Rectilinear position as a function of time for constant acceleration
\\ \midrule \\
Units & m
\\ \midrule \\
Equation & \begin{displaymath}
           p={p^{i}}+{v^{i}} t+\frac{{a^{c}} t^{2}}{2}
           \end{displaymath}
\\ \midrule \\
Description & \begin{symbDescription}
              \item{$p$ is the scalar position (m)}
              \item{${p^{i}}$ is the initial position (m)}
              \item{${v^{i}}$ is the initial speed ($\frac{\text{m}}{\text{s}}$)}
              \item{$t$ is the time (s)}
              \item{${a^{c}}$ is the constant acceleration ($\frac{\text{m}}{\text{s}^{2}}$)}
              \end{symbDescription}
\\ \midrule \\
Notes & 
\\ \midrule \\
Source & --
\\ \midrule \\
RefBy & \hyperref[GD:posVec]{GD: posVec}.
\\ \bottomrule \end{tabular}
\end{minipage}
Detailed derivation of rectilinear position:
Assume we have rectilinear motion of a particle (of negligible size and shape \hyperref[pointMass]{A: pointMass}); that is, motion in a straight line. The position is $p$ and the velocity is $v$. The motion in \hyperref[TM:velocity]{TM: velocity} is now one-dimensional. The initial position (at $t=0$, from \hyperref[timeStartZero]{A: timeStartZero}) is represented by ${p^{i}}$. From \hyperref[TM:velocity]{TM: velocity}, using the above symbols we have:
\begin{displaymath}
v=\frac{d\,p}{d\,t}
\end{displaymath}
Rearranging and integrating, we have:
\begin{displaymath}
\int_{{p^{i}}}^{p}{1}\,dp=\int_{0}^{t}{v}\,dt
\end{displaymath}
From \hyperref[GD:rectVel]{GD: rectVel} we can replace $v$:
\begin{displaymath}
\int_{{p^{i}}}^{p}{1}\,dp=\int_{0}^{t}{{v^{i}}+{a^{c}} t}\,dt
\end{displaymath}
Performing the integration, we have:
\begin{displaymath}
p={p^{i}}+{v^{i}} t+\frac{{a^{c}} t^{2}}{2}
\end{displaymath}
\par~

\noindent \begin{minipage}{\textwidth}
\begin{tabular}{p{0.2\textwidth} p{0.73\textwidth}}
\toprule \textbf{Refname} & \textbf{GD:velVec}
\phantomsection 
\label{GD:velVec}
\\ \midrule \\
Label & Velocity vector as a function of time
\\ \midrule \\
Units & $\frac{\text{m}}{\text{s}}$
\\ \midrule \\
Equation & \begin{displaymath}
           \mathbf{v}=\begin{bmatrix}
{{v_{x}}^{i}}+{{a_{x}}^{c}} t\\
{{v_{y}}^{i}}+{{a_{y}}^{c}} t
\end{bmatrix}
           \end{displaymath}
\\ \midrule \\
Description & \begin{symbDescription}
              \item{$\mathbf{v}$ is the velocity ($\frac{\text{m}}{\text{s}}$)}
              \item{${{v_{x}}^{i}}$ is the x-component of initial velocity ($\frac{\text{m}}{\text{s}}$)}
              \item{${{a_{x}}^{c}}$ is the x-component of constant acceleration ($\frac{\text{m}}{\text{s}^{2}}$)}
              \item{$t$ is the time (s)}
              \item{${{v_{y}}^{i}}$ is the y-component of initial velocity ($\frac{\text{m}}{\text{s}}$)}
              \item{${{a_{y}}^{c}}$ is the y-component of constant acceleration ($\frac{\text{m}}{\text{s}^{2}}$)}
              \end{symbDescription}
\\ \midrule \\
Notes & 
\\ \midrule \\
Source & --
\\ \midrule \\
RefBy & 
\\ \bottomrule \end{tabular}
\end{minipage}
Detailed derivation of velocity vector:
For a two-dimensional Cartesian coordinate system (\hyperref[twoDMotion]{A: twoDMotion} and \hyperref[cartSyst]{A: cartSyst}), we can represent the velocity vector as $\mathbf{v}=\begin{bmatrix}
{v_{x}}\\
{v_{y}}
\end{bmatrix}$ and the acceleration vector as $\mathbf{a}=\begin{bmatrix}
{a_{x}}\\
{a_{y}}
\end{bmatrix}$. The acceleration is assumed to be constant (\hyperref[constAccel]{A: constAccel}) and the constant acceleration vector is represented as ${\mathbf{a}^{c}}=\begin{bmatrix}
{{a_{x}}^{c}}\\
{{a_{y}}^{c}}
\end{bmatrix}$. The initial velocity (at $t=0$) is represented by ${\mathbf{v}^{i}}=\begin{bmatrix}
{{v_{x}}^{i}}\\
{{v_{y}}^{i}}
\end{bmatrix}$. Since we have a Cartesian coordinate system, \hyperref[GD:rectVel]{GD: rectVel} can be applied to each coordinate of the velocity vector to yield:
\begin{displaymath}
\mathbf{v}=\begin{bmatrix}
{{v_{x}}^{i}}+{{a_{x}}^{c}} t\\
{{v_{y}}^{i}}+{{a_{y}}^{c}} t
\end{bmatrix}
\end{displaymath}
\par~

\noindent \begin{minipage}{\textwidth}
\begin{tabular}{p{0.2\textwidth} p{0.73\textwidth}}
\toprule \textbf{Refname} & \textbf{GD:posVec}
\phantomsection 
\label{GD:posVec}
\\ \midrule \\
Label & Position vector as a function of time
\\ \midrule \\
Units & m
\\ \midrule \\
Equation & \begin{displaymath}
           \mathbf{p}=\begin{bmatrix}
{{p_{x}}^{i}}+{{v_{x}}^{i}} t+\frac{{{a_{x}}^{c}} t^{2}}{2}\\
{{p_{y}}^{i}}+{{v_{y}}^{i}} t+\frac{{{a_{y}}^{c}} t^{2}}{2}
\end{bmatrix}
           \end{displaymath}
\\ \midrule \\
Description & \begin{symbDescription}
              \item{$\mathbf{p}$ is the position (m)}
              \item{${{p_{x}}^{i}}$ is the x-component of initial position ($\frac{\text{m}}{\text{s}}$)}
              \item{${{v_{x}}^{i}}$ is the x-component of initial velocity ($\frac{\text{m}}{\text{s}}$)}
              \item{$t$ is the time (s)}
              \item{${{a_{x}}^{c}}$ is the x-component of constant acceleration ($\frac{\text{m}}{\text{s}^{2}}$)}
              \item{${{p_{y}}^{i}}$ is the y-component of initial position ($\frac{\text{m}}{\text{s}}$)}
              \item{${{v_{y}}^{i}}$ is the y-component of initial velocity ($\frac{\text{m}}{\text{s}}$)}
              \item{${{a_{y}}^{c}}$ is the y-component of constant acceleration ($\frac{\text{m}}{\text{s}^{2}}$)}
              \end{symbDescription}
\\ \midrule \\
Notes & 
\\ \midrule \\
Source & --
\\ \midrule \\
RefBy & \hyperref[IM:calOfLandingTime]{IM: calOfLandingTime} \hyperref[IM:calOfLandingDist]{IM: calOfLandingDist}.
\\ \bottomrule \end{tabular}
\end{minipage}
Detailed derivation of position vector:
For a two-dimensional Cartesian coordinate system (\hyperref[twoDMotion]{A: twoDMotion} and \hyperref[cartSyst]{A: cartSyst}), we can represent the position vector as $\mathbf{p}=\begin{bmatrix}
{p_{x}}\\
{p_{y}}
\end{bmatrix}$, the velocity vector as $\mathbf{v}=\begin{bmatrix}
{v_{x}}\\
{v_{y}}
\end{bmatrix}$, and the acceleration vector as $\mathbf{a}=\begin{bmatrix}
{a_{x}}\\
{a_{y}}
\end{bmatrix}$. The acceleration is assumed to be constant (\hyperref[constAccel]{A: constAccel}) and the constant acceleration vector is represented as ${\mathbf{a}^{c}}=\begin{bmatrix}
{{a_{x}}^{c}}\\
{{a_{y}}^{c}}
\end{bmatrix}$. The initial velocity (at $t=0$) is represented by ${\mathbf{v}^{i}}=\begin{bmatrix}
{{v_{x}}^{i}}\\
{{v_{y}}^{i}}
\end{bmatrix}$. Since we have a Cartesian coordinate system, \hyperref[GD:rectPos]{GD: rectPos} can be applied to each coordinate of the position vector to yield:
\begin{displaymath}
\mathbf{p}=\begin{bmatrix}
{{p_{x}}^{i}}+{{v_{x}}^{i}} t+\frac{{{a_{x}}^{c}} t^{2}}{2}\\
{{p_{y}}^{i}}+{{v_{y}}^{i}} t+\frac{{{a_{y}}^{c}} t^{2}}{2}
\end{bmatrix}
\end{displaymath}
\subsubsection{Data Definitions}
\label{Sec:DDs}
This section collects and defines all the data needed to build the instance models.
\par~

\noindent \begin{minipage}{\textwidth}
\begin{tabular}{p{0.2\textwidth} p{0.73\textwidth}}
\toprule \textbf{Refname} & \textbf{DD:speedIX}
\phantomsection 
\label{DD:speedIX}
\\ \midrule \\
Label & X-component of initial velocity
\\ \midrule \\
Symbol & ${{v_{x}}^{i}}$
\\ \midrule \\
Units & $\frac{\text{m}}{\text{s}}$
\\ \midrule \\
Equation & \begin{displaymath}
           {{v_{x}}^{i}}=|{\mathbf{v}^{i}}| \cos\left(θ\right)
           \end{displaymath}
\\ \midrule \\
Description & \begin{symbDescription}
              \item{${{v_{x}}^{i}}$ is the x-component of initial velocity ($\frac{\text{m}}{\text{s}}$)}
              \item{${\mathbf{v}^{i}}$ is the initial velocity ($\frac{\text{m}}{\text{s}}$)}
              \item{$θ$ is the launch angle (${}^{\circ}$)}
              \end{symbDescription}
\\ \midrule \\
Source & --
\\ \midrule \\
RefBy & \hyperref[IM:calOfLandingDist]{IM: calOfLandingDist}.
\\ \bottomrule \end{tabular}
\end{minipage}
\par~

\noindent \begin{minipage}{\textwidth}
\begin{tabular}{p{0.2\textwidth} p{0.73\textwidth}}
\toprule \textbf{Refname} & \textbf{DD:speedIY}
\phantomsection 
\label{DD:speedIY}
\\ \midrule \\
Label & Y-component of initial velocity
\\ \midrule \\
Symbol & ${{v_{y}}^{i}}$
\\ \midrule \\
Units & $\frac{\text{m}}{\text{s}}$
\\ \midrule \\
Equation & \begin{displaymath}
           {{v_{y}}^{i}}=|{\mathbf{v}^{i}}| \sin\left(θ\right)
           \end{displaymath}
\\ \midrule \\
Description & \begin{symbDescription}
              \item{${{v_{y}}^{i}}$ is the y-component of initial velocity ($\frac{\text{m}}{\text{s}}$)}
              \item{${\mathbf{v}^{i}}$ is the initial velocity ($\frac{\text{m}}{\text{s}}$)}
              \item{$θ$ is the launch angle (${}^{\circ}$)}
              \end{symbDescription}
\\ \midrule \\
Source & --
\\ \midrule \\
RefBy & \hyperref[IM:calOfLandingTime]{IM: calOfLandingTime}.
\\ \bottomrule \end{tabular}
\end{minipage}
\subsubsection{Instance Models}
\label{Sec:IMs}
This section transforms the problem defined in \hyperref[Sec:ProbDesc]{Section: Problem Description} into one which is expressed in mathematical terms. It uses concrete symbols defined in \hyperref[Sec:DDs]{Section: Data Definitions} to replace the abstract symbols in the models identified in \hyperref[Sec:TMs]{Section: Theoretical Models} and \hyperref[Sec:GDs]{Section: General Definitions}.
\par~

\noindent \begin{minipage}{\textwidth}
\begin{tabular}{p{0.2\textwidth} p{0.73\textwidth}}
\toprule \textbf{Refname} & \textbf{IM:calOfLandingTime}
\phantomsection 
\label{IM:calOfLandingTime}
\\ \midrule \\
Label & Calculation of landing time
\\ \midrule \\
Input & ${v^{i}}$, $θ$
\\ \midrule \\
Output & $t'$
\\ \midrule \\
Input Constraints & \begin{displaymath}
                    {v^{i}}>0
                    \end{displaymath}
                    \begin{displaymath}
                    0<θ<90
                    \end{displaymath}
\\ \midrule \\
Output Constraints & \begin{displaymath}
                     t'>0
                     \end{displaymath}
\\ \midrule \\
Equation & \begin{displaymath}
           t'=\frac{2 {v^{i}} \sin\left(θ\right)}{g}
           \end{displaymath}
\\ \midrule \\
Description & \begin{symbDescription}
              \item{$t'$ is the launch duration (s)}
              \item{${v^{i}}$ is the initial speed ($\frac{\text{m}}{\text{s}}$)}
              \item{$θ$ is the launch angle (${}^{\circ}$)}
              \item{$g$ is the gravitational acceleration ($\frac{\text{m}}{\text{s}^{2}}$)}
              \end{symbDescription}
\\ \midrule \\
Notes & 
\\ \midrule \\
Source & --
\\ \midrule \\
RefBy & \hyperref[IM:calOfLandingDist]{IM: calOfLandingDist} \hyperref[calcValues]{FR: Calculate-Values}.
\\ \bottomrule \end{tabular}
\end{minipage}
Detailed derivation of launch duration:
We know that ${{p_{y}}^{i}}=0$ (\hyperref[launchOrigin]{A: launchOrigin}) and ${{a_{y}}^{c}}=-g$ (\hyperref[accelYGravity]{A: accelYGravity}). Substituting these values into the y-direction of \hyperref[GD:posVec]{GD: posVec} gives us:
\begin{displaymath}
{p_{y}}={{v_{y}}^{i}} t-\frac{g t^{2}}{2}
\end{displaymath}
To find the time that the projectile lands, we want to find the $t$ value ($t'$) where ${p_{y}}=0$ (since the target is on the x-axis from \hyperref[targetXAxis]{A: targetXAxis}). From the equation above we get:
\begin{displaymath}
{{v_{y}}^{i}} t'-\frac{g t'^{2}}{2}=0
\end{displaymath}
Divide by $t'$ (with the constraint $t'>0$) to get:
\begin{displaymath}
{{v_{y}}^{i}}-\frac{g t'}{2}=0
\end{displaymath}
Solving for $t'$ gives us:
\begin{displaymath}
t'=\frac{2 {{v_{y}}^{i}}}{g}
\end{displaymath}
From \hyperref[DD:speedIY]{DD: speedIY} we can replace ${{v_{y}}^{i}}$:
\begin{displaymath}
t'=\frac{2 {v^{i}} \sin\left(θ\right)}{g}
\end{displaymath}
\par~

\noindent \begin{minipage}{\textwidth}
\begin{tabular}{p{0.2\textwidth} p{0.73\textwidth}}
\toprule \textbf{Refname} & \textbf{IM:calOfLandingDist}
\phantomsection 
\label{IM:calOfLandingDist}
\\ \midrule \\
Label & Calculation of landing distance
\\ \midrule \\
Input & ${v^{i}}$, $θ$
\\ \midrule \\
Output & $p'$
\\ \midrule \\
Input Constraints & \begin{displaymath}
                    {v^{i}}>0
                    \end{displaymath}
                    \begin{displaymath}
                    0<θ<90
                    \end{displaymath}
\\ \midrule \\
Output Constraints & \begin{displaymath}
                     p'>0
                     \end{displaymath}
\\ \midrule \\
Equation & \begin{displaymath}
           p'=\frac{2 {v^{i}}^{2} \sin\left(θ\right) \cos\left(θ\right)}{g}
           \end{displaymath}
\\ \midrule \\
Description & \begin{symbDescription}
              \item{$p'$ is the landing position (m)}
              \item{${v^{i}}$ is the initial speed ($\frac{\text{m}}{\text{s}}$)}
              \item{$θ$ is the launch angle (${}^{\circ}$)}
              \item{$g$ is the gravitational acceleration ($\frac{\text{m}}{\text{s}^{2}}$)}
              \end{symbDescription}
\\ \midrule \\
Notes & The constraint $p'>0$ is from \hyperref[posXDirection]{A: posXDirection}.
\\ \midrule \\
Source & --
\\ \midrule \\
RefBy & \hyperref[calcValues]{FR: Calculate-Values}.
\\ \bottomrule \end{tabular}
\end{minipage}
Detailed derivation of landing position:
We know that ${{p_{x}}^{i}}=0$ (\hyperref[launchOrigin]{A: launchOrigin}) and ${{a_{x}}^{c}}=0$ (\hyperref[accelXZero]{A: accelXZero}). Substituting these values into the x-direction of \hyperref[GD:posVec]{GD: posVec} gives us:
\begin{displaymath}
{p_{x}}={{v_{x}}^{i}} t
\end{displaymath}
To find the horizontal distance travelled, we want to find the ${p_{x}}$ value ($p'$) at launch duration (from \hyperref[IM:calOfLandingTime]{IM: calOfLandingTime}):
\begin{displaymath}
p'=\frac{{{v_{x}}^{i}}\cdot{}2 {v^{i}} \sin\left(θ\right)}{g}
\end{displaymath}
From \hyperref[DD:speedIX]{DD: speedIX} we can replace ${{v_{x}}^{i}}$:
\begin{displaymath}
p'=\frac{{v^{i}} \cos\left(θ\right)\cdot{}2 {v^{i}} \sin\left(θ\right)}{g}
\end{displaymath}
Rearranging this equation gives us:
\begin{displaymath}
p'=\frac{2 {v^{i}}^{2} \sin\left(θ\right) \cos\left(θ\right)}{g}
\end{displaymath}
\par~

\noindent \begin{minipage}{\textwidth}
\begin{tabular}{p{0.2\textwidth} p{0.73\textwidth}}
\toprule \textbf{Refname} & \textbf{IM:shortIM}
\phantomsection 
\label{IM:shortIM}
\\ \midrule \\
Label & IsShort
\\ \midrule \\
Input & ${p_{target}}$, $p'$
\\ \midrule \\
Output & $isShort$
\\ \midrule \\
Input Constraints & \begin{displaymath}
                    {p_{target}}>0
                    \end{displaymath}
                    \begin{displaymath}
                    p'>0
                    \end{displaymath}
\\ \midrule \\
Output Constraints & 
\\ \midrule \\
Equation & \begin{displaymath}
           isShort={p_{target}}>p'
           \end{displaymath}
\\ \midrule \\
Description & \begin{symbDescription}
              \item{$isShort$ is the variable that is assigned true when the target position is greater than the landing position (Unitless)}
              \item{${p_{target}}$ is the target position (m)}
              \item{$p'$ is the landing position (m)}
              \end{symbDescription}
\\ \midrule \\
Notes & 
\\ \midrule \\
Source & --
\\ \midrule \\
RefBy & \hyperref[calcValues]{FR: Calculate-Values}.
\\ \bottomrule \end{tabular}
\end{minipage}
\par~

\noindent \begin{minipage}{\textwidth}
\begin{tabular}{p{0.2\textwidth} p{0.73\textwidth}}
\toprule \textbf{Refname} & \textbf{IM:offsetIM}
\phantomsection 
\label{IM:offsetIM}
\\ \midrule \\
Label & Offset
\\ \midrule \\
Input & ${p_{target}}$, $p'$
\\ \midrule \\
Output & ${d_{offset}}$
\\ \midrule \\
Input Constraints & \begin{displaymath}
                    {p_{target}}>0
                    \end{displaymath}
                    \begin{displaymath}
                    p'>0
                    \end{displaymath}
\\ \midrule \\
Output Constraints & 
\\ \midrule \\
Equation & \begin{displaymath}
           {d_{offset}}=|{p_{target}}-p'|
           \end{displaymath}
\\ \midrule \\
Description & \begin{symbDescription}
              \item{${d_{offset}}$ is the offset between the target position and the landing position (m)}
              \item{${p_{target}}$ is the target position (m)}
              \item{$p'$ is the landing position (m)}
              \end{symbDescription}
\\ \midrule \\
Notes & 
\\ \midrule \\
Source & --
\\ \midrule \\
RefBy & \hyperref[calcValues]{FR: Calculate-Values}.
\\ \bottomrule \end{tabular}
\end{minipage}
\par~

\noindent \begin{minipage}{\textwidth}
\begin{tabular}{p{0.2\textwidth} p{0.73\textwidth}}
\toprule \textbf{Refname} & \textbf{IM:hitIM}
\phantomsection 
\label{IM:hitIM}
\\ \midrule \\
Label & IsHit
\\ \midrule \\
Input & ${d_{offset}}$, ${p_{target}}$
\\ \midrule \\
Output & $isHit$
\\ \midrule \\
Input Constraints & \begin{displaymath}
                    {d_{offset}}>0
                    \end{displaymath}
                    \begin{displaymath}
                    {p_{target}}>0
                    \end{displaymath}
\\ \midrule \\
Output Constraints & 
\\ \midrule \\
Equation & \begin{displaymath}
           isHit={d_{offset}}<\frac{1}{50} {p_{target}}
           \end{displaymath}
\\ \midrule \\
Description & \begin{symbDescription}
              \item{$isHit$ is the variable that is assigned true when the landing position is within a degree of tolerance of the target position (Unitless)}
              \item{${d_{offset}}$ is the offset between the target position and the landing position (m)}
              \item{${p_{target}}$ is the target position (m)}
              \end{symbDescription}
\\ \midrule \\
Notes & 
\\ \midrule \\
Source & --
\\ \midrule \\
RefBy & \hyperref[outputValues]{FR: Output-Values} \hyperref[outputValues]{FR: Output-Values} \hyperref[calcValues]{FR: Calculate-Values}.
\\ \bottomrule \end{tabular}
\end{minipage}
\subsubsection{Data Constraints}
\label{Sec:DataConstraints}
\hyperref[Table:InDataConstraints]{Table:InDataConstraints} and \hyperref[Table:OutDataConstraints]{Table:OutDataConstraints} show the data constraints on the input and output variables, respectively. The column for physical constraints gives the physical limitations on the range of values that can be taken by the variable. The uncertainty column provides an estimate of the confidence with which the physical quantities can be measured. This information would be part of the input if one were performing an uncertainty quantification exercise. The constraints are conservative, to give the user of the model the flexibility to experiment with unusual situations. The column of typical values is intended to provide a feel for a common scenario.
\begin{longtable}{l l l l}
\toprule
Var & Physical Constraints & Typical Value & Uncert.
\\
\midrule
\endhead
${p_{target}}$ & ${p_{target}}>0$ & $100$ m & 10$\%$
\\
${v^{i}}$ & ${v^{i}}>0$ & $100$ $\frac{\text{m}}{\text{s}}$ & 10$\%$
\\
$θ$ & $0<θ<90$ & $45$ ${}^{\circ}$ & 10$\%$
\\
\bottomrule
\caption{Input Data Constraints}
\label{Table:InDataConstraints}
\end{longtable}
\begin{longtable}{l l}
\toprule
Var & Physical Constraints
\\
\midrule
\endhead
$p'$ & $p'>0$
\\
${d_{offset}}$ & ${d_{offset}}>0$
\\
\bottomrule
\caption{Output Data Constraints}
\label{Table:OutDataConstraints}
\end{longtable}
\subsubsection{Properties of a Correct Solution}
\label{Sec:CorSolProps}
FIXME.
\section{Requirements}
\label{Sec:Requirements}
This section provides the functional requirements, the tasks and behaviours that the software is expected to complete, and the non-functional requirements, the qualities that the software is expected to exhibit.
\subsection{Functional Requirements}
\label{Sec:FRs}
This section provides the functional requirements, the tasks and behaviours that the software is expected to complete.
\begin{itemize}
\item[Input-Parameters:\phantomsection\label{inputParams}]Input the quantities from \hyperref[Table:ReqInputs]{Table:ReqInputs}, which define the launch angle, launch speed, and target position.
\item[Verify-Parameters:\phantomsection\label{verifyParams}]Check the entered input parameters to ensure that they do not exceed the data constraints mentioned in \hyperref[Sec:DataConstraints]{Section: Data Constraints}. If any of the input parameters are out of bounds, an error message is displayed and the calculations stop.
\item[Calculate-Values:\phantomsection\label{calcValues}]Calculate the following quantities: $t'$ (from \hyperref[IM:calOfLandingTime]{IM: calOfLandingTime}), $p'$ (from \hyperref[IM:calOfLandingDist]{IM: calOfLandingDist}), $isShort$ (from \hyperref[IM:shortIM]{IM: shortIM}), ${d_{offset}}$ (from \hyperref[IM:offsetIM]{IM: offsetIM}), and $isHit$ (from \hyperref[IM:hitIM]{IM: hitIM}).
\item[Output-Values:\phantomsection\label{outputValues}]If $isHit$ (from \hyperref[IM:hitIM]{IM: hitIM}), output the message ``The target was hit.'' Otherwise, if $isShort$ (from \hyperref[IM:hitIM]{IM: hitIM}), output the message ``The projectile fell short.'' and ${d_{offset}}$. Otherwise, output the message ``The projectile went long.'' and ${d_{offset}}$.
\end{itemize}
\begin{longtable}{l l l}
\toprule
Symbol & Description & Units
\\
\midrule
\endhead
${p_{target}}$ & Target position & m
\\
${v^{i}}$ & Launch speed & $\frac{\text{m}}{\text{s}}$
\\
$θ$ & Launch angle & ${}^{\circ}$
\\
\bottomrule
\caption{Required Inputs following \hyperref[inputParams]{FR: Input-Parameters}}
\label{Table:ReqInputs}
\end{longtable}
\subsection{Non-Functional Requirements}
\label{Sec:NFRs}
This section provides the non-functional requirements, the qualities that the software is expected to exhibit.
\begin{itemize}
\item[Correct:\phantomsection\label{correct}]The outputs of the code have the properties described in \hyperref[Sec:CorSolProps]{Section: Properties of a Correct Solution}.
\item[Verifiable:\phantomsection\label{verifiable}]The code is tested with complete verification and validation plan.
\item[Understandable:\phantomsection\label{understandable}]The code is modularized with complete module guide and module interface specification.
\item[Reusable:\phantomsection\label{reusable}]The code is modularized.
\item[Maintainable:\phantomsection\label{maintainable}]The traceability between requirements, assumptions, theoretical models, general definitions, data definitions, instance models, likely changes, unlikely changes, and modules is completely recorded in traceability matrices in the SRS and module guide.
\item[Portable:\phantomsection\label{portable}]The code is able to be run in different environments.
\end{itemize}
\end{document}
