\documentclass[12pt]{article}
\usepackage{fontspec}
\usepackage{fullpage}
\usepackage{hyperref}
\hypersetup{bookmarks=true,colorlinks=true,linkcolor=red,citecolor=blue,filecolor=magenta,urlcolor=cyan}
\usepackage{amsmath}
\usepackage{amssymb}
\usepackage{mathtools}
\usepackage{unicode-math}
\usepackage{enumitem}
\usepackage{tabu}
\usepackage{longtable}
\usepackage{booktabs}
\setmathfont{Latin Modern Math}
\newlist{symbDescription}{description}{1}
\setlist[symbDescription]{noitemsep, topsep=0pt, parsep=0pt, partopsep=0pt}
\global\tabulinesep=1mm
\title{Software Requirements Specification for Projectile}
\author{Samuel J. Crawford}
\begin{document}
\maketitle
\tableofcontents
\newpage
\section{Specific System Description}
\label{Sec:SpecSystDesc}
This section first presents the problem description, which gives a high-level view of the problem to be solved. This is followed by the solution characteristics specification, which presents the assumptions, theories, and definitions that are used.
\subsection{Problem Description}
\label{Sec:ProbDesc}
A system is needed to efficiently and correctly predict the landing position of a projectile. Projectile is a computer program developed to interpret the inputs to give out the outputs which predict the landing position of a projectile.
\subsubsection{Goal Statements}
\label{Sec:GoalStmt}
Given the angle and speed of the projectile, the goal statements are:
\begin{itemize}
\item[calcLandingPosition:\phantomsection\label{calcPosition}]Calculate the landing position of the projectile.
\end{itemize}
\subsection{Solution Characteristics Specification}
\label{Sec:SolCharSpec}
The instance models that govern Projectile are presented in \hyperref[Sec:IMs]{Section: Instance Models}. The information to understand the meaning of the instance models and their derivation is also presented, so that the instance models can be verified.
\subsubsection{Assumptions}
\label{Sec:Assumps}
This section simplifies the original problem and helps in developing the theoretical model by filling in the missing information for the physical system. The numbers given in the square brackets refer to the Theoretical Models \hyperref[Sec:TMs]{Section: Theoretical Models}, General Definitions \hyperref[Sec:GDs]{Section: General Definitions}, Data Definitions \hyperref[Sec:DDs]{Section: Data Definitions}, Instance Models \hyperref[Sec:IMs]{Section: Instance Models}, Likely Changes \hyperref[Sec:LCs]{Section: Likely Changes}, or Unlikely Changes \hyperref[Sec:UCs]{Section: Unlikely Changes}, in which the respective assumption is used.
\begin{itemize}
\item[twoDMotion:\phantomsection\label{twoDMotion}]The projectile motion is in 2D. \hyperref[GD:velVec]{GD: velVec} \hyperref[GD:posVec]{GD: posVec}.
\item[cartSyst:\phantomsection\label{cartSyst}]A Cartesian coordinate system is used. \hyperref[GD:velVec]{GD: velVec} \hyperref[GD:posVec]{GD: posVec}.
\item[yAxisPerpend:\phantomsection\label{yAxisPerpend}]The y-axis is perpendicular to the x-axis.
\item[rightHandAxes:\phantomsection\label{rightHandAxes}]The axes are defined using a right-handed coordinate system.
\item[launchOrigin:\phantomsection\label{launchOrigin}]The launcher is coincident with the origin. \hyperref[GD:airTime]{GD: airTime}.
\item[targetXAxis:\phantomsection\label{targetXAxis}]The target lies on the x-axis. \hyperref[GD:airTime]{GD: airTime}.
\item[posXDirection:\phantomsection\label{posXDirection}]The positive x-direction is from the launcher to the target.
\item[constAccel:\phantomsection\label{constAccel}]The acceleration is constant. \hyperref[GD:velVec]{GD: velVec} \hyperref[GD:posVec]{GD: posVec}.
\item[accelXZero:\phantomsection\label{accelXZero}]The acceleration in the x-direction is zero. \hyperref[GD:distanceRefined]{GD: distanceRefined}.
\item[accelYGravity:\phantomsection\label{accelYGravity}]The acceleration in the y-direction is the acceleration due to gravity. \hyperref[GD:distance]{GD: distance}.
\item[neglectDrag:\phantomsection\label{neglectDrag}]Air drag is neglected.
\item[constMass:\phantomsection\label{constMass}]The mass of the projectile is constant.
\item[pointMass:\phantomsection\label{pointMass}]The size and shape of the projectile are negligible, so that it can be modelled as a point mass. \hyperref[GD:rectVel]{GD: rectVel}.
\item[freeFlight:\phantomsection\label{freeFlight}]The flight is free; there are no collisions during the trajectory of the projectile.
\item[timeStartZero:\phantomsection\label{timeStartZero}]Time starts at zero.
\end{itemize}
\subsubsection{Theoretical Models}
\label{Sec:TMs}
This section focuses on the general equations and laws that Projectile is based on.
\par~

\noindent \begin{minipage}{\textwidth}
\begin{tabular}{p{0.2\textwidth} p{0.73\textwidth}}
\toprule \textbf{Refname} & \textbf{TM:acceleration}
\phantomsection 
\label{TM:acceleration}
\\ \midrule \\
Label & Acceleration
\\ \midrule \\
Equation & \begin{displaymath}
           \mathbf{a}=\frac{d\,\mathbf{v}}{d\,t}
           \end{displaymath}
\\ \midrule \\
Description & \begin{symbDescription}
              \item{$\mathbf{a}$ is the acceleration ($\frac{\text{m}}{\text{s}^{2}}$)}
              \item{$t$ is the time (s)}
              \item{$\mathbf{v}$ is the velocity ($\frac{\text{m}}{\text{s}}$)}
              \end{symbDescription}
\\ \midrule \\
Source & --
\\ \midrule \\
RefBy & \hyperref[GD:rectVel]{GD: rectVel} \hyperref[GD:rectVel]{GD: rectVel}.
\\ \bottomrule \end{tabular}
\end{minipage}
\par~

\noindent \begin{minipage}{\textwidth}
\begin{tabular}{p{0.2\textwidth} p{0.73\textwidth}}
\toprule \textbf{Refname} & \textbf{TM:velocity}
\phantomsection 
\label{TM:velocity}
\\ \midrule \\
Label & Velocity
\\ \midrule \\
Equation & \begin{displaymath}
           \mathbf{v}=\frac{d\,\mathbf{r}}{d\,t}
           \end{displaymath}
\\ \midrule \\
Description & \begin{symbDescription}
              \item{$\mathbf{v}$ is the velocity ($\frac{\text{m}}{\text{s}}$)}
              \item{$t$ is the time (s)}
              \item{$\mathbf{r}$ is the displacement (m)}
              \end{symbDescription}
\\ \midrule \\
Source & --
\\ \midrule \\
RefBy & \hyperref[GD:rectPos]{GD: rectPos}.
\\ \bottomrule \end{tabular}
\end{minipage}
\subsubsection{General Definitions}
\label{Sec:GDs}
This section collects the laws and equations that will be used to build the instance models.
\par~

\noindent \begin{minipage}{\textwidth}
\begin{tabular}{p{0.2\textwidth} p{0.73\textwidth}}
\toprule \textbf{Refname} & \textbf{GD:rectVel}
\phantomsection 
\label{GD:rectVel}
\\ \midrule \\
Label & Rectilinear velocity as a function of time for constant acceleration
\\ \midrule \\
Units & $\frac{\text{m}}{\text{s}}$
\\ \midrule \\
Equation & \begin{displaymath}
           {v_{f}}={v_{i}}+{a^{c}} t
           \end{displaymath}
\\ \midrule \\
Description & \begin{symbDescription}
              \item{${v_{f}}$ is the final speed ($\frac{\text{m}}{\text{s}}$)}
              \item{${v_{i}}$ is the initial speed ($\frac{\text{m}}{\text{s}}$)}
              \item{${a^{c}}$ is the constant acceleration ($\frac{\text{m}}{\text{s}^{2}}$)}
              \item{$t$ is the time (s)}
              \end{symbDescription}
\\ \midrule \\
Notes & 
\\ \midrule \\
Source & --
\\ \midrule \\
RefBy & \hyperref[GD:velVec]{GD: velVec} \hyperref[GD:rectPos]{GD: rectPos} \hyperref[GD:airTime]{GD: airTime}.
\\ \bottomrule \end{tabular}
\end{minipage}
Detailed derivation of rectilinear velocity:
Assume we have rectilinear motion of a particle (of negligible size and shape \hyperref[pointMass]{A: pointMass}); that is, motion in a straight line. The velocity is $\mathbf{v}$ and the acceleration is $\mathbf{a}$. The motion in \hyperref[TM:acceleration]{TM: acceleration} is now one-dimensional with a constant acceleration, represented by ${a^{c}}$. The initial velocity (at $t=0$) is represented by ${\mathbf{v}^{i}}$. From \hyperref[TM:acceleration]{TM: acceleration}, using the above symbols we have:
\begin{displaymath}
{a^{c}}=\frac{d\,\mathbf{v}}{d\,t}
\end{displaymath}
Rearranging and integrating, we have:
\begin{displaymath}
\int_{{\mathbf{v}^{i}}}^{\mathbf{v}}{1}\,d\mathbf{v}=\int_{0}^{t}{{a^{c}}}\,dt
\end{displaymath}
Performing the integration, we have:
\begin{displaymath}
{v_{f}}={v_{i}}+{a^{c}} t
\end{displaymath}
\par~

\noindent \begin{minipage}{\textwidth}
\begin{tabular}{p{0.2\textwidth} p{0.73\textwidth}}
\toprule \textbf{Refname} & \textbf{GD:rectPos}
\phantomsection 
\label{GD:rectPos}
\\ \midrule \\
Label & Rectilinear position as a function of time for constant acceleration
\\ \midrule \\
Units & m
\\ \midrule \\
Equation & \begin{displaymath}
           \mathbf{p}={p^{i}}+{\mathbf{v}^{i}} t+\frac{{a^{c}} t^{2}}{2}
           \end{displaymath}
\\ \midrule \\
Description & \begin{symbDescription}
              \item{$\mathbf{p}$ is the position (m)}
              \item{${p^{i}}$ is the initial position (m)}
              \item{${\mathbf{v}^{i}}$ is the initial velocity ($\frac{\text{m}}{\text{s}}$)}
              \item{$t$ is the time (s)}
              \item{${a^{c}}$ is the constant acceleration ($\frac{\text{m}}{\text{s}^{2}}$)}
              \end{symbDescription}
\\ \midrule \\
Notes & 
\\ \midrule \\
Source & --
\\ \midrule \\
RefBy & \hyperref[GD:posVec]{GD: posVec}.
\\ \bottomrule \end{tabular}
\end{minipage}
Detailed derivation of rectilinear position:
From \hyperref[TM:velocity]{TM: velocity}, using the symbols ${a^{c}}$ for constant acceleration, ${\mathbf{v}^{i}}$ for initial velocity, and ${p^{i}}$ for initial position we have:.
\begin{displaymath}
\mathbf{v}=\frac{d\,\mathbf{p}}{d\,t}
\end{displaymath}
Rearranging and integrating, we have:
\begin{displaymath}
\int_{{p^{i}}}^{\mathbf{p}}{1}\,d\mathbf{p}=\int_{0}^{t}{\mathbf{v}}\,dt
\end{displaymath}
From \hyperref[GD:rectVel]{GD: rectVel} we can replace $\mathbf{v}$:
\begin{displaymath}
\int_{{p^{i}}}^{\mathbf{p}}{1}\,d\mathbf{p}=\int_{0}^{t}{{\mathbf{v}^{i}}+{a^{c}} t}\,dt
\end{displaymath}
Performing the integration, we have:
\begin{displaymath}
\mathbf{p}={p^{i}}+{\mathbf{v}^{i}} t+\frac{{a^{c}} t^{2}}{2}
\end{displaymath}
\par~

\noindent \begin{minipage}{\textwidth}
\begin{tabular}{p{0.2\textwidth} p{0.73\textwidth}}
\toprule \textbf{Refname} & \textbf{GD:velVec}
\phantomsection 
\label{GD:velVec}
\\ \midrule \\
Label & Velocity vector as a function of time
\\ \midrule \\
Units & $\frac{\text{m}}{\text{s}}$
\\ \midrule \\
Equation & \begin{displaymath}
           \mathbf{v}=\begin{bmatrix}
{{v_{x}}^{i}}+{{a_{x}}^{c}} t\\
{{v_{y}}^{i}}+{{a_{y}}^{c}} t
\end{bmatrix}
           \end{displaymath}
\\ \midrule \\
Description & \begin{symbDescription}
              \item{$\mathbf{v}$ is the velocity ($\frac{\text{m}}{\text{s}}$)}
              \item{${{v_{x}}^{i}}$ is the x-component of initial velocity ($\frac{\text{m}}{\text{s}}$)}
              \item{${{a_{x}}^{c}}$ is the x-component of constant acceleration ($\frac{\text{m}}{\text{s}^{2}}$)}
              \item{$t$ is the time (s)}
              \item{${{v_{y}}^{i}}$ is the y-component of initial velocity ($\frac{\text{m}}{\text{s}}$)}
              \item{${{a_{y}}^{c}}$ is the y-component of constant acceleration ($\frac{\text{m}}{\text{s}^{2}}$)}
              \end{symbDescription}
\\ \midrule \\
Notes & 
\\ \midrule \\
Source & --
\\ \midrule \\
RefBy & 
\\ \bottomrule \end{tabular}
\end{minipage}
Detailed derivation of velocity vector:
For a two-dimensional Cartesian coordinate system (\hyperref[twoDMotion]{A: twoDMotion} and \hyperref[cartSyst]{A: cartSyst}), we can represent the velocity vector as $\mathbf{v}=\begin{bmatrix}
{v_{x}}\\
{v_{y}}
\end{bmatrix}$ and the acceleration vector as $\mathbf{a}=\begin{bmatrix}
{a_{x}}\\
{a_{y}}
\end{bmatrix}$. The acceleration is assumed to be constant (\hyperref[constAccel]{A: constAccel}) and the constant acceleration is represented as ${a^{c}}=\begin{bmatrix}
{{a_{x}}^{c}}\\
{{a_{y}}^{c}}
\end{bmatrix}$. The initial velocity (at $t=0$) is represented by ${\mathbf{v}^{i}}=\begin{bmatrix}
{{v_{x}}^{i}}\\
{{v_{y}}^{i}}
\end{bmatrix}$. Since we have a Cartesian coordinate system, \hyperref[GD:rectVel]{GD: rectVel} can be applied to each coordinate direction, to yield:
\begin{displaymath}
\mathbf{v}=\begin{bmatrix}
{v_{x}}\\
{v_{y}}
\end{bmatrix}=\begin{bmatrix}
{{v_{x}}^{i}}+{{a_{x}}^{c}} t\\
{{v_{y}}^{i}}+{{a_{y}}^{c}} t
\end{bmatrix}
\end{displaymath}
\par~

\noindent \begin{minipage}{\textwidth}
\begin{tabular}{p{0.2\textwidth} p{0.73\textwidth}}
\toprule \textbf{Refname} & \textbf{GD:posVec}
\phantomsection 
\label{GD:posVec}
\\ \midrule \\
Label & Position vector as a function of time
\\ \midrule \\
Units & m
\\ \midrule \\
Equation & \begin{displaymath}
           \mathbf{p}=\begin{bmatrix}
{{p_{x}}^{i}}+{{v_{x}}^{i}} t+\frac{{{a_{x}}^{c}} t^{2}}{2}\\
{{p_{y}}^{i}}+{{v_{y}}^{i}} t+\frac{{{a_{y}}^{c}} t^{2}}{2}
\end{bmatrix}
           \end{displaymath}
\\ \midrule \\
Description & \begin{symbDescription}
              \item{$\mathbf{p}$ is the position (m)}
              \item{${{p_{x}}^{i}}$ is the x-component of initial position ($\frac{\text{m}}{\text{s}}$)}
              \item{${{v_{x}}^{i}}$ is the x-component of initial velocity ($\frac{\text{m}}{\text{s}}$)}
              \item{$t$ is the time (s)}
              \item{${{a_{x}}^{c}}$ is the x-component of constant acceleration ($\frac{\text{m}}{\text{s}^{2}}$)}
              \item{${{p_{y}}^{i}}$ is the y-component of initial position ($\frac{\text{m}}{\text{s}}$)}
              \item{${{v_{y}}^{i}}$ is the y-component of initial velocity ($\frac{\text{m}}{\text{s}}$)}
              \item{${{a_{y}}^{c}}$ is the y-component of constant acceleration ($\frac{\text{m}}{\text{s}^{2}}$)}
              \end{symbDescription}
\\ \midrule \\
Notes & 
\\ \midrule \\
Source & --
\\ \midrule \\
RefBy & 
\\ \bottomrule \end{tabular}
\end{minipage}
Detailed derivation of position vector:
For a two-dimensional Cartesian coordinate system (\hyperref[twoDMotion]{A: twoDMotion} and \hyperref[cartSyst]{A: cartSyst}), we can represent the position vector as $\mathbf{p}=\begin{bmatrix}
{p_{x}}\\
{p_{y}}
\end{bmatrix}$, the velocity vector as $\mathbf{v}=\begin{bmatrix}
{v_{x}}\\
{v_{y}}
\end{bmatrix}$ and the acceleration vector as $\mathbf{a}=\begin{bmatrix}
{a_{x}}\\
{a_{y}}
\end{bmatrix}$. The acceleration is assumed to be constant (\hyperref[constAccel]{A: constAccel}) and the constant acceleration is represented as ${a^{c}}=\begin{bmatrix}
{{a_{x}}^{c}}\\
{{a_{y}}^{c}}
\end{bmatrix}$. The initial velocity (at $t=0$) is represented by ${\mathbf{v}^{i}}=\begin{bmatrix}
{{v_{x}}^{i}}\\
{{v_{y}}^{i}}
\end{bmatrix}$. Since we have a Cartesian coordinate system, \hyperref[GD:rectPos]{GD: rectPos} can be applied to each coordinate direction, to yield:
\begin{displaymath}
\mathbf{p}=\begin{bmatrix}
{p_{x}}\\
{p_{x}}
\end{bmatrix}=\begin{bmatrix}
{{p_{x}}^{i}}+{{v_{x}}^{i}} t+\frac{{{a_{x}}^{c}} t^{2}}{2}\\
{{p_{y}}^{i}}+{{v_{y}}^{i}} t+\frac{{{a_{y}}^{c}} t^{2}}{2}
\end{bmatrix}
\end{displaymath}
\par~

\noindent \begin{minipage}{\textwidth}
\begin{tabular}{p{0.2\textwidth} p{0.73\textwidth}}
\toprule \textbf{Refname} & \textbf{GD:airTime}
\phantomsection 
\label{GD:airTime}
\\ \midrule \\
Label & Air time
\\ \midrule \\
Units & s
\\ \midrule \\
Equation & \begin{displaymath}
           t=\frac{2 {v_{i}} \sin\left(θ\right)}{{a_{y}}}
           \end{displaymath}
\\ \midrule \\
Description & \begin{symbDescription}
              \item{$t$ is the time (s)}
              \item{${v_{i}}$ is the initial speed ($\frac{\text{m}}{\text{s}}$)}
              \item{$θ$ is the launch angle (${}^{\circ}$)}
              \item{${a_{y}}$ is the y-component of acceleration ($\frac{\text{m}}{\text{s}^{2}}$)}
              \end{symbDescription}
\\ \midrule \\
Notes & 
\\ \midrule \\
Source & --
\\ \midrule \\
RefBy & \hyperref[GD:distanceRefined]{GD: distanceRefined}.
\\ \bottomrule \end{tabular}
\end{minipage}
Air time is derived from \hyperref[DD:speedY]{DD: speedY}. and \hyperref[GD:rectVel]{GD: rectVel} It also comes from the fact that the y-component of velocity the at the maximum height is zero and that the maximum height is halfway point of the trajectory (from \hyperref[launchOrigin]{A: launchOrigin} and \hyperref[targetXAxis]{A: targetXAxis}).
\par~

\noindent \begin{minipage}{\textwidth}
\begin{tabular}{p{0.2\textwidth} p{0.73\textwidth}}
\toprule \textbf{Refname} & \textbf{GD:distance}
\phantomsection 
\label{GD:distance}
\\ \midrule \\
Label & Distance in the x-direction
\\ \midrule \\
Units & m
\\ \midrule \\
Equation & \begin{displaymath}
           {r_{x}}={{v_{x}}^{i}} t+\frac{{a_{x}} t^{2}}{2}
           \end{displaymath}
\\ \midrule \\
Description & \begin{symbDescription}
              \item{${r_{x}}$ is the distance in the x-direction (m)}
              \item{${{v_{x}}^{i}}$ is the x-component of initial velocity ($\frac{\text{m}}{\text{s}}$)}
              \item{$t$ is the time (s)}
              \item{${a_{x}}$ is the x-component of acceleration ($\frac{\text{m}}{\text{s}^{2}}$)}
              \end{symbDescription}
\\ \midrule \\
Notes & \hyperref[accelYGravity]{A: accelYGravity}
\\ \midrule \\
Source & --
\\ \midrule \\
RefBy & \hyperref[GD:distanceRefined]{GD: distanceRefined}.
\\ \bottomrule \end{tabular}
\end{minipage}
\par~

\noindent \begin{minipage}{\textwidth}
\begin{tabular}{p{0.2\textwidth} p{0.73\textwidth}}
\toprule \textbf{Refname} & \textbf{GD:distanceRefined}
\phantomsection 
\label{GD:distanceRefined}
\\ \midrule \\
Label & Distance in the x-direction (refined)
\\ \midrule \\
Units & m
\\ \midrule \\
Equation & \begin{displaymath}
           {r_{x}}=\frac{2 {v_{i}}^{2} \sin\left(θ\right) \cos\left(θ\right)}{{a_{y}}}
           \end{displaymath}
\\ \midrule \\
Description & \begin{symbDescription}
              \item{${r_{x}}$ is the distance in the x-direction (m)}
              \item{${v_{i}}$ is the initial speed ($\frac{\text{m}}{\text{s}}$)}
              \item{$θ$ is the launch angle (${}^{\circ}$)}
              \item{${a_{y}}$ is the y-component of acceleration ($\frac{\text{m}}{\text{s}^{2}}$)}
              \end{symbDescription}
\\ \midrule \\
Notes & \hyperref[accelXZero]{A: accelXZero}
\\ \midrule \\
Source & --
\\ \midrule \\
RefBy & 
\\ \bottomrule \end{tabular}
\end{minipage}
Distance in the x-direction (refined) is derived from \hyperref[GD:airTime]{GD: airTime} and \hyperref[GD:distance]{GD: distance}.
\subsubsection{Data Definitions}
\label{Sec:DDs}
This section collects and defines all the data needed to build the instance models.
\par~

\noindent \begin{minipage}{\textwidth}
\begin{tabular}{p{0.2\textwidth} p{0.73\textwidth}}
\toprule \textbf{Refname} & \textbf{DD:speedX}
\phantomsection 
\label{DD:speedX}
\\ \midrule \\
Label & X-component of velocity
\\ \midrule \\
Symbol & ${v_{x}}$
\\ \midrule \\
Units & $\frac{\text{m}}{\text{s}}$
\\ \midrule \\
Equation & \begin{displaymath}
           {v_{x}}=v \cos\left(θ\right)
           \end{displaymath}
\\ \midrule \\
Description & \begin{symbDescription}
              \item{${v_{x}}$ is the x-component of velocity ($\frac{\text{m}}{\text{s}}$)}
              \item{$v$ is the speed ($\frac{\text{m}}{\text{s}}$)}
              \item{$θ$ is the launch angle (${}^{\circ}$)}
              \end{symbDescription}
\\ \midrule \\
Source & --
\\ \midrule \\
RefBy & 
\\ \bottomrule \end{tabular}
\end{minipage}
\par~

\noindent \begin{minipage}{\textwidth}
\begin{tabular}{p{0.2\textwidth} p{0.73\textwidth}}
\toprule \textbf{Refname} & \textbf{DD:speedY}
\phantomsection 
\label{DD:speedY}
\\ \midrule \\
Label & Y-component of velocity
\\ \midrule \\
Symbol & ${v_{y}}$
\\ \midrule \\
Units & $\frac{\text{m}}{\text{s}}$
\\ \midrule \\
Equation & \begin{displaymath}
           {v_{y}}=v \sin\left(θ\right)
           \end{displaymath}
\\ \midrule \\
Description & \begin{symbDescription}
              \item{${v_{y}}$ is the y-component of velocity ($\frac{\text{m}}{\text{s}}$)}
              \item{$v$ is the speed ($\frac{\text{m}}{\text{s}}$)}
              \item{$θ$ is the launch angle (${}^{\circ}$)}
              \end{symbDescription}
\\ \midrule \\
Source & --
\\ \midrule \\
RefBy & \hyperref[GD:airTime]{GD: airTime}.
\\ \bottomrule \end{tabular}
\end{minipage}
\subsubsection{Instance Models}
\label{Sec:IMs}
This section transforms the problem defined in \hyperref[Sec:ProbDesc]{Section: Problem Description} into one which is expressed in mathematical terms. It uses concrete symbols defined in \hyperref[Sec:DDs]{Section: Data Definitions} to replace the abstract symbols in the models identified in \hyperref[Sec:TMs]{Section: Theoretical Models} and \hyperref[Sec:GDs]{Section: General Definitions}.
\par~

\noindent \begin{minipage}{\textwidth}
\begin{tabular}{p{0.2\textwidth} p{0.73\textwidth}}
\toprule \textbf{Refname} & \textbf{IM:shortIM}
\phantomsection 
\label{IM:shortIM}
\\ \midrule \\
Label & IsShort
\\ \midrule \\
Input & ${r_{target}}$, ${r_{x}}$
\\ \midrule \\
Output & $isShort$
\\ \midrule \\
Input Constraints & \begin{displaymath}
                    {r_{target}}>0
                    \end{displaymath}
                    \begin{displaymath}
                    {r_{x}}>0
                    \end{displaymath}
\\ \midrule \\
Output Constraints & 
\\ \midrule \\
Equation & \begin{displaymath}
           isShort={r_{target}}>{r_{x}}
           \end{displaymath}
\\ \midrule \\
Description & \begin{symbDescription}
              \item{$isShort$ is the variable that is assigned true when the target distance is greater than the launch distance (Unitless)}
              \item{${r_{target}}$ is the target distance (m)}
              \item{${r_{x}}$ is the launch distance (m)}
              \end{symbDescription}
\\ \midrule \\
Notes & 
\\ \midrule \\
Source & --
\\ \midrule \\
RefBy & 
\\ \bottomrule \end{tabular}
\end{minipage}
\par~

\noindent \begin{minipage}{\textwidth}
\begin{tabular}{p{0.2\textwidth} p{0.73\textwidth}}
\toprule \textbf{Refname} & \textbf{IM:offsetIM}
\phantomsection 
\label{IM:offsetIM}
\\ \midrule \\
Label & Offset
\\ \midrule \\
Input & ${r_{target}}$, ${r_{x}}$
\\ \midrule \\
Output & ${r_{offset}}$
\\ \midrule \\
Input Constraints & \begin{displaymath}
                    {r_{target}}>0
                    \end{displaymath}
                    \begin{displaymath}
                    {r_{x}}>0
                    \end{displaymath}
\\ \midrule \\
Output Constraints & 
\\ \midrule \\
Equation & \begin{displaymath}
           {r_{offset}}=|{r_{target}}-{r_{x}}|
           \end{displaymath}
\\ \midrule \\
Description & \begin{symbDescription}
              \item{${r_{offset}}$ is the offset between the target distance and the launch distance (m)}
              \item{${r_{target}}$ is the target distance (m)}
              \item{${r_{x}}$ is the launch distance (m)}
              \end{symbDescription}
\\ \midrule \\
Notes & 
\\ \midrule \\
Source & --
\\ \midrule \\
RefBy & 
\\ \bottomrule \end{tabular}
\end{minipage}
\par~

\noindent \begin{minipage}{\textwidth}
\begin{tabular}{p{0.2\textwidth} p{0.73\textwidth}}
\toprule \textbf{Refname} & \textbf{IM:hitIM}
\phantomsection 
\label{IM:hitIM}
\\ \midrule \\
Label & IsHit
\\ \midrule \\
Input & ${r_{offset}}$, ${r_{target}}$
\\ \midrule \\
Output & $isHit$
\\ \midrule \\
Input Constraints & \begin{displaymath}
                    {r_{offset}}>0
                    \end{displaymath}
                    \begin{displaymath}
                    {r_{target}}>0
                    \end{displaymath}
\\ \midrule \\
Output Constraints & 
\\ \midrule \\
Equation & \begin{displaymath}
           isHit={r_{offset}}<\frac{1}{50} {r_{target}}
           \end{displaymath}
\\ \midrule \\
Description & \begin{symbDescription}
              \item{$isHit$ is the variable that is assigned true when the launch distance is within a degree of tolerance of the target distance (Unitless)}
              \item{${r_{offset}}$ is the offset between the target distance and the launch distance (m)}
              \item{${r_{target}}$ is the target distance (m)}
              \end{symbDescription}
\\ \midrule \\
Notes & 
\\ \midrule \\
Source & --
\\ \midrule \\
RefBy & 
\\ \bottomrule \end{tabular}
\end{minipage}
\end{document}
