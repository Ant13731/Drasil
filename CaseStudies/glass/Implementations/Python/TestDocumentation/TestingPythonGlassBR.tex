\documentclass[12pt]{article}
\usepackage[legalpaper, landscape, margin=0.5in]{geometry}
\usepackage{adjustbox}
\usepackage{graphicx}
\begin{document}
\title{Testing GlassBR}
\maketitle
%Table 1: testCalculations
\begin{table}[h!]
\centering
\caption{testCalculations}
\label{testCalculations}
\begin{adjustbox}{max width=\textwidth}
\begin{tabular}{*{10}{|c}}
\hline
\textbf{Ref} & \textbf{Test Name} & \textbf{fileName.py} & \textbf{Test Purpose} & \textbf{Traceability} & \textbf{Input File} & \textbf{Significant Input} & \textbf{Expected Output} & \textbf{Notes} \\
\hline
\hline
1 & Tst\_Pb\_DefaultValues & testCalculations & to make sure expected pb values is returned & uses equations from DD1's B and IM1's Pb & defaultInput.txt & see Input File & 'For the given input parameters, the glass is considered safe' & Improve: instead of equality of floats (assertEqual), should use some epsilon error 
\\
2 & Tst\_Pb\_SmallDimensionValues & testCalculations2 & " & " & testInput1.txt & " & " & " 
\\
3 & Tst\_Pb\_LargeDimensionValues & testCalculations3 & " & " & testInput2.txt & " & " & " 
\\
4 & Tst\_Pb\_LowPbTol & testCalculations4 & " & " & testInput3.txt & " & " & " 
\\
5 & Tst\_Pb\_DiffSDValues & testCalculations5 & " & " & testInput4.txt & " & " & " 
\\
6 & Tst\_Pb\_HighChgWght & testCalculations6 & " & " & testInput5.txt & " & " & " 
\\
7 & Tst\_Pb\_LowThickness & testCalculations7 & " & " & testInput6.txt & " & " & " 
\\
\hline
\end{tabular}
\end{adjustbox}
\end{table}
%Table 2: testCheckConstraints
\begin{table}[h!]
\centering
\caption{testCheckConstraints}
\label{testCheckConstraints}
\begin{adjustbox}{max width=\textwidth}
\begin{tabular}{*{10}{|c}}
\hline
\textbf{Ref} & \textbf{Test Name} & \textbf{fileName.py} & \textbf{Test Purpose} & \textbf{Traceability} & \textbf{Input File} & \textbf{Significant Input} & \textbf{Expected Output} & \textbf{Notes} \\
\hline
\hline
8 & checkAPositiveTest & testCheckConstraints & to ensure a (i.e. length) \textgreater 0 & Following A1 (glass must be of rectangular shape); following physical constraint from Table 2 where “a \textgreater 0” and software constraint from Table 2 where “a =\textgreater {d\textsubscript{min}}” & testInvalidInput1.txt  & a = -1600 & InputError: a and b must be greater than 0 & 
\\
9 & checkBPositiveTest & testCheckConstraints2 & to ensure b (i.e. breadth) \textgreater 0 & Following physical constraint from Table 2 where “b \textgreater 0” and software constraint from Table 2 where “b =\textgreater {d\textsubscript{min}}” & testInvalidInput2.txt & b = -1500 & InputError: a and b must be greater than 0 &
\\
10 & checkSmallAspectRTest & testCheckConstraints3 & to ensure 1 \textless a/b \textless 5 & length should pertain to the longer side, following physical constraint from Table 2 where “b \textless a” & testInvalidInput3.txt  & b = 2000 & (a/b=0.8\textless1); InputError: a/b must be between 1 and 5 &
\\
11 & checkLargeAspectRTest & testCheckConstraints4 & to ensure a/b (i.e. aspect ratio) \textless 5 & following software constraint from Table 2 where “a/b \textless {AR\textsubscript{max}}” & testInvalidInput4.txt & b = 200 & (a/b=8\textgreater5); InputError: a/b must be between 1 and 5 & 
\\
12 & checkValidThicknessTest & testCheckConstraints5 & to ensure input t value (i.e. nominal thickness) is one of the industrial standard thicknesses & following R1 (t description) & testInvalidInput5.txt & t = 7 & InputError: t must be in {[}2.5,2.7,3.0,4.0,,5.0,6.0,8.0, 10.0,12.0,16.0,,19.0,22.0{]} & 
\\
13 & checkLowerConstrOnWTest & testCheckConstraints6 & to ensure input w value (i.e. weight of charge) is \textgreater minimum permissible input charge weight & following value of {w\textsubscript{min}} (4.5 kg) from Table 3 & testInvalidInput6.txt & w = 3 & InputError: wtnt must be between 4.5 and 910 &
\\
14 & checkUpperConstrOnWTest & testCheckConstraints7 & to ensure input w value (i.e. weight of charge) is \textless maximum permissible input charge weight & following value of {w\textsubscript{max}} (910 kg) from Table 3 & testInvalidInput7.txt & w = 1000 & InputError: wtnt must be between 4.5 and 910 & 
\\
15 & checkTNTPositiveTest & testCheckConstraints8 & to ensure input tnt value (i.e. TNT equivalent factor) \textgreater 0 & following physical constraint from Table 2 where “TNT \textgreater 0” & testInvalidInput8.txt & tnt = -2 & InputError: TNT must be greater than 0 &
\\
16 & checkLowerConstrOnSDTest & testCheckConstraints9 & to see if input SD (i.e. Stand off Distance) is \textgreater minimum stand off distance permissible for input & following value of {SD\textsubscript{min}} (6 m) from Table 3 & testInvalidInput9.txt & sdx = 0; sdy = 1.0; sdz = 2.0 & InputError: SD must be between 6 and 130 &
\\
17 & checkUpperConstrOnSDTest & testCheckConstraints10 & to see if input SD (i.e. Stand off Distance) is \textless maximum stand off distance permissible for input & following value of {SD\textsubscript{max}} (130 m) from Table 3 & testInvalidInput10.txt & sdx = 0; sdy = 200; sdz = 100 & InputError: SD must be between 6 and 130 &
\\
18 & incorrectA0Test & testCheckConstraints11 & see 8 & see 8 & testInvalidInput11.txt & a = 0 & InputError: a and b must be greater than 0 &
\\
19 & incorrectB0Test & testCheckConstraints12 & see 9 & see 9 & testInvalidInput12.txt & b = 0 & InputError: a and b must be greater than 0 & RuntimeWarning: divide by zero encountered in double\_scalars params.asprat = params.a /params.b
\\
20 & incorrectTNT0Test & testCheckConstraints13 & see 15 & see 15 & testInvalidInput13.txt & tnt = 0 & InputError: TNT must be greater than 0 &
\\
21 & incorrectAspectREqLwrBndTest & testCheckConstraints14 & see 10 & see 10 & testInput7.txt & a = 1500; b = 1500 & (a/b = 1);  "Encountered an unexpected exception" à why not the same error as 10? &
\\
22 & incorrectAspectREqUpprBndTest & testCheckConstraints15 & see 11 & see 11 & testInput8.txt & a = 7500; b = 1500 & (a/b = 5); "Encountered an unexpected exception" &
\\
23 incorrectWEqLwrBndTest & testCheckConstraints16 & see 13 & see 13 & testInput9.txt & w = 4.5 & "Encountered an unexpected exception" & 
\\
24 & incorrectWEqUpprBndTest & testCheckConstraints17 & see 14 & see 14 & testInput10.txt & w = 910 & "Encountered an unexpected exception" & 
\\
25 & ? & testCheckConstraints18 & " & " & " & " & REMOVE? Or was it supposed to follow the pattern and have tnt = 0? like \#15 &
\\
26 & incorrectSDEqLwrBndTest & testCheckConstraints19 & see 16 & see 16 & testInput11.txt & sdx = 0; sdy = 6; sdz = 0 & "Encountered an unexpected exception" & 
\\
27 & incorrectWEqUpprBndTest & testCheckConstraints20 & see 17 & see 17 & testInput12.txt & sdx = 130; sdy = 0; sdz = 0 & "Encountered an unexpected exception" & 
\\
\hline
\end{tabular}
\end{adjustbox}
\end{table}
%Table 3: testDerivedValues
\begin{table}[h!]
\centering
\caption{testDerivedValues}
\label{testDerivedValues}
\begin{adjustbox}{max width=\textwidth}
\begin{tabular}{*{10}{|c}}
\hline
\textbf{Ref} & \textbf{Test Name} & \textbf{fileName.py} & \textbf{Test Purpose} & \textbf{Traceability} & \textbf{Input File} & \textbf{Significant Input} & \textbf{Expected Output} & \textbf{Notes} \\
\hline
\hline
28 & TstDrvdVals_HSGlTy & testDerivedValues & to ensure initial inputs have been correctly converted into derived quantities & following term definitions and equations from Data Definitions & defaultInput.txt & see Input File & - &
\\
29 & TstDrvdVals_ANGlTy & testDerivedValues2 & " & " & testInput1.txt & " & - & 
\\
30 & TstDrvdVals_FTGlTy & testDerivedValues3 & " & " & testInput2.txt & " & - &
\\               
\hline
\end{tabular}
\end{adjustbox}
\end{table}
%Table 4: testInputFormat
\begin{table}[h!]
\centering
\caption{testInputFormat}
\label{testInputFormat}
\begin{adjustbox}{max width=\textwidth}
\begin{tabular}{*{10}{|c}}
\hline
\textbf{Ref} & \textbf{Test Name} & \textbf{fileName.py} & \textbf{Test Purpose} & \textbf{Traceability} & \textbf{Input File} & \textbf{Significant Input} & \textbf{Expected Output} & \textbf{Notes} \\
\hline
\hline
31 & TstInFmt_1 & testInputFormat & to ensure data is being read in from the input file correctly & - & defaultInput.txt & see Input File & - &
\\
32 & TstInFmt_2 & testInputFormat2 & " & - & testInput1.txt & " & - & 
\\
33 & TstInFmt_3 & testInputFormat3 & " & - & testInput2.txt & " & - &               
\\               
\hline
\end{tabular}
\end{adjustbox}
\end{table}
%Table 5: testInterp
\begin{table}[h!]
\centering
\caption{testInterp}
\label{testInterp}
\begin{adjustbox}{max width=\textwidth}
\begin{tabular}{*{10}{|c}}
\hline
\textbf{Ref} & \textbf{Test Name} & \textbf{fileName.py} & \textbf{Test Purpose} & \textbf{Traceability} & \textbf{Input File} & \textbf{Significant Input} & \textbf{Expected Output} & \textbf{Notes} \\
\hline
\hline
34 & ? & testInterp & to ensure interpolated values are correctly calculated & - & - & x in {[}x1, x2{]} & & 
\\
35 & ? & testInterp2 & " & - & - & x in {[}x1, x2{]},where x is a Float & & 
\\
36 & ? & testInterp3 & " & - & - & x1 \textless x2 \textless x & &  
\\
37 & ? & testInterp4 & " & - & - & x \textless x2 \textless x1 & &  
\\
38 & ? & testInterp5 & " & - & - & negative, x \textless x2 \textless x1 & &  
\\
39 & ? & testInterp6 & " & - & - & negative, x in {[}x1, x2{]} & &  
\\
40 & ? & testInterp7 & " & - & - & negative, x1 \textless x2 \textless x & &  
\\
41 & ? & testInterp8 & " & - & - & value1 in data1 and jdx == 0? & &  
\\
42 & ? & testInterp9 & " & - & - & value1 in data1 and jdx == 0 and data2{[}jdx, idx{]} == data2{[}jdx+1, idx{]} & &  
\\
43 & ? & testInterp10 & " & - & - & value1 in data1 and data2{[}jdx, idx{]} == data2{[}jdx+1, idx{]} and value2 \textgreater data2{[}jdx, idx{]} & &  
\\
44 & ? & testInterp11 & " & - & - & value1 in data1 and data2{[}jdx, idx{]} == data2{[}jdx+1, idx{]} and value2 \textless data2{[}jdx, idx{]} & &  
\\
45 & ? & testInterp12 & " & - & - & value1 in data1 and data2{[}jdx, idx{]} == data2{[}jdx+1, idx{]} and jdx+1 == (data2{[}:, idx{]}).argmax and value2 \textgreater data2{[}jdx, idx{]} & &  
\\
46 & ? & testInterp13 & " & - & - & value1 not in data1 and value2 in data2{[}:, idx{]} and kdx == 0 & &  
\\
47 & ? & testInterp14 & " & - & - & value1 not in data1 and value2 in data2{[}:, idx{]} and data2{[}:, idx+1{]} and data2{[}kdx, idx+1{]} = data2 {[}kdx+1, idx+1{]} & &  
\\
\hline
\end{tabular}
\end{adjustbox}
\end{table}
%Table 6: testMainFun
\begin{table}[h!]
\centering
\caption{testMainFun}
\label{testMainFun}
\begin{adjustbox}{max width=\textwidth}
\begin{tabular}{*{10}{|c}}
\hline
\textbf{Ref} & \textbf{Test Name} & \textbf{fileName.py} & \textbf{Test Purpose} & \textbf{Traceability} & \textbf{Input File} & \textbf{Significant Input} & \textbf{Expected Output} & \textbf{Notes} \\
\hline
\hline
48 & TstMain_1 & testMainFun & to determine if the main program produces the correct output & coordinates the running of the program & defaultInput.txt & see Input File & outputfile.txt = output.txt &
\\
49 & TstMain_2 & testMainFun2 & " & " & testInput1.txt & " & outputfile.txt = output1.txt &
\\
50 & TstMain_3 & testMainFun3 & " & " & testInput2.txt & " & outputfile.txt = output2.txt &
\\
51 & TstMain_4 & testMainFun4 & " & " & testInput3.txt & " & outputfile.txt = output3.txt &
\\
52 & TstMain_5 & testMainFun5 & " & " & testInput4.txt & " & outputfile.txt = output4.txt &
\\
53 & TstMain_6 & testMainFun6 & " & " & testInput5.txt & " & outputfile.txt = output5.txt &
\\
54 & TstMain_7 & testMainFun7 & " & " & testInput6.txt & " & outputfile.txt = output6.txt &
\\
\hline
\end{tabular}
\end{adjustbox}
\end{table}
%Table 7: testOutputFormat
\begin{table}[h!]
\centering
\caption{testOutputFormat}
\label{testOutputFormat}
\begin{adjustbox}{max width=\textwidth}
\begin{tabular}{*{10}{|c}}
\hline
\textbf{Ref} & \textbf{Test Name} & \textbf{fileName.py} & \textbf{Test Purpose} & \textbf{Traceability} & \textbf{Input File} & \textbf{Significant Input} & \textbf{Expected Output} & \textbf{Notes} \\
\hline
\hline
55 & TstOutFmt_1 & testOutputFormat & to ensure structure of output data matches desired output formatting & - & defaultInput.txt & see Input File & testoutput.txt = output.txt &
\\
56 & TstOutFmt_2 & testOutputFormat2 & " & - & testInput3.txt & " & testoutput.txt = output3.txt &
\\
57 & TstOutFmt_3 & testOutputFormat3 & " & - & testInput4.txt & " & testoutput.txt = output4.txt &
\\
\hline
\end{tabular}
\end{adjustbox}
\end{table}
%Table 8: testReadTable
\begin{table}[h!]
\centering
\caption{testReadTable}
\label{testReadTable}
\begin{adjustbox}{max width=\textwidth}
\begin{tabular}{*{10}{|c}}
\hline
\textbf{Ref} & \textbf{Test Name} & \textbf{fileName.py} & \textbf{Test Purpose} & \textbf{Traceability} & \textbf{Input File} & \textbf{Significant Input} & \textbf{Expected Output} & \textbf{Notes} \\
\hline
\hline
59 & ? & testReadTable & & - & testTable1.txt & see Input File & &
\\
60 & ? & testReadTable2 & " & - & testTable2.txt & " & &
\\
\hline     
\end{tabular}
\end{adjustbox}
\end{table}
\end{document}
