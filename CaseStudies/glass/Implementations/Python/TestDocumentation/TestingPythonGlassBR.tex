\documentclass[12pt]{article}
\usepackage[legalpaper, landscape, margin=0.5in]{geometry}
\usepackage{adjustbox}
\usepackage{graphicx}
\begin{document}
\title{Testing GlassBR}
\maketitle
%\begin{table}[]
%\caption{testCalculations}
%\label{testCalculations}
%\begin{tabular}{l|l|l|l|l|l|l|l|l|l}
%Ref & Test Name & fileName.py & Test Purpose & Traceability & Input File & %Significant Input & Expected Output & Notes \\
%1 & ? & testCalculations & to make sure returns expected value of pb & %function uses equations from DD1's B and IM1's Pb & defaultInput.txt & see %Input File & `For the given input parameters, the glass is considered safe' %& Improve: instead of equality of floats (assertEqual), should use some %epsilon error \\
%\end{tabular}
%\end{table}
\begin{table}[h!]
\centering
\caption{testCalculations}
\label{testCalculations}
\begin{adjustbox}{max width=\textwidth}
\begin{tabular}{*{10}{|c}}
\hline
\textbf{Ref} & \textbf{Test Name} & \textbf{fileName.py} & \textbf{Test Purpose} & \textbf{Traceability} & \textbf{Input File} & \textbf{Significant Input} & \textbf{Expected Output} & \textbf{Notes} \\
\hline
\hline
1 & ? & testCalculations & to make sure expected pb values is returned & uses equations from DD1's B and IM1's Pb & defaultInput.txt & see Input File & 'For the given input parameters, the glass is considered safe' & Improve: instead of equality of floats (assertEqual), should use some epsilon error \\
2 & ? & testCalculations2 & " & " & testInput1.txt & " & " & " \\
3 & ? & testCalculations3 & " & " & testInput2.txt & " & " & " \\
4 & ? & testCalculations4 & " & " & testInput3.txt & " & " & " \\
5 & ? & testCalculations5 & " & " & testInput4.txt & " & " & " \\
6 & ? & testCalculations6 & " & " & testInput5.txt & " & " & " \\
7 & ? & testCalculations7 & " & " & testInput6.txt & " & " & " \\
\hline
\end{tabular}
\end{adjustbox}
\end{table}
\end{document}
